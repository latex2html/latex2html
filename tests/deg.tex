\documentclass[10pt]{article}
\usepackage[T1]{fontenc}
\usepackage[utf8]{inputenc}
\usepackage[english]{babel}
\usepackage{gensymb}

% Testing math picture generation
%
% pdflatex test (may be do it twice to be sure)
%
% latex2html -html_version 4.0 -use_pdftex -use_dvipng -image_type gif -local_icons test.tex
% In index.html you will see '180 degree' as an embedded image in the first line
% and something like '>_._' as an image around the last line.
%
% rm -rf test
% latex2html -html_version 4.0,math -use_pdftex -use_dvipng -image_type gif -local_icons test.tex
% Now in the first line '180' will be text, and the degree symbol will be
% not near the top of a digit, but at the baseline at the bottom.
% In the last line both underscores will be invisible. There is underscore
% in the corresponding inlined image, but it is totally transparent.
%
% The underscores appear normally with either of the following options:
% \documentclass[12pt]{article} (instead of [10pt])
% -image_type png (instead of gif)
% -notransparent (but -noantialias and -noantialias_text do not help)
%
% -nouse_dvipng (with $MATH_SCALE_FACTOR=1.6) gives good aligned images.
% latex2html version 2018.2 with -use_dvipng gives good aligned images.
% latex2html version 2020.2 corrected with patch gives good aligned images.

\begin{document}

This effect acts only in one direction which can vary up to an angle
of $180\degree$ with these parameters:

\begin{description}
\item[Length:] distance between original image and final blur step;
  corresponds to the distance of the fields.
\item[Angle:] angle of motion in one direction for linear blur
\item[Steps:] number of blur steps to be used in the calculation.
  Increasing the number takes more CPU.
\item[Channels:] R,G,B,A.
\item[Clear] With the Clear buttons we can bring the slider to default
  values without affecting the other parameters.
\end{description}

In order to provide some types of help, the Menu Bar Shell Commands
are available for customization purposes. In the main window on the
top line containing the \textit{File}, \textit{Edit}, {\dots}
\textit{Window} pulldown menus, all the way to the right hand side is
the \textit{shell cmds} icon. You might see a small gold-color
bordered box with the $>\_.\_$ inside and if you mouse over it, the
tooltip says \textit{shell cmds}.

\end{document}
