\begin{htmlonly}
\documentclass[dvips,a4paper]{article}
\usepackage{html,htmllist,makeidx}

%this font is used in  features.tex
\font\wncyr = wncyr10


\begin{htmlonly}
 \usepackage[dvips]{graphicx}
 \usepackage[dvips]{color}
% \usepackage{frames}
 \usepackage[dvips,leftbars]{changebar}
 \newcommand{\FoilTeX}{\env{FoilTeX}}
%
% \newcommand{\htmlurl}[1]{\htmladdnormallink{#1}{#1}}
 \def\Email#1{\htmladdnormallink{<#1>}{mailto:#1}}
 \def\path#1{\texttt{#1}}
% \def\urldef#1#2#3{\def#1{#2{#3}}}
% \newcommand{\onlinedoc}{\url{http://www-dsed.llnl.gov/files/programs/unix/latex2html/manual/}}
 \newcommand{\onlinedoc}{\url{http://www-texdev.ics.mq.edu.au/l2h/docs/manual/}}
% \def\patches{\htmlurl{http://www-dsed.llnl.gov/files/programs/unix/latex2html/}}
 \def\patches{\htmlurl{http://www.latex2html.org/current/}}
 \def\sourceA{%
  \url{http://www.latex2html.org/current/}}
%  \url{http://www-dsed.llnl.gov/files/programs/unix/latex2html/sources/}}
% \def\sourceB{\htmlurl{ftp://ftp.mpn.com/pub/nikos/latex2html-98.1.tar.gz}}
 \def\sourceB{\CTANtug{\CTANA}}
 \def\sourceC{\htmlurl{http://ftp.rzg.mpg.de/pub/software/latex2html/sources/}}
% \def\CVSsite{\url{http://cdc-server.cdc.informatik.tu-darmstadt.de/\~{}latex2html/}}
% \def\CVSrepos{\url{http://cdc-server.cdc.informatik.tu-darmstadt.de/\~{}latex2html/user/}}
% \def\CVSlatest{\url{http://cdc-server.cdc.informatik.tu-darmstadt.de/\~{}latex2html/l2h-latest.tar.gz}}
% \def\CVSsite{\htmlurl{http://saftsack.fs.uni-bayreuth.de/~latex2ht/}}
% \def\CVSrepos{\htmlurl{http://saftsack.fs.uni-bayreuth.de/~latex2ht/user/}}
 \def\CVSrepos{\htmlurl{https://www.github.com/latex2html/latex2html/}}
% \def\CVSlatest{\htmlurl{http://saftsack.fs.uni-bayreuth.de/~latex2ht/l2h-latest.tar.gz}}
 \def\CVSlatest{\htmlurl{https://www.github.com/latex2html/latex2html/archive/master.zip}}
 \newcommand{\CTANtug}[1]{\htmlurl{http://mirrors.ctan.org/#1}}
 \def\CTANA{support/latex2html}
 \def\tugURL{\htmlurl{http://www.tug.org/}}
 \def\danteURL{\htmlurl{http://www.dante.de/}}
% \def\ListURL{\htmlurl{http://www.xray.mpe.mpg.de/mailing-lists/latex2html/}}
 \def\ListURL{\htmlurl{http://www.tug.org/mailman/listinfo/latex2html/}}
%
% \def\glossary#1{\index{#1@\texttt{#1} \label{III#1}\htmlref{(G)}{GGG#1}}}
 \newcommand{\glossary}[1]{\index{#1@\texttt{#1} \label{III#1}\htmlref{(G)}{GGG#1}}}
 \def\Glossary#1#2{\index{#1@{#2} \label{III#1}\htmlref{(G)}{GGG#1}}}
% \newcommand{\Glossary}[2]{\index{#1@{#2} \label{III#1}\htmlref{(G)}{GGG#1}}}
%
% hack to suppress changebar entries read from .aux file  !!! bug in changebar.perl !!! 
 \makeatletter
 \def\cb@barpoint#1#2#3{}
 \makeatother
%
\newcommand{\sameas}[1]{\textcolor{red}{Same as setting: #1}}
\newcommand{\onlinedocRM}{\url{http://www-texdev.ics.mq.edu.au/l2h/docs/manual/}}
\newcommand{\EXcolors}{\url{http://www-texdev.ics.mq.edu.au/l2h/crayola/}}
%
\newcommand{\Lc}[1]{\texttt{\char92#1}} % LaTeX command
\newcommand{\Tc}[1]{\texttt{\char92#1}} % TeX command
\newcommand{\Cs}[1]{ \texttt{-#1} }   % command-line switch
\newcommand{\Ve}[1]{\index{#1@\texttt{#1}}\texttt{#1}} % version extension
\newcommand{\gsl}[1]{#1}
\newcommand{\indexentry}[2]{\item #1 #2}

%
\internal{}%
\internal{O}%
\internal{S}%
\internal{E}%
\internal{M}%
\internal{H}%
\internal{P}%

\newcommand{\texdev}[1]{\htmladdnormallink
 {\texdevURL/#1}{\texdevURL/#1}}
\newcommand{\ctan}{http://ctan.tug.org/ctan}
\newcommand{\ctanTUG}{\htmladdnormallinkfoot
 {TUG's searchable CTAN site}{\ctan}}
\newcommand{\ctanURL}[1]{\htmladdnormallink
 {CTAN:\texttt{.../#1}}{\ctan/#1}}
\newcommand{\ctanURLbr}[1]{\htmladdnormallink
 {CTAN:\texttt{.../#1}}{\ctan/#1}}
\newcommand{\ctanTUGurl}[2]{\ctanURL{#2}}
\newcommand{\indichtml}{l2h/indic/IndicHTML/}

\end{htmlonly}

\newcommand{\TUGurl}{\htmladdnormallink{TUG}{\htmlurl{http://www.tug.org}}}

\newcommand{\texdevURL}{http://www-texdev.ics.mq.edu.au/}
\newcommand{\Unicode}{\htmladdnormallinkfoot
 {Unicode}{http://www.unicode.org/}}
\newcommand{\IndicHTML}{
 {Indic\TeX/HTML}}

%begin{latexonly}
\usepackage{array}

\newcommand{\Cs}[1]{{\upshape`\,\texttt{-#1}\,'}}
\newcommand{\Ve}[1]{\index{#1@\texttt{#1}}{\upshape`\,\texttt{#1}\,'}}
\newcommand{\Lc}[1]{{\upshape\ttfamily\char92#1}}
\newcommand{\Tc}[1]{{\upshape\ttfamily\char92#1}}

\newcommand{\ctanTUG}[1]{TUG's searchable CTAN site\footnote{\ctanTUGurl}}
\newcommand{\ctanURL}{CTAN: \penalty-200\ctanurl}
\newcommand{\ctanURLbr}{CTAN: \newline\ctanurl}
\urldef\texdev\url{http://www-texdev.ics.mq.edu.au/}
\urldef\ctanTUGurl\url{http://mirrors.ctan.org/}
\urldef\indichtml\url{l2h/indic/IndicHTML/}

\urldef\NetscapeHome\url{http://home.netscape.com}
\urldef\giftransURL\url{http://ftp.ost.eltele.no/pub/multimedia/giftrans-1.12.2.tar.gz}
\urldef\WinZipURL\url{http://www.winzip.com/getsite.cgi?winzip70.exe}
\urldef\MikTeXURL\url{http://www.miktex.de}
\urldef\ActiveStateURL\url{http://www.activestate.com}
\urldef\NetpbmWinURL\url{ftp://ftp.ese-metz.fr/pub/TeX/win32/}
%end{latexonly}

\begin{htmlonly}
\newcommand{\NetscapeHome}{\url{http://home.netscape.com}}
\newcommand{\giftransURL}{\url{http://ftp.ost.eltele.no/pub/multimedia/giftrans-1.12.2.tar.gz}}
\newcommand{\WinZipURL}{\url{http://www.winzip.com/getsite.cgi?winzip70.exe}}
\newcommand{\MikTeXURL}{\url{http://www.miktex.de}}
\newcommand{\ActiveStateURL}{\url{http://www.activestate.com}}
\newcommand{\NetpbmWinURL}{\url{ftp://ftp.ese-metz.fr/pub/TeX/win32/}}
\end{htmlonly}



% thanks to \KrisRose for this macro:
%\def\smiley{\hbox{\rlap{$\bigcirc$}\kern1.3pt$\scriptstyle\ddot\smile\;$}}
\def\mathsmiley{\smiley\,}


\newcommand{\Lcs}[1]{{\upshape\ttfamily\char92#1}}

\renewcommand{\thefootnote}{\arabic{footnote}}

%\newcommand{\godown}[1]{{\htmlref
% {\htmladdimg[left BORDER=0]{../psfiles/dn.gif}}{#1}}}
\newcommand{\godown}[1]{}

%\newcommand{\goback}[1]{{\htmlref
% {\htmladdimg[left BORDER=0]{../psfiles/up.gif}}{#1}}}
\newcommand{\goback}[1]{}


\newcommand{\WiiiC}{\htmladdnormallinkfoot{World Wide Web Consortium}{http://www.w3c.org/}}
\newcommand{\latextohtmlNG}{\latextohtml-NG}
\newcommand{\maillist}{\htmladdnormallinkfoot{\latextohtml{} mailing list}%
{http://cbl.leeds.ac.uk/nikos/tex2html/doc/mail/mail.html}}
%
\newcommand{\HTMLiii}{\textup{\texttt{HTML} 3.2}}%
\newcommand{\HTMLIII}{\htmladdnormallink{\HTMLiii}%
{http://www.w3.org/TR/REC-html32.html}}%

\newcommand{\HTMLiv}{\textup{\texttt{HTML} 4.0}}%
\newcommand{\HTMLIV}{\htmladdnormallink{\HTMLiv}%
{http://www.w3.org/TR/REC-html40/}}%

%\newcommand{\latextohtml}{\textup{\LaTeX 2{\ttfamily HTML}}}%
\newcommand{\Perl}{\htmlref{\textsl{Perl}}{GGGPerl}}%  
\newcommand{\PS}{\htmlref{\textup{Post\-Script}}{GGGpostscript}\Glossary{PostScript}{PostScript}{}}%
\providecommand{\MF}{}
\renewcommand{\MF}{\htmlref{\textsl{Metafont}}{GGGMetafont}\Glossary{Metafont}{\textsl{Metafont}}{}}%
\newcommand{\fn}[1]{\htmlref{\texttt{#1}}{GGG#1}\glossary{#1}}%  file names, with link to glossary
\newcommand{\gn}[1]{\texttt{#1}\label{GGG#1}\htmlref{\^{}}{III#1}}%  file names, labelled within glossary
%\newcommand{\appl}[1]{\htmlref{\textsl{#1}}{GGG#1}\Glossary{#1}{\textsl{#1}}{}}%  application software names
\newcommand{\appl}[1]{\htmlref{\textsl{#1}}{GGG#1}\Glossary{#1}{\gsl{#1}}{}}%  application software names
%
\newcommand{\env}[1]{{\upshape\sffamily #1}}%  LaTeX environment and package names
\newcommand{\HTMLtag}[1]{\path{<#1>}}%  HTML tag
\newcommand{\Meta}[1]{\texttt{\upshape<\textit{#1}>}}%  Meta string

%
% developer names and addresses:
%
\newcommand{\NikosDrakos}{\index{Drakos, Nikos}%\Email{nikos@cbl.leeds.ac.uk}
{Nikos Drakos}}
%
\newcommand{\RossMoore}{\index{Moore, Ross}%\Email{ross@maths.mq.edu.au}
{Ross Moore}}
\newcommand{\Macquarie}{\htmladdnormallink
{Macquarie University}{http://www.mq.edu.au/}}
%
\newcommand{\Hennecke}{\index{Hennecke, Marcus}%\Email{hennecke@dbag.ulm.DaimlerBenz.COM}
{Marcus Hennecke}}
%
\newcommand{\Noworolski}{\index{Noworolski, Mark}%\Email{jmn@eecs.berkeley.edu}
{Mark Noworolski}}
%
\newcommand{\Isani}{\index{Isani, Sidik}%\Email{isani@cfht.hawaii.edu}
{Sidik Isani}}
%
\newcommand{\Goossens}{\index{Goossens, Michel}%\Email{goossens@cern.ch}
{Michel Goossens}}
%
\newcommand{\Wilck}{\index{Wilck, Martin}%\Email{martin@tropos.de}
{Martin Wilck}}
%
\newcommand{\PatrickDaly}{\index{Daly, Patrick}%\Email{daly@linmpi.dnet.gwdg.de}
{Patrick Daly}}
%
\newcommand{\Arseneau}{\index{Arseneau, Donald}%
{Donald Arseneau}}
%
\newcommand{\JohannesBraams}{\index{Braams, Johannes}%
{Johannes Braams}}
%
\newcommand{\HerbSwan}{\index{Swan, Herb}%\Email{herb.swan@perc.Arco.COM}
%\htmladdnormallink{Herb Swan}{mailto:herb.swan@perc.Arco.COM}}
{Herb Swan}}
%
\newcommand{\Lippmann}{\index{Lippmann, Jens}%\Email{lippmann@rbg.informatik.tu-darmstadt.de}
{Jens Lippmann}}
%
\newcommand{\Rouchal}{\index{Rouchal, Marek}%\Email{marek@saftsack.fs.uni-bayreuth.de}
{Marek Rouchal}}
%
\newcommand{\Bohnet}{\index{Bohnet, Achim}%\Email{ach@rosat.mpe-garching.mpg.de}
{Achim Bohnet}}
%
\newcommand{\Nelson}{\index{Nelson, Scott}%\Email{nelson18@llnl.gov}
{Scott Nelson}}
%
\newcommand{\AxelRamge}{\index{Ramge, Axel}%\Email{axel@ramge.de}
{Axel Ramge}}
%
\newcommand{\Popineau}{\index{Popineau, Fabrice}%\Email{popineau@esemetz.ese-metz.fr}
{Fabrice Popineau}}
%
\newcommand{\Wortmann}{\index{Wortmann, Uli}%\Email{uli12@bonk.ethz.ch}
{Uli Wortmann}}
%
\newcommand{\Veytsman}{\index{Veytsman, Boris}%\Email{boris@plmsc.psu.edu}
{Boris Veytsman}}
%
\newcommand{\Taupin}{\index{Taupin, Daniel}%\Email{taupin@lps.u-psud.fr}
{Daniel Taupin}}
%
\newcommand{\ArtOgawa}{\index{Ogawa, Art}%\Email{ogawa@teleport.com}
{Art Ogawa}}
%
\newcommand{\Perbandt}{\index{Perbandt, Adalbert}%\Email{perbandt@hl.siemens.de}
{Adalbert Perbandt}}
%
%\endinput

%
% (La)TeX related URLs
%
\newcommand{\CSEP}{\index{Computer Science Education Project!CSEP}%
{Computational Science Education Project (ORNL)}%
} 
%
\newcommand{\CBLU}{\index{Computer Based Learning Unit!CBLU}%
{Computer Based Learning Unit}}%
%
\newcommand{\CERN}{\index{CERN}%
\htmladdnormallink{CERN}{http://www.cern.ch}} 
%
\newcommand{\KrisRose}{\index{Rose, Kristoffer}%\Email{krisrose@brics.dk}
{Kristoffer Rose}}
%
\newcommand{\XypicDK}{\index{Xy-pic@\Xy-pic graphics package}%
\index{Xy-pic@\Xy-pic graphics package!home page}%
{Xy-pic}}
%
\newcommand{\LiPS}{\index{LiPS Design Team}%
{LiPS Design Team}}
\newcommand{\FIDarmstadt}{\index{Fachbereich Informatik, Darmstadt}%
\htmladdnormallink{Fachbereich Informatik}{http://www.informatik.tu-darmstadt.de/}}
\newcommand{\Darmstadt}{\index{Darmstadt!Fachbereich Informatik}%
{Darmstadt}}
%
\newcommand{\DANTE}{\index{DANTE}%
\htmladdnormallink{DANTE e.V.}{http://www.dante.de/}}
\newcommand{\Praesidium}{\index{DANTE!Praesidium}%
\htmladdnormallink{Praesidium}{http://www.dante.de/dante/Organe.html}}
\newcommand{\LaTeXiii}{\index{LaTeX@\LaTeX!LaTeX3@\LaTeX{3}}%
\htmladdnormallink{\LaTeX{3}}{http://www.tex.ac.uk/CTAN/latex/latex3}}
%
\newcommand{\Engberg}{\index{Uffe Engberg}%
\htmladdnormallinkfoot{Uffe Engberg}{http://www.brics.dk/\~{}engberg}} 

\newcommand{\texdevINDIC}{\texdev\indic}
%\endinput


%% earlier contributions from...
\newcommand{\AndrewCole}{\index{Computer Based Learning Unit!Andrew Cole}%
{Andrew Cole}}
%
\newcommand{\AnaPaiva}{\index{Computer Based Learning Unit!Ana Maria Paiva}%
{Ana Maria Paiva}}
%
\newcommand{\RodWilliams}{\index{Computer Based Learning Unit!Roderick Williams}%
{Roderick Williams}}
%
\newcommand{\JamilSawar}{\index{Computer Based Learning Unit!Jamil Sawar}%
{Jamil Sawar}}

\newcommand{\RobertThau}{\index{Thau, Robert S.}%
{Robert S. Thau}}

\newcommand{\RobertCailliau}{\index{Cailliau, Robert}%
{Robert Cailliau}}

\newcommand{\EricCarroll}{\index{Carroll, Eric}%
{Eric Carroll}}

\newcommand{\FranzVojik}{\index{Vojik, Franz}%
{Franz Vojik}}

\newcommand{\AxelBelinfante}{\index{Belinfante, Axel}%
{Axel Belinfante}}

\newcommand{\ToddLittle}{\index{Little, Todd}%
{Todd Little}}

\newcommand{\VerenaUmar}{\index{Umar, Verena}%
{Verena Umar}}

\newcommand{\IanFoster}{\index{Foster, Ian}%
{Ian Foster}}

\newcommand{\BobOlson}{\index{Olson, Bob}%
{Bob Olson}}

\endinput


%
% Ian Foster \Email{itf@mcs.anl.gov}
% Bob Olson \Email{olson@mcs.anl.gov}
% Verena Umar \Email{verena@edu.vanderbilt.cas.compsci}
% Axel Belinfante \Email{Axel.Belinfante@cs.utwente.nl}
% Todd Little \Email{little@com.dec.enet.nuts2u}
% Franz Vojik \Email{vojik@de.tu-muenchen.informatik}
% Eric Carroll \Email{eric@ca.utoronto.utcc.enfm}
% Roderick Williams \Email{rodw@cbl.leeds.ac.uk}
% Robert Cailliau \Email{cailliau@cernnext.cern.ch}
% Toni Lantunen at CERN
% Ana Maria Paiva, Jamil Sawar, Andrew Cole at CBLU Leeds
% Phillip Conrad (Perfect Byte, Inc.)
% L. Peter Deutsch (

%
\begin{htmlonly}
\documentclass[dvips,a4paper]{article}
\usepackage{html,htmllist,makeidx}

%this font is used in  features.tex
\font\wncyr = wncyr10


\begin{htmlonly}
 \usepackage[dvips]{graphicx}
 \usepackage[dvips]{color}
% \usepackage{frames}
 \usepackage[dvips,leftbars]{changebar}
 \newcommand{\FoilTeX}{\env{FoilTeX}}
%
% \newcommand{\htmlurl}[1]{\htmladdnormallink{#1}{#1}}
 \def\Email#1{\htmladdnormallink{<#1>}{mailto:#1}}
 \def\path#1{\texttt{#1}}
% \def\urldef#1#2#3{\def#1{#2{#3}}}
% \newcommand{\onlinedoc}{\url{http://www-dsed.llnl.gov/files/programs/unix/latex2html/manual/}}
 \newcommand{\onlinedoc}{\url{http://www-texdev.ics.mq.edu.au/l2h/docs/manual/}}
% \def\patches{\htmlurl{http://www-dsed.llnl.gov/files/programs/unix/latex2html/}}
 \def\patches{\htmlurl{http://www.latex2html.org/current/}}
 \def\sourceA{%
  \url{http://www.latex2html.org/current/}}
%  \url{http://www-dsed.llnl.gov/files/programs/unix/latex2html/sources/}}
% \def\sourceB{\htmlurl{ftp://ftp.mpn.com/pub/nikos/latex2html-98.1.tar.gz}}
 \def\sourceB{\CTANtug{\CTANA}}
 \def\sourceC{\htmlurl{http://ftp.rzg.mpg.de/pub/software/latex2html/sources/}}
% \def\CVSsite{\url{http://cdc-server.cdc.informatik.tu-darmstadt.de/\~{}latex2html/}}
% \def\CVSrepos{\url{http://cdc-server.cdc.informatik.tu-darmstadt.de/\~{}latex2html/user/}}
% \def\CVSlatest{\url{http://cdc-server.cdc.informatik.tu-darmstadt.de/\~{}latex2html/l2h-latest.tar.gz}}
% \def\CVSsite{\htmlurl{http://saftsack.fs.uni-bayreuth.de/~latex2ht/}}
% \def\CVSrepos{\htmlurl{http://saftsack.fs.uni-bayreuth.de/~latex2ht/user/}}
 \def\CVSrepos{\htmlurl{https://www.github.com/latex2html/latex2html/}}
% \def\CVSlatest{\htmlurl{http://saftsack.fs.uni-bayreuth.de/~latex2ht/l2h-latest.tar.gz}}
 \def\CVSlatest{\htmlurl{https://www.github.com/latex2html/latex2html/archive/master.zip}}
 \newcommand{\CTANtug}[1]{\htmlurl{http://mirrors.ctan.org/#1}}
 \def\CTANA{support/latex2html}
 \def\tugURL{\htmlurl{http://www.tug.org/}}
 \def\danteURL{\htmlurl{http://www.dante.de/}}
% \def\ListURL{\htmlurl{http://www.xray.mpe.mpg.de/mailing-lists/latex2html/}}
 \def\ListURL{\htmlurl{http://www.tug.org/mailman/listinfo/latex2html/}}
%
% \def\glossary#1{\index{#1@\texttt{#1} \label{III#1}\htmlref{(G)}{GGG#1}}}
 \newcommand{\glossary}[1]{\index{#1@\texttt{#1} \label{III#1}\htmlref{(G)}{GGG#1}}}
 \def\Glossary#1#2{\index{#1@{#2} \label{III#1}\htmlref{(G)}{GGG#1}}}
% \newcommand{\Glossary}[2]{\index{#1@{#2} \label{III#1}\htmlref{(G)}{GGG#1}}}
%
% hack to suppress changebar entries read from .aux file  !!! bug in changebar.perl !!! 
 \makeatletter
 \def\cb@barpoint#1#2#3{}
 \makeatother
%
\newcommand{\sameas}[1]{\textcolor{red}{Same as setting: #1}}
\newcommand{\onlinedocRM}{\url{http://www-texdev.ics.mq.edu.au/l2h/docs/manual/}}
\newcommand{\EXcolors}{\url{http://www-texdev.ics.mq.edu.au/l2h/crayola/}}
%
\newcommand{\Lc}[1]{\texttt{\char92#1}} % LaTeX command
\newcommand{\Tc}[1]{\texttt{\char92#1}} % TeX command
\newcommand{\Cs}[1]{ \texttt{-#1} }   % command-line switch
\newcommand{\Ve}[1]{\index{#1@\texttt{#1}}\texttt{#1}} % version extension
\newcommand{\gsl}[1]{#1}
\newcommand{\indexentry}[2]{\item #1 #2}

%
\internal{}%
\internal{O}%
\internal{S}%
\internal{E}%
\internal{M}%
\internal{H}%
\internal{P}%

\newcommand{\texdev}[1]{\htmladdnormallink
 {\texdevURL/#1}{\texdevURL/#1}}
\newcommand{\ctan}{http://ctan.tug.org/ctan}
\newcommand{\ctanTUG}{\htmladdnormallinkfoot
 {TUG's searchable CTAN site}{\ctan}}
\newcommand{\ctanURL}[1]{\htmladdnormallink
 {CTAN:\texttt{.../#1}}{\ctan/#1}}
\newcommand{\ctanURLbr}[1]{\htmladdnormallink
 {CTAN:\texttt{.../#1}}{\ctan/#1}}
\newcommand{\ctanTUGurl}[2]{\ctanURL{#2}}
\newcommand{\indichtml}{l2h/indic/IndicHTML/}

\end{htmlonly}

\newcommand{\TUGurl}{\htmladdnormallink{TUG}{\htmlurl{http://www.tug.org}}}

\newcommand{\texdevURL}{http://www-texdev.ics.mq.edu.au/}
\newcommand{\Unicode}{\htmladdnormallinkfoot
 {Unicode}{http://www.unicode.org/}}
\newcommand{\IndicHTML}{
 {Indic\TeX/HTML}}

%begin{latexonly}
\usepackage{array}

\newcommand{\Cs}[1]{{\upshape`\,\texttt{-#1}\,'}}
\newcommand{\Ve}[1]{\index{#1@\texttt{#1}}{\upshape`\,\texttt{#1}\,'}}
\newcommand{\Lc}[1]{{\upshape\ttfamily\char92#1}}
\newcommand{\Tc}[1]{{\upshape\ttfamily\char92#1}}

\newcommand{\ctanTUG}[1]{TUG's searchable CTAN site\footnote{\ctanTUGurl}}
\newcommand{\ctanURL}{CTAN: \penalty-200\ctanurl}
\newcommand{\ctanURLbr}{CTAN: \newline\ctanurl}
\urldef\texdev\url{http://www-texdev.ics.mq.edu.au/}
\urldef\ctanTUGurl\url{http://mirrors.ctan.org/}
\urldef\indichtml\url{l2h/indic/IndicHTML/}

\urldef\NetscapeHome\url{http://home.netscape.com}
\urldef\giftransURL\url{http://ftp.ost.eltele.no/pub/multimedia/giftrans-1.12.2.tar.gz}
\urldef\WinZipURL\url{http://www.winzip.com/getsite.cgi?winzip70.exe}
\urldef\MikTeXURL\url{http://www.miktex.de}
\urldef\ActiveStateURL\url{http://www.activestate.com}
\urldef\NetpbmWinURL\url{ftp://ftp.ese-metz.fr/pub/TeX/win32/}
%end{latexonly}

\begin{htmlonly}
\newcommand{\NetscapeHome}{\url{http://home.netscape.com}}
\newcommand{\giftransURL}{\url{http://ftp.ost.eltele.no/pub/multimedia/giftrans-1.12.2.tar.gz}}
\newcommand{\WinZipURL}{\url{http://www.winzip.com/getsite.cgi?winzip70.exe}}
\newcommand{\MikTeXURL}{\url{http://www.miktex.de}}
\newcommand{\ActiveStateURL}{\url{http://www.activestate.com}}
\newcommand{\NetpbmWinURL}{\url{ftp://ftp.ese-metz.fr/pub/TeX/win32/}}
\end{htmlonly}



% thanks to \KrisRose for this macro:
%\def\smiley{\hbox{\rlap{$\bigcirc$}\kern1.3pt$\scriptstyle\ddot\smile\;$}}
\def\mathsmiley{\smiley\,}


\newcommand{\Lcs}[1]{{\upshape\ttfamily\char92#1}}

\renewcommand{\thefootnote}{\arabic{footnote}}

%\newcommand{\godown}[1]{{\htmlref
% {\htmladdimg[left BORDER=0]{../psfiles/dn.gif}}{#1}}}
\newcommand{\godown}[1]{}

%\newcommand{\goback}[1]{{\htmlref
% {\htmladdimg[left BORDER=0]{../psfiles/up.gif}}{#1}}}
\newcommand{\goback}[1]{}


\newcommand{\WiiiC}{\htmladdnormallinkfoot{World Wide Web Consortium}{http://www.w3c.org/}}
\newcommand{\latextohtmlNG}{\latextohtml-NG}
\newcommand{\maillist}{\htmladdnormallinkfoot{\latextohtml{} mailing list}%
{http://cbl.leeds.ac.uk/nikos/tex2html/doc/mail/mail.html}}
%
\newcommand{\HTMLiii}{\textup{\texttt{HTML} 3.2}}%
\newcommand{\HTMLIII}{\htmladdnormallink{\HTMLiii}%
{http://www.w3.org/TR/REC-html32.html}}%

\newcommand{\HTMLiv}{\textup{\texttt{HTML} 4.0}}%
\newcommand{\HTMLIV}{\htmladdnormallink{\HTMLiv}%
{http://www.w3.org/TR/REC-html40/}}%

%\newcommand{\latextohtml}{\textup{\LaTeX 2{\ttfamily HTML}}}%
\newcommand{\Perl}{\htmlref{\textsl{Perl}}{GGGPerl}}%  
\newcommand{\PS}{\htmlref{\textup{Post\-Script}}{GGGpostscript}\Glossary{PostScript}{PostScript}{}}%
\providecommand{\MF}{}
\renewcommand{\MF}{\htmlref{\textsl{Metafont}}{GGGMetafont}\Glossary{Metafont}{\textsl{Metafont}}{}}%
\newcommand{\fn}[1]{\htmlref{\texttt{#1}}{GGG#1}\glossary{#1}}%  file names, with link to glossary
\newcommand{\gn}[1]{\texttt{#1}\label{GGG#1}\htmlref{\^{}}{III#1}}%  file names, labelled within glossary
%\newcommand{\appl}[1]{\htmlref{\textsl{#1}}{GGG#1}\Glossary{#1}{\textsl{#1}}{}}%  application software names
\newcommand{\appl}[1]{\htmlref{\textsl{#1}}{GGG#1}\Glossary{#1}{\gsl{#1}}{}}%  application software names
%
\newcommand{\env}[1]{{\upshape\sffamily #1}}%  LaTeX environment and package names
\newcommand{\HTMLtag}[1]{\path{<#1>}}%  HTML tag
\newcommand{\Meta}[1]{\texttt{\upshape<\textit{#1}>}}%  Meta string

%
% developer names and addresses:
%
\newcommand{\NikosDrakos}{\index{Drakos, Nikos}%\Email{nikos@cbl.leeds.ac.uk}
{Nikos Drakos}}
%
\newcommand{\RossMoore}{\index{Moore, Ross}%\Email{ross@maths.mq.edu.au}
{Ross Moore}}
\newcommand{\Macquarie}{\htmladdnormallink
{Macquarie University}{http://www.mq.edu.au/}}
%
\newcommand{\Hennecke}{\index{Hennecke, Marcus}%\Email{hennecke@dbag.ulm.DaimlerBenz.COM}
{Marcus Hennecke}}
%
\newcommand{\Noworolski}{\index{Noworolski, Mark}%\Email{jmn@eecs.berkeley.edu}
{Mark Noworolski}}
%
\newcommand{\Isani}{\index{Isani, Sidik}%\Email{isani@cfht.hawaii.edu}
{Sidik Isani}}
%
\newcommand{\Goossens}{\index{Goossens, Michel}%\Email{goossens@cern.ch}
{Michel Goossens}}
%
\newcommand{\Wilck}{\index{Wilck, Martin}%\Email{martin@tropos.de}
{Martin Wilck}}
%
\newcommand{\PatrickDaly}{\index{Daly, Patrick}%\Email{daly@linmpi.dnet.gwdg.de}
{Patrick Daly}}
%
\newcommand{\Arseneau}{\index{Arseneau, Donald}%
{Donald Arseneau}}
%
\newcommand{\JohannesBraams}{\index{Braams, Johannes}%
{Johannes Braams}}
%
\newcommand{\HerbSwan}{\index{Swan, Herb}%\Email{herb.swan@perc.Arco.COM}
%\htmladdnormallink{Herb Swan}{mailto:herb.swan@perc.Arco.COM}}
{Herb Swan}}
%
\newcommand{\Lippmann}{\index{Lippmann, Jens}%\Email{lippmann@rbg.informatik.tu-darmstadt.de}
{Jens Lippmann}}
%
\newcommand{\Rouchal}{\index{Rouchal, Marek}%\Email{marek@saftsack.fs.uni-bayreuth.de}
{Marek Rouchal}}
%
\newcommand{\Bohnet}{\index{Bohnet, Achim}%\Email{ach@rosat.mpe-garching.mpg.de}
{Achim Bohnet}}
%
\newcommand{\Nelson}{\index{Nelson, Scott}%\Email{nelson18@llnl.gov}
{Scott Nelson}}
%
\newcommand{\AxelRamge}{\index{Ramge, Axel}%\Email{axel@ramge.de}
{Axel Ramge}}
%
\newcommand{\Popineau}{\index{Popineau, Fabrice}%\Email{popineau@esemetz.ese-metz.fr}
{Fabrice Popineau}}
%
\newcommand{\Wortmann}{\index{Wortmann, Uli}%\Email{uli12@bonk.ethz.ch}
{Uli Wortmann}}
%
\newcommand{\Veytsman}{\index{Veytsman, Boris}%\Email{boris@plmsc.psu.edu}
{Boris Veytsman}}
%
\newcommand{\Taupin}{\index{Taupin, Daniel}%\Email{taupin@lps.u-psud.fr}
{Daniel Taupin}}
%
\newcommand{\ArtOgawa}{\index{Ogawa, Art}%\Email{ogawa@teleport.com}
{Art Ogawa}}
%
\newcommand{\Perbandt}{\index{Perbandt, Adalbert}%\Email{perbandt@hl.siemens.de}
{Adalbert Perbandt}}
%
%\endinput

%
% (La)TeX related URLs
%
\newcommand{\CSEP}{\index{Computer Science Education Project!CSEP}%
{Computational Science Education Project (ORNL)}%
} 
%
\newcommand{\CBLU}{\index{Computer Based Learning Unit!CBLU}%
{Computer Based Learning Unit}}%
%
\newcommand{\CERN}{\index{CERN}%
\htmladdnormallink{CERN}{http://www.cern.ch}} 
%
\newcommand{\KrisRose}{\index{Rose, Kristoffer}%\Email{krisrose@brics.dk}
{Kristoffer Rose}}
%
\newcommand{\XypicDK}{\index{Xy-pic@\Xy-pic graphics package}%
\index{Xy-pic@\Xy-pic graphics package!home page}%
{Xy-pic}}
%
\newcommand{\LiPS}{\index{LiPS Design Team}%
{LiPS Design Team}}
\newcommand{\FIDarmstadt}{\index{Fachbereich Informatik, Darmstadt}%
\htmladdnormallink{Fachbereich Informatik}{http://www.informatik.tu-darmstadt.de/}}
\newcommand{\Darmstadt}{\index{Darmstadt!Fachbereich Informatik}%
{Darmstadt}}
%
\newcommand{\DANTE}{\index{DANTE}%
\htmladdnormallink{DANTE e.V.}{http://www.dante.de/}}
\newcommand{\Praesidium}{\index{DANTE!Praesidium}%
\htmladdnormallink{Praesidium}{http://www.dante.de/dante/Organe.html}}
\newcommand{\LaTeXiii}{\index{LaTeX@\LaTeX!LaTeX3@\LaTeX{3}}%
\htmladdnormallink{\LaTeX{3}}{http://www.tex.ac.uk/CTAN/latex/latex3}}
%
\newcommand{\Engberg}{\index{Uffe Engberg}%
\htmladdnormallinkfoot{Uffe Engberg}{http://www.brics.dk/\~{}engberg}} 

\newcommand{\texdevINDIC}{\texdev\indic}
%\endinput


%% earlier contributions from...
\newcommand{\AndrewCole}{\index{Computer Based Learning Unit!Andrew Cole}%
{Andrew Cole}}
%
\newcommand{\AnaPaiva}{\index{Computer Based Learning Unit!Ana Maria Paiva}%
{Ana Maria Paiva}}
%
\newcommand{\RodWilliams}{\index{Computer Based Learning Unit!Roderick Williams}%
{Roderick Williams}}
%
\newcommand{\JamilSawar}{\index{Computer Based Learning Unit!Jamil Sawar}%
{Jamil Sawar}}

\newcommand{\RobertThau}{\index{Thau, Robert S.}%
{Robert S. Thau}}

\newcommand{\RobertCailliau}{\index{Cailliau, Robert}%
{Robert Cailliau}}

\newcommand{\EricCarroll}{\index{Carroll, Eric}%
{Eric Carroll}}

\newcommand{\FranzVojik}{\index{Vojik, Franz}%
{Franz Vojik}}

\newcommand{\AxelBelinfante}{\index{Belinfante, Axel}%
{Axel Belinfante}}

\newcommand{\ToddLittle}{\index{Little, Todd}%
{Todd Little}}

\newcommand{\VerenaUmar}{\index{Umar, Verena}%
{Verena Umar}}

\newcommand{\IanFoster}{\index{Foster, Ian}%
{Ian Foster}}

\newcommand{\BobOlson}{\index{Olson, Bob}%
{Bob Olson}}

\endinput


%
% Ian Foster \Email{itf@mcs.anl.gov}
% Bob Olson \Email{olson@mcs.anl.gov}
% Verena Umar \Email{verena@edu.vanderbilt.cas.compsci}
% Axel Belinfante \Email{Axel.Belinfante@cs.utwente.nl}
% Todd Little \Email{little@com.dec.enet.nuts2u}
% Franz Vojik \Email{vojik@de.tu-muenchen.informatik}
% Eric Carroll \Email{eric@ca.utoronto.utcc.enfm}
% Roderick Williams \Email{rodw@cbl.leeds.ac.uk}
% Robert Cailliau \Email{cailliau@cernnext.cern.ch}
% Toni Lantunen at CERN
% Ana Maria Paiva, Jamil Sawar, Andrew Cole at CBLU Leeds
% Phillip Conrad (Perfect Byte, Inc.)
% L. Peter Deutsch (

%
\begin{htmlonly}
\documentclass[dvips,a4paper]{article}
\usepackage{html,htmllist,makeidx}

%this font is used in  features.tex
\font\wncyr = wncyr10


\begin{htmlonly}
 \usepackage[dvips]{graphicx}
 \usepackage[dvips]{color}
% \usepackage{frames}
 \usepackage[dvips,leftbars]{changebar}
 \newcommand{\FoilTeX}{\env{FoilTeX}}
%
% \newcommand{\htmlurl}[1]{\htmladdnormallink{#1}{#1}}
 \def\Email#1{\htmladdnormallink{<#1>}{mailto:#1}}
 \def\path#1{\texttt{#1}}
% \def\urldef#1#2#3{\def#1{#2{#3}}}
% \newcommand{\onlinedoc}{\url{http://www-dsed.llnl.gov/files/programs/unix/latex2html/manual/}}
 \newcommand{\onlinedoc}{\url{http://www-texdev.ics.mq.edu.au/l2h/docs/manual/}}
% \def\patches{\htmlurl{http://www-dsed.llnl.gov/files/programs/unix/latex2html/}}
 \def\patches{\htmlurl{http://www.latex2html.org/current/}}
 \def\sourceA{%
  \url{http://www.latex2html.org/current/}}
%  \url{http://www-dsed.llnl.gov/files/programs/unix/latex2html/sources/}}
% \def\sourceB{\htmlurl{ftp://ftp.mpn.com/pub/nikos/latex2html-98.1.tar.gz}}
 \def\sourceB{\CTANtug{\CTANA}}
 \def\sourceC{\htmlurl{http://ftp.rzg.mpg.de/pub/software/latex2html/sources/}}
% \def\CVSsite{\url{http://cdc-server.cdc.informatik.tu-darmstadt.de/\~{}latex2html/}}
% \def\CVSrepos{\url{http://cdc-server.cdc.informatik.tu-darmstadt.de/\~{}latex2html/user/}}
% \def\CVSlatest{\url{http://cdc-server.cdc.informatik.tu-darmstadt.de/\~{}latex2html/l2h-latest.tar.gz}}
% \def\CVSsite{\htmlurl{http://saftsack.fs.uni-bayreuth.de/~latex2ht/}}
% \def\CVSrepos{\htmlurl{http://saftsack.fs.uni-bayreuth.de/~latex2ht/user/}}
 \def\CVSrepos{\htmlurl{https://www.github.com/latex2html/latex2html/}}
% \def\CVSlatest{\htmlurl{http://saftsack.fs.uni-bayreuth.de/~latex2ht/l2h-latest.tar.gz}}
 \def\CVSlatest{\htmlurl{https://www.github.com/latex2html/latex2html/archive/master.zip}}
 \newcommand{\CTANtug}[1]{\htmlurl{http://mirrors.ctan.org/#1}}
 \def\CTANA{support/latex2html}
 \def\tugURL{\htmlurl{http://www.tug.org/}}
 \def\danteURL{\htmlurl{http://www.dante.de/}}
% \def\ListURL{\htmlurl{http://www.xray.mpe.mpg.de/mailing-lists/latex2html/}}
 \def\ListURL{\htmlurl{http://www.tug.org/mailman/listinfo/latex2html/}}
%
% \def\glossary#1{\index{#1@\texttt{#1} \label{III#1}\htmlref{(G)}{GGG#1}}}
 \newcommand{\glossary}[1]{\index{#1@\texttt{#1} \label{III#1}\htmlref{(G)}{GGG#1}}}
 \def\Glossary#1#2{\index{#1@{#2} \label{III#1}\htmlref{(G)}{GGG#1}}}
% \newcommand{\Glossary}[2]{\index{#1@{#2} \label{III#1}\htmlref{(G)}{GGG#1}}}
%
% hack to suppress changebar entries read from .aux file  !!! bug in changebar.perl !!! 
 \makeatletter
 \def\cb@barpoint#1#2#3{}
 \makeatother
%
\newcommand{\sameas}[1]{\textcolor{red}{Same as setting: #1}}
\newcommand{\onlinedocRM}{\url{http://www-texdev.ics.mq.edu.au/l2h/docs/manual/}}
\newcommand{\EXcolors}{\url{http://www-texdev.ics.mq.edu.au/l2h/crayola/}}
%
\newcommand{\Lc}[1]{\texttt{\char92#1}} % LaTeX command
\newcommand{\Tc}[1]{\texttt{\char92#1}} % TeX command
\newcommand{\Cs}[1]{ \texttt{-#1} }   % command-line switch
\newcommand{\Ve}[1]{\index{#1@\texttt{#1}}\texttt{#1}} % version extension
\newcommand{\gsl}[1]{#1}
\newcommand{\indexentry}[2]{\item #1 #2}

%
\internal{}%
\internal{O}%
\internal{S}%
\internal{E}%
\internal{M}%
\internal{H}%
\internal{P}%

\newcommand{\texdev}[1]{\htmladdnormallink
 {\texdevURL/#1}{\texdevURL/#1}}
\newcommand{\ctan}{http://ctan.tug.org/ctan}
\newcommand{\ctanTUG}{\htmladdnormallinkfoot
 {TUG's searchable CTAN site}{\ctan}}
\newcommand{\ctanURL}[1]{\htmladdnormallink
 {CTAN:\texttt{.../#1}}{\ctan/#1}}
\newcommand{\ctanURLbr}[1]{\htmladdnormallink
 {CTAN:\texttt{.../#1}}{\ctan/#1}}
\newcommand{\ctanTUGurl}[2]{\ctanURL{#2}}
\newcommand{\indichtml}{l2h/indic/IndicHTML/}

\end{htmlonly}

\newcommand{\TUGurl}{\htmladdnormallink{TUG}{\htmlurl{http://www.tug.org}}}

\newcommand{\texdevURL}{http://www-texdev.ics.mq.edu.au/}
\newcommand{\Unicode}{\htmladdnormallinkfoot
 {Unicode}{http://www.unicode.org/}}
\newcommand{\IndicHTML}{
 {Indic\TeX/HTML}}

%begin{latexonly}
\usepackage{array}

\newcommand{\Cs}[1]{{\upshape`\,\texttt{-#1}\,'}}
\newcommand{\Ve}[1]{\index{#1@\texttt{#1}}{\upshape`\,\texttt{#1}\,'}}
\newcommand{\Lc}[1]{{\upshape\ttfamily\char92#1}}
\newcommand{\Tc}[1]{{\upshape\ttfamily\char92#1}}

\newcommand{\ctanTUG}[1]{TUG's searchable CTAN site\footnote{\ctanTUGurl}}
\newcommand{\ctanURL}{CTAN: \penalty-200\ctanurl}
\newcommand{\ctanURLbr}{CTAN: \newline\ctanurl}
\urldef\texdev\url{http://www-texdev.ics.mq.edu.au/}
\urldef\ctanTUGurl\url{http://mirrors.ctan.org/}
\urldef\indichtml\url{l2h/indic/IndicHTML/}

\urldef\NetscapeHome\url{http://home.netscape.com}
\urldef\giftransURL\url{http://ftp.ost.eltele.no/pub/multimedia/giftrans-1.12.2.tar.gz}
\urldef\WinZipURL\url{http://www.winzip.com/getsite.cgi?winzip70.exe}
\urldef\MikTeXURL\url{http://www.miktex.de}
\urldef\ActiveStateURL\url{http://www.activestate.com}
\urldef\NetpbmWinURL\url{ftp://ftp.ese-metz.fr/pub/TeX/win32/}
%end{latexonly}

\begin{htmlonly}
\newcommand{\NetscapeHome}{\url{http://home.netscape.com}}
\newcommand{\giftransURL}{\url{http://ftp.ost.eltele.no/pub/multimedia/giftrans-1.12.2.tar.gz}}
\newcommand{\WinZipURL}{\url{http://www.winzip.com/getsite.cgi?winzip70.exe}}
\newcommand{\MikTeXURL}{\url{http://www.miktex.de}}
\newcommand{\ActiveStateURL}{\url{http://www.activestate.com}}
\newcommand{\NetpbmWinURL}{\url{ftp://ftp.ese-metz.fr/pub/TeX/win32/}}
\end{htmlonly}



% thanks to \KrisRose for this macro:
%\def\smiley{\hbox{\rlap{$\bigcirc$}\kern1.3pt$\scriptstyle\ddot\smile\;$}}
\def\mathsmiley{\smiley\,}


\newcommand{\Lcs}[1]{{\upshape\ttfamily\char92#1}}

\renewcommand{\thefootnote}{\arabic{footnote}}

%\newcommand{\godown}[1]{{\htmlref
% {\htmladdimg[left BORDER=0]{../psfiles/dn.gif}}{#1}}}
\newcommand{\godown}[1]{}

%\newcommand{\goback}[1]{{\htmlref
% {\htmladdimg[left BORDER=0]{../psfiles/up.gif}}{#1}}}
\newcommand{\goback}[1]{}


\newcommand{\WiiiC}{\htmladdnormallinkfoot{World Wide Web Consortium}{http://www.w3c.org/}}
\newcommand{\latextohtmlNG}{\latextohtml-NG}
\newcommand{\maillist}{\htmladdnormallinkfoot{\latextohtml{} mailing list}%
{http://cbl.leeds.ac.uk/nikos/tex2html/doc/mail/mail.html}}
%
\newcommand{\HTMLiii}{\textup{\texttt{HTML} 3.2}}%
\newcommand{\HTMLIII}{\htmladdnormallink{\HTMLiii}%
{http://www.w3.org/TR/REC-html32.html}}%

\newcommand{\HTMLiv}{\textup{\texttt{HTML} 4.0}}%
\newcommand{\HTMLIV}{\htmladdnormallink{\HTMLiv}%
{http://www.w3.org/TR/REC-html40/}}%

%\newcommand{\latextohtml}{\textup{\LaTeX 2{\ttfamily HTML}}}%
\newcommand{\Perl}{\htmlref{\textsl{Perl}}{GGGPerl}}%  
\newcommand{\PS}{\htmlref{\textup{Post\-Script}}{GGGpostscript}\Glossary{PostScript}{PostScript}{}}%
\providecommand{\MF}{}
\renewcommand{\MF}{\htmlref{\textsl{Metafont}}{GGGMetafont}\Glossary{Metafont}{\textsl{Metafont}}{}}%
\newcommand{\fn}[1]{\htmlref{\texttt{#1}}{GGG#1}\glossary{#1}}%  file names, with link to glossary
\newcommand{\gn}[1]{\texttt{#1}\label{GGG#1}\htmlref{\^{}}{III#1}}%  file names, labelled within glossary
%\newcommand{\appl}[1]{\htmlref{\textsl{#1}}{GGG#1}\Glossary{#1}{\textsl{#1}}{}}%  application software names
\newcommand{\appl}[1]{\htmlref{\textsl{#1}}{GGG#1}\Glossary{#1}{\gsl{#1}}{}}%  application software names
%
\newcommand{\env}[1]{{\upshape\sffamily #1}}%  LaTeX environment and package names
\newcommand{\HTMLtag}[1]{\path{<#1>}}%  HTML tag
\newcommand{\Meta}[1]{\texttt{\upshape<\textit{#1}>}}%  Meta string

%
% developer names and addresses:
%
\newcommand{\NikosDrakos}{\index{Drakos, Nikos}%\Email{nikos@cbl.leeds.ac.uk}
{Nikos Drakos}}
%
\newcommand{\RossMoore}{\index{Moore, Ross}%\Email{ross@maths.mq.edu.au}
{Ross Moore}}
\newcommand{\Macquarie}{\htmladdnormallink
{Macquarie University}{http://www.mq.edu.au/}}
%
\newcommand{\Hennecke}{\index{Hennecke, Marcus}%\Email{hennecke@dbag.ulm.DaimlerBenz.COM}
{Marcus Hennecke}}
%
\newcommand{\Noworolski}{\index{Noworolski, Mark}%\Email{jmn@eecs.berkeley.edu}
{Mark Noworolski}}
%
\newcommand{\Isani}{\index{Isani, Sidik}%\Email{isani@cfht.hawaii.edu}
{Sidik Isani}}
%
\newcommand{\Goossens}{\index{Goossens, Michel}%\Email{goossens@cern.ch}
{Michel Goossens}}
%
\newcommand{\Wilck}{\index{Wilck, Martin}%\Email{martin@tropos.de}
{Martin Wilck}}
%
\newcommand{\PatrickDaly}{\index{Daly, Patrick}%\Email{daly@linmpi.dnet.gwdg.de}
{Patrick Daly}}
%
\newcommand{\Arseneau}{\index{Arseneau, Donald}%
{Donald Arseneau}}
%
\newcommand{\JohannesBraams}{\index{Braams, Johannes}%
{Johannes Braams}}
%
\newcommand{\HerbSwan}{\index{Swan, Herb}%\Email{herb.swan@perc.Arco.COM}
%\htmladdnormallink{Herb Swan}{mailto:herb.swan@perc.Arco.COM}}
{Herb Swan}}
%
\newcommand{\Lippmann}{\index{Lippmann, Jens}%\Email{lippmann@rbg.informatik.tu-darmstadt.de}
{Jens Lippmann}}
%
\newcommand{\Rouchal}{\index{Rouchal, Marek}%\Email{marek@saftsack.fs.uni-bayreuth.de}
{Marek Rouchal}}
%
\newcommand{\Bohnet}{\index{Bohnet, Achim}%\Email{ach@rosat.mpe-garching.mpg.de}
{Achim Bohnet}}
%
\newcommand{\Nelson}{\index{Nelson, Scott}%\Email{nelson18@llnl.gov}
{Scott Nelson}}
%
\newcommand{\AxelRamge}{\index{Ramge, Axel}%\Email{axel@ramge.de}
{Axel Ramge}}
%
\newcommand{\Popineau}{\index{Popineau, Fabrice}%\Email{popineau@esemetz.ese-metz.fr}
{Fabrice Popineau}}
%
\newcommand{\Wortmann}{\index{Wortmann, Uli}%\Email{uli12@bonk.ethz.ch}
{Uli Wortmann}}
%
\newcommand{\Veytsman}{\index{Veytsman, Boris}%\Email{boris@plmsc.psu.edu}
{Boris Veytsman}}
%
\newcommand{\Taupin}{\index{Taupin, Daniel}%\Email{taupin@lps.u-psud.fr}
{Daniel Taupin}}
%
\newcommand{\ArtOgawa}{\index{Ogawa, Art}%\Email{ogawa@teleport.com}
{Art Ogawa}}
%
\newcommand{\Perbandt}{\index{Perbandt, Adalbert}%\Email{perbandt@hl.siemens.de}
{Adalbert Perbandt}}
%
%\endinput

%
% (La)TeX related URLs
%
\newcommand{\CSEP}{\index{Computer Science Education Project!CSEP}%
{Computational Science Education Project (ORNL)}%
} 
%
\newcommand{\CBLU}{\index{Computer Based Learning Unit!CBLU}%
{Computer Based Learning Unit}}%
%
\newcommand{\CERN}{\index{CERN}%
\htmladdnormallink{CERN}{http://www.cern.ch}} 
%
\newcommand{\KrisRose}{\index{Rose, Kristoffer}%\Email{krisrose@brics.dk}
{Kristoffer Rose}}
%
\newcommand{\XypicDK}{\index{Xy-pic@\Xy-pic graphics package}%
\index{Xy-pic@\Xy-pic graphics package!home page}%
{Xy-pic}}
%
\newcommand{\LiPS}{\index{LiPS Design Team}%
{LiPS Design Team}}
\newcommand{\FIDarmstadt}{\index{Fachbereich Informatik, Darmstadt}%
\htmladdnormallink{Fachbereich Informatik}{http://www.informatik.tu-darmstadt.de/}}
\newcommand{\Darmstadt}{\index{Darmstadt!Fachbereich Informatik}%
{Darmstadt}}
%
\newcommand{\DANTE}{\index{DANTE}%
\htmladdnormallink{DANTE e.V.}{http://www.dante.de/}}
\newcommand{\Praesidium}{\index{DANTE!Praesidium}%
\htmladdnormallink{Praesidium}{http://www.dante.de/dante/Organe.html}}
\newcommand{\LaTeXiii}{\index{LaTeX@\LaTeX!LaTeX3@\LaTeX{3}}%
\htmladdnormallink{\LaTeX{3}}{http://www.tex.ac.uk/CTAN/latex/latex3}}
%
\newcommand{\Engberg}{\index{Uffe Engberg}%
\htmladdnormallinkfoot{Uffe Engberg}{http://www.brics.dk/\~{}engberg}} 

\newcommand{\texdevINDIC}{\texdev\indic}
%\endinput


%% earlier contributions from...
\newcommand{\AndrewCole}{\index{Computer Based Learning Unit!Andrew Cole}%
{Andrew Cole}}
%
\newcommand{\AnaPaiva}{\index{Computer Based Learning Unit!Ana Maria Paiva}%
{Ana Maria Paiva}}
%
\newcommand{\RodWilliams}{\index{Computer Based Learning Unit!Roderick Williams}%
{Roderick Williams}}
%
\newcommand{\JamilSawar}{\index{Computer Based Learning Unit!Jamil Sawar}%
{Jamil Sawar}}

\newcommand{\RobertThau}{\index{Thau, Robert S.}%
{Robert S. Thau}}

\newcommand{\RobertCailliau}{\index{Cailliau, Robert}%
{Robert Cailliau}}

\newcommand{\EricCarroll}{\index{Carroll, Eric}%
{Eric Carroll}}

\newcommand{\FranzVojik}{\index{Vojik, Franz}%
{Franz Vojik}}

\newcommand{\AxelBelinfante}{\index{Belinfante, Axel}%
{Axel Belinfante}}

\newcommand{\ToddLittle}{\index{Little, Todd}%
{Todd Little}}

\newcommand{\VerenaUmar}{\index{Umar, Verena}%
{Verena Umar}}

\newcommand{\IanFoster}{\index{Foster, Ian}%
{Ian Foster}}

\newcommand{\BobOlson}{\index{Olson, Bob}%
{Bob Olson}}

\endinput


%
% Ian Foster \Email{itf@mcs.anl.gov}
% Bob Olson \Email{olson@mcs.anl.gov}
% Verena Umar \Email{verena@edu.vanderbilt.cas.compsci}
% Axel Belinfante \Email{Axel.Belinfante@cs.utwente.nl}
% Todd Little \Email{little@com.dec.enet.nuts2u}
% Franz Vojik \Email{vojik@de.tu-muenchen.informatik}
% Eric Carroll \Email{eric@ca.utoronto.utcc.enfm}
% Roderick Williams \Email{rodw@cbl.leeds.ac.uk}
% Robert Cailliau \Email{cailliau@cernnext.cern.ch}
% Toni Lantunen at CERN
% Ana Maria Paiva, Jamil Sawar, Andrew Cole at CBLU Leeds
% Phillip Conrad (Perfect Byte, Inc.)
% L. Peter Deutsch (

%
\begin{htmlonly}
\documentclass[dvips,a4paper]{article}
\usepackage{html,htmllist,makeidx}
\input{manhtml.tex}
%
\input{overview.ptr}	% Input counters and section
\end{htmlonly}
\startdocument
%
%\section{Overview\label{sec:ovw}\index{overview}}
\label{sec:ovw}\index{overview}
\noindent 
This manual describes the \latextohtml{} translator which is
used to create Web pages from document source written for
the \LaTeX{} typesetting system, or simply containing \LaTeX{} commands.

\godown{quest0}\medskip
\noindent
To use \latextohtml{} to translate a file~\Meta{file}\texttt{.tex}
containing \LaTeX{} commands, simply type:
\begin{quote}
\texttt{latex2html} \Meta{file}\texttt{.tex} 
\end{quote}
\noindent
This will create a new directory called \Meta{file} which will contain 
the generated \texttt{HTML} files, some log files and possibly some images.

\medskip\godown{quest1}\label{quest0}\smallskip
\noindent
Basically the translator reads the source document and creates a linked
set of \texttt{HTML} pages, displaying the information it contains.
The \LaTeX{} commands and environments that are found are interpreted
either as ``markup'' instructions, or as macros expanding into more text 
or markup commands. 
Where such markup corresponds to the intended use for markup tags
in the \texttt{HTML} language, a direct translation is made.
If there is no natural way to present the information using simple text
embellished with \texttt{HTML} markup tags, then an image is generated, 
using \LaTeX{} itself to interpret the portion of code.

Of course this is a drastically over-simplified description of what
\latextohtml{} actually does. Many questions spring readily to mind.
The answers to these and the options available to handle
particular situations are discussed elsewhere in this manual.

\godown{quest2}\label{quest1}%
\begin{itemize}
\item \latexhtml
{\emph{What does ``natural way to present the information'' really mean? }}
{{\large What does ``natural way to present the information'' really mean? }}
\end{itemize}
\noindent
Text and paragraphing clearly should appear as such, whether printed
or on-screen. Different font sizes and styles such as ``bold-face''
or ``italic'' are generally rendered accordingly.
However, whereas \LaTeX{} has access to appropriate fonts for specialised
purposes such as mathematical symbols, these cannot be guaranteed to be
available with all Web-browsers. So for information requiring such things,
\latextohtml{} will generally resort to making an image,
using \LaTeX{} itself to typeset the material required for that image. 

\hyperref{The next page}{Section~}{}{features:ovw} contains a brief overview
of how \LaTeX's standard environments are handled within \latextohtml.
It also mentions some of the extra features that are available. \html{\\}%
In general \latextohtml{} attempts to use textual constructions to represent
the required information. Generation of an image is done only when there is
no adequate textual construction with the required version of \texttt{HTML},
or when specifically requested to do so.
Various extensions, to cope with the different \texttt{HTML} versions and
extra features, are discussed \hyperref{elsewhere}{in Section~}{}{sec:fea}.
That describes what to expect on the \texttt{HTML} pages, with little
or no changes required to the \LaTeX{} source.

Just as \LaTeX{} has various packages which can be used to present specific
types of information in appropriate ways, 
so is \latextohtml{} capable of handling the commands from many of these packages. 
See \hyperref{this table}{Table~}{}{styles} for a listing of those 
packages which currently have special support.

\godown{quest3}\label{quest2}%
\begin{itemize}
\item \latexhtml
{\emph{Some features of \texttt{HTML} have no direct counterpart in
a \LaTeX{} typeset document.\\Can such features be used with \latextohtml? }}
{{\large Some features of \texttt{HTML} have no direct counterpart in
a \LaTeX{} typeset document.\\Can such features be used with \latextohtml? }}
\end{itemize}
\noindent
Any effect currently available with any version of the \texttt{HTML} 
standard can be specified for a document processed by \latextohtml.
New \LaTeX{} commands are defined in the \fn{html.sty} package;
the features that these commands allow are the subject of
\hyperref{a whole section of}{Section~}{ in}{sec:hyp} this manual.
Some of the new commands provide improved strategies for effects 
already existing in \LaTeX; e.g. 
\htmlref{cross-references}{hyperref} and \htmlref{citations}{hypercite}.
To use these effectively requires only small changes to the \LaTeX{} source.

Other commands define new environments which are completely
ignored when processed by \LaTeX. 
Indeed the \htmlref{full scope}{HTMLtag} of \HTMLiii{} is available, 
using \LaTeX-like macros to help structure the source, 
reduce the tedium of repetitious use of tags, and ensure that
all appropriate tags are correctly closed.


\godown{quest4}\label{quest3}%
\begin{itemize}
\item \latexhtml
{\emph{What determines the amount of information that goes onto
a single \texttt{HTML} page?\\How are different pages linked? }}
{{\large What determines the amount of information that goes onto
a single \texttt{HTML} page?\\How are different pages linked? }}
\end{itemize}
\noindent
The \texttt{HTML} pages can contain whole chapters, sections,
(sub)subsections or (sub)paragraphs. This is fully customisable
using the command-line options discussed in detail in
\hyperref{a separate section}{Section~}{}{options} of this manual.


\godown{quest5}\label{quest4}%
\begin{itemize}
\item \latexhtml
{\emph{Does the original document have to be a valid \LaTeX{} document,
typesetting without errors? If not, does it help if it is? }}
{{\large Does the original document have to be a valid \LaTeX{} document,
typesetting without errors? If not, does it help if it is? }}
\end{itemize}
\noindent
In fact any document can be fed to the \latextohtml{} processor,
but it is designed specifically to recognise and sensibly translate
the intentions expressed by \LaTeX{} markup commands. Although sensible
results can be obtained even when the \LaTeX{} source is not valid,
the most reliable translations are obtained when it is. 
Relevant issues are discussed 
\hyperref{in a later section}{in Section~}{}{devel}.


\godown{quest6}\label{quest5}%
\begin{itemize}
\item \latexhtml
{\emph{When developing a document which contains special \texttt{HTML}
features, is it best to regularly test it in \LaTeX{} or with \latextohtml{}? }}
{{\large When developing a document which contains special \texttt{HTML}
features, is it best to regularly test it in \LaTeX{} or with \latextohtml{}? }}
\end{itemize}
\noindent
The answer to such a question changes as the developer gains
more experience with the available tools.
Some aspects to be considered are discussed 
\hyperref{in a later section}{in Section~}{}{devel} of this manual.


\medskip\htmlrule\goback{overview}\label{quest6}\medskip
\noindent
Information relevant to obtaining the latest version of \latextohtml, 
installation within the local environment, and where to look for
help when things do not go as expected, can be found in 
\hyperref{the support section}{Section~}{}{sec:sup}.


%%% END XTRACTFAQ (START is within outer document)

\bigskip\noindent
What follows next is a brief summary of the features supported
within \latextohtml{}.\html{\dots}


\subsection{List of Features\label{features:ovw}\index{features!listing}}
%\section{List of Features\label{features:ovw}\index{features!listing}}
%
Following is a listing of the main features of the translator;
more specific details on these is given elsewhere in this manual.

\smallskip\noindent
The \latextohtml{} translator \dots
%
\index{components!specify depth}
\begin{itemize}
\item 
breaks up a document into one or more components as specified
by the user\footnote{The user can specify the \emph{depth} at which 
the document should not be broken up any further.};

\index{navigation panel!optional}\index{navigation panel!customisable}%
\item 
provides optional, customisable iconic navigation
panels on every page which contain links to other parts of the
document, or other documents;

\index{equations!inlined}\index{equations!alignment}%
\index{equations!right-justified}\index{equations!numbered}%
\index{arbitrary environments}%
\item 
handles inlined equations ( \(\sum_{i=1}^{n} x_{i} = \int_{0}^{1} f \)), 
handles equation alignment ($A_{B_{C+D}}$), 
right-justified numbered equations (see \hyperref{example}{equation~}{}{eq:demo}), 
tables (see \hyperref{example}{Table~}{}{tab}), 
figures (see \hyperref{example}{Figure~}{}{fig:example}),
and \emph{any arbitrary environment}.
Either the complete environment or sub-parts thereof\html{\dots}
are passed to \LaTeX{}  for conversion to images, which are then either included
in the document or are made available through hypertext links.

\index{figures}\index{tables}\index{thumbnail}%
\index{figures!arbitrarily scaled}\index{tables!arbitrarily scaled}%
\index{figures!oriented}\index{tables!oriented}%
\item 
figures or tables can be arbitrarily scaled and oriented, 
and shown either as inlined images or ``thumbnail'' sketches\html{\dots}
or their contents displayed within a table constructed
using the \HTMLtag{TABLE} tags of \HTMLiii.

\index{environment!theorem-like}%
\index{theorems!theorem-like environments}%
\index{counter!automatic}%
\index{counter!dependency}%
\item
theorem-like environments are supported, along with
automatic numbering and counter dependencies.

\index{browser@browser\label{IIIbrowser}!supports images}%
\index{browser!character-based}%
\item 
can produce output suitable for browsers that support inlined images 
or character-based browsers (as specified by the user).
In particular the \TeX{} or \LaTeX{} code for mathematical expressions
and formulas will be displayed in character-based browsers,
such as \env{lynx}.

\index{color!coloured text}%
\index{color!coloured background}%
\index{image!backdrop}\index{backdrop!tiled with an image}%
\item 
coloured text and/or background is fully supported, as is the
ability to use an image to create a tiled backdrop.

\index{extensions@extensions\label{IIIexts}!new commands}%
\index{extensions!new environments}%
\index{extensions!new theorems}\index{extensions!in style-files}%
\item 
handles definitions of new commands, environments and counters
even when these are defined in external files for input\footnote{%
This allows the definition of \texttt{HTML} macros in \LaTeX{} !};

\index{footnotes}\index{table of contents}\index{list of figures}%
\index{list of tables}\index{bibliography}\index{index}%
\item 
handles footnotes\footnote{Like this!}, 
tables of contents, lists of figures and tables, 
bibliographies and can generate an index.
By including hyperlinks between index entries, 
simple navigation aids can be built into the index, for easy browsing.

%\index{references@references\label{IIIrefs}!hyper-links}%
\index{references@references!hyper-links}%
\index{references!between documents}%
\item 
automatically translates cross-references and citations into hyper-links,
and extends the \LaTeX{}  cross-referencing mechanism to work
not just within a document but \emph{between documents} 
which may reside in remote locations;

\index{character set@character set\label{IIIcharset}}%
\index{character set!ISO-8879@ISO--8879 (ISO--Latin--1)}%
\index{character set!ISO-10646@ISO--10646 (Unicode)}%
\index{special!characters|see{\htmlref{character set}{IIIcharset}}}%
\index{accents}%%
\item 
translates \LaTeX{}  accent and special character 
commands (e.g. \. {A} \O {\"o} \pounds \copyright \P) to
the equivalent ISO--Latin--1 or Unicode character set,
else an image can be created;

\index{hypertext links}\index{hypertext links!multi-media resources}%
\index{hypertext links!Internet services}%
\index{hypertext links!news/sound/video}%
\item 
recognises hypertext links (to multi-media resources or
arbitrary Internet services such as 
\textsl{sound}, \textsl{video}, \textsl{ftp}, \textsl{http}, \textsl{news}) and
links which invoke arbitrary program scripts---all expressed as \LaTeX{}  
commands;

\index{conditional text}%
\item 
recognises \emph{conditional text}  which is intended only for
the hypertext version, or only for the paper (\texttt{.dvi}) version;\par
%
\index{HTML@\texttt{HTML}!raw@raw \texttt{HTML} commands}%
\index{HTML@\texttt{HTML}!interactive forms}%
\item 
can include raw \texttt{HTML} in a \LaTeX{}  document 
(e.g. in order to specify interactive forms);

\index{LaTeX commands@\LaTeX{} commands}\label{hypcites}%
\index{Common LaTeX@Common \LaTeX{} Commands!latex blue@\LaTeX{} blue book}%
\index{LaTeX blue book@\LaTeX{} blue book}%
\item 
can deal sensibly with 
\strikeout{at least the \textit{Common }\LaTeX{} \textit{Commands}
summarised at the back of\\}%
virtually all of the concepts and commands described in
the \LaTeX{} \htmlcite{blue book}{lamp:latex}, 
where there is a meaningful interpretation appropriate to
an \texttt{HTML} document.
Also many other \LaTeX{} constructions are handled, including many described in the 
\LaTeX{} \hypercite{\textit{Companion}}{\textit{Companion}}{}{goossens:latex}
and \LaTeX{} \hypercite{\textit{Graphics Companion} (e.g. \Xy-pic)}%
{\textit{Graphics Companion}}{\Xy-pic}{goossens:latexGraphics};

%\index{images@images\label{IIIimages}!equations}%
\index{images@images!equations}%
\index{GIF|see{\htmlref{images}{IIIimages}}}%
\index{images!tables}%
\index{tables!as images}\index{tables!as HTML@as \texttt{HTML} mark-up}%
\index{HTML@\texttt{HTML}!Version 2.1}%
\index{HTML@\texttt{HTML}!Version 3.0}%
\item 
can be configured to translate equations either
as GIF images or as \texttt{HTML} 3.0 mark-up
(as browsers become available which are suitable for the task),
or by making images of subparts of equations, as required.

\index{references!symbolic}\index{references!document segments}%
\index{document!illustrative examples}%
\item 
links symbolic references across document segments which have been
independently processed;%

\index{LaTeX commands@\LaTeX{} commands!embedded}%
\index{LaTeX commands@\LaTeX{} commands!not syntactically legal}%
\item 
will try to translate any document with embedded \LaTeX{} commands, 
irrespective of whether it is complete or syntactically legal.

\end{itemize}


\subsection{Exemplary Documents\label{exemplary}}
%\section{Exemplary Documents\label{exemplary}}
Here is a selection of documents illustrating different contexts in 
which \latextohtml{} has been used. This list is by no means exhaustive,
but all links were valid as of June 1997.

\medskip\noindent
An earlier listing of converted documents can be found at:
\url{http://cbl.leeds.ac.uk/~nikos/tex2html/doc/latex2html/node6.html#sample}.
However some of the links are no longer valid.

\begin{description}
%
\item [\latextohtml{} documentation and usage]~%
\begin{itemize}
\item 
The \latextohtml{} Translator, User Manual\\
\url{http://www-texdev.ics.mq.edu.au/l2h/docs/manual/}\\
\onlinedoc
\html{\smallskip}%

\item 
Mathematics with \latextohtml{}\\
\url{http://www-texdev.ics.mq.edu.au/l2h/mathdocs/amsmath2/}
\html{\smallskip}%

\item 
\latextohtml, Style Sheets, \texttt{XML}\\
\url{http://www.ramge.de/ax/latex2html/latex2html.html}
\html{\smallskip}%

\item
\textit{All about }\latextohtml, by \htmladdnormallinkfoot{Nikos Drakos}%
{http://cbl.leeds.ac.uk/\~{}nikos/tex2html/doc/latex2html/latex2html.html}
\html{\smallskip}%

\item 
What is \latextohtml{}? (with \Xy-pic diagrams)\\
\url{http://www.maths.mq.edu.au/texdev/Xypic/L2Htalk/}
\html{\smallskip}%

\item 
German specials in \TeX\\
\url{http://www.maths.mq.edu.au/texdev/tests/harn/node1.html}
\html{\smallskip}%

\item
Crayola Colours\\
\url{http://www.maths.mq.edu.au/~ross/latex/crayola/crayola.html}
\html{\smallskip}%


%

\end{itemize}

\htmlrule[50\% center]


\item[Software Manuals, Computing Resources]~%
\begin{itemize}
%
\item 
\Xy-pic User's Guide, accessible from the 
\htmladdnormallinkfoot{\Xy-pic Home Page}{http://www.brics.dk/\~{}krisrose/Xy-pic.html}
or \htmladdnormallinkfoot{``down-under''}{http://www.maths.mq.edu.au/texdev/xyguide-html/}
\html{\smallskip}%

\item 
LiPS --- \htmladdnormallinkfoot{A System for Distributed Processing on Workstations}
{http://cdc-server.cdc.informatik.tu-darmstadt.de/}\\
\htmladdnormallink{Manual}{http://cdc-server.cdc.informatik.tu-darmstadt.de%
/home/zohar/LiPS/documentation/objects/doc/html/Manual2.1/lipsdist/}, 
\htmladdnormallink{Lecture Notes}{http://cdc-server.cdc.informatik.tu-darmstadt.de%
/home/zohar/LiPS/documentation/objects/doc/html/Talks/LectureSS97/LectureSS97/}
and various theses
\html{\smallskip}%

\item
LINUX Documentation Project\\
\url{http://linuxwww.db.erau.edu/ldp/linux.html/}\\
\htmladdnormallink{Getting Started}%
{http://linuxwww.db.erau.edu/ldp/LDP/gs/gs.html}, 
\htmladdnormallink{The LINUX Kernel}%
{http://linuxwww.db.erau.edu/ldp/LDP/tlk/}, 
\htmladdnormallink{Network Administrator's Guide}%
{http://linuxwww.db.erau.edu/ldp/LDP/nag/nag.html}\\
\htmladdnormallink{Programmer's Guide}%
{http://linuxwww.db.erau.edu/LPG/}, 
\htmladdnormallink{System Administrator's Guide}%
{http://linuxwww.db.erau.edu/ldp/LDP/sag/index.html}
\html{\smallskip}%

\item
ECLiPSE --- \htmladdnormallinkfoot{The ECRC Constraint Logic Parallel System}%
{http://www.ecrc.de/eclipse/eclipse.html}\\
\htmladdnormallink{User Manual}%
{http://www.ecrc.de/eclipse/html/umsroot/umsroot.html}, 
\htmladdnormallink{Extensions User Manual}%
{http://www.ecrc.de/eclipse/html/umsroot/umsroot.html}
\html{\smallskip}%

\item
MPQC --- \htmladdnormallinkfoot{Massively Parallel Quantum Chemistry Program}
{http://midway.ca.sandia.gov/\~{}cljanss/mpqc.html}\\
\htmladdnormallink{User Guide}%
{http://midway.ca.sandia.gov/\~{}cljanss/mpqc/mpqcuser/}, 
\htmladdnormallinkfoot{Scientific Computing library}%
{http://midway.ca.sandia.gov/\~{}cljanss/mpqc/prog/prog.html}
\html{\smallskip}%

\item
Xgraphics\\
\url{http://www-theorie.physik.uni-wuerzburg.de/~lueders/Xgraphics/}
\html{\smallskip}%

\item
\textit{Glish} 2.6, User Manual\\
\url{http://aips2.nrao.edu/aips++/docs/html/aips++.html}
\html{\smallskip}%

\item
UTCC --- \htmladdnormallinkfoot{University of Tennessee, Computing Center}%
{http://www.utcc.utk.edu/utcc/docs/}\\
\htmladdnormallink{Facilities and Services}%
{http://www.utcc.utk.edu/utcc/docs/u010003/u010003.html}, 
\htmladdnormallink{GNU Emacs}
{http://www.utcc.utk.edu/utcc/docs/u010599/u010599.html},
\htmladdnormallink{Help Sheet}
{http://www.utcc.utk.edu/utcc/docs/u010627/u010627.html},
\htmladdnormallink{Intro to Unix}
{http://www.utcc.utk.edu/utcc/docs/u010634/u010634.html},
\htmladdnormallink{\texttt{vi} editor}
{http://www.utcc.utk.edu/utcc/docs/u010648/u010648.html}\\
\htmladdnormallink{Software Support}
{http://www.utcc.utk.edu/utcc/docs/u010673/u010673.html}
\html{\smallskip}%


\end{itemize}


\htmlrule[50\% center]


\item [Journals, Conference Proceedings, Newsletters]~%
\begin{itemize}
%
\item ``CERN Computer Newsletter''\\
\url{http://consult.cern.ch/cnls/} issues 214--224.
\html{\smallskip}%

\item ``National Symposium in Mathematics''\\
\url{http://www.maths.mq.edu.au/texdev/MathSymp/}
\html{\smallskip}

\item
American Mathematical Society (\AmS)\quad(\url{http://www.ams.org/journals/})\\
Articles in the following electronic journals use a variant of \latextohtml{}:\\
\htmladdnormallink{\textit{Bulletin of the }\AmS}{http://www.ams.org/bull/}, 
\htmladdnormallink{\textit{Conformal Geometry and Dynamics}}{http://www.ams.org/ecgd/}\\ 
\htmladdnormallink{\textit{Electronic Research Announcements}}{http://www.ams.org/era/}

\smallskip
Other \AmS\ journals are available only to subscribers:\\
\htmladdnormallink{\textit{Journal of the }\AmS}{http://www.ams.org/jams/}, 
\htmladdnormallink{\textit{Mathematics of Computation}}{http://www.ams.org/mcom/},
\htmladdnormallink{\textit{Representation Theory}}{http://www.ams.org/ert/}\\
\htmladdnormallink{\textit{Proceedings of the }\AmS}{http://www.ams.org/proc/}, 
\htmladdnormallink{\textit{Transactions of the }\AmS}{http://www.ams.org/tran/}

%

\end{itemize}


\htmlrule[50\% center]

\item[Encyclop\ae{}dic Reference Material]~%
\begin{itemize}
\item
\htmladdnormallinkfoot{\textit{Eric's Treasure Troves of Science}:}%
{http://www.astro.virginia.edu/\~{}eww6n\#TreasureTroves} 
\htmladdnormallink{Astronomy}%
{http://www.astro.virginia.edu/\~{}eww6n/astro/astro.html}, 
\htmladdnormallink{Chemistry}%
{http://www.astro.virginia.edu/\~{}eww6n/chem/chem.html}, 
\htmladdnormallink{Math}%
{http://www.astro.virginia.edu/\~{}eww6n/math/math.html}\\ 
\htmladdnormallink{Music}%
{http://www.astro.virginia.edu/\~{}eww6n/music/music.html}, 
\htmladdnormallink{Physics}%
{http://www.astro.virginia.edu/\~{}eww6n/physics/physics.html}, 
\htmladdnormallink{Rocket History}%
{http://www.astro.virginia.edu/\~{}eww6n/rockets/rockets.html}, 
\htmladdnormallink{Scientific Books}%
{http://www.astro.virginia.edu/\~{}eww6n/books/books.html}, 
\htmladdnormallink{Scientific Biographies}%
{http://www.astro.virginia.edu/\~{}eww6n/bios/bios.html}
\html{\smallskip}

\item
Resources for Economists on the Internet\\
\url{http://wueconb.wustl.edu/EconFAQ/}
\html{\smallskip}

\item
\textit{Les math{\`e}mes de Lacan} by Jacques Siboni\\
\url{http:www.shef.ac.uk/~psysc/thesaur3/index.html}
\html{\smallskip}

\end{itemize}

\htmlrule[50\% center]


\item[Course Materials]~%
\begin{itemize}
\item
Mathematics Courses at \htmladdnormallinkfoot{Macquarie University}%
{http://www.maths.mq.edu.au/\~{}ross/index.html\#courses}\\
\htmladdnormallink{MATH130}%
{http://www.maths.mq.edu.au/texdev/MATHS/MATH130/study_guide99/}, 
\htmladdnormallink{MATH132}%
{http://www.maths.mq.edu.au/texdev/MATHS/MATH132/study_guide99/}, 
\htmladdnormallink{MATH233}%
{http://www.maths.mq.edu.au/texdev/MATHS/MATH233/study_guide/}, 
\htmladdnormallink{MATH300}%
{http://www.maths.mq.edu.au/texdev/MATHS/MATH300/study_guide/}
\htmladdnormallink{MATH337}%
{http://www.maths.mq.edu.au/texdev/MATHS/MATH337/study_guide99/}, 
\html{\smallskip}

\item 
\htmladdnormallinkfoot{Engineering Science 100}%
{http://fas.sfu.ca/ensc/people/Faculty/jones/personal/ensc100/}%
, Simon Fraser University.
\html{\smallskip}%

\end{itemize}

\htmlrule[50\% center]



\item[Other Interesting Sites]~%
\begin{itemize}
%
\item {\large rel\char64 X} \quad
(\url{http://www.ramge.de/ax/ax.html})
\html{\smallskip}%

\item \Wortmann's thesis and Geological papers use anti-aliased images;\\ 
\htmladdnormallinkfoot{\textit{Die untere und mittlere Kreide}}
{http://bonk.ethz.ch/papers/diss/main-node78.html}, 
\htmladdnormallinkfoot{\textit{The Barium Problem}}
{http://bonk.ethz.ch/papers/lecture-notes-97/ln97-node10.html}.
\html{\smallskip}%

\item 
Stanford Computer Graphics Laboratory, 
\htmladdnormallinkfoot{Publications}{http:/www-graphics.stanford.edu/papers/}\\
\htmladdnormallink{Web Visualization in Hyperbolic Space}
{http:/www-graphics.stanford.edu/papers/webviz/}, 
\htmladdnormallink{Topology of the MBONE}
{http:/www-graphics.stanford.edu/papers/mbone/}, 
\htmladdnormallink{Complex Models from Range Images}
{http:/www-graphics.stanford.edu/papers/volrange/},
\htmladdnormallink{Optical Triangulation}
{http:/www-graphics.stanford.edu/papers/spacetime/},
\htmladdnormallink{Metamorphosis}
{http:/www-graphics.stanford.edu/papers/morph/}


\item
The RGO Worldwide Guide to Public Videoconference Centers\\
\url{http://www.ast.cam.ac.uk/~ralf/vcguide/}
\html{\smallskip}%

\end{itemize}

\end{description}













	% Input counters and section
\end{htmlonly}
\startdocument
%
%\section{Overview\label{sec:ovw}\index{overview}}
\label{sec:ovw}\index{overview}
\noindent 
This manual describes the \latextohtml{} translator which is
used to create Web pages from document source written for
the \LaTeX{} typesetting system, or simply containing \LaTeX{} commands.

\godown{quest0}\medskip
\noindent
To use \latextohtml{} to translate a file~\Meta{file}\texttt{.tex}
containing \LaTeX{} commands, simply type:
\begin{quote}
\texttt{latex2html} \Meta{file}\texttt{.tex} 
\end{quote}
\noindent
This will create a new directory called \Meta{file} which will contain 
the generated \texttt{HTML} files, some log files and possibly some images.

\medskip\godown{quest1}\label{quest0}\smallskip
\noindent
Basically the translator reads the source document and creates a linked
set of \texttt{HTML} pages, displaying the information it contains.
The \LaTeX{} commands and environments that are found are interpreted
either as ``markup'' instructions, or as macros expanding into more text 
or markup commands. 
Where such markup corresponds to the intended use for markup tags
in the \texttt{HTML} language, a direct translation is made.
If there is no natural way to present the information using simple text
embellished with \texttt{HTML} markup tags, then an image is generated, 
using \LaTeX{} itself to interpret the portion of code.

Of course this is a drastically over-simplified description of what
\latextohtml{} actually does. Many questions spring readily to mind.
The answers to these and the options available to handle
particular situations are discussed elsewhere in this manual.

\godown{quest2}\label{quest1}%
\begin{itemize}
\item \latexhtml
{\emph{What does ``natural way to present the information'' really mean? }}
{{\large What does ``natural way to present the information'' really mean? }}
\end{itemize}
\noindent
Text and paragraphing clearly should appear as such, whether printed
or on-screen. Different font sizes and styles such as ``bold-face''
or ``italic'' are generally rendered accordingly.
However, whereas \LaTeX{} has access to appropriate fonts for specialised
purposes such as mathematical symbols, these cannot be guaranteed to be
available with all Web-browsers. So for information requiring such things,
\latextohtml{} will generally resort to making an image,
using \LaTeX{} itself to typeset the material required for that image. 

\hyperref{The next page}{Section~}{}{features:ovw} contains a brief overview
of how \LaTeX's standard environments are handled within \latextohtml.
It also mentions some of the extra features that are available. \html{\\}%
In general \latextohtml{} attempts to use textual constructions to represent
the required information. Generation of an image is done only when there is
no adequate textual construction with the required version of \texttt{HTML},
or when specifically requested to do so.
Various extensions, to cope with the different \texttt{HTML} versions and
extra features, are discussed \hyperref{elsewhere}{in Section~}{}{sec:fea}.
That describes what to expect on the \texttt{HTML} pages, with little
or no changes required to the \LaTeX{} source.

Just as \LaTeX{} has various packages which can be used to present specific
types of information in appropriate ways, 
so is \latextohtml{} capable of handling the commands from many of these packages. 
See \hyperref{this table}{Table~}{}{styles} for a listing of those 
packages which currently have special support.

\godown{quest3}\label{quest2}%
\begin{itemize}
\item \latexhtml
{\emph{Some features of \texttt{HTML} have no direct counterpart in
a \LaTeX{} typeset document.\\Can such features be used with \latextohtml? }}
{{\large Some features of \texttt{HTML} have no direct counterpart in
a \LaTeX{} typeset document.\\Can such features be used with \latextohtml? }}
\end{itemize}
\noindent
Any effect currently available with any version of the \texttt{HTML} 
standard can be specified for a document processed by \latextohtml.
New \LaTeX{} commands are defined in the \fn{html.sty} package;
the features that these commands allow are the subject of
\hyperref{a whole section of}{Section~}{ in}{sec:hyp} this manual.
Some of the new commands provide improved strategies for effects 
already existing in \LaTeX; e.g. 
\htmlref{cross-references}{hyperref} and \htmlref{citations}{hypercite}.
To use these effectively requires only small changes to the \LaTeX{} source.

Other commands define new environments which are completely
ignored when processed by \LaTeX. 
Indeed the \htmlref{full scope}{HTMLtag} of \HTMLiii{} is available, 
using \LaTeX-like macros to help structure the source, 
reduce the tedium of repetitious use of tags, and ensure that
all appropriate tags are correctly closed.


\godown{quest4}\label{quest3}%
\begin{itemize}
\item \latexhtml
{\emph{What determines the amount of information that goes onto
a single \texttt{HTML} page?\\How are different pages linked? }}
{{\large What determines the amount of information that goes onto
a single \texttt{HTML} page?\\How are different pages linked? }}
\end{itemize}
\noindent
The \texttt{HTML} pages can contain whole chapters, sections,
(sub)subsections or (sub)paragraphs. This is fully customisable
using the command-line options discussed in detail in
\hyperref{a separate section}{Section~}{}{options} of this manual.


\godown{quest5}\label{quest4}%
\begin{itemize}
\item \latexhtml
{\emph{Does the original document have to be a valid \LaTeX{} document,
typesetting without errors? If not, does it help if it is? }}
{{\large Does the original document have to be a valid \LaTeX{} document,
typesetting without errors? If not, does it help if it is? }}
\end{itemize}
\noindent
In fact any document can be fed to the \latextohtml{} processor,
but it is designed specifically to recognise and sensibly translate
the intentions expressed by \LaTeX{} markup commands. Although sensible
results can be obtained even when the \LaTeX{} source is not valid,
the most reliable translations are obtained when it is. 
Relevant issues are discussed 
\hyperref{in a later section}{in Section~}{}{devel}.


\godown{quest6}\label{quest5}%
\begin{itemize}
\item \latexhtml
{\emph{When developing a document which contains special \texttt{HTML}
features, is it best to regularly test it in \LaTeX{} or with \latextohtml{}? }}
{{\large When developing a document which contains special \texttt{HTML}
features, is it best to regularly test it in \LaTeX{} or with \latextohtml{}? }}
\end{itemize}
\noindent
The answer to such a question changes as the developer gains
more experience with the available tools.
Some aspects to be considered are discussed 
\hyperref{in a later section}{in Section~}{}{devel} of this manual.


\medskip\htmlrule\goback{overview}\label{quest6}\medskip
\noindent
Information relevant to obtaining the latest version of \latextohtml, 
installation within the local environment, and where to look for
help when things do not go as expected, can be found in 
\hyperref{the support section}{Section~}{}{sec:sup}.


%%% END XTRACTFAQ (START is within outer document)

\bigskip\noindent
What follows next is a brief summary of the features supported
within \latextohtml{}.\html{\dots}


\subsection{List of Features\label{features:ovw}\index{features!listing}}
%\section{List of Features\label{features:ovw}\index{features!listing}}
%
Following is a listing of the main features of the translator;
more specific details on these is given elsewhere in this manual.

\smallskip\noindent
The \latextohtml{} translator \dots
%
\index{components!specify depth}
\begin{itemize}
\item 
breaks up a document into one or more components as specified
by the user\footnote{The user can specify the \emph{depth} at which 
the document should not be broken up any further.};

\index{navigation panel!optional}\index{navigation panel!customisable}%
\item 
provides optional, customisable iconic navigation
panels on every page which contain links to other parts of the
document, or other documents;

\index{equations!inlined}\index{equations!alignment}%
\index{equations!right-justified}\index{equations!numbered}%
\index{arbitrary environments}%
\item 
handles inlined equations ( \(\sum_{i=1}^{n} x_{i} = \int_{0}^{1} f \)), 
handles equation alignment ($A_{B_{C+D}}$), 
right-justified numbered equations (see \hyperref{example}{equation~}{}{eq:demo}), 
tables (see \hyperref{example}{Table~}{}{tab}), 
figures (see \hyperref{example}{Figure~}{}{fig:example}),
and \emph{any arbitrary environment}.
Either the complete environment or sub-parts thereof\html{\dots}
are passed to \LaTeX{}  for conversion to images, which are then either included
in the document or are made available through hypertext links.

\index{figures}\index{tables}\index{thumbnail}%
\index{figures!arbitrarily scaled}\index{tables!arbitrarily scaled}%
\index{figures!oriented}\index{tables!oriented}%
\item 
figures or tables can be arbitrarily scaled and oriented, 
and shown either as inlined images or ``thumbnail'' sketches\html{\dots}
or their contents displayed within a table constructed
using the \HTMLtag{TABLE} tags of \HTMLiii.

\index{environment!theorem-like}%
\index{theorems!theorem-like environments}%
\index{counter!automatic}%
\index{counter!dependency}%
\item
theorem-like environments are supported, along with
automatic numbering and counter dependencies.

\index{browser@browser\label{IIIbrowser}!supports images}%
\index{browser!character-based}%
\item 
can produce output suitable for browsers that support inlined images 
or character-based browsers (as specified by the user).
In particular the \TeX{} or \LaTeX{} code for mathematical expressions
and formulas will be displayed in character-based browsers,
such as \env{lynx}.

\index{color!coloured text}%
\index{color!coloured background}%
\index{image!backdrop}\index{backdrop!tiled with an image}%
\item 
coloured text and/or background is fully supported, as is the
ability to use an image to create a tiled backdrop.

\index{extensions@extensions\label{IIIexts}!new commands}%
\index{extensions!new environments}%
\index{extensions!new theorems}\index{extensions!in style-files}%
\item 
handles definitions of new commands, environments and counters
even when these are defined in external files for input\footnote{%
This allows the definition of \texttt{HTML} macros in \LaTeX{} !};

\index{footnotes}\index{table of contents}\index{list of figures}%
\index{list of tables}\index{bibliography}\index{index}%
\item 
handles footnotes\footnote{Like this!}, 
tables of contents, lists of figures and tables, 
bibliographies and can generate an index.
By including hyperlinks between index entries, 
simple navigation aids can be built into the index, for easy browsing.

%\index{references@references\label{IIIrefs}!hyper-links}%
\index{references@references!hyper-links}%
\index{references!between documents}%
\item 
automatically translates cross-references and citations into hyper-links,
and extends the \LaTeX{}  cross-referencing mechanism to work
not just within a document but \emph{between documents} 
which may reside in remote locations;

\index{character set@character set\label{IIIcharset}}%
\index{character set!ISO-8879@ISO--8879 (ISO--Latin--1)}%
\index{character set!ISO-10646@ISO--10646 (Unicode)}%
\index{special!characters|see{\htmlref{character set}{IIIcharset}}}%
\index{accents}%%
\item 
translates \LaTeX{}  accent and special character 
commands (e.g. \. {A} \O {\"o} \pounds \copyright \P) to
the equivalent ISO--Latin--1 or Unicode character set,
else an image can be created;

\index{hypertext links}\index{hypertext links!multi-media resources}%
\index{hypertext links!Internet services}%
\index{hypertext links!news/sound/video}%
\item 
recognises hypertext links (to multi-media resources or
arbitrary Internet services such as 
\textsl{sound}, \textsl{video}, \textsl{ftp}, \textsl{http}, \textsl{news}) and
links which invoke arbitrary program scripts---all expressed as \LaTeX{}  
commands;

\index{conditional text}%
\item 
recognises \emph{conditional text}  which is intended only for
the hypertext version, or only for the paper (\texttt{.dvi}) version;\par
%
\index{HTML@\texttt{HTML}!raw@raw \texttt{HTML} commands}%
\index{HTML@\texttt{HTML}!interactive forms}%
\item 
can include raw \texttt{HTML} in a \LaTeX{}  document 
(e.g. in order to specify interactive forms);

\index{LaTeX commands@\LaTeX{} commands}\label{hypcites}%
\index{Common LaTeX@Common \LaTeX{} Commands!latex blue@\LaTeX{} blue book}%
\index{LaTeX blue book@\LaTeX{} blue book}%
\item 
can deal sensibly with 
\strikeout{at least the \textit{Common }\LaTeX{} \textit{Commands}
summarised at the back of\\}%
virtually all of the concepts and commands described in
the \LaTeX{} \htmlcite{blue book}{lamp:latex}, 
where there is a meaningful interpretation appropriate to
an \texttt{HTML} document.
Also many other \LaTeX{} constructions are handled, including many described in the 
\LaTeX{} \hypercite{\textit{Companion}}{\textit{Companion}}{}{goossens:latex}
and \LaTeX{} \hypercite{\textit{Graphics Companion} (e.g. \Xy-pic)}%
{\textit{Graphics Companion}}{\Xy-pic}{goossens:latexGraphics};

%\index{images@images\label{IIIimages}!equations}%
\index{images@images!equations}%
\index{GIF|see{\htmlref{images}{IIIimages}}}%
\index{images!tables}%
\index{tables!as images}\index{tables!as HTML@as \texttt{HTML} mark-up}%
\index{HTML@\texttt{HTML}!Version 2.1}%
\index{HTML@\texttt{HTML}!Version 3.0}%
\item 
can be configured to translate equations either
as GIF images or as \texttt{HTML} 3.0 mark-up
(as browsers become available which are suitable for the task),
or by making images of subparts of equations, as required.

\index{references!symbolic}\index{references!document segments}%
\index{document!illustrative examples}%
\item 
links symbolic references across document segments which have been
independently processed;%

\index{LaTeX commands@\LaTeX{} commands!embedded}%
\index{LaTeX commands@\LaTeX{} commands!not syntactically legal}%
\item 
will try to translate any document with embedded \LaTeX{} commands, 
irrespective of whether it is complete or syntactically legal.

\end{itemize}


\subsection{Exemplary Documents\label{exemplary}}
%\section{Exemplary Documents\label{exemplary}}
Here is a selection of documents illustrating different contexts in 
which \latextohtml{} has been used. This list is by no means exhaustive,
but all links were valid as of June 1997.

\medskip\noindent
An earlier listing of converted documents can be found at:
\url{http://cbl.leeds.ac.uk/~nikos/tex2html/doc/latex2html/node6.html#sample}.
However some of the links are no longer valid.

\begin{description}
%
\item [\latextohtml{} documentation and usage]~%
\begin{itemize}
\item 
The \latextohtml{} Translator, User Manual\\
\url{http://www-texdev.ics.mq.edu.au/l2h/docs/manual/}\\
\onlinedoc
\html{\smallskip}%

\item 
Mathematics with \latextohtml{}\\
\url{http://www-texdev.ics.mq.edu.au/l2h/mathdocs/amsmath2/}
\html{\smallskip}%

\item 
\latextohtml, Style Sheets, \texttt{XML}\\
\url{http://www.ramge.de/ax/latex2html/latex2html.html}
\html{\smallskip}%

\item
\textit{All about }\latextohtml, by \htmladdnormallinkfoot{Nikos Drakos}%
{http://cbl.leeds.ac.uk/\~{}nikos/tex2html/doc/latex2html/latex2html.html}
\html{\smallskip}%

\item 
What is \latextohtml{}? (with \Xy-pic diagrams)\\
\url{http://www.maths.mq.edu.au/texdev/Xypic/L2Htalk/}
\html{\smallskip}%

\item 
German specials in \TeX\\
\url{http://www.maths.mq.edu.au/texdev/tests/harn/node1.html}
\html{\smallskip}%

\item
Crayola Colours\\
\url{http://www.maths.mq.edu.au/~ross/latex/crayola/crayola.html}
\html{\smallskip}%


%

\end{itemize}

\htmlrule[50\% center]


\item[Software Manuals, Computing Resources]~%
\begin{itemize}
%
\item 
\Xy-pic User's Guide, accessible from the 
\htmladdnormallinkfoot{\Xy-pic Home Page}{http://www.brics.dk/\~{}krisrose/Xy-pic.html}
or \htmladdnormallinkfoot{``down-under''}{http://www.maths.mq.edu.au/texdev/xyguide-html/}
\html{\smallskip}%

\item 
LiPS --- \htmladdnormallinkfoot{A System for Distributed Processing on Workstations}
{http://cdc-server.cdc.informatik.tu-darmstadt.de/}\\
\htmladdnormallink{Manual}{http://cdc-server.cdc.informatik.tu-darmstadt.de%
/home/zohar/LiPS/documentation/objects/doc/html/Manual2.1/lipsdist/}, 
\htmladdnormallink{Lecture Notes}{http://cdc-server.cdc.informatik.tu-darmstadt.de%
/home/zohar/LiPS/documentation/objects/doc/html/Talks/LectureSS97/LectureSS97/}
and various theses
\html{\smallskip}%

\item
LINUX Documentation Project\\
\url{http://linuxwww.db.erau.edu/ldp/linux.html/}\\
\htmladdnormallink{Getting Started}%
{http://linuxwww.db.erau.edu/ldp/LDP/gs/gs.html}, 
\htmladdnormallink{The LINUX Kernel}%
{http://linuxwww.db.erau.edu/ldp/LDP/tlk/}, 
\htmladdnormallink{Network Administrator's Guide}%
{http://linuxwww.db.erau.edu/ldp/LDP/nag/nag.html}\\
\htmladdnormallink{Programmer's Guide}%
{http://linuxwww.db.erau.edu/LPG/}, 
\htmladdnormallink{System Administrator's Guide}%
{http://linuxwww.db.erau.edu/ldp/LDP/sag/index.html}
\html{\smallskip}%

\item
ECLiPSE --- \htmladdnormallinkfoot{The ECRC Constraint Logic Parallel System}%
{http://www.ecrc.de/eclipse/eclipse.html}\\
\htmladdnormallink{User Manual}%
{http://www.ecrc.de/eclipse/html/umsroot/umsroot.html}, 
\htmladdnormallink{Extensions User Manual}%
{http://www.ecrc.de/eclipse/html/umsroot/umsroot.html}
\html{\smallskip}%

\item
MPQC --- \htmladdnormallinkfoot{Massively Parallel Quantum Chemistry Program}
{http://midway.ca.sandia.gov/\~{}cljanss/mpqc.html}\\
\htmladdnormallink{User Guide}%
{http://midway.ca.sandia.gov/\~{}cljanss/mpqc/mpqcuser/}, 
\htmladdnormallinkfoot{Scientific Computing library}%
{http://midway.ca.sandia.gov/\~{}cljanss/mpqc/prog/prog.html}
\html{\smallskip}%

\item
Xgraphics\\
\url{http://www-theorie.physik.uni-wuerzburg.de/~lueders/Xgraphics/}
\html{\smallskip}%

\item
\textit{Glish} 2.6, User Manual\\
\url{http://aips2.nrao.edu/aips++/docs/html/aips++.html}
\html{\smallskip}%

\item
UTCC --- \htmladdnormallinkfoot{University of Tennessee, Computing Center}%
{http://www.utcc.utk.edu/utcc/docs/}\\
\htmladdnormallink{Facilities and Services}%
{http://www.utcc.utk.edu/utcc/docs/u010003/u010003.html}, 
\htmladdnormallink{GNU Emacs}
{http://www.utcc.utk.edu/utcc/docs/u010599/u010599.html},
\htmladdnormallink{Help Sheet}
{http://www.utcc.utk.edu/utcc/docs/u010627/u010627.html},
\htmladdnormallink{Intro to Unix}
{http://www.utcc.utk.edu/utcc/docs/u010634/u010634.html},
\htmladdnormallink{\texttt{vi} editor}
{http://www.utcc.utk.edu/utcc/docs/u010648/u010648.html}\\
\htmladdnormallink{Software Support}
{http://www.utcc.utk.edu/utcc/docs/u010673/u010673.html}
\html{\smallskip}%


\end{itemize}


\htmlrule[50\% center]


\item [Journals, Conference Proceedings, Newsletters]~%
\begin{itemize}
%
\item ``CERN Computer Newsletter''\\
\url{http://consult.cern.ch/cnls/} issues 214--224.
\html{\smallskip}%

\item ``National Symposium in Mathematics''\\
\url{http://www.maths.mq.edu.au/texdev/MathSymp/}
\html{\smallskip}

\item
American Mathematical Society (\AmS)\quad(\url{http://www.ams.org/journals/})\\
Articles in the following electronic journals use a variant of \latextohtml{}:\\
\htmladdnormallink{\textit{Bulletin of the }\AmS}{http://www.ams.org/bull/}, 
\htmladdnormallink{\textit{Conformal Geometry and Dynamics}}{http://www.ams.org/ecgd/}\\ 
\htmladdnormallink{\textit{Electronic Research Announcements}}{http://www.ams.org/era/}

\smallskip
Other \AmS\ journals are available only to subscribers:\\
\htmladdnormallink{\textit{Journal of the }\AmS}{http://www.ams.org/jams/}, 
\htmladdnormallink{\textit{Mathematics of Computation}}{http://www.ams.org/mcom/},
\htmladdnormallink{\textit{Representation Theory}}{http://www.ams.org/ert/}\\
\htmladdnormallink{\textit{Proceedings of the }\AmS}{http://www.ams.org/proc/}, 
\htmladdnormallink{\textit{Transactions of the }\AmS}{http://www.ams.org/tran/}

%

\end{itemize}


\htmlrule[50\% center]

\item[Encyclop\ae{}dic Reference Material]~%
\begin{itemize}
\item
\htmladdnormallinkfoot{\textit{Eric's Treasure Troves of Science}:}%
{http://www.astro.virginia.edu/\~{}eww6n\#TreasureTroves} 
\htmladdnormallink{Astronomy}%
{http://www.astro.virginia.edu/\~{}eww6n/astro/astro.html}, 
\htmladdnormallink{Chemistry}%
{http://www.astro.virginia.edu/\~{}eww6n/chem/chem.html}, 
\htmladdnormallink{Math}%
{http://www.astro.virginia.edu/\~{}eww6n/math/math.html}\\ 
\htmladdnormallink{Music}%
{http://www.astro.virginia.edu/\~{}eww6n/music/music.html}, 
\htmladdnormallink{Physics}%
{http://www.astro.virginia.edu/\~{}eww6n/physics/physics.html}, 
\htmladdnormallink{Rocket History}%
{http://www.astro.virginia.edu/\~{}eww6n/rockets/rockets.html}, 
\htmladdnormallink{Scientific Books}%
{http://www.astro.virginia.edu/\~{}eww6n/books/books.html}, 
\htmladdnormallink{Scientific Biographies}%
{http://www.astro.virginia.edu/\~{}eww6n/bios/bios.html}
\html{\smallskip}

\item
Resources for Economists on the Internet\\
\url{http://wueconb.wustl.edu/EconFAQ/}
\html{\smallskip}

\item
\textit{Les math{\`e}mes de Lacan} by Jacques Siboni\\
\url{http:www.shef.ac.uk/~psysc/thesaur3/index.html}
\html{\smallskip}

\end{itemize}

\htmlrule[50\% center]


\item[Course Materials]~%
\begin{itemize}
\item
Mathematics Courses at \htmladdnormallinkfoot{Macquarie University}%
{http://www.maths.mq.edu.au/\~{}ross/index.html\#courses}\\
\htmladdnormallink{MATH130}%
{http://www.maths.mq.edu.au/texdev/MATHS/MATH130/study_guide99/}, 
\htmladdnormallink{MATH132}%
{http://www.maths.mq.edu.au/texdev/MATHS/MATH132/study_guide99/}, 
\htmladdnormallink{MATH233}%
{http://www.maths.mq.edu.au/texdev/MATHS/MATH233/study_guide/}, 
\htmladdnormallink{MATH300}%
{http://www.maths.mq.edu.au/texdev/MATHS/MATH300/study_guide/}
\htmladdnormallink{MATH337}%
{http://www.maths.mq.edu.au/texdev/MATHS/MATH337/study_guide99/}, 
\html{\smallskip}

\item 
\htmladdnormallinkfoot{Engineering Science 100}%
{http://fas.sfu.ca/ensc/people/Faculty/jones/personal/ensc100/}%
, Simon Fraser University.
\html{\smallskip}%

\end{itemize}

\htmlrule[50\% center]



\item[Other Interesting Sites]~%
\begin{itemize}
%
\item {\large rel\char64 X} \quad
(\url{http://www.ramge.de/ax/ax.html})
\html{\smallskip}%

\item \Wortmann's thesis and Geological papers use anti-aliased images;\\ 
\htmladdnormallinkfoot{\textit{Die untere und mittlere Kreide}}
{http://bonk.ethz.ch/papers/diss/main-node78.html}, 
\htmladdnormallinkfoot{\textit{The Barium Problem}}
{http://bonk.ethz.ch/papers/lecture-notes-97/ln97-node10.html}.
\html{\smallskip}%

\item 
Stanford Computer Graphics Laboratory, 
\htmladdnormallinkfoot{Publications}{http:/www-graphics.stanford.edu/papers/}\\
\htmladdnormallink{Web Visualization in Hyperbolic Space}
{http:/www-graphics.stanford.edu/papers/webviz/}, 
\htmladdnormallink{Topology of the MBONE}
{http:/www-graphics.stanford.edu/papers/mbone/}, 
\htmladdnormallink{Complex Models from Range Images}
{http:/www-graphics.stanford.edu/papers/volrange/},
\htmladdnormallink{Optical Triangulation}
{http:/www-graphics.stanford.edu/papers/spacetime/},
\htmladdnormallink{Metamorphosis}
{http:/www-graphics.stanford.edu/papers/morph/}


\item
The RGO Worldwide Guide to Public Videoconference Centers\\
\url{http://www.ast.cam.ac.uk/~ralf/vcguide/}
\html{\smallskip}%

\end{itemize}

\end{description}













	% Input counters and section
\end{htmlonly}
\startdocument
%
%\section{Overview\label{sec:ovw}\index{overview}}
\label{sec:ovw}\index{overview}
\noindent 
This manual describes the \latextohtml{} translator which is
used to create Web pages from document source written for
the \LaTeX{} typesetting system, or simply containing \LaTeX{} commands.

\godown{quest0}\medskip
\noindent
To use \latextohtml{} to translate a file~\Meta{file}\texttt{.tex}
containing \LaTeX{} commands, simply type:
\begin{quote}
\texttt{latex2html} \Meta{file}\texttt{.tex} 
\end{quote}
\noindent
This will create a new directory called \Meta{file} which will contain 
the generated \texttt{HTML} files, some log files and possibly some images.

\medskip\godown{quest1}\label{quest0}\smallskip
\noindent
Basically the translator reads the source document and creates a linked
set of \texttt{HTML} pages, displaying the information it contains.
The \LaTeX{} commands and environments that are found are interpreted
either as ``markup'' instructions, or as macros expanding into more text 
or markup commands. 
Where such markup corresponds to the intended use for markup tags
in the \texttt{HTML} language, a direct translation is made.
If there is no natural way to present the information using simple text
embellished with \texttt{HTML} markup tags, then an image is generated, 
using \LaTeX{} itself to interpret the portion of code.

Of course this is a drastically over-simplified description of what
\latextohtml{} actually does. Many questions spring readily to mind.
The answers to these and the options available to handle
particular situations are discussed elsewhere in this manual.

\godown{quest2}\label{quest1}%
\begin{itemize}
\item \latexhtml
{\emph{What does ``natural way to present the information'' really mean? }}
{{\large What does ``natural way to present the information'' really mean? }}
\end{itemize}
\noindent
Text and paragraphing clearly should appear as such, whether printed
or on-screen. Different font sizes and styles such as ``bold-face''
or ``italic'' are generally rendered accordingly.
However, whereas \LaTeX{} has access to appropriate fonts for specialised
purposes such as mathematical symbols, these cannot be guaranteed to be
available with all Web-browsers. So for information requiring such things,
\latextohtml{} will generally resort to making an image,
using \LaTeX{} itself to typeset the material required for that image. 

\hyperref{The next page}{Section~}{}{features:ovw} contains a brief overview
of how \LaTeX's standard environments are handled within \latextohtml.
It also mentions some of the extra features that are available. \html{\\}%
In general \latextohtml{} attempts to use textual constructions to represent
the required information. Generation of an image is done only when there is
no adequate textual construction with the required version of \texttt{HTML},
or when specifically requested to do so.
Various extensions, to cope with the different \texttt{HTML} versions and
extra features, are discussed \hyperref{elsewhere}{in Section~}{}{sec:fea}.
That describes what to expect on the \texttt{HTML} pages, with little
or no changes required to the \LaTeX{} source.

Just as \LaTeX{} has various packages which can be used to present specific
types of information in appropriate ways, 
so is \latextohtml{} capable of handling the commands from many of these packages. 
See \hyperref{this table}{Table~}{}{styles} for a listing of those 
packages which currently have special support.

\godown{quest3}\label{quest2}%
\begin{itemize}
\item \latexhtml
{\emph{Some features of \texttt{HTML} have no direct counterpart in
a \LaTeX{} typeset document.\\Can such features be used with \latextohtml? }}
{{\large Some features of \texttt{HTML} have no direct counterpart in
a \LaTeX{} typeset document.\\Can such features be used with \latextohtml? }}
\end{itemize}
\noindent
Any effect currently available with any version of the \texttt{HTML} 
standard can be specified for a document processed by \latextohtml.
New \LaTeX{} commands are defined in the \fn{html.sty} package;
the features that these commands allow are the subject of
\hyperref{a whole section of}{Section~}{ in}{sec:hyp} this manual.
Some of the new commands provide improved strategies for effects 
already existing in \LaTeX; e.g. 
\htmlref{cross-references}{hyperref} and \htmlref{citations}{hypercite}.
To use these effectively requires only small changes to the \LaTeX{} source.

Other commands define new environments which are completely
ignored when processed by \LaTeX. 
Indeed the \htmlref{full scope}{HTMLtag} of \HTMLiii{} is available, 
using \LaTeX-like macros to help structure the source, 
reduce the tedium of repetitious use of tags, and ensure that
all appropriate tags are correctly closed.


\godown{quest4}\label{quest3}%
\begin{itemize}
\item \latexhtml
{\emph{What determines the amount of information that goes onto
a single \texttt{HTML} page?\\How are different pages linked? }}
{{\large What determines the amount of information that goes onto
a single \texttt{HTML} page?\\How are different pages linked? }}
\end{itemize}
\noindent
The \texttt{HTML} pages can contain whole chapters, sections,
(sub)subsections or (sub)paragraphs. This is fully customisable
using the command-line options discussed in detail in
\hyperref{a separate section}{Section~}{}{options} of this manual.


\godown{quest5}\label{quest4}%
\begin{itemize}
\item \latexhtml
{\emph{Does the original document have to be a valid \LaTeX{} document,
typesetting without errors? If not, does it help if it is? }}
{{\large Does the original document have to be a valid \LaTeX{} document,
typesetting without errors? If not, does it help if it is? }}
\end{itemize}
\noindent
In fact any document can be fed to the \latextohtml{} processor,
but it is designed specifically to recognise and sensibly translate
the intentions expressed by \LaTeX{} markup commands. Although sensible
results can be obtained even when the \LaTeX{} source is not valid,
the most reliable translations are obtained when it is. 
Relevant issues are discussed 
\hyperref{in a later section}{in Section~}{}{devel}.


\godown{quest6}\label{quest5}%
\begin{itemize}
\item \latexhtml
{\emph{When developing a document which contains special \texttt{HTML}
features, is it best to regularly test it in \LaTeX{} or with \latextohtml{}? }}
{{\large When developing a document which contains special \texttt{HTML}
features, is it best to regularly test it in \LaTeX{} or with \latextohtml{}? }}
\end{itemize}
\noindent
The answer to such a question changes as the developer gains
more experience with the available tools.
Some aspects to be considered are discussed 
\hyperref{in a later section}{in Section~}{}{devel} of this manual.


\medskip\htmlrule\goback{overview}\label{quest6}\medskip
\noindent
Information relevant to obtaining the latest version of \latextohtml, 
installation within the local environment, and where to look for
help when things do not go as expected, can be found in 
\hyperref{the support section}{Section~}{}{sec:sup}.


%%% END XTRACTFAQ (START is within outer document)

\bigskip\noindent
What follows next is a brief summary of the features supported
within \latextohtml{}.\html{\dots}


\subsection{List of Features\label{features:ovw}\index{features!listing}}
%\section{List of Features\label{features:ovw}\index{features!listing}}
%
Following is a listing of the main features of the translator;
more specific details on these is given elsewhere in this manual.

\smallskip\noindent
The \latextohtml{} translator \dots
%
\index{components!specify depth}
\begin{itemize}
\item 
breaks up a document into one or more components as specified
by the user\footnote{The user can specify the \emph{depth} at which 
the document should not be broken up any further.};

\index{navigation panel!optional}\index{navigation panel!customisable}%
\item 
provides optional, customisable iconic navigation
panels on every page which contain links to other parts of the
document, or other documents;

\index{equations!inlined}\index{equations!alignment}%
\index{equations!right-justified}\index{equations!numbered}%
\index{arbitrary environments}%
\item 
handles inlined equations ( \(\sum_{i=1}^{n} x_{i} = \int_{0}^{1} f \)), 
handles equation alignment ($A_{B_{C+D}}$), 
right-justified numbered equations (see \hyperref{example}{equation~}{}{eq:demo}), 
tables (see \hyperref{example}{Table~}{}{tab}), 
figures (see \hyperref{example}{Figure~}{}{fig:example}),
and \emph{any arbitrary environment}.
Either the complete environment or sub-parts thereof\html{\dots}
are passed to \LaTeX{}  for conversion to images, which are then either included
in the document or are made available through hypertext links.

\index{figures}\index{tables}\index{thumbnail}%
\index{figures!arbitrarily scaled}\index{tables!arbitrarily scaled}%
\index{figures!oriented}\index{tables!oriented}%
\item 
figures or tables can be arbitrarily scaled and oriented, 
and shown either as inlined images or ``thumbnail'' sketches\html{\dots}
or their contents displayed within a table constructed
using the \HTMLtag{TABLE} tags of \HTMLiii.

\index{environment!theorem-like}%
\index{theorems!theorem-like environments}%
\index{counter!automatic}%
\index{counter!dependency}%
\item
theorem-like environments are supported, along with
automatic numbering and counter dependencies.

\index{browser@browser\label{IIIbrowser}!supports images}%
\index{browser!character-based}%
\item 
can produce output suitable for browsers that support inlined images 
or character-based browsers (as specified by the user).
In particular the \TeX{} or \LaTeX{} code for mathematical expressions
and formulas will be displayed in character-based browsers,
such as \env{lynx}.

\index{color!coloured text}%
\index{color!coloured background}%
\index{image!backdrop}\index{backdrop!tiled with an image}%
\item 
coloured text and/or background is fully supported, as is the
ability to use an image to create a tiled backdrop.

\index{extensions@extensions\label{IIIexts}!new commands}%
\index{extensions!new environments}%
\index{extensions!new theorems}\index{extensions!in style-files}%
\item 
handles definitions of new commands, environments and counters
even when these are defined in external files for input\footnote{%
This allows the definition of \texttt{HTML} macros in \LaTeX{} !};

\index{footnotes}\index{table of contents}\index{list of figures}%
\index{list of tables}\index{bibliography}\index{index}%
\item 
handles footnotes\footnote{Like this!}, 
tables of contents, lists of figures and tables, 
bibliographies and can generate an index.
By including hyperlinks between index entries, 
simple navigation aids can be built into the index, for easy browsing.

%\index{references@references\label{IIIrefs}!hyper-links}%
\index{references@references!hyper-links}%
\index{references!between documents}%
\item 
automatically translates cross-references and citations into hyper-links,
and extends the \LaTeX{}  cross-referencing mechanism to work
not just within a document but \emph{between documents} 
which may reside in remote locations;

\index{character set@character set\label{IIIcharset}}%
\index{character set!ISO-8879@ISO--8879 (ISO--Latin--1)}%
\index{character set!ISO-10646@ISO--10646 (Unicode)}%
\index{special!characters|see{\htmlref{character set}{IIIcharset}}}%
\index{accents}%%
\item 
translates \LaTeX{}  accent and special character 
commands (e.g. \. {A} \O {\"o} \pounds \copyright \P) to
the equivalent ISO--Latin--1 or Unicode character set,
else an image can be created;

\index{hypertext links}\index{hypertext links!multi-media resources}%
\index{hypertext links!Internet services}%
\index{hypertext links!news/sound/video}%
\item 
recognises hypertext links (to multi-media resources or
arbitrary Internet services such as 
\textsl{sound}, \textsl{video}, \textsl{ftp}, \textsl{http}, \textsl{news}) and
links which invoke arbitrary program scripts---all expressed as \LaTeX{}  
commands;

\index{conditional text}%
\item 
recognises \emph{conditional text}  which is intended only for
the hypertext version, or only for the paper (\texttt{.dvi}) version;\par
%
\index{HTML@\texttt{HTML}!raw@raw \texttt{HTML} commands}%
\index{HTML@\texttt{HTML}!interactive forms}%
\item 
can include raw \texttt{HTML} in a \LaTeX{}  document 
(e.g. in order to specify interactive forms);

\index{LaTeX commands@\LaTeX{} commands}\label{hypcites}%
\index{Common LaTeX@Common \LaTeX{} Commands!latex blue@\LaTeX{} blue book}%
\index{LaTeX blue book@\LaTeX{} blue book}%
\item 
can deal sensibly with 
\strikeout{at least the \textit{Common }\LaTeX{} \textit{Commands}
summarised at the back of\\}%
virtually all of the concepts and commands described in
the \LaTeX{} \htmlcite{blue book}{lamp:latex}, 
where there is a meaningful interpretation appropriate to
an \texttt{HTML} document.
Also many other \LaTeX{} constructions are handled, including many described in the 
\LaTeX{} \hypercite{\textit{Companion}}{\textit{Companion}}{}{goossens:latex}
and \LaTeX{} \hypercite{\textit{Graphics Companion} (e.g. \Xy-pic)}%
{\textit{Graphics Companion}}{\Xy-pic}{goossens:latexGraphics};

%\index{images@images\label{IIIimages}!equations}%
\index{images@images!equations}%
\index{GIF|see{\htmlref{images}{IIIimages}}}%
\index{images!tables}%
\index{tables!as images}\index{tables!as HTML@as \texttt{HTML} mark-up}%
\index{HTML@\texttt{HTML}!Version 2.1}%
\index{HTML@\texttt{HTML}!Version 3.0}%
\item 
can be configured to translate equations either
as GIF images or as \texttt{HTML} 3.0 mark-up
(as browsers become available which are suitable for the task),
or by making images of subparts of equations, as required.

\index{references!symbolic}\index{references!document segments}%
\index{document!illustrative examples}%
\item 
links symbolic references across document segments which have been
independently processed;%

\index{LaTeX commands@\LaTeX{} commands!embedded}%
\index{LaTeX commands@\LaTeX{} commands!not syntactically legal}%
\item 
will try to translate any document with embedded \LaTeX{} commands, 
irrespective of whether it is complete or syntactically legal.

\end{itemize}


\subsection{Exemplary Documents\label{exemplary}}
%\section{Exemplary Documents\label{exemplary}}
Here is a selection of documents illustrating different contexts in 
which \latextohtml{} has been used. This list is by no means exhaustive,
but all links were valid as of June 1997.

\medskip\noindent
An earlier listing of converted documents can be found at:
\url{http://cbl.leeds.ac.uk/~nikos/tex2html/doc/latex2html/node6.html#sample}.
However some of the links are no longer valid.

\begin{description}
%
\item [\latextohtml{} documentation and usage]~%
\begin{itemize}
\item 
The \latextohtml{} Translator, User Manual\\
\url{http://www-texdev.ics.mq.edu.au/l2h/docs/manual/}\\
\onlinedoc
\html{\smallskip}%

\item 
Mathematics with \latextohtml{}\\
\url{http://www-texdev.ics.mq.edu.au/l2h/mathdocs/amsmath2/}
\html{\smallskip}%

\item 
\latextohtml, Style Sheets, \texttt{XML}\\
\url{http://www.ramge.de/ax/latex2html/latex2html.html}
\html{\smallskip}%

\item
\textit{All about }\latextohtml, by \htmladdnormallinkfoot{Nikos Drakos}%
{http://cbl.leeds.ac.uk/\~{}nikos/tex2html/doc/latex2html/latex2html.html}
\html{\smallskip}%

\item 
What is \latextohtml{}? (with \Xy-pic diagrams)\\
\url{http://www.maths.mq.edu.au/texdev/Xypic/L2Htalk/}
\html{\smallskip}%

\item 
German specials in \TeX\\
\url{http://www.maths.mq.edu.au/texdev/tests/harn/node1.html}
\html{\smallskip}%

\item
Crayola Colours\\
\url{http://www.maths.mq.edu.au/~ross/latex/crayola/crayola.html}
\html{\smallskip}%


%

\end{itemize}

\htmlrule[50\% center]


\item[Software Manuals, Computing Resources]~%
\begin{itemize}
%
\item 
\Xy-pic User's Guide, accessible from the 
\htmladdnormallinkfoot{\Xy-pic Home Page}{http://www.brics.dk/\~{}krisrose/Xy-pic.html}
or \htmladdnormallinkfoot{``down-under''}{http://www.maths.mq.edu.au/texdev/xyguide-html/}
\html{\smallskip}%

\item 
LiPS --- \htmladdnormallinkfoot{A System for Distributed Processing on Workstations}
{http://cdc-server.cdc.informatik.tu-darmstadt.de/}\\
\htmladdnormallink{Manual}{http://cdc-server.cdc.informatik.tu-darmstadt.de%
/home/zohar/LiPS/documentation/objects/doc/html/Manual2.1/lipsdist/}, 
\htmladdnormallink{Lecture Notes}{http://cdc-server.cdc.informatik.tu-darmstadt.de%
/home/zohar/LiPS/documentation/objects/doc/html/Talks/LectureSS97/LectureSS97/}
and various theses
\html{\smallskip}%

\item
LINUX Documentation Project\\
\url{http://linuxwww.db.erau.edu/ldp/linux.html/}\\
\htmladdnormallink{Getting Started}%
{http://linuxwww.db.erau.edu/ldp/LDP/gs/gs.html}, 
\htmladdnormallink{The LINUX Kernel}%
{http://linuxwww.db.erau.edu/ldp/LDP/tlk/}, 
\htmladdnormallink{Network Administrator's Guide}%
{http://linuxwww.db.erau.edu/ldp/LDP/nag/nag.html}\\
\htmladdnormallink{Programmer's Guide}%
{http://linuxwww.db.erau.edu/LPG/}, 
\htmladdnormallink{System Administrator's Guide}%
{http://linuxwww.db.erau.edu/ldp/LDP/sag/index.html}
\html{\smallskip}%

\item
ECLiPSE --- \htmladdnormallinkfoot{The ECRC Constraint Logic Parallel System}%
{http://www.ecrc.de/eclipse/eclipse.html}\\
\htmladdnormallink{User Manual}%
{http://www.ecrc.de/eclipse/html/umsroot/umsroot.html}, 
\htmladdnormallink{Extensions User Manual}%
{http://www.ecrc.de/eclipse/html/umsroot/umsroot.html}
\html{\smallskip}%

\item
MPQC --- \htmladdnormallinkfoot{Massively Parallel Quantum Chemistry Program}
{http://midway.ca.sandia.gov/\~{}cljanss/mpqc.html}\\
\htmladdnormallink{User Guide}%
{http://midway.ca.sandia.gov/\~{}cljanss/mpqc/mpqcuser/}, 
\htmladdnormallinkfoot{Scientific Computing library}%
{http://midway.ca.sandia.gov/\~{}cljanss/mpqc/prog/prog.html}
\html{\smallskip}%

\item
Xgraphics\\
\url{http://www-theorie.physik.uni-wuerzburg.de/~lueders/Xgraphics/}
\html{\smallskip}%

\item
\textit{Glish} 2.6, User Manual\\
\url{http://aips2.nrao.edu/aips++/docs/html/aips++.html}
\html{\smallskip}%

\item
UTCC --- \htmladdnormallinkfoot{University of Tennessee, Computing Center}%
{http://www.utcc.utk.edu/utcc/docs/}\\
\htmladdnormallink{Facilities and Services}%
{http://www.utcc.utk.edu/utcc/docs/u010003/u010003.html}, 
\htmladdnormallink{GNU Emacs}
{http://www.utcc.utk.edu/utcc/docs/u010599/u010599.html},
\htmladdnormallink{Help Sheet}
{http://www.utcc.utk.edu/utcc/docs/u010627/u010627.html},
\htmladdnormallink{Intro to Unix}
{http://www.utcc.utk.edu/utcc/docs/u010634/u010634.html},
\htmladdnormallink{\texttt{vi} editor}
{http://www.utcc.utk.edu/utcc/docs/u010648/u010648.html}\\
\htmladdnormallink{Software Support}
{http://www.utcc.utk.edu/utcc/docs/u010673/u010673.html}
\html{\smallskip}%


\end{itemize}


\htmlrule[50\% center]


\item [Journals, Conference Proceedings, Newsletters]~%
\begin{itemize}
%
\item ``CERN Computer Newsletter''\\
\url{http://consult.cern.ch/cnls/} issues 214--224.
\html{\smallskip}%

\item ``National Symposium in Mathematics''\\
\url{http://www.maths.mq.edu.au/texdev/MathSymp/}
\html{\smallskip}

\item
American Mathematical Society (\AmS)\quad(\url{http://www.ams.org/journals/})\\
Articles in the following electronic journals use a variant of \latextohtml{}:\\
\htmladdnormallink{\textit{Bulletin of the }\AmS}{http://www.ams.org/bull/}, 
\htmladdnormallink{\textit{Conformal Geometry and Dynamics}}{http://www.ams.org/ecgd/}\\ 
\htmladdnormallink{\textit{Electronic Research Announcements}}{http://www.ams.org/era/}

\smallskip
Other \AmS\ journals are available only to subscribers:\\
\htmladdnormallink{\textit{Journal of the }\AmS}{http://www.ams.org/jams/}, 
\htmladdnormallink{\textit{Mathematics of Computation}}{http://www.ams.org/mcom/},
\htmladdnormallink{\textit{Representation Theory}}{http://www.ams.org/ert/}\\
\htmladdnormallink{\textit{Proceedings of the }\AmS}{http://www.ams.org/proc/}, 
\htmladdnormallink{\textit{Transactions of the }\AmS}{http://www.ams.org/tran/}

%

\end{itemize}


\htmlrule[50\% center]

\item[Encyclop\ae{}dic Reference Material]~%
\begin{itemize}
\item
\htmladdnormallinkfoot{\textit{Eric's Treasure Troves of Science}:}%
{http://www.astro.virginia.edu/\~{}eww6n\#TreasureTroves} 
\htmladdnormallink{Astronomy}%
{http://www.astro.virginia.edu/\~{}eww6n/astro/astro.html}, 
\htmladdnormallink{Chemistry}%
{http://www.astro.virginia.edu/\~{}eww6n/chem/chem.html}, 
\htmladdnormallink{Math}%
{http://www.astro.virginia.edu/\~{}eww6n/math/math.html}\\ 
\htmladdnormallink{Music}%
{http://www.astro.virginia.edu/\~{}eww6n/music/music.html}, 
\htmladdnormallink{Physics}%
{http://www.astro.virginia.edu/\~{}eww6n/physics/physics.html}, 
\htmladdnormallink{Rocket History}%
{http://www.astro.virginia.edu/\~{}eww6n/rockets/rockets.html}, 
\htmladdnormallink{Scientific Books}%
{http://www.astro.virginia.edu/\~{}eww6n/books/books.html}, 
\htmladdnormallink{Scientific Biographies}%
{http://www.astro.virginia.edu/\~{}eww6n/bios/bios.html}
\html{\smallskip}

\item
Resources for Economists on the Internet\\
\url{http://wueconb.wustl.edu/EconFAQ/}
\html{\smallskip}

\item
\textit{Les math{\`e}mes de Lacan} by Jacques Siboni\\
\url{http:www.shef.ac.uk/~psysc/thesaur3/index.html}
\html{\smallskip}

\end{itemize}

\htmlrule[50\% center]


\item[Course Materials]~%
\begin{itemize}
\item
Mathematics Courses at \htmladdnormallinkfoot{Macquarie University}%
{http://www.maths.mq.edu.au/\~{}ross/index.html\#courses}\\
\htmladdnormallink{MATH130}%
{http://www.maths.mq.edu.au/texdev/MATHS/MATH130/study_guide99/}, 
\htmladdnormallink{MATH132}%
{http://www.maths.mq.edu.au/texdev/MATHS/MATH132/study_guide99/}, 
\htmladdnormallink{MATH233}%
{http://www.maths.mq.edu.au/texdev/MATHS/MATH233/study_guide/}, 
\htmladdnormallink{MATH300}%
{http://www.maths.mq.edu.au/texdev/MATHS/MATH300/study_guide/}
\htmladdnormallink{MATH337}%
{http://www.maths.mq.edu.au/texdev/MATHS/MATH337/study_guide99/}, 
\html{\smallskip}

\item 
\htmladdnormallinkfoot{Engineering Science 100}%
{http://fas.sfu.ca/ensc/people/Faculty/jones/personal/ensc100/}%
, Simon Fraser University.
\html{\smallskip}%

\end{itemize}

\htmlrule[50\% center]



\item[Other Interesting Sites]~%
\begin{itemize}
%
\item {\large rel\char64 X} \quad
(\url{http://www.ramge.de/ax/ax.html})
\html{\smallskip}%

\item \Wortmann's thesis and Geological papers use anti-aliased images;\\ 
\htmladdnormallinkfoot{\textit{Die untere und mittlere Kreide}}
{http://bonk.ethz.ch/papers/diss/main-node78.html}, 
\htmladdnormallinkfoot{\textit{The Barium Problem}}
{http://bonk.ethz.ch/papers/lecture-notes-97/ln97-node10.html}.
\html{\smallskip}%

\item 
Stanford Computer Graphics Laboratory, 
\htmladdnormallinkfoot{Publications}{http:/www-graphics.stanford.edu/papers/}\\
\htmladdnormallink{Web Visualization in Hyperbolic Space}
{http:/www-graphics.stanford.edu/papers/webviz/}, 
\htmladdnormallink{Topology of the MBONE}
{http:/www-graphics.stanford.edu/papers/mbone/}, 
\htmladdnormallink{Complex Models from Range Images}
{http:/www-graphics.stanford.edu/papers/volrange/},
\htmladdnormallink{Optical Triangulation}
{http:/www-graphics.stanford.edu/papers/spacetime/},
\htmladdnormallink{Metamorphosis}
{http:/www-graphics.stanford.edu/papers/morph/}


\item
The RGO Worldwide Guide to Public Videoconference Centers\\
\url{http://www.ast.cam.ac.uk/~ralf/vcguide/}
\html{\smallskip}%

\end{itemize}

\end{description}













	% Input counters and section
\end{htmlonly}
\startdocument
%
%\section{Overview\label{sec:ovw}\index{overview}}
\label{sec:ovw}\index{overview}
\noindent 
This manual describes the \latextohtml{} translator which is
used to create Web pages from document source written for
the \LaTeX{} typesetting system, or simply containing \LaTeX{} commands.

\godown{quest0}\medskip
\noindent
To use \latextohtml{} to translate a file~\Meta{file}\texttt{.tex}
containing \LaTeX{} commands, simply type:
\begin{quote}
\texttt{latex2html} \Meta{file}\texttt{.tex} 
\end{quote}
\noindent
This will create a new directory called \Meta{file} which will contain 
the generated \texttt{HTML} files, some log files and possibly some images.

\medskip\godown{quest1}\label{quest0}\smallskip
\noindent
Basically the translator reads the source document and creates a linked
set of \texttt{HTML} pages, displaying the information it contains.
The \LaTeX{} commands and environments that are found are interpreted
either as ``markup'' instructions, or as macros expanding into more text 
or markup commands. 
Where such markup corresponds to the intended use for markup tags
in the \texttt{HTML} language, a direct translation is made.
If there is no natural way to present the information using simple text
embellished with \texttt{HTML} markup tags, then an image is generated, 
using \LaTeX{} itself to interpret the portion of code.

Of course this is a drastically over-simplified description of what
\latextohtml{} actually does. Many questions spring readily to mind.
The answers to these and the options available to handle
particular situations are discussed elsewhere in this manual.

\godown{quest2}\label{quest1}%
\begin{itemize}
\item \latexhtml
{\emph{What does ``natural way to present the information'' really mean? }}
{{\large What does ``natural way to present the information'' really mean? }}
\end{itemize}
\noindent
Text and paragraphing clearly should appear as such, whether printed
or on-screen. Different font sizes and styles such as ``bold-face''
or ``italic'' are generally rendered accordingly.
However, whereas \LaTeX{} has access to appropriate fonts for specialised
purposes such as mathematical symbols, these cannot be guaranteed to be
available with all Web-browsers. So for information requiring such things,
\latextohtml{} will generally resort to making an image,
using \LaTeX{} itself to typeset the material required for that image. 

\hyperref{The next page}{Section~}{}{features:ovw} contains a brief overview
of how \LaTeX's standard environments are handled within \latextohtml.
It also mentions some of the extra features that are available. \html{\\}%
In general \latextohtml{} attempts to use textual constructions to represent
the required information. Generation of an image is done only when there is
no adequate textual construction with the required version of \texttt{HTML},
or when specifically requested to do so.
Various extensions, to cope with the different \texttt{HTML} versions and
extra features, are discussed \hyperref{elsewhere}{in Section~}{}{sec:fea}.
That describes what to expect on the \texttt{HTML} pages, with little
or no changes required to the \LaTeX{} source.

Just as \LaTeX{} has various packages which can be used to present specific
types of information in appropriate ways, 
so is \latextohtml{} capable of handling the commands from many of these packages. 
See \hyperref{this table}{Table~}{}{styles} for a listing of those 
packages which currently have special support.

\godown{quest3}\label{quest2}%
\begin{itemize}
\item \latexhtml
{\emph{Some features of \texttt{HTML} have no direct counterpart in
a \LaTeX{} typeset document.\\Can such features be used with \latextohtml? }}
{{\large Some features of \texttt{HTML} have no direct counterpart in
a \LaTeX{} typeset document.\\Can such features be used with \latextohtml? }}
\end{itemize}
\noindent
Any effect currently available with any version of the \texttt{HTML} 
standard can be specified for a document processed by \latextohtml.
New \LaTeX{} commands are defined in the \fn{html.sty} package;
the features that these commands allow are the subject of
\hyperref{a whole section of}{Section~}{ in}{sec:hyp} this manual.
Some of the new commands provide improved strategies for effects 
already existing in \LaTeX; e.g. 
\htmlref{cross-references}{hyperref} and \htmlref{citations}{hypercite}.
To use these effectively requires only small changes to the \LaTeX{} source.

Other commands define new environments which are completely
ignored when processed by \LaTeX. 
Indeed the \htmlref{full scope}{HTMLtag} of \HTMLiii{} is available, 
using \LaTeX-like macros to help structure the source, 
reduce the tedium of repetitious use of tags, and ensure that
all appropriate tags are correctly closed.


\godown{quest4}\label{quest3}%
\begin{itemize}
\item \latexhtml
{\emph{What determines the amount of information that goes onto
a single \texttt{HTML} page?\\How are different pages linked? }}
{{\large What determines the amount of information that goes onto
a single \texttt{HTML} page?\\How are different pages linked? }}
\end{itemize}
\noindent
The \texttt{HTML} pages can contain whole chapters, sections,
(sub)subsections or (sub)paragraphs. This is fully customisable
using the command-line options discussed in detail in
\hyperref{a separate section}{Section~}{}{options} of this manual.


\godown{quest5}\label{quest4}%
\begin{itemize}
\item \latexhtml
{\emph{Does the original document have to be a valid \LaTeX{} document,
typesetting without errors? If not, does it help if it is? }}
{{\large Does the original document have to be a valid \LaTeX{} document,
typesetting without errors? If not, does it help if it is? }}
\end{itemize}
\noindent
In fact any document can be fed to the \latextohtml{} processor,
but it is designed specifically to recognise and sensibly translate
the intentions expressed by \LaTeX{} markup commands. Although sensible
results can be obtained even when the \LaTeX{} source is not valid,
the most reliable translations are obtained when it is. 
Relevant issues are discussed 
\hyperref{in a later section}{in Section~}{}{devel}.


\godown{quest6}\label{quest5}%
\begin{itemize}
\item \latexhtml
{\emph{When developing a document which contains special \texttt{HTML}
features, is it best to regularly test it in \LaTeX{} or with \latextohtml{}? }}
{{\large When developing a document which contains special \texttt{HTML}
features, is it best to regularly test it in \LaTeX{} or with \latextohtml{}? }}
\end{itemize}
\noindent
The answer to such a question changes as the developer gains
more experience with the available tools.
Some aspects to be considered are discussed 
\hyperref{in a later section}{in Section~}{}{devel} of this manual.


\medskip\htmlrule\goback{overview}\label{quest6}\medskip
\noindent
Information relevant to obtaining the latest version of \latextohtml, 
installation within the local environment, and where to look for
help when things do not go as expected, can be found in 
\hyperref{the support section}{Section~}{}{sec:sup}.


%%% END XTRACTFAQ (START is within outer document)

\bigskip\noindent
What follows next is a brief summary of the features supported
within \latextohtml{}.\html{\dots}


\subsection{List of Features\label{features:ovw}\index{features!listing}}
%\section{List of Features\label{features:ovw}\index{features!listing}}
%
Following is a listing of the main features of the translator;
more specific details on these is given elsewhere in this manual.

\smallskip\noindent
The \latextohtml{} translator \dots
%
\index{components!specify depth}
\begin{itemize}
\item 
breaks up a document into one or more components as specified
by the user\footnote{The user can specify the \emph{depth} at which 
the document should not be broken up any further.};

\index{navigation panel!optional}\index{navigation panel!customisable}%
\item 
provides optional, customisable iconic navigation
panels on every page which contain links to other parts of the
document, or other documents;

\index{equations!inlined}\index{equations!alignment}%
\index{equations!right-justified}\index{equations!numbered}%
\index{arbitrary environments}%
\item 
handles inlined equations ( \(\sum_{i=1}^{n} x_{i} = \int_{0}^{1} f \)), 
handles equation alignment ($A_{B_{C+D}}$), 
right-justified numbered equations (see \hyperref{example}{equation~}{}{eq:demo}), 
tables (see \hyperref{example}{Table~}{}{tab}), 
figures (see \hyperref{example}{Figure~}{}{fig:example}),
and \emph{any arbitrary environment}.
Either the complete environment or sub-parts thereof\html{\dots}
are passed to \LaTeX{}  for conversion to images, which are then either included
in the document or are made available through hypertext links.

\index{figures}\index{tables}\index{thumbnail}%
\index{figures!arbitrarily scaled}\index{tables!arbitrarily scaled}%
\index{figures!oriented}\index{tables!oriented}%
\item 
figures or tables can be arbitrarily scaled and oriented, 
and shown either as inlined images or ``thumbnail'' sketches\html{\dots}
or their contents displayed within a table constructed
using the \HTMLtag{TABLE} tags of \HTMLiii.

\index{environment!theorem-like}%
\index{theorems!theorem-like environments}%
\index{counter!automatic}%
\index{counter!dependency}%
\item
theorem-like environments are supported, along with
automatic numbering and counter dependencies.

\index{browser@browser\label{IIIbrowser}!supports images}%
\index{browser!character-based}%
\item 
can produce output suitable for browsers that support inlined images 
or character-based browsers (as specified by the user).
In particular the \TeX{} or \LaTeX{} code for mathematical expressions
and formulas will be displayed in character-based browsers,
such as \env{lynx}.

\index{color!coloured text}%
\index{color!coloured background}%
\index{image!backdrop}\index{backdrop!tiled with an image}%
\item 
coloured text and/or background is fully supported, as is the
ability to use an image to create a tiled backdrop.

\index{extensions@extensions\label{IIIexts}!new commands}%
\index{extensions!new environments}%
\index{extensions!new theorems}\index{extensions!in style-files}%
\item 
handles definitions of new commands, environments and counters
even when these are defined in external files for input\footnote{%
This allows the definition of \texttt{HTML} macros in \LaTeX{} !};

\index{footnotes}\index{table of contents}\index{list of figures}%
\index{list of tables}\index{bibliography}\index{index}%
\item 
handles footnotes\footnote{Like this!}, 
tables of contents, lists of figures and tables, 
bibliographies and can generate an index.
By including hyperlinks between index entries, 
simple navigation aids can be built into the index, for easy browsing.

%\index{references@references\label{IIIrefs}!hyper-links}%
\index{references@references!hyper-links}%
\index{references!between documents}%
\item 
automatically translates cross-references and citations into hyper-links,
and extends the \LaTeX{}  cross-referencing mechanism to work
not just within a document but \emph{between documents} 
which may reside in remote locations;

\index{character set@character set\label{IIIcharset}}%
\index{character set!ISO-8879@ISO--8879 (ISO--Latin--1)}%
\index{character set!ISO-10646@ISO--10646 (Unicode)}%
\index{special!characters|see{\htmlref{character set}{IIIcharset}}}%
\index{accents}%%
\item 
translates \LaTeX{}  accent and special character 
commands (e.g. \. {A} \O {\"o} \pounds \copyright \P) to
the equivalent ISO--Latin--1 or Unicode character set,
else an image can be created;

\index{hypertext links}\index{hypertext links!multi-media resources}%
\index{hypertext links!Internet services}%
\index{hypertext links!news/sound/video}%
\item 
recognises hypertext links (to multi-media resources or
arbitrary Internet services such as 
\textsl{sound}, \textsl{video}, \textsl{ftp}, \textsl{http}, \textsl{news}) and
links which invoke arbitrary program scripts---all expressed as \LaTeX{}  
commands;

\index{conditional text}%
\item 
recognises \emph{conditional text}  which is intended only for
the hypertext version, or only for the paper (\texttt{.dvi}) version;\par
%
\index{HTML@\texttt{HTML}!raw@raw \texttt{HTML} commands}%
\index{HTML@\texttt{HTML}!interactive forms}%
\item 
can include raw \texttt{HTML} in a \LaTeX{}  document 
(e.g. in order to specify interactive forms);

\index{LaTeX commands@\LaTeX{} commands}\label{hypcites}%
\index{Common LaTeX@Common \LaTeX{} Commands!latex blue@\LaTeX{} blue book}%
\index{LaTeX blue book@\LaTeX{} blue book}%
\item 
can deal sensibly with 
\strikeout{at least the \textit{Common }\LaTeX{} \textit{Commands}
summarised at the back of\\}%
virtually all of the concepts and commands described in
the \LaTeX{} \htmlcite{blue book}{lamp:latex}, 
where there is a meaningful interpretation appropriate to
an \texttt{HTML} document.
Also many other \LaTeX{} constructions are handled, including many described in the 
\LaTeX{} \hypercite{\textit{Companion}}{\textit{Companion}}{}{goossens:latex}
and \LaTeX{} \hypercite{\textit{Graphics Companion} (e.g. \Xy-pic)}%
{\textit{Graphics Companion}}{\Xy-pic}{goossens:latexGraphics};

%\index{images@images\label{IIIimages}!equations}%
\index{images@images!equations}%
\index{GIF|see{\htmlref{images}{IIIimages}}}%
\index{images!tables}%
\index{tables!as images}\index{tables!as HTML@as \texttt{HTML} mark-up}%
\index{HTML@\texttt{HTML}!Version 2.1}%
\index{HTML@\texttt{HTML}!Version 3.0}%
\item 
can be configured to translate equations either
as GIF images or as \texttt{HTML} 3.0 mark-up
(as browsers become available which are suitable for the task),
or by making images of subparts of equations, as required.

\index{references!symbolic}\index{references!document segments}%
\index{document!illustrative examples}%
\item 
links symbolic references across document segments which have been
independently processed;%

\index{LaTeX commands@\LaTeX{} commands!embedded}%
\index{LaTeX commands@\LaTeX{} commands!not syntactically legal}%
\item 
will try to translate any document with embedded \LaTeX{} commands, 
irrespective of whether it is complete or syntactically legal.

\end{itemize}


\subsection{Exemplary Documents\label{exemplary}}
%\section{Exemplary Documents\label{exemplary}}
Here is a selection of documents illustrating different contexts in 
which \latextohtml{} has been used. This list is by no means exhaustive,
but all links were valid as of June 1997.

\medskip\noindent
An earlier listing of converted documents can be found at:
\url{http://cbl.leeds.ac.uk/~nikos/tex2html/doc/latex2html/node6.html#sample}.
However some of the links are no longer valid.

\begin{description}
%
\item [\latextohtml{} documentation and usage]~%
\begin{itemize}
\item 
The \latextohtml{} Translator, User Manual\\
\url{http://www-texdev.ics.mq.edu.au/l2h/docs/manual/}\\
\onlinedoc
\html{\smallskip}%

\item 
Mathematics with \latextohtml{}\\
\url{http://www-texdev.ics.mq.edu.au/l2h/mathdocs/amsmath2/}
\html{\smallskip}%

\item 
\latextohtml, Style Sheets, \texttt{XML}\\
\url{http://www.ramge.de/ax/latex2html/latex2html.html}
\html{\smallskip}%

\item
\textit{All about }\latextohtml, by \htmladdnormallinkfoot{Nikos Drakos}%
{http://cbl.leeds.ac.uk/\~{}nikos/tex2html/doc/latex2html/latex2html.html}
\html{\smallskip}%

\item 
What is \latextohtml{}? (with \Xy-pic diagrams)\\
\url{http://www.maths.mq.edu.au/texdev/Xypic/L2Htalk/}
\html{\smallskip}%

\item 
German specials in \TeX\\
\url{http://www.maths.mq.edu.au/texdev/tests/harn/node1.html}
\html{\smallskip}%

\item
Crayola Colours\\
\url{http://www.maths.mq.edu.au/~ross/latex/crayola/crayola.html}
\html{\smallskip}%


%

\end{itemize}

\htmlrule[50\% center]


\item[Software Manuals, Computing Resources]~%
\begin{itemize}
%
\item 
\Xy-pic User's Guide, accessible from the 
\htmladdnormallinkfoot{\Xy-pic Home Page}{http://www.brics.dk/\~{}krisrose/Xy-pic.html}
or \htmladdnormallinkfoot{``down-under''}{http://www.maths.mq.edu.au/texdev/xyguide-html/}
\html{\smallskip}%

\item 
LiPS --- \htmladdnormallinkfoot{A System for Distributed Processing on Workstations}
{http://cdc-server.cdc.informatik.tu-darmstadt.de/}\\
\htmladdnormallink{Manual}{http://cdc-server.cdc.informatik.tu-darmstadt.de%
/home/zohar/LiPS/documentation/objects/doc/html/Manual2.1/lipsdist/}, 
\htmladdnormallink{Lecture Notes}{http://cdc-server.cdc.informatik.tu-darmstadt.de%
/home/zohar/LiPS/documentation/objects/doc/html/Talks/LectureSS97/LectureSS97/}
and various theses
\html{\smallskip}%

\item
LINUX Documentation Project\\
\url{http://linuxwww.db.erau.edu/ldp/linux.html/}\\
\htmladdnormallink{Getting Started}%
{http://linuxwww.db.erau.edu/ldp/LDP/gs/gs.html}, 
\htmladdnormallink{The LINUX Kernel}%
{http://linuxwww.db.erau.edu/ldp/LDP/tlk/}, 
\htmladdnormallink{Network Administrator's Guide}%
{http://linuxwww.db.erau.edu/ldp/LDP/nag/nag.html}\\
\htmladdnormallink{Programmer's Guide}%
{http://linuxwww.db.erau.edu/LPG/}, 
\htmladdnormallink{System Administrator's Guide}%
{http://linuxwww.db.erau.edu/ldp/LDP/sag/index.html}
\html{\smallskip}%

\item
ECLiPSE --- \htmladdnormallinkfoot{The ECRC Constraint Logic Parallel System}%
{http://www.ecrc.de/eclipse/eclipse.html}\\
\htmladdnormallink{User Manual}%
{http://www.ecrc.de/eclipse/html/umsroot/umsroot.html}, 
\htmladdnormallink{Extensions User Manual}%
{http://www.ecrc.de/eclipse/html/umsroot/umsroot.html}
\html{\smallskip}%

\item
MPQC --- \htmladdnormallinkfoot{Massively Parallel Quantum Chemistry Program}
{http://midway.ca.sandia.gov/\~{}cljanss/mpqc.html}\\
\htmladdnormallink{User Guide}%
{http://midway.ca.sandia.gov/\~{}cljanss/mpqc/mpqcuser/}, 
\htmladdnormallinkfoot{Scientific Computing library}%
{http://midway.ca.sandia.gov/\~{}cljanss/mpqc/prog/prog.html}
\html{\smallskip}%

\item
Xgraphics\\
\url{http://www-theorie.physik.uni-wuerzburg.de/~lueders/Xgraphics/}
\html{\smallskip}%

\item
\textit{Glish} 2.6, User Manual\\
\url{http://aips2.nrao.edu/aips++/docs/html/aips++.html}
\html{\smallskip}%

\item
UTCC --- \htmladdnormallinkfoot{University of Tennessee, Computing Center}%
{http://www.utcc.utk.edu/utcc/docs/}\\
\htmladdnormallink{Facilities and Services}%
{http://www.utcc.utk.edu/utcc/docs/u010003/u010003.html}, 
\htmladdnormallink{GNU Emacs}
{http://www.utcc.utk.edu/utcc/docs/u010599/u010599.html},
\htmladdnormallink{Help Sheet}
{http://www.utcc.utk.edu/utcc/docs/u010627/u010627.html},
\htmladdnormallink{Intro to Unix}
{http://www.utcc.utk.edu/utcc/docs/u010634/u010634.html},
\htmladdnormallink{\texttt{vi} editor}
{http://www.utcc.utk.edu/utcc/docs/u010648/u010648.html}\\
\htmladdnormallink{Software Support}
{http://www.utcc.utk.edu/utcc/docs/u010673/u010673.html}
\html{\smallskip}%


\end{itemize}


\htmlrule[50\% center]


\item [Journals, Conference Proceedings, Newsletters]~%
\begin{itemize}
%
\item ``CERN Computer Newsletter''\\
\url{http://consult.cern.ch/cnls/} issues 214--224.
\html{\smallskip}%

\item ``National Symposium in Mathematics''\\
\url{http://www.maths.mq.edu.au/texdev/MathSymp/}
\html{\smallskip}

\item
American Mathematical Society (\AmS)\quad(\url{http://www.ams.org/journals/})\\
Articles in the following electronic journals use a variant of \latextohtml{}:\\
\htmladdnormallink{\textit{Bulletin of the }\AmS}{http://www.ams.org/bull/}, 
\htmladdnormallink{\textit{Conformal Geometry and Dynamics}}{http://www.ams.org/ecgd/}\\ 
\htmladdnormallink{\textit{Electronic Research Announcements}}{http://www.ams.org/era/}

\smallskip
Other \AmS\ journals are available only to subscribers:\\
\htmladdnormallink{\textit{Journal of the }\AmS}{http://www.ams.org/jams/}, 
\htmladdnormallink{\textit{Mathematics of Computation}}{http://www.ams.org/mcom/},
\htmladdnormallink{\textit{Representation Theory}}{http://www.ams.org/ert/}\\
\htmladdnormallink{\textit{Proceedings of the }\AmS}{http://www.ams.org/proc/}, 
\htmladdnormallink{\textit{Transactions of the }\AmS}{http://www.ams.org/tran/}

%

\end{itemize}


\htmlrule[50\% center]

\item[Encyclop\ae{}dic Reference Material]~%
\begin{itemize}
\item
\htmladdnormallinkfoot{\textit{Eric's Treasure Troves of Science}:}%
{http://www.astro.virginia.edu/\~{}eww6n\#TreasureTroves} 
\htmladdnormallink{Astronomy}%
{http://www.astro.virginia.edu/\~{}eww6n/astro/astro.html}, 
\htmladdnormallink{Chemistry}%
{http://www.astro.virginia.edu/\~{}eww6n/chem/chem.html}, 
\htmladdnormallink{Math}%
{http://www.astro.virginia.edu/\~{}eww6n/math/math.html}\\ 
\htmladdnormallink{Music}%
{http://www.astro.virginia.edu/\~{}eww6n/music/music.html}, 
\htmladdnormallink{Physics}%
{http://www.astro.virginia.edu/\~{}eww6n/physics/physics.html}, 
\htmladdnormallink{Rocket History}%
{http://www.astro.virginia.edu/\~{}eww6n/rockets/rockets.html}, 
\htmladdnormallink{Scientific Books}%
{http://www.astro.virginia.edu/\~{}eww6n/books/books.html}, 
\htmladdnormallink{Scientific Biographies}%
{http://www.astro.virginia.edu/\~{}eww6n/bios/bios.html}
\html{\smallskip}

\item
Resources for Economists on the Internet\\
\url{http://wueconb.wustl.edu/EconFAQ/}
\html{\smallskip}

\item
\textit{Les math{\`e}mes de Lacan} by Jacques Siboni\\
\url{http:www.shef.ac.uk/~psysc/thesaur3/index.html}
\html{\smallskip}

\end{itemize}

\htmlrule[50\% center]


\item[Course Materials]~%
\begin{itemize}
\item
Mathematics Courses at \htmladdnormallinkfoot{Macquarie University}%
{http://www.maths.mq.edu.au/\~{}ross/index.html\#courses}\\
\htmladdnormallink{MATH130}%
{http://www.maths.mq.edu.au/texdev/MATHS/MATH130/study_guide99/}, 
\htmladdnormallink{MATH132}%
{http://www.maths.mq.edu.au/texdev/MATHS/MATH132/study_guide99/}, 
\htmladdnormallink{MATH233}%
{http://www.maths.mq.edu.au/texdev/MATHS/MATH233/study_guide/}, 
\htmladdnormallink{MATH300}%
{http://www.maths.mq.edu.au/texdev/MATHS/MATH300/study_guide/}
\htmladdnormallink{MATH337}%
{http://www.maths.mq.edu.au/texdev/MATHS/MATH337/study_guide99/}, 
\html{\smallskip}

\item 
\htmladdnormallinkfoot{Engineering Science 100}%
{http://fas.sfu.ca/ensc/people/Faculty/jones/personal/ensc100/}%
, Simon Fraser University.
\html{\smallskip}%

\end{itemize}

\htmlrule[50\% center]



\item[Other Interesting Sites]~%
\begin{itemize}
%
\item {\large rel\char64 X} \quad
(\url{http://www.ramge.de/ax/ax.html})
\html{\smallskip}%

\item \Wortmann's thesis and Geological papers use anti-aliased images;\\ 
\htmladdnormallinkfoot{\textit{Die untere und mittlere Kreide}}
{http://bonk.ethz.ch/papers/diss/main-node78.html}, 
\htmladdnormallinkfoot{\textit{The Barium Problem}}
{http://bonk.ethz.ch/papers/lecture-notes-97/ln97-node10.html}.
\html{\smallskip}%

\item 
Stanford Computer Graphics Laboratory, 
\htmladdnormallinkfoot{Publications}{http:/www-graphics.stanford.edu/papers/}\\
\htmladdnormallink{Web Visualization in Hyperbolic Space}
{http:/www-graphics.stanford.edu/papers/webviz/}, 
\htmladdnormallink{Topology of the MBONE}
{http:/www-graphics.stanford.edu/papers/mbone/}, 
\htmladdnormallink{Complex Models from Range Images}
{http:/www-graphics.stanford.edu/papers/volrange/},
\htmladdnormallink{Optical Triangulation}
{http:/www-graphics.stanford.edu/papers/spacetime/},
\htmladdnormallink{Metamorphosis}
{http:/www-graphics.stanford.edu/papers/morph/}


\item
The RGO Worldwide Guide to Public Videoconference Centers\\
\url{http://www.ast.cam.ac.uk/~ralf/vcguide/}
\html{\smallskip}%

\end{itemize}

\end{description}













