
\begin{htmlonly}
\documentclass{article}
\usepackage{html,htmllist,makeidx}
\usepackage{longtable}

%this font is used in  features.tex
\font\wncyr = wncyr10


\begin{htmlonly}
 \usepackage[dvips]{graphicx}
 \usepackage[dvips]{color}
% \usepackage{frames}
 \usepackage[dvips,leftbars]{changebar}
 \newcommand{\FoilTeX}{\env{FoilTeX}}
%
% \newcommand{\htmlurl}[1]{\htmladdnormallink{#1}{#1}}
 \def\Email#1{\htmladdnormallink{<#1>}{mailto:#1}}
 \def\path#1{\texttt{#1}}
% \def\urldef#1#2#3{\def#1{#2{#3}}}
% \newcommand{\onlinedoc}{\url{http://www-dsed.llnl.gov/files/programs/unix/latex2html/manual/}}
 \newcommand{\onlinedoc}{\url{http://www-texdev.ics.mq.edu.au/l2h/docs/manual/}}
% \def\patches{\htmlurl{http://www-dsed.llnl.gov/files/programs/unix/latex2html/}}
 \def\patches{\htmlurl{http://www.latex2html.org/current/}}
 \def\sourceA{%
  \url{http://www.latex2html.org/current/}}
%  \url{http://www-dsed.llnl.gov/files/programs/unix/latex2html/sources/}}
% \def\sourceB{\htmlurl{ftp://ftp.mpn.com/pub/nikos/latex2html-98.1.tar.gz}}
 \def\sourceB{\CTANtug{\CTANA}}
 \def\sourceC{\htmlurl{http://ftp.rzg.mpg.de/pub/software/latex2html/sources/}}
% \def\CVSsite{\url{http://cdc-server.cdc.informatik.tu-darmstadt.de/\~{}latex2html/}}
% \def\CVSrepos{\url{http://cdc-server.cdc.informatik.tu-darmstadt.de/\~{}latex2html/user/}}
% \def\CVSlatest{\url{http://cdc-server.cdc.informatik.tu-darmstadt.de/\~{}latex2html/l2h-latest.tar.gz}}
% \def\CVSsite{\htmlurl{http://saftsack.fs.uni-bayreuth.de/~latex2ht/}}
% \def\CVSrepos{\htmlurl{http://saftsack.fs.uni-bayreuth.de/~latex2ht/user/}}
 \def\CVSrepos{\htmlurl{https://www.github.com/latex2html/latex2html/}}
% \def\CVSlatest{\htmlurl{http://saftsack.fs.uni-bayreuth.de/~latex2ht/l2h-latest.tar.gz}}
 \def\CVSlatest{\htmlurl{https://www.github.com/latex2html/latex2html/archive/master.zip}}
 \newcommand{\CTANtug}[1]{\htmlurl{http://mirrors.ctan.org/#1}}
 \def\CTANA{support/latex2html}
 \def\tugURL{\htmlurl{http://www.tug.org/}}
 \def\danteURL{\htmlurl{http://www.dante.de/}}
% \def\ListURL{\htmlurl{http://www.xray.mpe.mpg.de/mailing-lists/latex2html/}}
 \def\ListURL{\htmlurl{http://www.tug.org/mailman/listinfo/latex2html/}}
%
% \def\glossary#1{\index{#1@\texttt{#1} \label{III#1}\htmlref{(G)}{GGG#1}}}
 \newcommand{\glossary}[1]{\index{#1@\texttt{#1} \label{III#1}\htmlref{(G)}{GGG#1}}}
 \def\Glossary#1#2{\index{#1@{#2} \label{III#1}\htmlref{(G)}{GGG#1}}}
% \newcommand{\Glossary}[2]{\index{#1@{#2} \label{III#1}\htmlref{(G)}{GGG#1}}}
%
% hack to suppress changebar entries read from .aux file  !!! bug in changebar.perl !!! 
 \makeatletter
 \def\cb@barpoint#1#2#3{}
 \makeatother
%
\newcommand{\sameas}[1]{\textcolor{red}{Same as setting: #1}}
\newcommand{\onlinedocRM}{\url{http://www-texdev.ics.mq.edu.au/l2h/docs/manual/}}
\newcommand{\EXcolors}{\url{http://www-texdev.ics.mq.edu.au/l2h/crayola/}}
%
\newcommand{\Lc}[1]{\texttt{\char92#1}} % LaTeX command
\newcommand{\Tc}[1]{\texttt{\char92#1}} % TeX command
\newcommand{\Cs}[1]{ \texttt{-#1} }   % command-line switch
\newcommand{\Ve}[1]{\index{#1@\texttt{#1}}\texttt{#1}} % version extension
\newcommand{\gsl}[1]{#1}
\newcommand{\indexentry}[2]{\item #1 #2}

%
\internal{}%
\internal{O}%
\internal{S}%
\internal{E}%
\internal{M}%
\internal{H}%
\internal{P}%

\newcommand{\texdev}[1]{\htmladdnormallink
 {\texdevURL/#1}{\texdevURL/#1}}
\newcommand{\ctan}{http://ctan.tug.org/ctan}
\newcommand{\ctanTUG}{\htmladdnormallinkfoot
 {TUG's searchable CTAN site}{\ctan}}
\newcommand{\ctanURL}[1]{\htmladdnormallink
 {CTAN:\texttt{.../#1}}{\ctan/#1}}
\newcommand{\ctanURLbr}[1]{\htmladdnormallink
 {CTAN:\texttt{.../#1}}{\ctan/#1}}
\newcommand{\ctanTUGurl}[2]{\ctanURL{#2}}
\newcommand{\indichtml}{l2h/indic/IndicHTML/}

\end{htmlonly}

\newcommand{\TUGurl}{\htmladdnormallink{TUG}{\htmlurl{http://www.tug.org}}}

\newcommand{\texdevURL}{http://www-texdev.ics.mq.edu.au/}
\newcommand{\Unicode}{\htmladdnormallinkfoot
 {Unicode}{http://www.unicode.org/}}
\newcommand{\IndicHTML}{
 {Indic\TeX/HTML}}

%begin{latexonly}
\usepackage{array}

\newcommand{\Cs}[1]{{\upshape`\,\texttt{-#1}\,'}}
\newcommand{\Ve}[1]{\index{#1@\texttt{#1}}{\upshape`\,\texttt{#1}\,'}}
\newcommand{\Lc}[1]{{\upshape\ttfamily\char92#1}}
\newcommand{\Tc}[1]{{\upshape\ttfamily\char92#1}}

\newcommand{\ctanTUG}[1]{TUG's searchable CTAN site\footnote{\ctanTUGurl}}
\newcommand{\ctanURL}{CTAN: \penalty-200\ctanurl}
\newcommand{\ctanURLbr}{CTAN: \newline\ctanurl}
\urldef\texdev\url{http://www-texdev.ics.mq.edu.au/}
\urldef\ctanTUGurl\url{http://mirrors.ctan.org/}
\urldef\indichtml\url{l2h/indic/IndicHTML/}

\urldef\NetscapeHome\url{http://home.netscape.com}
\urldef\giftransURL\url{http://ftp.ost.eltele.no/pub/multimedia/giftrans-1.12.2.tar.gz}
\urldef\WinZipURL\url{http://www.winzip.com/getsite.cgi?winzip70.exe}
\urldef\MikTeXURL\url{http://www.miktex.de}
\urldef\ActiveStateURL\url{http://www.activestate.com}
\urldef\NetpbmWinURL\url{ftp://ftp.ese-metz.fr/pub/TeX/win32/}
%end{latexonly}

\begin{htmlonly}
\newcommand{\NetscapeHome}{\url{http://home.netscape.com}}
\newcommand{\giftransURL}{\url{http://ftp.ost.eltele.no/pub/multimedia/giftrans-1.12.2.tar.gz}}
\newcommand{\WinZipURL}{\url{http://www.winzip.com/getsite.cgi?winzip70.exe}}
\newcommand{\MikTeXURL}{\url{http://www.miktex.de}}
\newcommand{\ActiveStateURL}{\url{http://www.activestate.com}}
\newcommand{\NetpbmWinURL}{\url{ftp://ftp.ese-metz.fr/pub/TeX/win32/}}
\end{htmlonly}



% thanks to \KrisRose for this macro:
%\def\smiley{\hbox{\rlap{$\bigcirc$}\kern1.3pt$\scriptstyle\ddot\smile\;$}}
\def\mathsmiley{\smiley\,}


\newcommand{\Lcs}[1]{{\upshape\ttfamily\char92#1}}

\renewcommand{\thefootnote}{\arabic{footnote}}

%\newcommand{\godown}[1]{{\htmlref
% {\htmladdimg[left BORDER=0]{../psfiles/dn.gif}}{#1}}}
\newcommand{\godown}[1]{}

%\newcommand{\goback}[1]{{\htmlref
% {\htmladdimg[left BORDER=0]{../psfiles/up.gif}}{#1}}}
\newcommand{\goback}[1]{}


\newcommand{\WiiiC}{\htmladdnormallinkfoot{World Wide Web Consortium}{http://www.w3c.org/}}
\newcommand{\latextohtmlNG}{\latextohtml-NG}
\newcommand{\maillist}{\htmladdnormallinkfoot{\latextohtml{} mailing list}%
{http://cbl.leeds.ac.uk/nikos/tex2html/doc/mail/mail.html}}
%
\newcommand{\HTMLiii}{\textup{\texttt{HTML} 3.2}}%
\newcommand{\HTMLIII}{\htmladdnormallink{\HTMLiii}%
{http://www.w3.org/TR/REC-html32.html}}%

\newcommand{\HTMLiv}{\textup{\texttt{HTML} 4.0}}%
\newcommand{\HTMLIV}{\htmladdnormallink{\HTMLiv}%
{http://www.w3.org/TR/REC-html40/}}%

%\newcommand{\latextohtml}{\textup{\LaTeX 2{\ttfamily HTML}}}%
\newcommand{\Perl}{\htmlref{\textsl{Perl}}{GGGPerl}}%  
\newcommand{\PS}{\htmlref{\textup{Post\-Script}}{GGGpostscript}\Glossary{PostScript}{PostScript}{}}%
\providecommand{\MF}{}
\renewcommand{\MF}{\htmlref{\textsl{Metafont}}{GGGMetafont}\Glossary{Metafont}{\textsl{Metafont}}{}}%
\newcommand{\fn}[1]{\htmlref{\texttt{#1}}{GGG#1}\glossary{#1}}%  file names, with link to glossary
\newcommand{\gn}[1]{\texttt{#1}\label{GGG#1}\htmlref{\^{}}{III#1}}%  file names, labelled within glossary
%\newcommand{\appl}[1]{\htmlref{\textsl{#1}}{GGG#1}\Glossary{#1}{\textsl{#1}}{}}%  application software names
\newcommand{\appl}[1]{\htmlref{\textsl{#1}}{GGG#1}\Glossary{#1}{\gsl{#1}}{}}%  application software names
%
\newcommand{\env}[1]{{\upshape\sffamily #1}}%  LaTeX environment and package names
\newcommand{\HTMLtag}[1]{\path{<#1>}}%  HTML tag
\newcommand{\Meta}[1]{\texttt{\upshape<\textit{#1}>}}%  Meta string

%
% developer names and addresses:
%
\newcommand{\NikosDrakos}{\index{Drakos, Nikos}%\Email{nikos@cbl.leeds.ac.uk}
{Nikos Drakos}}
%
\newcommand{\RossMoore}{\index{Moore, Ross}%\Email{ross@maths.mq.edu.au}
{Ross Moore}}
\newcommand{\Macquarie}{\htmladdnormallink
{Macquarie University}{http://www.mq.edu.au/}}
%
\newcommand{\Hennecke}{\index{Hennecke, Marcus}%\Email{hennecke@dbag.ulm.DaimlerBenz.COM}
{Marcus Hennecke}}
%
\newcommand{\Noworolski}{\index{Noworolski, Mark}%\Email{jmn@eecs.berkeley.edu}
{Mark Noworolski}}
%
\newcommand{\Isani}{\index{Isani, Sidik}%\Email{isani@cfht.hawaii.edu}
{Sidik Isani}}
%
\newcommand{\Goossens}{\index{Goossens, Michel}%\Email{goossens@cern.ch}
{Michel Goossens}}
%
\newcommand{\Wilck}{\index{Wilck, Martin}%\Email{martin@tropos.de}
{Martin Wilck}}
%
\newcommand{\PatrickDaly}{\index{Daly, Patrick}%\Email{daly@linmpi.dnet.gwdg.de}
{Patrick Daly}}
%
\newcommand{\Arseneau}{\index{Arseneau, Donald}%
{Donald Arseneau}}
%
\newcommand{\JohannesBraams}{\index{Braams, Johannes}%
{Johannes Braams}}
%
\newcommand{\HerbSwan}{\index{Swan, Herb}%\Email{herb.swan@perc.Arco.COM}
%\htmladdnormallink{Herb Swan}{mailto:herb.swan@perc.Arco.COM}}
{Herb Swan}}
%
\newcommand{\Lippmann}{\index{Lippmann, Jens}%\Email{lippmann@rbg.informatik.tu-darmstadt.de}
{Jens Lippmann}}
%
\newcommand{\Rouchal}{\index{Rouchal, Marek}%\Email{marek@saftsack.fs.uni-bayreuth.de}
{Marek Rouchal}}
%
\newcommand{\Bohnet}{\index{Bohnet, Achim}%\Email{ach@rosat.mpe-garching.mpg.de}
{Achim Bohnet}}
%
\newcommand{\Nelson}{\index{Nelson, Scott}%\Email{nelson18@llnl.gov}
{Scott Nelson}}
%
\newcommand{\AxelRamge}{\index{Ramge, Axel}%\Email{axel@ramge.de}
{Axel Ramge}}
%
\newcommand{\Popineau}{\index{Popineau, Fabrice}%\Email{popineau@esemetz.ese-metz.fr}
{Fabrice Popineau}}
%
\newcommand{\Wortmann}{\index{Wortmann, Uli}%\Email{uli12@bonk.ethz.ch}
{Uli Wortmann}}
%
\newcommand{\Veytsman}{\index{Veytsman, Boris}%\Email{boris@plmsc.psu.edu}
{Boris Veytsman}}
%
\newcommand{\Taupin}{\index{Taupin, Daniel}%\Email{taupin@lps.u-psud.fr}
{Daniel Taupin}}
%
\newcommand{\ArtOgawa}{\index{Ogawa, Art}%\Email{ogawa@teleport.com}
{Art Ogawa}}
%
\newcommand{\Perbandt}{\index{Perbandt, Adalbert}%\Email{perbandt@hl.siemens.de}
{Adalbert Perbandt}}
%
%\endinput

%
% (La)TeX related URLs
%
\newcommand{\CSEP}{\index{Computer Science Education Project!CSEP}%
{Computational Science Education Project (ORNL)}%
} 
%
\newcommand{\CBLU}{\index{Computer Based Learning Unit!CBLU}%
{Computer Based Learning Unit}}%
%
\newcommand{\CERN}{\index{CERN}%
\htmladdnormallink{CERN}{http://www.cern.ch}} 
%
\newcommand{\KrisRose}{\index{Rose, Kristoffer}%\Email{krisrose@brics.dk}
{Kristoffer Rose}}
%
\newcommand{\XypicDK}{\index{Xy-pic@\Xy-pic graphics package}%
\index{Xy-pic@\Xy-pic graphics package!home page}%
{Xy-pic}}
%
\newcommand{\LiPS}{\index{LiPS Design Team}%
{LiPS Design Team}}
\newcommand{\FIDarmstadt}{\index{Fachbereich Informatik, Darmstadt}%
\htmladdnormallink{Fachbereich Informatik}{http://www.informatik.tu-darmstadt.de/}}
\newcommand{\Darmstadt}{\index{Darmstadt!Fachbereich Informatik}%
{Darmstadt}}
%
\newcommand{\DANTE}{\index{DANTE}%
\htmladdnormallink{DANTE e.V.}{http://www.dante.de/}}
\newcommand{\Praesidium}{\index{DANTE!Praesidium}%
\htmladdnormallink{Praesidium}{http://www.dante.de/dante/Organe.html}}
\newcommand{\LaTeXiii}{\index{LaTeX@\LaTeX!LaTeX3@\LaTeX{3}}%
\htmladdnormallink{\LaTeX{3}}{http://www.tex.ac.uk/CTAN/latex/latex3}}
%
\newcommand{\Engberg}{\index{Uffe Engberg}%
\htmladdnormallinkfoot{Uffe Engberg}{http://www.brics.dk/\~{}engberg}} 

\newcommand{\texdevINDIC}{\texdev\indic}
%\endinput


%% earlier contributions from...
\newcommand{\AndrewCole}{\index{Computer Based Learning Unit!Andrew Cole}%
{Andrew Cole}}
%
\newcommand{\AnaPaiva}{\index{Computer Based Learning Unit!Ana Maria Paiva}%
{Ana Maria Paiva}}
%
\newcommand{\RodWilliams}{\index{Computer Based Learning Unit!Roderick Williams}%
{Roderick Williams}}
%
\newcommand{\JamilSawar}{\index{Computer Based Learning Unit!Jamil Sawar}%
{Jamil Sawar}}

\newcommand{\RobertThau}{\index{Thau, Robert S.}%
{Robert S. Thau}}

\newcommand{\RobertCailliau}{\index{Cailliau, Robert}%
{Robert Cailliau}}

\newcommand{\EricCarroll}{\index{Carroll, Eric}%
{Eric Carroll}}

\newcommand{\FranzVojik}{\index{Vojik, Franz}%
{Franz Vojik}}

\newcommand{\AxelBelinfante}{\index{Belinfante, Axel}%
{Axel Belinfante}}

\newcommand{\ToddLittle}{\index{Little, Todd}%
{Todd Little}}

\newcommand{\VerenaUmar}{\index{Umar, Verena}%
{Verena Umar}}

\newcommand{\IanFoster}{\index{Foster, Ian}%
{Ian Foster}}

\newcommand{\BobOlson}{\index{Olson, Bob}%
{Bob Olson}}

\endinput


%
% Ian Foster \Email{itf@mcs.anl.gov}
% Bob Olson \Email{olson@mcs.anl.gov}
% Verena Umar \Email{verena@edu.vanderbilt.cas.compsci}
% Axel Belinfante \Email{Axel.Belinfante@cs.utwente.nl}
% Todd Little \Email{little@com.dec.enet.nuts2u}
% Franz Vojik \Email{vojik@de.tu-muenchen.informatik}
% Eric Carroll \Email{eric@ca.utoronto.utcc.enfm}
% Roderick Williams \Email{rodw@cbl.leeds.ac.uk}
% Robert Cailliau \Email{cailliau@cernnext.cern.ch}
% Toni Lantunen at CERN
% Ana Maria Paiva, Jamil Sawar, Andrew Cole at CBLU Leeds
% Phillip Conrad (Perfect Byte, Inc.)
% L. Peter Deutsch (

%

\begin{htmlonly}
\documentclass{article}
\usepackage{html,htmllist,makeidx}
\usepackage{longtable}

%this font is used in  features.tex
\font\wncyr = wncyr10


\begin{htmlonly}
 \usepackage[dvips]{graphicx}
 \usepackage[dvips]{color}
% \usepackage{frames}
 \usepackage[dvips,leftbars]{changebar}
 \newcommand{\FoilTeX}{\env{FoilTeX}}
%
% \newcommand{\htmlurl}[1]{\htmladdnormallink{#1}{#1}}
 \def\Email#1{\htmladdnormallink{<#1>}{mailto:#1}}
 \def\path#1{\texttt{#1}}
% \def\urldef#1#2#3{\def#1{#2{#3}}}
% \newcommand{\onlinedoc}{\url{http://www-dsed.llnl.gov/files/programs/unix/latex2html/manual/}}
 \newcommand{\onlinedoc}{\url{http://www-texdev.ics.mq.edu.au/l2h/docs/manual/}}
% \def\patches{\htmlurl{http://www-dsed.llnl.gov/files/programs/unix/latex2html/}}
 \def\patches{\htmlurl{http://www.latex2html.org/current/}}
 \def\sourceA{%
  \url{http://www.latex2html.org/current/}}
%  \url{http://www-dsed.llnl.gov/files/programs/unix/latex2html/sources/}}
% \def\sourceB{\htmlurl{ftp://ftp.mpn.com/pub/nikos/latex2html-98.1.tar.gz}}
 \def\sourceB{\CTANtug{\CTANA}}
 \def\sourceC{\htmlurl{http://ftp.rzg.mpg.de/pub/software/latex2html/sources/}}
% \def\CVSsite{\url{http://cdc-server.cdc.informatik.tu-darmstadt.de/\~{}latex2html/}}
% \def\CVSrepos{\url{http://cdc-server.cdc.informatik.tu-darmstadt.de/\~{}latex2html/user/}}
% \def\CVSlatest{\url{http://cdc-server.cdc.informatik.tu-darmstadt.de/\~{}latex2html/l2h-latest.tar.gz}}
% \def\CVSsite{\htmlurl{http://saftsack.fs.uni-bayreuth.de/~latex2ht/}}
% \def\CVSrepos{\htmlurl{http://saftsack.fs.uni-bayreuth.de/~latex2ht/user/}}
 \def\CVSrepos{\htmlurl{https://www.github.com/latex2html/latex2html/}}
% \def\CVSlatest{\htmlurl{http://saftsack.fs.uni-bayreuth.de/~latex2ht/l2h-latest.tar.gz}}
 \def\CVSlatest{\htmlurl{https://www.github.com/latex2html/latex2html/archive/master.zip}}
 \newcommand{\CTANtug}[1]{\htmlurl{http://mirrors.ctan.org/#1}}
 \def\CTANA{support/latex2html}
 \def\tugURL{\htmlurl{http://www.tug.org/}}
 \def\danteURL{\htmlurl{http://www.dante.de/}}
% \def\ListURL{\htmlurl{http://www.xray.mpe.mpg.de/mailing-lists/latex2html/}}
 \def\ListURL{\htmlurl{http://www.tug.org/mailman/listinfo/latex2html/}}
%
% \def\glossary#1{\index{#1@\texttt{#1} \label{III#1}\htmlref{(G)}{GGG#1}}}
 \newcommand{\glossary}[1]{\index{#1@\texttt{#1} \label{III#1}\htmlref{(G)}{GGG#1}}}
 \def\Glossary#1#2{\index{#1@{#2} \label{III#1}\htmlref{(G)}{GGG#1}}}
% \newcommand{\Glossary}[2]{\index{#1@{#2} \label{III#1}\htmlref{(G)}{GGG#1}}}
%
% hack to suppress changebar entries read from .aux file  !!! bug in changebar.perl !!! 
 \makeatletter
 \def\cb@barpoint#1#2#3{}
 \makeatother
%
\newcommand{\sameas}[1]{\textcolor{red}{Same as setting: #1}}
\newcommand{\onlinedocRM}{\url{http://www-texdev.ics.mq.edu.au/l2h/docs/manual/}}
\newcommand{\EXcolors}{\url{http://www-texdev.ics.mq.edu.au/l2h/crayola/}}
%
\newcommand{\Lc}[1]{\texttt{\char92#1}} % LaTeX command
\newcommand{\Tc}[1]{\texttt{\char92#1}} % TeX command
\newcommand{\Cs}[1]{ \texttt{-#1} }   % command-line switch
\newcommand{\Ve}[1]{\index{#1@\texttt{#1}}\texttt{#1}} % version extension
\newcommand{\gsl}[1]{#1}
\newcommand{\indexentry}[2]{\item #1 #2}

%
\internal{}%
\internal{O}%
\internal{S}%
\internal{E}%
\internal{M}%
\internal{H}%
\internal{P}%

\newcommand{\texdev}[1]{\htmladdnormallink
 {\texdevURL/#1}{\texdevURL/#1}}
\newcommand{\ctan}{http://ctan.tug.org/ctan}
\newcommand{\ctanTUG}{\htmladdnormallinkfoot
 {TUG's searchable CTAN site}{\ctan}}
\newcommand{\ctanURL}[1]{\htmladdnormallink
 {CTAN:\texttt{.../#1}}{\ctan/#1}}
\newcommand{\ctanURLbr}[1]{\htmladdnormallink
 {CTAN:\texttt{.../#1}}{\ctan/#1}}
\newcommand{\ctanTUGurl}[2]{\ctanURL{#2}}
\newcommand{\indichtml}{l2h/indic/IndicHTML/}

\end{htmlonly}

\newcommand{\TUGurl}{\htmladdnormallink{TUG}{\htmlurl{http://www.tug.org}}}

\newcommand{\texdevURL}{http://www-texdev.ics.mq.edu.au/}
\newcommand{\Unicode}{\htmladdnormallinkfoot
 {Unicode}{http://www.unicode.org/}}
\newcommand{\IndicHTML}{
 {Indic\TeX/HTML}}

%begin{latexonly}
\usepackage{array}

\newcommand{\Cs}[1]{{\upshape`\,\texttt{-#1}\,'}}
\newcommand{\Ve}[1]{\index{#1@\texttt{#1}}{\upshape`\,\texttt{#1}\,'}}
\newcommand{\Lc}[1]{{\upshape\ttfamily\char92#1}}
\newcommand{\Tc}[1]{{\upshape\ttfamily\char92#1}}

\newcommand{\ctanTUG}[1]{TUG's searchable CTAN site\footnote{\ctanTUGurl}}
\newcommand{\ctanURL}{CTAN: \penalty-200\ctanurl}
\newcommand{\ctanURLbr}{CTAN: \newline\ctanurl}
\urldef\texdev\url{http://www-texdev.ics.mq.edu.au/}
\urldef\ctanTUGurl\url{http://mirrors.ctan.org/}
\urldef\indichtml\url{l2h/indic/IndicHTML/}

\urldef\NetscapeHome\url{http://home.netscape.com}
\urldef\giftransURL\url{http://ftp.ost.eltele.no/pub/multimedia/giftrans-1.12.2.tar.gz}
\urldef\WinZipURL\url{http://www.winzip.com/getsite.cgi?winzip70.exe}
\urldef\MikTeXURL\url{http://www.miktex.de}
\urldef\ActiveStateURL\url{http://www.activestate.com}
\urldef\NetpbmWinURL\url{ftp://ftp.ese-metz.fr/pub/TeX/win32/}
%end{latexonly}

\begin{htmlonly}
\newcommand{\NetscapeHome}{\url{http://home.netscape.com}}
\newcommand{\giftransURL}{\url{http://ftp.ost.eltele.no/pub/multimedia/giftrans-1.12.2.tar.gz}}
\newcommand{\WinZipURL}{\url{http://www.winzip.com/getsite.cgi?winzip70.exe}}
\newcommand{\MikTeXURL}{\url{http://www.miktex.de}}
\newcommand{\ActiveStateURL}{\url{http://www.activestate.com}}
\newcommand{\NetpbmWinURL}{\url{ftp://ftp.ese-metz.fr/pub/TeX/win32/}}
\end{htmlonly}



% thanks to \KrisRose for this macro:
%\def\smiley{\hbox{\rlap{$\bigcirc$}\kern1.3pt$\scriptstyle\ddot\smile\;$}}
\def\mathsmiley{\smiley\,}


\newcommand{\Lcs}[1]{{\upshape\ttfamily\char92#1}}

\renewcommand{\thefootnote}{\arabic{footnote}}

%\newcommand{\godown}[1]{{\htmlref
% {\htmladdimg[left BORDER=0]{../psfiles/dn.gif}}{#1}}}
\newcommand{\godown}[1]{}

%\newcommand{\goback}[1]{{\htmlref
% {\htmladdimg[left BORDER=0]{../psfiles/up.gif}}{#1}}}
\newcommand{\goback}[1]{}


\newcommand{\WiiiC}{\htmladdnormallinkfoot{World Wide Web Consortium}{http://www.w3c.org/}}
\newcommand{\latextohtmlNG}{\latextohtml-NG}
\newcommand{\maillist}{\htmladdnormallinkfoot{\latextohtml{} mailing list}%
{http://cbl.leeds.ac.uk/nikos/tex2html/doc/mail/mail.html}}
%
\newcommand{\HTMLiii}{\textup{\texttt{HTML} 3.2}}%
\newcommand{\HTMLIII}{\htmladdnormallink{\HTMLiii}%
{http://www.w3.org/TR/REC-html32.html}}%

\newcommand{\HTMLiv}{\textup{\texttt{HTML} 4.0}}%
\newcommand{\HTMLIV}{\htmladdnormallink{\HTMLiv}%
{http://www.w3.org/TR/REC-html40/}}%

%\newcommand{\latextohtml}{\textup{\LaTeX 2{\ttfamily HTML}}}%
\newcommand{\Perl}{\htmlref{\textsl{Perl}}{GGGPerl}}%  
\newcommand{\PS}{\htmlref{\textup{Post\-Script}}{GGGpostscript}\Glossary{PostScript}{PostScript}{}}%
\providecommand{\MF}{}
\renewcommand{\MF}{\htmlref{\textsl{Metafont}}{GGGMetafont}\Glossary{Metafont}{\textsl{Metafont}}{}}%
\newcommand{\fn}[1]{\htmlref{\texttt{#1}}{GGG#1}\glossary{#1}}%  file names, with link to glossary
\newcommand{\gn}[1]{\texttt{#1}\label{GGG#1}\htmlref{\^{}}{III#1}}%  file names, labelled within glossary
%\newcommand{\appl}[1]{\htmlref{\textsl{#1}}{GGG#1}\Glossary{#1}{\textsl{#1}}{}}%  application software names
\newcommand{\appl}[1]{\htmlref{\textsl{#1}}{GGG#1}\Glossary{#1}{\gsl{#1}}{}}%  application software names
%
\newcommand{\env}[1]{{\upshape\sffamily #1}}%  LaTeX environment and package names
\newcommand{\HTMLtag}[1]{\path{<#1>}}%  HTML tag
\newcommand{\Meta}[1]{\texttt{\upshape<\textit{#1}>}}%  Meta string

%
% developer names and addresses:
%
\newcommand{\NikosDrakos}{\index{Drakos, Nikos}%\Email{nikos@cbl.leeds.ac.uk}
{Nikos Drakos}}
%
\newcommand{\RossMoore}{\index{Moore, Ross}%\Email{ross@maths.mq.edu.au}
{Ross Moore}}
\newcommand{\Macquarie}{\htmladdnormallink
{Macquarie University}{http://www.mq.edu.au/}}
%
\newcommand{\Hennecke}{\index{Hennecke, Marcus}%\Email{hennecke@dbag.ulm.DaimlerBenz.COM}
{Marcus Hennecke}}
%
\newcommand{\Noworolski}{\index{Noworolski, Mark}%\Email{jmn@eecs.berkeley.edu}
{Mark Noworolski}}
%
\newcommand{\Isani}{\index{Isani, Sidik}%\Email{isani@cfht.hawaii.edu}
{Sidik Isani}}
%
\newcommand{\Goossens}{\index{Goossens, Michel}%\Email{goossens@cern.ch}
{Michel Goossens}}
%
\newcommand{\Wilck}{\index{Wilck, Martin}%\Email{martin@tropos.de}
{Martin Wilck}}
%
\newcommand{\PatrickDaly}{\index{Daly, Patrick}%\Email{daly@linmpi.dnet.gwdg.de}
{Patrick Daly}}
%
\newcommand{\Arseneau}{\index{Arseneau, Donald}%
{Donald Arseneau}}
%
\newcommand{\JohannesBraams}{\index{Braams, Johannes}%
{Johannes Braams}}
%
\newcommand{\HerbSwan}{\index{Swan, Herb}%\Email{herb.swan@perc.Arco.COM}
%\htmladdnormallink{Herb Swan}{mailto:herb.swan@perc.Arco.COM}}
{Herb Swan}}
%
\newcommand{\Lippmann}{\index{Lippmann, Jens}%\Email{lippmann@rbg.informatik.tu-darmstadt.de}
{Jens Lippmann}}
%
\newcommand{\Rouchal}{\index{Rouchal, Marek}%\Email{marek@saftsack.fs.uni-bayreuth.de}
{Marek Rouchal}}
%
\newcommand{\Bohnet}{\index{Bohnet, Achim}%\Email{ach@rosat.mpe-garching.mpg.de}
{Achim Bohnet}}
%
\newcommand{\Nelson}{\index{Nelson, Scott}%\Email{nelson18@llnl.gov}
{Scott Nelson}}
%
\newcommand{\AxelRamge}{\index{Ramge, Axel}%\Email{axel@ramge.de}
{Axel Ramge}}
%
\newcommand{\Popineau}{\index{Popineau, Fabrice}%\Email{popineau@esemetz.ese-metz.fr}
{Fabrice Popineau}}
%
\newcommand{\Wortmann}{\index{Wortmann, Uli}%\Email{uli12@bonk.ethz.ch}
{Uli Wortmann}}
%
\newcommand{\Veytsman}{\index{Veytsman, Boris}%\Email{boris@plmsc.psu.edu}
{Boris Veytsman}}
%
\newcommand{\Taupin}{\index{Taupin, Daniel}%\Email{taupin@lps.u-psud.fr}
{Daniel Taupin}}
%
\newcommand{\ArtOgawa}{\index{Ogawa, Art}%\Email{ogawa@teleport.com}
{Art Ogawa}}
%
\newcommand{\Perbandt}{\index{Perbandt, Adalbert}%\Email{perbandt@hl.siemens.de}
{Adalbert Perbandt}}
%
%\endinput

%
% (La)TeX related URLs
%
\newcommand{\CSEP}{\index{Computer Science Education Project!CSEP}%
{Computational Science Education Project (ORNL)}%
} 
%
\newcommand{\CBLU}{\index{Computer Based Learning Unit!CBLU}%
{Computer Based Learning Unit}}%
%
\newcommand{\CERN}{\index{CERN}%
\htmladdnormallink{CERN}{http://www.cern.ch}} 
%
\newcommand{\KrisRose}{\index{Rose, Kristoffer}%\Email{krisrose@brics.dk}
{Kristoffer Rose}}
%
\newcommand{\XypicDK}{\index{Xy-pic@\Xy-pic graphics package}%
\index{Xy-pic@\Xy-pic graphics package!home page}%
{Xy-pic}}
%
\newcommand{\LiPS}{\index{LiPS Design Team}%
{LiPS Design Team}}
\newcommand{\FIDarmstadt}{\index{Fachbereich Informatik, Darmstadt}%
\htmladdnormallink{Fachbereich Informatik}{http://www.informatik.tu-darmstadt.de/}}
\newcommand{\Darmstadt}{\index{Darmstadt!Fachbereich Informatik}%
{Darmstadt}}
%
\newcommand{\DANTE}{\index{DANTE}%
\htmladdnormallink{DANTE e.V.}{http://www.dante.de/}}
\newcommand{\Praesidium}{\index{DANTE!Praesidium}%
\htmladdnormallink{Praesidium}{http://www.dante.de/dante/Organe.html}}
\newcommand{\LaTeXiii}{\index{LaTeX@\LaTeX!LaTeX3@\LaTeX{3}}%
\htmladdnormallink{\LaTeX{3}}{http://www.tex.ac.uk/CTAN/latex/latex3}}
%
\newcommand{\Engberg}{\index{Uffe Engberg}%
\htmladdnormallinkfoot{Uffe Engberg}{http://www.brics.dk/\~{}engberg}} 

\newcommand{\texdevINDIC}{\texdev\indic}
%\endinput


%% earlier contributions from...
\newcommand{\AndrewCole}{\index{Computer Based Learning Unit!Andrew Cole}%
{Andrew Cole}}
%
\newcommand{\AnaPaiva}{\index{Computer Based Learning Unit!Ana Maria Paiva}%
{Ana Maria Paiva}}
%
\newcommand{\RodWilliams}{\index{Computer Based Learning Unit!Roderick Williams}%
{Roderick Williams}}
%
\newcommand{\JamilSawar}{\index{Computer Based Learning Unit!Jamil Sawar}%
{Jamil Sawar}}

\newcommand{\RobertThau}{\index{Thau, Robert S.}%
{Robert S. Thau}}

\newcommand{\RobertCailliau}{\index{Cailliau, Robert}%
{Robert Cailliau}}

\newcommand{\EricCarroll}{\index{Carroll, Eric}%
{Eric Carroll}}

\newcommand{\FranzVojik}{\index{Vojik, Franz}%
{Franz Vojik}}

\newcommand{\AxelBelinfante}{\index{Belinfante, Axel}%
{Axel Belinfante}}

\newcommand{\ToddLittle}{\index{Little, Todd}%
{Todd Little}}

\newcommand{\VerenaUmar}{\index{Umar, Verena}%
{Verena Umar}}

\newcommand{\IanFoster}{\index{Foster, Ian}%
{Ian Foster}}

\newcommand{\BobOlson}{\index{Olson, Bob}%
{Bob Olson}}

\endinput


%
% Ian Foster \Email{itf@mcs.anl.gov}
% Bob Olson \Email{olson@mcs.anl.gov}
% Verena Umar \Email{verena@edu.vanderbilt.cas.compsci}
% Axel Belinfante \Email{Axel.Belinfante@cs.utwente.nl}
% Todd Little \Email{little@com.dec.enet.nuts2u}
% Franz Vojik \Email{vojik@de.tu-muenchen.informatik}
% Eric Carroll \Email{eric@ca.utoronto.utcc.enfm}
% Roderick Williams \Email{rodw@cbl.leeds.ac.uk}
% Robert Cailliau \Email{cailliau@cernnext.cern.ch}
% Toni Lantunen at CERN
% Ana Maria Paiva, Jamil Sawar, Andrew Cole at CBLU Leeds
% Phillip Conrad (Perfect Byte, Inc.)
% L. Peter Deutsch (

%

\begin{htmlonly}
\documentclass{article}
\usepackage{html,htmllist,makeidx}
\usepackage{longtable}

%this font is used in  features.tex
\font\wncyr = wncyr10


\begin{htmlonly}
 \usepackage[dvips]{graphicx}
 \usepackage[dvips]{color}
% \usepackage{frames}
 \usepackage[dvips,leftbars]{changebar}
 \newcommand{\FoilTeX}{\env{FoilTeX}}
%
% \newcommand{\htmlurl}[1]{\htmladdnormallink{#1}{#1}}
 \def\Email#1{\htmladdnormallink{<#1>}{mailto:#1}}
 \def\path#1{\texttt{#1}}
% \def\urldef#1#2#3{\def#1{#2{#3}}}
% \newcommand{\onlinedoc}{\url{http://www-dsed.llnl.gov/files/programs/unix/latex2html/manual/}}
 \newcommand{\onlinedoc}{\url{http://www-texdev.ics.mq.edu.au/l2h/docs/manual/}}
% \def\patches{\htmlurl{http://www-dsed.llnl.gov/files/programs/unix/latex2html/}}
 \def\patches{\htmlurl{http://www.latex2html.org/current/}}
 \def\sourceA{%
  \url{http://www.latex2html.org/current/}}
%  \url{http://www-dsed.llnl.gov/files/programs/unix/latex2html/sources/}}
% \def\sourceB{\htmlurl{ftp://ftp.mpn.com/pub/nikos/latex2html-98.1.tar.gz}}
 \def\sourceB{\CTANtug{\CTANA}}
 \def\sourceC{\htmlurl{http://ftp.rzg.mpg.de/pub/software/latex2html/sources/}}
% \def\CVSsite{\url{http://cdc-server.cdc.informatik.tu-darmstadt.de/\~{}latex2html/}}
% \def\CVSrepos{\url{http://cdc-server.cdc.informatik.tu-darmstadt.de/\~{}latex2html/user/}}
% \def\CVSlatest{\url{http://cdc-server.cdc.informatik.tu-darmstadt.de/\~{}latex2html/l2h-latest.tar.gz}}
% \def\CVSsite{\htmlurl{http://saftsack.fs.uni-bayreuth.de/~latex2ht/}}
% \def\CVSrepos{\htmlurl{http://saftsack.fs.uni-bayreuth.de/~latex2ht/user/}}
 \def\CVSrepos{\htmlurl{https://www.github.com/latex2html/latex2html/}}
% \def\CVSlatest{\htmlurl{http://saftsack.fs.uni-bayreuth.de/~latex2ht/l2h-latest.tar.gz}}
 \def\CVSlatest{\htmlurl{https://www.github.com/latex2html/latex2html/archive/master.zip}}
 \newcommand{\CTANtug}[1]{\htmlurl{http://mirrors.ctan.org/#1}}
 \def\CTANA{support/latex2html}
 \def\tugURL{\htmlurl{http://www.tug.org/}}
 \def\danteURL{\htmlurl{http://www.dante.de/}}
% \def\ListURL{\htmlurl{http://www.xray.mpe.mpg.de/mailing-lists/latex2html/}}
 \def\ListURL{\htmlurl{http://www.tug.org/mailman/listinfo/latex2html/}}
%
% \def\glossary#1{\index{#1@\texttt{#1} \label{III#1}\htmlref{(G)}{GGG#1}}}
 \newcommand{\glossary}[1]{\index{#1@\texttt{#1} \label{III#1}\htmlref{(G)}{GGG#1}}}
 \def\Glossary#1#2{\index{#1@{#2} \label{III#1}\htmlref{(G)}{GGG#1}}}
% \newcommand{\Glossary}[2]{\index{#1@{#2} \label{III#1}\htmlref{(G)}{GGG#1}}}
%
% hack to suppress changebar entries read from .aux file  !!! bug in changebar.perl !!! 
 \makeatletter
 \def\cb@barpoint#1#2#3{}
 \makeatother
%
\newcommand{\sameas}[1]{\textcolor{red}{Same as setting: #1}}
\newcommand{\onlinedocRM}{\url{http://www-texdev.ics.mq.edu.au/l2h/docs/manual/}}
\newcommand{\EXcolors}{\url{http://www-texdev.ics.mq.edu.au/l2h/crayola/}}
%
\newcommand{\Lc}[1]{\texttt{\char92#1}} % LaTeX command
\newcommand{\Tc}[1]{\texttt{\char92#1}} % TeX command
\newcommand{\Cs}[1]{ \texttt{-#1} }   % command-line switch
\newcommand{\Ve}[1]{\index{#1@\texttt{#1}}\texttt{#1}} % version extension
\newcommand{\gsl}[1]{#1}
\newcommand{\indexentry}[2]{\item #1 #2}

%
\internal{}%
\internal{O}%
\internal{S}%
\internal{E}%
\internal{M}%
\internal{H}%
\internal{P}%

\newcommand{\texdev}[1]{\htmladdnormallink
 {\texdevURL/#1}{\texdevURL/#1}}
\newcommand{\ctan}{http://ctan.tug.org/ctan}
\newcommand{\ctanTUG}{\htmladdnormallinkfoot
 {TUG's searchable CTAN site}{\ctan}}
\newcommand{\ctanURL}[1]{\htmladdnormallink
 {CTAN:\texttt{.../#1}}{\ctan/#1}}
\newcommand{\ctanURLbr}[1]{\htmladdnormallink
 {CTAN:\texttt{.../#1}}{\ctan/#1}}
\newcommand{\ctanTUGurl}[2]{\ctanURL{#2}}
\newcommand{\indichtml}{l2h/indic/IndicHTML/}

\end{htmlonly}

\newcommand{\TUGurl}{\htmladdnormallink{TUG}{\htmlurl{http://www.tug.org}}}

\newcommand{\texdevURL}{http://www-texdev.ics.mq.edu.au/}
\newcommand{\Unicode}{\htmladdnormallinkfoot
 {Unicode}{http://www.unicode.org/}}
\newcommand{\IndicHTML}{
 {Indic\TeX/HTML}}

%begin{latexonly}
\usepackage{array}

\newcommand{\Cs}[1]{{\upshape`\,\texttt{-#1}\,'}}
\newcommand{\Ve}[1]{\index{#1@\texttt{#1}}{\upshape`\,\texttt{#1}\,'}}
\newcommand{\Lc}[1]{{\upshape\ttfamily\char92#1}}
\newcommand{\Tc}[1]{{\upshape\ttfamily\char92#1}}

\newcommand{\ctanTUG}[1]{TUG's searchable CTAN site\footnote{\ctanTUGurl}}
\newcommand{\ctanURL}{CTAN: \penalty-200\ctanurl}
\newcommand{\ctanURLbr}{CTAN: \newline\ctanurl}
\urldef\texdev\url{http://www-texdev.ics.mq.edu.au/}
\urldef\ctanTUGurl\url{http://mirrors.ctan.org/}
\urldef\indichtml\url{l2h/indic/IndicHTML/}

\urldef\NetscapeHome\url{http://home.netscape.com}
\urldef\giftransURL\url{http://ftp.ost.eltele.no/pub/multimedia/giftrans-1.12.2.tar.gz}
\urldef\WinZipURL\url{http://www.winzip.com/getsite.cgi?winzip70.exe}
\urldef\MikTeXURL\url{http://www.miktex.de}
\urldef\ActiveStateURL\url{http://www.activestate.com}
\urldef\NetpbmWinURL\url{ftp://ftp.ese-metz.fr/pub/TeX/win32/}
%end{latexonly}

\begin{htmlonly}
\newcommand{\NetscapeHome}{\url{http://home.netscape.com}}
\newcommand{\giftransURL}{\url{http://ftp.ost.eltele.no/pub/multimedia/giftrans-1.12.2.tar.gz}}
\newcommand{\WinZipURL}{\url{http://www.winzip.com/getsite.cgi?winzip70.exe}}
\newcommand{\MikTeXURL}{\url{http://www.miktex.de}}
\newcommand{\ActiveStateURL}{\url{http://www.activestate.com}}
\newcommand{\NetpbmWinURL}{\url{ftp://ftp.ese-metz.fr/pub/TeX/win32/}}
\end{htmlonly}



% thanks to \KrisRose for this macro:
%\def\smiley{\hbox{\rlap{$\bigcirc$}\kern1.3pt$\scriptstyle\ddot\smile\;$}}
\def\mathsmiley{\smiley\,}


\newcommand{\Lcs}[1]{{\upshape\ttfamily\char92#1}}

\renewcommand{\thefootnote}{\arabic{footnote}}

%\newcommand{\godown}[1]{{\htmlref
% {\htmladdimg[left BORDER=0]{../psfiles/dn.gif}}{#1}}}
\newcommand{\godown}[1]{}

%\newcommand{\goback}[1]{{\htmlref
% {\htmladdimg[left BORDER=0]{../psfiles/up.gif}}{#1}}}
\newcommand{\goback}[1]{}


\newcommand{\WiiiC}{\htmladdnormallinkfoot{World Wide Web Consortium}{http://www.w3c.org/}}
\newcommand{\latextohtmlNG}{\latextohtml-NG}
\newcommand{\maillist}{\htmladdnormallinkfoot{\latextohtml{} mailing list}%
{http://cbl.leeds.ac.uk/nikos/tex2html/doc/mail/mail.html}}
%
\newcommand{\HTMLiii}{\textup{\texttt{HTML} 3.2}}%
\newcommand{\HTMLIII}{\htmladdnormallink{\HTMLiii}%
{http://www.w3.org/TR/REC-html32.html}}%

\newcommand{\HTMLiv}{\textup{\texttt{HTML} 4.0}}%
\newcommand{\HTMLIV}{\htmladdnormallink{\HTMLiv}%
{http://www.w3.org/TR/REC-html40/}}%

%\newcommand{\latextohtml}{\textup{\LaTeX 2{\ttfamily HTML}}}%
\newcommand{\Perl}{\htmlref{\textsl{Perl}}{GGGPerl}}%  
\newcommand{\PS}{\htmlref{\textup{Post\-Script}}{GGGpostscript}\Glossary{PostScript}{PostScript}{}}%
\providecommand{\MF}{}
\renewcommand{\MF}{\htmlref{\textsl{Metafont}}{GGGMetafont}\Glossary{Metafont}{\textsl{Metafont}}{}}%
\newcommand{\fn}[1]{\htmlref{\texttt{#1}}{GGG#1}\glossary{#1}}%  file names, with link to glossary
\newcommand{\gn}[1]{\texttt{#1}\label{GGG#1}\htmlref{\^{}}{III#1}}%  file names, labelled within glossary
%\newcommand{\appl}[1]{\htmlref{\textsl{#1}}{GGG#1}\Glossary{#1}{\textsl{#1}}{}}%  application software names
\newcommand{\appl}[1]{\htmlref{\textsl{#1}}{GGG#1}\Glossary{#1}{\gsl{#1}}{}}%  application software names
%
\newcommand{\env}[1]{{\upshape\sffamily #1}}%  LaTeX environment and package names
\newcommand{\HTMLtag}[1]{\path{<#1>}}%  HTML tag
\newcommand{\Meta}[1]{\texttt{\upshape<\textit{#1}>}}%  Meta string

%
% developer names and addresses:
%
\newcommand{\NikosDrakos}{\index{Drakos, Nikos}%\Email{nikos@cbl.leeds.ac.uk}
{Nikos Drakos}}
%
\newcommand{\RossMoore}{\index{Moore, Ross}%\Email{ross@maths.mq.edu.au}
{Ross Moore}}
\newcommand{\Macquarie}{\htmladdnormallink
{Macquarie University}{http://www.mq.edu.au/}}
%
\newcommand{\Hennecke}{\index{Hennecke, Marcus}%\Email{hennecke@dbag.ulm.DaimlerBenz.COM}
{Marcus Hennecke}}
%
\newcommand{\Noworolski}{\index{Noworolski, Mark}%\Email{jmn@eecs.berkeley.edu}
{Mark Noworolski}}
%
\newcommand{\Isani}{\index{Isani, Sidik}%\Email{isani@cfht.hawaii.edu}
{Sidik Isani}}
%
\newcommand{\Goossens}{\index{Goossens, Michel}%\Email{goossens@cern.ch}
{Michel Goossens}}
%
\newcommand{\Wilck}{\index{Wilck, Martin}%\Email{martin@tropos.de}
{Martin Wilck}}
%
\newcommand{\PatrickDaly}{\index{Daly, Patrick}%\Email{daly@linmpi.dnet.gwdg.de}
{Patrick Daly}}
%
\newcommand{\Arseneau}{\index{Arseneau, Donald}%
{Donald Arseneau}}
%
\newcommand{\JohannesBraams}{\index{Braams, Johannes}%
{Johannes Braams}}
%
\newcommand{\HerbSwan}{\index{Swan, Herb}%\Email{herb.swan@perc.Arco.COM}
%\htmladdnormallink{Herb Swan}{mailto:herb.swan@perc.Arco.COM}}
{Herb Swan}}
%
\newcommand{\Lippmann}{\index{Lippmann, Jens}%\Email{lippmann@rbg.informatik.tu-darmstadt.de}
{Jens Lippmann}}
%
\newcommand{\Rouchal}{\index{Rouchal, Marek}%\Email{marek@saftsack.fs.uni-bayreuth.de}
{Marek Rouchal}}
%
\newcommand{\Bohnet}{\index{Bohnet, Achim}%\Email{ach@rosat.mpe-garching.mpg.de}
{Achim Bohnet}}
%
\newcommand{\Nelson}{\index{Nelson, Scott}%\Email{nelson18@llnl.gov}
{Scott Nelson}}
%
\newcommand{\AxelRamge}{\index{Ramge, Axel}%\Email{axel@ramge.de}
{Axel Ramge}}
%
\newcommand{\Popineau}{\index{Popineau, Fabrice}%\Email{popineau@esemetz.ese-metz.fr}
{Fabrice Popineau}}
%
\newcommand{\Wortmann}{\index{Wortmann, Uli}%\Email{uli12@bonk.ethz.ch}
{Uli Wortmann}}
%
\newcommand{\Veytsman}{\index{Veytsman, Boris}%\Email{boris@plmsc.psu.edu}
{Boris Veytsman}}
%
\newcommand{\Taupin}{\index{Taupin, Daniel}%\Email{taupin@lps.u-psud.fr}
{Daniel Taupin}}
%
\newcommand{\ArtOgawa}{\index{Ogawa, Art}%\Email{ogawa@teleport.com}
{Art Ogawa}}
%
\newcommand{\Perbandt}{\index{Perbandt, Adalbert}%\Email{perbandt@hl.siemens.de}
{Adalbert Perbandt}}
%
%\endinput

%
% (La)TeX related URLs
%
\newcommand{\CSEP}{\index{Computer Science Education Project!CSEP}%
{Computational Science Education Project (ORNL)}%
} 
%
\newcommand{\CBLU}{\index{Computer Based Learning Unit!CBLU}%
{Computer Based Learning Unit}}%
%
\newcommand{\CERN}{\index{CERN}%
\htmladdnormallink{CERN}{http://www.cern.ch}} 
%
\newcommand{\KrisRose}{\index{Rose, Kristoffer}%\Email{krisrose@brics.dk}
{Kristoffer Rose}}
%
\newcommand{\XypicDK}{\index{Xy-pic@\Xy-pic graphics package}%
\index{Xy-pic@\Xy-pic graphics package!home page}%
{Xy-pic}}
%
\newcommand{\LiPS}{\index{LiPS Design Team}%
{LiPS Design Team}}
\newcommand{\FIDarmstadt}{\index{Fachbereich Informatik, Darmstadt}%
\htmladdnormallink{Fachbereich Informatik}{http://www.informatik.tu-darmstadt.de/}}
\newcommand{\Darmstadt}{\index{Darmstadt!Fachbereich Informatik}%
{Darmstadt}}
%
\newcommand{\DANTE}{\index{DANTE}%
\htmladdnormallink{DANTE e.V.}{http://www.dante.de/}}
\newcommand{\Praesidium}{\index{DANTE!Praesidium}%
\htmladdnormallink{Praesidium}{http://www.dante.de/dante/Organe.html}}
\newcommand{\LaTeXiii}{\index{LaTeX@\LaTeX!LaTeX3@\LaTeX{3}}%
\htmladdnormallink{\LaTeX{3}}{http://www.tex.ac.uk/CTAN/latex/latex3}}
%
\newcommand{\Engberg}{\index{Uffe Engberg}%
\htmladdnormallinkfoot{Uffe Engberg}{http://www.brics.dk/\~{}engberg}} 

\newcommand{\texdevINDIC}{\texdev\indic}
%\endinput


%% earlier contributions from...
\newcommand{\AndrewCole}{\index{Computer Based Learning Unit!Andrew Cole}%
{Andrew Cole}}
%
\newcommand{\AnaPaiva}{\index{Computer Based Learning Unit!Ana Maria Paiva}%
{Ana Maria Paiva}}
%
\newcommand{\RodWilliams}{\index{Computer Based Learning Unit!Roderick Williams}%
{Roderick Williams}}
%
\newcommand{\JamilSawar}{\index{Computer Based Learning Unit!Jamil Sawar}%
{Jamil Sawar}}

\newcommand{\RobertThau}{\index{Thau, Robert S.}%
{Robert S. Thau}}

\newcommand{\RobertCailliau}{\index{Cailliau, Robert}%
{Robert Cailliau}}

\newcommand{\EricCarroll}{\index{Carroll, Eric}%
{Eric Carroll}}

\newcommand{\FranzVojik}{\index{Vojik, Franz}%
{Franz Vojik}}

\newcommand{\AxelBelinfante}{\index{Belinfante, Axel}%
{Axel Belinfante}}

\newcommand{\ToddLittle}{\index{Little, Todd}%
{Todd Little}}

\newcommand{\VerenaUmar}{\index{Umar, Verena}%
{Verena Umar}}

\newcommand{\IanFoster}{\index{Foster, Ian}%
{Ian Foster}}

\newcommand{\BobOlson}{\index{Olson, Bob}%
{Bob Olson}}

\endinput


%
% Ian Foster \Email{itf@mcs.anl.gov}
% Bob Olson \Email{olson@mcs.anl.gov}
% Verena Umar \Email{verena@edu.vanderbilt.cas.compsci}
% Axel Belinfante \Email{Axel.Belinfante@cs.utwente.nl}
% Todd Little \Email{little@com.dec.enet.nuts2u}
% Franz Vojik \Email{vojik@de.tu-muenchen.informatik}
% Eric Carroll \Email{eric@ca.utoronto.utcc.enfm}
% Roderick Williams \Email{rodw@cbl.leeds.ac.uk}
% Robert Cailliau \Email{cailliau@cernnext.cern.ch}
% Toni Lantunen at CERN
% Ana Maria Paiva, Jamil Sawar, Andrew Cole at CBLU Leeds
% Phillip Conrad (Perfect Byte, Inc.)
% L. Peter Deutsch (

%

\begin{htmlonly}
\documentclass{article}
\usepackage{html,htmllist,makeidx}
\usepackage{longtable}
\input{manhtml.tex}
%
\input{userman.ptr}             % Input counters and section
\label{sec:man}\index{manual}
\end{htmlonly}
\startdocument
%
%\section{User Manual\label{sec:man}\index{manual}}
\label{sec:man}
\index{customised layout}
\index{getting started}
\html{\\}\noindent
\latextohtml{} is a program for creating hyperlinked sets of \texttt{HTML} pages
from a document marked-up using \LaTeX{} commands.
Previous sections have discussed the results of specific \LaTeX{} commands.
In this section we discuss instead the extensive range
of command-line switches and options, and other aspects of \Perl{} code,
that affect the way the translation is performed.

\medskip\noindent
To use \latextohtml{} to translate a file~\Meta{file}\texttt{.tex }
containing \LaTeX{} commands, type:
\begin{quote}
\texttt{latex2html} \Meta{file}\texttt{.tex} 
\end{quote}

\index{browser@browser\label{IIIbrowser}}%
\index{viewer|see{\htmlref{browser}{IIIbrowser}}}%
\index{browser!NCSA Mosaic@\textsl{NCSA Mosaic}}%
\index{browser!Netscape Navigator@\textsl{Netscape Navigator}}\html{\\}%
\noindent
This will create a new directory called \Meta{file} which will contain 
the generated \texttt{HTML} files, some log files and possibly some images.
To view the result use an \texttt{HTML} browser, such as 
\appl{NCSA Mosaic} or \appl{Netscape Navigator},
on the main \texttt{HTML} file which is \texttt{\Meta{file}/\Meta{file}.html}\,. 
The file will contain navigation links 
to the other parts of the generated document.
%
\index{file suffix!optional}%
\index{file suffix!.tex@\texttt{.tex} or\texttt{.doc}}%
The \texttt{.tex} suffix is optional and will be supplied
by the program if it is omitted by the user.
Other suffixes are acceptable also, such as \texttt{.doc}\,.

\medskip
\index{options!command-line options}\html{\\}\noindent
It is possible to customise the output from \latextohtml{} using a number of 
\hyperref{command-line options}{command-line options (see Section~}{)}{options}
with which you can specify:
\begin{itemize}
\item
how to break up the document; 
\item
where to put the generated files, and what are their names;
\item
the title for the whole document; 
\item
the signature at the end of each page is; 
\item
how many navigation panels to provide, what links to put in them;
\item
what other documents this one links to;
\item 
extra information to include about the document;
\item 
whether to retain the original \LaTeX{} section-numbering scheme;
\item
and many other things that affect how the information is obtained,
processed or displayed in the resulting \texttt{.html} files and images.
\end{itemize}

\medskip
\index{manual!short on-line}\html{\\}\noindent
The \latextohtml{} script includes a short manual
which can be viewed with the command: 
\begin{verbatim}
  perldoc latex2html
\end{verbatim}


\subsection{Developing Documents using \latextohtml{}\label{devel}}
%\section{Developing Documents using \latextohtml{}\label{devel}}
%
Although any document containing \LaTeX{} commands can be translated
by the \latextohtml{} translator, the best results are obtained when
that document is itself a valid \LaTeX{} document.
Indeed it is generally a good idea to develop documents so that
they produce good readable results in both the \LaTeX{} typeset
version as well as a set of \texttt{HTML} pages.
This is not just a nicety; 
there are several good practical reasons for doing this.

\begin{htmllist}\htmlitemmark{BlueBall}
%
\item[\LaTeX{} macros: ]
The macro commands that \latextohtml{} recognises are based upon
corresponding commands for \LaTeX. If one tries to use syntax that is
incorrect for \LaTeX{} then there is no reason why \latextohtml{} should
be able to ``get it right'', by somehow recognising the true intent.


\index{error checking!using LaTeX@using \LaTeX}%
\item[error checking: ]
Processing the document first using \LaTeX{} is the easiest, 
and quickest, way to check for valid syntax.\html{\\}
Whereas \LaTeX{} stops at each error (when run in interactive mode),
allowing a fix to be made ``on the spot'' or a ``stop-fix-restart'',
\latextohtml{} does not stop when it detects an error in \LaTeX{} syntax.
Useful messages are given concerning missing or unmatched braces,
but other apparent anomalies generate only warning messages, 
which are saved to the end. 
(Some warnings are also shown immediately when the \fn{\$VERBOSITY} 
\htmlref{variable}{verbosity} is set to at least 3.)\html{\\}
In practice it can be much quicker to test for invalid syntax using \LaTeX{} before 
attempting to use the \latextohtml{} translator.

Furthermore, \LaTeX{} warns of cross-reference labels that have not been defined. 
This is useful to help avoid having hyperlinks which point to nowhere.

\index{error checking!missing braces}%
\index{error checking!unmatched brace}\html{\\}\noindent
The case of missing braces, or an unmatched opening brace, 
is an error that \latextohtml{} actually handles better than \LaTeX{} 
(or rather, the underlying \TeX{} processor).
Whereas \TeX{} only detects an error when something else goes wrong
later in the processing, \latextohtml{} shows where the
unmatched brace itself occurs. 


\index{auxiliary file}
\item[auxiliary file: ]
Some information that \latextohtml{} might need is normally
read from the \texttt{.aux} file for the document being processed;
or perhaps from the \texttt{.aux} file of a different document,
of which the current document is just a portion. 
Clearly a valid \LaTeX{} run is required to produce the \texttt{.aux} file.

\html{\\}\noindent
Of course if no information in the \texttt{.aux} file is actually used,
then this \LaTeX{} run can be neglected.
Also, if the \texttt{.aux} file has already been created, and edits
are made on the source which do not alter the information stored
within the \texttt{.aux} file, then there is no need for a fresh
\LaTeX{} run (except for the purposes of error-checking).


\index{bibliography}
\item[bibliography: ]
Suppose the document requires a bibliography, or list of references,
which is to be prepared using Bib\TeX.
\latextohtml{} reads citation information from the \texttt{.aux} file,
and can import the bibliography itself from the \texttt{.bbl} file.
However these must first be created using \LaTeX.



\index{document segmentation}
\item[document segmentation: ]
With the document segmentation technique, 
discussed fully in \hyperref{a later section}{Section~}{}{Segmentation},
it is vitally important that the full document processes correctly
in \LaTeX. The desired effect is that of a single large document,
whereas the pieces will actually be processed separately.
To achieve this, \LaTeX{} writes vital information into special
\texttt{.ptr} files. Like the \texttt{.aux} file, these files
are read by \latextohtml{} to get section-numbering and other
information to be used while processing each segment.


\index{printing!hyperlink to typeset version}%
\index{printing}\html{\\}\noindent
\item[print quality: ]
When a document contains automatically-generate images, these images
are usually bitmaps designed for viewing on-screen. In general the 
resolution of these is too poor to give a good result when printed
on a high-resolution laser-printer.
Thus if it is likely that the reader will want to obtain
a printed version of your document, then it is best to include
a hyperlink to the typeset \texttt{.dvi} version, 
or to a \PS\ conversion of the \texttt{.dvi} file. 
(In either case, a link to a compressed version is even better.)

\end{htmllist}



\subsection{Command-Line Options\label{options}}%
%\section{Command-Line Options\label{options}}%
\index{options@options\label{IIIoptions}}%
\index{switches|see{\htmlref{options}{IIIoptions}}}%
\index{options|(}%
\index{options!environment variables}%
\index{options!set in initialisation file}%
\html{\\}\noindent
The command-line options described here can be used to change the
default behaviour of \latextohtml. 
Alternatively, often there is a corresponding environment variable
whose value may be set or changed within a \fn{.latex2html-init} 
initialisation file, in order to achieve the same result.\html{\\}
There are so many options that they are listed here in groups, 
according to the nature of the effects they control.
When a large number of such options are required for the processing of a document,
it is usual to store the switches and their desired settings within a \fn{Makefile},
for use with the Unix \fn{make} utility. Now a simple command such as:
\begin{verbatim}
  make mydocument
\end{verbatim}
can initiate a call to \fn{latex2html} that would otherwise take many lines of typing.
Indeed it could instigate several such calls to \latextohtml, or to other programs
such as \LaTeX{} and Bib\TeX, \fn{dvips} and others.
The document segmentation feature, discussed in 
\hyperref{another section}{Section~}{}{Segmentation},
uses this technique extensively.


\subsubsection{Options controlling Titles, File-Names and Sectioning}
%\subsection{Options controlling Titles, File-Names and Sectioning}
\index{options!titles}\index{options!file-names}\index{options!sectioning}%
%
\index{document!title}%
%
\begin{htmllist}\htmlitemmark{PurpleBall}%
\item [ -t \Meta{top-page-title}\label{cs_toppagetitle}]
\sameas{\fn{\$TITLE}\texttt{ = \char34}\Meta{top-page-title}\texttt{\char34;}}\\
Name the document using this title.

\index{filenames!.htm@\texttt{.htm} extension}%

\item [ -short\_extn\label{cs_shortextn}]
\sameas{\fn{\$SHORTEXTN}\texttt{ = 1;}}\\
Use a filename prefix of \texttt{.htm } for the produced
\texttt{HTML} files. This is particularly useful for creating pages
to be stored on CD-ROM or other media, to be used with 
operating systems that require a 3-character extension.%

\index{filenames!long names}

\item [ -long\_titles \Meta{num}\label{cs_longtitles}]
\sameas{\fn{\$LONG\_TITLES}\texttt{ = }\Meta{num}\texttt{;}}\\
Instead of the standard  names: \texttt{node1.html}, \texttt{node2.html},... 
the filenames for each \texttt{HTML} page are constructed
from the first \Meta{num} words of the section heading
for that page, separated by the `\texttt{\_}' character.\html{\\}
Commas and common short words (\texttt{a an to by of and for the})
are omitted from both title and word-count.

\smallskip\noindent
\textbf{Warning:} Use this switch with great caution.
Currently there are no checks for uniqueness of names or overall length. 
Very long names can easily result from using this feature.%

\index{filenames!customised}%
\index{filenames!custom title hook@\texttt{custom\_title\_hook}}

\item [ -custom\_titles\label{cs_customtitles}]
\sameas{\fn{\$CUSTOM\_TITLES}\texttt{ = 1;}}\\
Instead of the standard  names: \texttt{node1.html}, \texttt{node2.html}, ... 
the filenames for each \texttt{HTML} page are constructed using a \Perl{}
subroutine named \texttt{ custom\_title\_hook}~. 
The user may define his/her own version of this subroutine, 
within a \fn{.latex2html-init} file say, 
to override the default (which uses the standard names).
This subroutine takes the section-heading as a parameter 
and must return the required name, or the empty string (default).

\index{output!redirect to directory}

\item [ -dir \Meta{output-directory}\label{cs_outputdir}]
\sameas{\fn{\$DESTDIR}\texttt{ = \char34}\Meta{output-directory}\texttt{\char34;}}\\
Redirect the output to the specified directory.\html{\\}
The default behaviour is to create (or reuse) a directory having the same
name as the prefix of the document being processed.%

\index{output!in current directory}%
\index{output!default directory}

\item [ -no\_subdir\label{cs_nosubdir}]
\sameas{\fn{\$NO\_SUBDIR}\texttt{ = 1;}}\\
Place the generated \texttt{HTML} files into the current directory.
This overrides any \fn{\$DESTDIR} setting.

\index{filenames!prefix}%
\index{temporary files}%

\item [ -prefix \Meta{filename-prefix}\label{cs_prefix}]%
\sameas{\fn{\$PREFIX}\texttt{ = \char34}\Meta{filename-prefix}\texttt{\char34;}}\\
The \Meta{filename-prefix} will be prepended to all 
\texttt{.gif}, \texttt{.pl} and \texttt{.html} files produced, 
except for the top-level \texttt{.html} file;  
it may include a (relative) directory path. 
This will enable multiple products of \latextohtml{} to
peacefully coexist in the same directory.
However, do \emph{not} attempt to \emph{simultaneously} run multiple 
instances of \latextohtml{} using the same output directory, 
else various temporary files will overwrite each other.

\index{filenames!auto-prefix}%

\item [ -auto\_prefix\label{cs_autoprefix}]
\sameas{\fn{\$AUTO\_PREFIX}\texttt{ = 1;}}\\
Constructs the prefix as `\Meta{title}--'
to be prepended to all the files produced, 
where \Meta{title} is the name of the \LaTeX{} file being processed.
(Note the `--' in this prefix.)\\
This overrides any \fn{\$PREFIX} setting.

\index{filenames!automatic link to directory name}%
\index{filenames!index.html@\fn{index.html}}%

\item [ -no\_auto\_link\label{cs_noautolink}]
\sameas{\fn{\$NO\_AUTO\_LINK}\texttt{ = 1;}}\\
If \fn{\$NO\_AUTO\_LINK} is empty and variables \fn{\$LINKPOINT} 
and \fn{\$LINKNAME} are defined appropriately
(as is the default in the \fn{latex2html.config} file),
then a hard link to the main \texttt{HTML} page is produced,
using the name supplied in \fn{\$LINKNAME}. Typically this is \fn{index.html}; 
on many systems a file of this name will be used, if it exists,
when a browser tries to view a URL which points to a directory.
On other systems a different value for \fn{\$LINKNAME} may be appropriate.
Typically \fn{\$LINKPOINT} has value \texttt{\$FILE.html}, but
this may also be changed to match whichever \texttt{HTML} page
is to become the target of the automatic link.\\
Use of the \Cs{no\_auto\_link} switch \emph{cancels} 
this automatic linking facility, 
when not required for a particular document.

\index{sections!in separate files}%

\item [ -split \Meta{num}\label{cs_splitdepth}]
\sameas{\fn{\$MAX\_SPLIT\_DEPTH}\texttt{ = }\Meta{num}\texttt{;}}
 (default is \texttt{8})\\
Stop splitting sections into separate files at this depth.
Specifying \Cs{split 0} will put the entire document into a 
single \texttt{HTML} file. See \htmlref{below}{seclevels} for
the different levels of sectioning. Also see the next item for
how to set a ``relative'' depth for splitting.%

\item [ -split +\Meta{num}\label{cs_reldepth}]
\sameas{\fn{\$MAX\_SPLIT\_DEPTH}\texttt{ = }\textrm{-}\Meta{num}\texttt{;}}
 (default is \texttt{8})\\
The level at which to stop splitting sections is calculated ``relative to'' 
the shallowest level of sectioning that occurs within the document.
For example, if the document contains \Lc{section} commands, but no \Lc{part}
or \Lc{chapter} commands, then \Cs{split +1} will cause splitting at each \Lc{section}
but not at any deeper level; whereas \Cs{split +2} or \Cs{split +3} also split down to 
\Lc{subsection} and \Lc{subsubsection} commands respectively.
Specifying \Cs{split +0} puts the entire document into a single \texttt{HTML} file.


\index{depth!child nodes revealed}%
\index{components!specify depth}

\index{child-links table}%
\item [ -link \Meta{num}\label{cs_linkdepth}]
\sameas{\fn{\$MAX\_LINK\_DEPTH}\texttt{ = }\Meta{num}\texttt{;}}
 (default is \texttt{4})\\
For each node, create links to child nodes down to this much 
deeper than the node's sectioning-level.\html{\\}
Specifying \Cs{link 0} will show no links to child nodes from that page,\html{\\}
\Cs{link 1} will show only the immediate descendents, etc.\html{\\}
A value at least as big as that of the 
\htmlref{\Cs{split \Meta{num}}}
{splitdepth} depth will
produce a mini table-of-contents (when not empty) on each page,
for the tree structure rooted at that node.%

\index{child-links table!at bottom of page}%

When the page has a sectioning-level less than the \Cs{split} depth,
so that the a mini table-of-contents has links to other \texttt{HTML} pages, 
this table is located at the bottom of the page, 
unless placed elsewhere using the \Lc{tableofchildlinks} command.%

\index{child-links table!at top of page}%
\index{child-links table!tableofchildlinks@\texttt{tableofchildlinks}}

On pages having a sectioning-level just less than the \Cs{split} depth 
the mini table-of-contents contains links to subsections etc.
occurring on the \emph{same} \texttt{HTML} page. 
Now the table is located at the \emph{top} of this page, 
unless placed elsewhere using the \Lc{tableofchildlinks} command.

\index{table of contents!depth}%

\item [ -toc\_depth  \Meta{num}\label{cs_tocdepth}]
\sameas{\fn{\$TOC\_DEPTH}\texttt{ = }\Meta{num}\texttt{;}}
 (default is \texttt{4})\\
Sectioning levels down to \Meta{num} are to be included 
within the Table-of-Contents tree.%

\index{table of contents!star-sections shown}%
\index{table of contents!star-sections not listed}

\item [ -toc\_stars\label{cs_tocstars}]
\sameas{\fn{\$TOC\_STARS}\texttt{ = 1;}}\\
Sections created using the starred-form of sectioning commands
are included within the Table-of-Contents.
As with \LaTeX, normally such sections are \emph{not} listed.
%

\index{sections!show section numbers}%
\index{sections!numbers not shown}

\item [ -show\_section\_numbers\label{cs_showsecnums}]
\sameas{\fn{\$SHOW\_SECTION\_NUMBERS}\texttt{ = 1;}}\\
Show section numbers. By default section numbers are \emph{not} shown, 
so as to encourage the use of particular sections as stand-alone documents. 
\strikeout{In order to be shown, 
section titles must be unique and must not contain inlined graphics.}

\index{segmentation!unsegment}

\item [ -unsegment\label{cs_unsegment}]
\sameas{\fn{\$UNSEGMENT}\texttt{ = 1;}}\\
Treat a segmented document (see the section about \htmlref{document
  segmentation}{Segmentation}) like it were not segmented.
This will cause the translator to concatenate all segments and process
them as a whole.
You might find this useful to check a segmented document for
consistency.
%
\end{htmllist}

\goodbreak
\index{sectioning levels}\index{levels!sectioning}\label{seclevels}%
\html{\\}\noindent
For all documents the sectioning levels referred to above are:\nobreak
\begin{center}
\begin{tabular}{ll}
\textbf{0} & document\\
\textbf{1} & part\\
\textbf{2} & chapter\\
\textbf{3} & section\\
\textbf{4} & subsection\\
\textbf{5} & subsubsection\\
\textbf{6} & paragraph\\
\textbf{7} & subparagraph\\
\textbf{8} & subsubparagraph
\end{tabular}
\end{center}
These levels apply even when the document contains no sectioning for
the shallower levels; e.g. no \Lc{part} or \Lc{chapter} commands is most common,
especially when using \LaTeX's \env{article} document-class.



\subsubsection{Options controlling Extensions and Special Features}
\index{extensions|(}%
\index{options!extensions}\index{options!special features}%
\html{\\}\noindent
The switches described here govern the type of \texttt{HTML} code that
can be generated, and how to choose between the available options 
when there are alternative strategies for implementing portions of \LaTeX{} code. 

\begin{htmllist}\htmlitemmark{PurpleBall}%
%
\index{HTML version@\texttt{HTML} version}%
\index{HTML@\texttt{HTML}!non-standard extensions}%
\item [ -html\_version \texttt{(2.0|3.0|3.2)[,(math|i18n|table)]*}\label{cs_htmlversion}]
~\\\sameas{\fn{\$HTML\_VERSION}\texttt{ = \char34 ... \char34;}}\\
This specifies both the \texttt{HTML}~\htmlref{version}{version} to generate, 
and any extra (non-standard) \texttt{HTML} features that may be required.\html{\\}
The version number corresponds to a published DTD for an \texttt{HTML} standard
(although 3.0 was never accepted and subsequently withdrawn).
A corresponding \Perl{} file in the \fn{versions/} subdirectory is loaded;
these files are named `\texttt{html}\Meta{num}\texttt{.pl}'.

Following the version number, a comma-separated list of extensions
can be given. Each corresponds to a file `\Meta{name}\texttt{.pl}' 
also located in the \fn{versions/} subdirectory. 
When such a file is loaded the resulting \texttt{HTML} code 
can no longer be expected to validate with the specified DTD. 
An exception is \Ve{math} when the \htmlref{\Cs{no\_math}}{nomath} 
switch is also used, which should still validate.

Currently, versions 2.0, 3.2 and 4.0 are available.
(and also 2.1, 2.2, 3.0 and 3.1, for hoistorical reasons). 
The extensions \Ve{i18n}, \Ve{tables}, \Ve{math}
correspond roughly to what used to be called versions `2.1', `2.2', `3.1'
respectively, in releases of \latextohtml{} up to 1996.
Now these extensions can be loaded with any of `2.0', `3.2' or `4.0' 
as the specified standard.\\  
The default version is usually set to be `3.2', within \fn{latex2html.config}\,.%

\index{extensions!disabled}%
\index{extensions!TeX defs@\TeX definitions}%

\item [ -no\_tex\_defs\label{cs_notexdefs}]
\sameas{\fn{\$TEXDEFS}\texttt{ = 0;}} (default is 1)\\
When \fn{\$TEXDEFS} is set (default) the file \fn{texdefs.perl} will be read.
This provides code to allow common \TeX{} commands like \Tc{def}, \Tc{newbox},
\Tc{newdimen} and others, to be recognised, especially within the document preamble.
In the case of \Tc{def}, the definition may even be fully interpreted,
but this requires the pattern-matching to be not too complicated.

If \fn{\$TEXDEFS} is `\texttt{0}' or empty, then \fn{texdefs.perl} will not be loaded;
the translator will make no attempt to interpret any raw \TeX{} commands.  
This feature is intended to enable sophisticated authors the ability to insert 
arbitrary \TeX{} commands in environments that are destined 
to be processed by \LaTeX{} anyway; 
e.g. \env{figure}s, \strikeout{\env{theorem}s}, \env{picture}s, etc.
However this should rarely be needed, as now there is better support for these
types of environment. There are now other methods to specify which chunks of code 
are to be passed to \LaTeX{} for explicit image-generation; 
see the discussion of the \env{makeimage} 
\hyperref[page]{environment}{environment on page~}{}{makeimage}.


\index{auxiliary file!which to read}%

\item [ -external\_file \Meta{filename}\label{cs_externalfile}] 
\sameas{\fn{\$EXTERNAL\_FILE}\texttt{ = \char34}\Meta{filename}\texttt{\char34;}}\\
Specifies the prefix of the \texttt{.aux} file that this document should read.
The \texttt{.aux} extension will be appended to this prefix 
to get the complete filename, with directory path if needed.\html{\\}  
This file could contain necessary information 
regarding citations, figure, table and section numbers from \LaTeX{} 
and perhaps other information also.
Use of this switch is vital for \htmlref{document segments}{Segmentation}, 
processed separately and linked to appear as if generated 
from a single \LaTeX{} document.%

\index{font size!for image generation}%
\index{font-size!in environments}\index{font-size!magnification}%

\item [ -font\_size \Meta{size}\label{cs_fontsize}]
\sameas{\fn{\$FONT\_SIZE}\texttt{ = \char34}\Meta{size}\texttt{\char34;}}\\
This option provides better control 
over the font size of environments made into images using \LaTeX.  
\Meta{size} must be one of the font sizes that \LaTeX{} recognizes; 
i.e. `\texttt{10pt}', `\texttt{11pt}', `\texttt{12pt}', etc.  
Default is `\texttt{10pt}', or whatever option may have been specified on the 
\Lc{documentclass} or \Lc{documentstyle} line.\html{\\}
Whatever size is selected, it will be magnified by the installation variables
\fn{\$MATH\_SCALE\_FACTOR}, \fn{\$FIGURE\_SCALE\_FACTOR}
and \fn{\$DISP\_SCALE\_FACTOR} as appropriate.

\smallskip\noindent
\textbf{Note: }This switch provides no control over the size of text
on the \texttt{HTML} pages. 
Such control is subject entirely to the user's choices
of settings for the browser windows.

\index{font size!scalable}%

\item [ -scalable\_fonts\label{cs_scalefonts}]
\sameas{\fn{\$SCALABLE\_FONTS}\texttt{ = 1;}}\\
This is used when scalable fonts, such as \PS\ versions
of the \TeX{} fonts, are available for image-generation.\html{\\}
It has the effect of setting \fn{\$PK\_GENERATION} to `\texttt{1}',
and \fn{\$DVIPS\_MODE} to be empty, 
overriding any \htmlref{previous settings}{pkgen} for these variables.

\index{simple math!cancelled}%

\item [ -no\_math\label{cs_nomath}]
\sameas{\fn{\$NO\_SIMPLE\_MATH}\texttt{ = 1;}}\\
Ordinarily simple mathematical expressions are set using the ordinary text font, 
but italiced. When part of the expression can not be represented this way,
an image is made \emph{of the whole formula}. 
This is called ``simple math''.
When \fn{\$NO\_SIMPLE\_MATH} is set, 
then \emph{all} mathematics is made into images, whether simple or not.

However, if the \Ve{math} extension is loaded, 
using the \Cs{html\_version} switch described \htmlref{earlier}{cs_htmlversion},
then specifying \Cs{no\_math} produces a quite different effect.
Now it is the special \HTMLtag{MATH} tags and entities which are cancelled. 
In their place a sophisticated scheme for parsing mathematical expressions 
is used. Images are made of those sub-parts of a formula which cannot be 
adequately expressed using (italiced) text characters and \HTMLtag{SUB} and
\HTMLtag{SUP} tags.
See \hyperref{the subsection on mathematics}{Section~}{}{maths} for more details.

\index{icons!copied to local directory}%

\item [-local\_icons\label{cs_localicons}]
\sameas{\fn{\$LOCAL\_ICONS}\texttt{ = 1;}}\\
A copy of each of the icons actually used within the document is placed in the
directory along with the \texttt{HTML} files and generated images.
This allows the whole document to be fully self-contained, within this directory;
otherwise the icons must be retrieved from a (perhaps remote) server.

\index{icons!alternative set}\index{icons!customised images}%
The icons are normally copied from a subdirectory of the \fn{\$LATEX2HTMLDIR},
set within \fn{latex2html.config}. An alternative set of icons can be used
by specifying a (relative) directory path in \fn{\$ALTERNATIVE\_ICONS}
to where the customised images can be found.


\index{initialisation!init file@\texttt{-init\_file}}%
\index{initialisation file!specified}%

\item [ -init\_file \Meta{file}\label{cs_initfile}]
Load the specified initialisation file. This \Perl{} file will be loaded after loading 
\fn{\$HOME/.latex2html-init}\,, or \fn{.latex2html-init} in the local
directory, if either file exists. It is read at the time the switch is processed,
so the contents of the file may change any of the values of any of the
variables that were previously established, as well as any default options.
More than one initialisation file can be read in this way.%

\index{index!codified links}%

\cbversion{98.1}\begin{changebar}
\index{forking!prevent}%
\item [ -no\_fork\label{cs_no_fork}]
\sameas{\fn{\$NOFORK}\texttt{ = 1;}}\\
When set this disables a feature in the early part of the processing
whereby some memory-intensive operations are performed by `forked'
child processes. Some single-task operating systems, such as \texttt{DOS},
do not support this feature. Having \fn{\$NOFORK} set then ensures
that unnecessary file-handles that are needed with the forked processes,
are not consumed unnecessarily, perhaps resulting in a fatal \Perl{} error.

\index{languages!dtd isolanguage}
\index{character set!dtd isolanguage}
\item [ -iso\_language \Meta{type}\label{cs_iso_language}]
This enables you to specify a different language type than
\texttt{'EN'} to be used in the DTD entries of the \texttt{HTML}
document, e.g. \texttt{'EN.US'}.
\end{changebar}%

\item [ -short\_index\label{cs_shortindex}]
\sameas{\fn{\$SHORT\_INDEX}\texttt{ = 1;}}\\
Creates shorter Index listings, using codified links; 
this is fully compatible with the \env{makeidx} \htmlref{package}{makeidx}.

\index{footnotes!on separate page}%
\index{footnotes!on same page}%
\index{footnotes!marker style}

\item [ -no\_footnode\label{cs_nofootnode}]
\sameas{\fn{\$NO\_FOOTNODE}\texttt{ = 1;}}\\
Suppresses use of a separate file for footnotes;
instead these are placed at the bottom of the \texttt{HTML} pages 
where the references occur.

When this option is used, it is frequently desirable to change the 
style of the marker used to indicate the presence of a footnote. 
This is done as in \LaTeX, using code such as follows.
\begin{quote}
\verb|\renewcommand{\thefootnote}{\arabic{footnote}}|
\end{quote}
All the styles \Lc{arabic}, \Lc{alph}, \Lc{roman}, \Lc{Alph} and \Lc{Roman}
are available.

\index{footnotes!numbering}
\cbversion{98.1}\begin{changebar}
\item [ -numbered\_footnotes\label{cs_numbered_footnotes}]
\sameas{\fn{\$NUMBERED\_FOOTNOTES}\texttt{ = 1;}}\\
If this is set you will get every footnote applied with a subsequent
number, to ease readability.
\end{changebar}

\index{address!signature}%
\index{address!using a subroutine}%

\item [ -address \Meta{author-address}\label{cs_address}]
\sameas{\fn{\$ADDRESS}\texttt{ = \char34}\Meta{author-address}\texttt{\char34;}}\\
Sign each page with this address.\html{\\}
See \fn{latex2html.config} for an example using \Perl{} code
to automatically include the date.

A user-defined \Perl{} subroutine called \texttt{\&custom\_address} 
can be used instead, if defined; it takes the value of \fn{\$ADDRESS} 
as a parameter, which may be used or ignored as desired.
At the time when this subroutine will be called, 
variables named \fn{\$depth}, \fn{\$title}, \fn{\$file}
hold the sectioning-level, title and filename of the \texttt{HTML} page
being produced; \fn{\$FILE} holds the name of the filename for the title-page
of the whole document.

\index{About this@\emph{About this document...}}

\item [ -info \Meta{string}\label{cs_infostring}]
\sameas{\fn{\$INFO}\texttt{ = \char34}\Meta{string}\texttt{\char34;}}\\
Generate a new section \emph{``About this document'' } containing information
about the document being translated. The default is to generate such a section 
with information on the original document, the date, the user and the translator.
An empty string (or the value `\texttt{0}') 
disables the creation of this extra section.\html{\\}
If a non-empty string is given, it will be placed as the contents of the 
\emph{``About this document'' } page instead of the default information.
%
\end{htmllist}


\subsubsection{Switches controlling Image Generation}
%\subsection[center]{Switches controlling Image Generation}
%
These switches affect whether images are created at all, whether
old images are reused on subsequent runs or new ones created afresh,
and whether anti-aliasing effects are used within the images themselves.

\begin{htmllist}\htmlitemmark{PinkBall}%
%
\index{ascii-mode}% 
\index{browser!character-based}%
\item [ -ascii\_mode \label{cs_asciimode}]
\sameas{\fn{\$ASCII\_MODE}\texttt{ = }\fn{\$EXTERNAL\_IMAGES}\texttt{ = 1;}}\\
Use only \texttt{ASCII} characters and do not include any images in the final output.
With \Cs{ascii\_mode} the output of the translator can be used on 
character-based browsers, such as \appl{lynx},
which do not support inlined images (via the \HTMLtag{IMG} tag).


\index{draft mode}%
\item [ -nolatex \label{cs_nolatex}]
\sameas{\fn{\$NOLATEX}\texttt{ = 1;}}\\
Disable the mechanism for passing unknown environments to \LaTeX{} for processing.
This can be thought of as ``draft mode'' which allows 
faster translation of the basic document structure and text, 
without fancy figures, equations or tables. 

(This option has been superseded by the \Cs{no\_images} option,
\htmlref{see below}{noimages}.)


\index{images!via hypertext links}%
\item [ -external\_images\label{cs_extimages}]
\sameas{\fn{\$EXTERNAL\_IMAGES}\texttt{ = 1;}}\\
Instead of including any generated images inside the document, 
leave them outside the document and provide hypertext links to them.


\index{images!links to PostScript@links to \PS}%
\item [ -ps\_images\label{cs_psimages}]
\sameas{\fn{\$PS\_IMAGES}\texttt{ = }\fn{\$EXTERNAL\_IMAGES}\texttt{ = 1;}}\\
Use links to external \PS\ files rather than inlined images
in the chosen graphics format.

\index{images!discard PostScript files@discard \PS\ files}%
\item [ -discard\label{cs_discard}]
\sameas{\fn{\$DISCARD\_PS}\texttt{ = 1;}}\\
The temporary \PS\ files are discarded immediately after they
have been used to create the image in the desired graphics format.

\index{images!disabled}%
\item [ -no\_images\label{cs_noimages}]
\sameas{\fn{\$NO\_IMAGES}\texttt{ = 1;}}\\
Do not attempt to produce any inlined images. 
The missing images can be generated ``off-line'' by restarting \latextohtml{}
with the option \Cs{images\_only}.

\index{images!generated off-line}%
\item [ -images\_only\label{cs_imagesonly}]
\sameas{\fn{\$IMAGES\_ONLY}\texttt{ = 1;}}\\
Try to convert any inlined images that were left over from previous
runs of \latextohtml. 


\index{image files!reuse option}%
\item [ -reuse \Meta{reuse\_option}\label{cs_reuseoptions}]
\sameas{\fn{\$REUSE}\texttt{ = }\Meta{reuse\_option}\texttt{;}}\\
This switch specifies the extent to which image files are to be shared
or recycled.\html{\\}
There are three valid options:
%
\begin{htmllist}\htmlitemmark{OrangeBall}
\index{image files!not recycled}\index{image-reuse!interactive}%
\item [\texttt{0}] 
Do not ever share or recycle image files.\\
This choice also invokes an interactive session 
prompting the user about what to do about
a pre-existing \texttt{HTML} directory, if it exists.
%
\index{image files!recycled}%
\item [\texttt{1}] 
Recycle image files from a previous run if they are available,\\
but do not share identical images that must be created in this run.
%
\index{image files!shared by default}%
\item [\texttt{2}] 
Recycle image files from a previous run and share identical
images from this run.\\
This is the default.
\end{htmllist}
\hyperref{A later section}{Section~}{}{recycling} provides 
additional information about image-reuse.


\index{image files!not reused}\index{image-reuse!interactive}%
\item [ -no\_reuse\label{cs_noreuse}]
\sameas{\fn{\$REUSE}\texttt{ = 0;}}\\
Do \emph{not} share or recycle images generated during previous translations.
This is equivalent to \Cs{reuse 0}.
(This will enable the initial interactive session during which the user is
asked whether to reuse the old directory, delete its contents or quit.)%


\index{images!anti-aliasing}\index{anti-aliasing!in figures}%
\item [ -antialias\label{cs_aalias}]
\sameas{\fn{\$ANTI\_ALIAS}\texttt{ = 1;}} (Default is 0.)\\
Generated images of \env{figure} environments and external \PS\ files
should use anti-aliasing. By default anti-aliasing is not
used with these images, since this may interfere with the contents
of the images themselves.

\index{images!anti-aliasing}\index{anti-aliasing!in math and text}%
\item [ -antialias\_text\label{cs_aaliastext}]
\sameas{\fn{\$ANTI\_ALIAS\_TEXT}\texttt{ = 1;}} (Default is 1.)\\
Generated images of typeset material such as text, mathematical formulas,
tables and the content of \env{makeimage} \htmlref{environments}{makeimage},
should use anti-aliasing effects.\html{\\}
The default is normally to use anti-aliasing for text, 
since the resulting images are much clearer on-screen. 
However the default may have been changed locally.


\index{images!no anti-aliasing}\index{anti-aliasing!turned off}%
\item [ -no\_antialias\label{cs_noaalias}]
\sameas{\fn{\$ANTI\_ALIAS}\texttt{ = 0;}} (Default is 0.)\\
Generated images of \env{figure} environments and external \PS\ files 
should not use anti-aliasing with images, 
though the local default may have been changed to use it.


\index{images!no anti-aliasing}\index{anti-aliasing!blurred printing}%
\item [ -no\_antialias\_text\label{cs_noaaliastext}]
\sameas{\fn{\$ANTI\_ALIAS\_TEXT}\texttt{ = 0;}} (Default is 1.)\\
Generated images of typeset material should \emph{not} use anti-aliasing effects.
Although on-screen images of text are definitely improved using anti-aliasing,
printed images can be badly blurred, even at 300dpi. 
Higher resolution printers do a much better job with the resulting grey-scale images.

\cbversion{98.1}\begin{changebar}
\index{images!white background}%
\item [ -white\label{cs_white}]
\sameas{\fn{\$WHITE\_BACKGROUND}\texttt{ = 1;}} (Default is 1.)\\
Ensures that images of \env{figure} environments have a white
background. Otherwise transparency effects may not work correctly.

\item [ -no\_white\label{cs_nowhite}]
\sameas{\fn{\$WHITE\_BACKGROUND}\texttt{ = '';}} (Default is 1.)\\
Cancels the requirement that \env{figure} environments have a white background.

\index{images!latex dump}%
\item [ -ldump\label{cs_ldump}]
\sameas{\fn{\$LATEX\_DUMP}\texttt{ = 1;}} (Default is 0.)\\
Use this if you want to speed up image processing during the 2nd
and subsequent runs of \latextohtml{} on the same document.
The translator now produces a \LaTeX{} format-dump of the preamble to
\fn{images.tex} which is used on subsequent runs.
This significantly reduces the startup time when
\LaTeX{} reads the \fn{images.tex} file for image-generation.

This process actually consumes additional time on the first run, 
since \LaTeX{} is called twice --- once to create the format-dump,
then again to load and use it.
The pay-off comes with the faster loading on subsequent runs. 
Approximately 1 Meg of disk space is consumed by the dump file.
\end{changebar}
%
\end{htmllist}


\subsubsection{Switches controlling Navigation Panels\label{navoptions}}
%\subsection[center]{Switches controlling Navigation Panels\label{navoptions}}
%
The following switches govern whether to include one or more navigation
panels on each \texttt{HTML} page, 
also which buttons to include within such a panel.

\begin{htmllist}\htmlitemmark{PinkBall}%

\index{navigation panel!disabled}
\item [ -no\_navigation\label{cs_nonavig}]
\sameas{\fn{\$NO\_NAVIGATION}\texttt{ = 1;}}\\
Disable the mechanism for putting navigation links in each page.\\
This overrides any settings of the \fn{\$TOP\_NAVIGATION},
\fn{\$BOTTOM\_NAVIGATION} and \fn{\$AUTO\_NAVIGATION} variables.

\index{navigation panel!at top of page}
\item [ -top\_navigation\label{cs_topnavig}]
\sameas{\fn{\$TOP\_NAVIGATION}\texttt{ = 1;}}\\
Put navigation links at the top of each page.    

\index{navigation panel!at bottom of page}%
\item [ -bottom\_navigation\label{cs_botnavig}]
\sameas{\fn{\$BOTTOM\_NAVIGATION}\texttt{ = 1;}}\\
Put navigation links at the bottom of each page as well as the top.

\index{navigation panel!according to number of words}
\item [ -auto\_navigation\label{cs_autonavig}]
\sameas{\fn{\$AUTO\_NAVIGATION}\texttt{ = 1;}}\\
Put navigation links at the top of each page. 
Also put one at the bottom of the page, if the page exceeds 
\fn{\$WORDS\_IN\_PAGE} number of words (default = 450).

\index{navigation panel!next logical page}
\item [ -next\_page\_in\_navigation\label{cs_nextinnavig}]
\sameas{\fn{\$NEXT\_PAGE\_IN\_NAVIGATION}\texttt{ = 1;}}\\
Put a link to the \textsl{next logical page} in the navigation panel.

\index{navigation panel!previous logical page}%
\item [ -previous\_page\_in\_navigation\label{cs_previnnavig}]~\\%
\sameas{\fn{\$PREVIOUS\_PAGE\_IN\_NAVIGATION}\texttt{ = 1;}}\\
Put a link to the \textsl{previous logical page} in the navigation panel.

\index{navigation panel!link to table-of-contents}
\item [ -contents\_in\_navigation\label{cs_continnavig}]
\sameas{\fn{\$CONTENTS\_IN\_NAVIGATION}\texttt{ = 1;}}\\
Put a link to the \textsl{table-of-contents} in the navigation panel if there is one.

\index{navigation panel!link to index}
\item [ -index\_in\_navigation\label{cs_indexinnavig}]
\sameas{\fn{\$INDEX\_IN\_NAVIGATION}\texttt{ = 1;}}\\
Put a link to the \textsl{index-page} in the navigation panel if there is an index.
%
\end{htmllist}


\subsubsection{Switches for Linking to other documents\label{otherdocs}}
%\subsubsection[center]{Switches for Linking to other documents\label{otherdocs}}
%
When processing a single stand-alone document, the switches described in this
section should not be needed at all, since the automatically generated navigation panels, 
described \hyperref{on the previous page}{in Section~}{, }{navoptions}
should generate all the required navigation links.
However if a document is to be regarded as part of a much larger document,
then links from its first and final pages, to locations in other parts
of the larger (virtual) document, 
need to be provided explicitly for some of the buttons in the navigation panel.

The following switches allow for such links to other documents,
by providing the title and URL for navigation panel hyperlinks.
In particular, the ``Document Segmentation''
\hyperref{feature}{feature of Section~}{}{Segmentation} 
necessarily makes great use of these switches.
It is usual for the text and targets of these navigation hyperlinks
to be recorded in a \fn{Makefile}, 
to avoid tedious typing of long command-lines having many switches.

\begin{htmllist}\htmlitemmark{PinkBall}%
%
\index{URL!universal resource locator}\index{URL!for UP@for \textsl{UP} button}%
\item [ -up\_url \Meta{URL}\label{cs_upURL}] 
\sameas{\fn{\$EXTERNAL\_UP\_LINK}\texttt{ = \char34}\Meta{URL}\texttt{\char34;}}\\
Specifies a universal resource locator (URL) to associate
with the \textsl{``UP''} button in the navigation panel(s).


\item [ -up\_title \Meta{string}\label{cs_uptitle}] 
\sameas{\fn{\$EXTERNAL\_UP\_TITLE}\texttt{ = \char34}\Meta{string}\texttt{\char34;}}\\
Specifies a title associated with this URL.%


\index{URL!for PREVIOUS@for \textsl{PREVIOUS} button}%
\item [ -prev\_url \Meta{URL}\label{cs_prevURL}] 
\sameas{\fn{\$EXTERNAL\_PREV\_LINK}\texttt{ = \char34}\Meta{URL}\texttt{\char34;}}\\
Specifies a URL to associate
with the \textsl{``PREVIOUS''} button in the navigation panel(s).%


\item [ -prev\_title \Meta{string}\label{cs_prevtitle}] 
\sameas{\fn{\$EXTERNAL\_PREV\_TITLE}\texttt{ = \char34}\Meta{string}\texttt{\char34;}}\\
Specifies a title associated with this URL.%


\index{URL!for NEXT@for \textsl{NEXT} button}%
\item [ -down\_url \Meta{URL}\label{cs_downURL}]
\sameas{\fn{\$EXTERNAL\_DOWN\_LINK}\texttt{ = \char34}\Meta{URL}\texttt{\char34;}}\\
Specifies a URL for the \textsl{``NEXT''} button in the navigation panel(s).%


\item [ -down\_title \Meta{string}\label{cs_downtitle}] 
\sameas{\fn{\$EXTERNAL\_DOWN\_TITLE}\texttt{ = \char34}\Meta{string}\texttt{\char34;}}\\
Specifies a title associated with this URL.%


\index{URL!for CONTENTS@for \textsl{CONTENTS} button}%
\item [ -contents \Meta{URL}\label{cs_contentsURL}] 
\sameas{\fn{\$EXTERNAL\_CONTENTS}\texttt{ = \char34}\Meta{URL}\texttt{\char34;}}\\
Specifies a URL for the \textsl{``CONTENTS''} button,
for document segments that would not otherwise have one.%


\index{URL!for INDEX@for \textsl{INDEX} button}%
\item [ -index \Meta{URL}\label{cs_indexURL}] 
\sameas{\fn{\$EXTERNAL\_INDEX}\texttt{ = \char34}\Meta{URL}\texttt{\char34;}}\\
Specifies a URL for the \textsl{``INDEX''} button, 
for document segments that otherwise would not have an index.

\index{URL!for external Bibliography}%
\item [ -biblio \Meta{URL}\label{cs_biblioURL}] 
\sameas{\fn{\$EXTERNAL\_BIBLIO}\texttt{ = \char34}\Meta{URL}\texttt{\char34;}}\\
Specifies the URL for the bibliography page to be used,
when not explicitly part of the document itself.
%
\end{htmllist}
\smallskip
\textbf{Warning: }
On some systems it is difficult to give text-strings \Meta{string} containing
space characters, on the command-line or via a \fn{Makefile}.
One way to overcome this is to use the corresponding variable.
Another way is to replace the spaces with underscores (\verb|_|).


\subsubsection{Switches for Help and Tracing}
%\subsubsection[center]{Switches for Help and Tracing}
%
The first two of the following switches are self-explanatory.
When problems arise in processing a document, 
the switches \Cs{debug} and \Cs{verbosity} will each
cause \latextohtml{} to generate more output to the screen.
These extra messages should help to locate the cause of the problem.

\begin{htmllist}\htmlitemmark{PinkBall}%
%
\index{options!defining temp directory}%
\item [ -tmp \Meta{path}\label{cs_tmpdir}]
Define a temporary directory to use for image generation.
If \Meta{path} is $0$, the standard temporary directory \texttt{/tmp}
is used.
%
\index{options!print option listing}%
\item [ -h(elp)\label{cs_optionlist}]
Print out the list of all command-line options.

\index{current version@current version of \latextohtml}%
\item [ -v\label{cs_showversion}] 
Print the current version of \latextohtml.

\index{debugging}\index{diagnostic information}%
\item [ -debug\label{cs_debugmode}]
\sameas{\fn{\$DEBUG}\texttt{ = 1;}}\\
Run in debug-mode, displaying messages and/or diagnostic information 
about files read, and utilities called by \latextohtml.
Shows any messages produced by these calls.%

\index{debugging!Perl@\Perl{} debugger}%

More extensive diagnostics, from the \Perl{} debugger, 
can be obtained by appending the string `\texttt{-w-}' 
to the 1st line of the \fn{latex2html} (and other) \Perl{} script(s).


\index{verbosity level}%
\item [ -verbosity \Meta{num}\label{cs_verbositylevel}] 
\sameas{\fn{\$VERBOSITY}\texttt{ = \Meta{num};}}\\
Display messages revealing certain aspects of the processing
performed by \latextohtml{} on the provided input file(s).
The \Meta{num} parameter can be an integer in the range 0 to 8.
Each higher value adds to the messages produced.

\begin{itemize}
\item[0. ]
No special tracing; as for versions of \latextohtml{} prior to \textsc{v97.1}\,.

\item[1. ] (This is the default.)
Show section-headings and the corresponding \texttt{HTML} file names,
and indicators that major stages in the processing have been completed.

\item[2. ]
Print environment names and identifier numbers, and new theorem-types.
Show warnings as they occur, and indicators for more stages of processing.
Print names of files for storing auxiliary data arrays.

\item[3. ]
Print command names as they are encountered and processed; also
any unknown commands encountered while pre-processing.
Show names of new commands, environments, theorems, counters and
counter-dependencies, for each document partition.

\item[4. ]
Indicate command-substitution the pre-process of math-environments.
Print the contents of unknown environments for processing in \LaTeX,
both before and after reverting to \LaTeX{} source.
Show all operations affecting the values of counters.
Also show links, labels and sectioning keys, at the stages of processing.

\item[5. ]
Detail the processing in the document preamble.
Show substitutions of new environments.
Show the contents of all recognised environments, both before
and after processing. Show the cached/encoded information for the image keys,
allowing two images to be tested for equality.

\item[6. ]
Show replacements of new commands, accents and wrapped commands.

\item[7. ]
Trace the processing of commands in math mode; both before and after.

\item[8. ]
Trace the processing of all commands, both before and after.

\end{itemize}
%
The command-line option sets an initial value only.
During processing the value of \fn{\$VERBOSITY} can be set
dynamically using the \verb|\htmltracing{...}| command, 
whose argument is the desired value, 
or by using the more general \Lc{HTMLset} command
as follows: \verb|\HTMLset{VERBOSITY}{|\Meta{num}\texttt{\char125}\,.
%
\end{htmllist}
\index{options|)}%
%

\subsubsection{Other Configuration Variables, without switches}
%\subsection[center]{Other Configuration Variables, without switches}
%
The configuration variables described here do not warrant having
a command-line switch to assign values.
Either they represent aspects of \latextohtml\ that are specific to the local site,
or they govern properties that should apply to \emph{all} documents,
rather than something that typically would change
for the different documents within a particular sub-directory.

Normally these variables have their value set within the \fn{latex2html.config} file.
In the following listing the defaults are shown, as the lines of \Perl{} code
used to establish these values.
If a different value is required, then these can be assigned from
a local \fn{.latex2html-init} initialisation file, 
without affecting the defaults for other users, 
or documents processed from other directories.

\begin{htmllist}\htmlitemmark{PurpleBall}%
%
\index{directory delimiter}%
\item [\fn{\$dd} ]
~holds the string to be used in file-names to delimit directories;
it is set internally to `\texttt{/}', unless the variable has already been given
a value within \fn{latex2html.config}~.

\noindent
\textbf{Note:} This value \emph{cannot} be set within a \fn{.latex2html-init} 
initialisation file, since its value \emph{needs to be known} 
in order to find such a file.

\index{installation variables!inserted@inserted by \fn{install-test}}%
\item [\fn{\$LATEX2HTMLDIR}] \latex{ }%
Read by the \fn{install-test} script from \fn{latex2html.config}, 
its value is inserted into the \fn{latex2html} \Perl{} script as part
of the installation process.

\item [\fn{\$LATEX2HTMLSTYLES}\texttt{ = \char34\$LATEX2HTMLDIR/styles\char34;}]
Read from the \fn{latex2html.config} file by \fn{install-test}\,,
its value is checked to locate the \fn{styles/} directory.

\item [\fn{\$LATEX2HTMLVERSIONS}\texttt{ = \char34\$LATEX2HTMLDIR/versions\char34;}]
The value of this variable should be set within \fn{latex2html.config} 
to specify the directory path where the version and extension files can be found.

\item [\fn{\$ALTERNATIVE\_ICONS}\texttt{ = '';}\label{alticons}]
This may contain the (relative) directory path to a set of customised icons
to be used in conjunction with the \htmlref{\Cs{local\_icons}}{cs_localicons} switch.

\index{installation variables!checked@checked by \fn{install-test}}%
\item [\fn{\$TEXEXPAND}\texttt{  = \char34\$LATEX2HTMLDIR/texexpand\char34;}]
Read by the \fn{install-test} \Perl{} script from \fn{latex2html.config}, 
its value is used to locate the \fn{texexpand} \Perl{} script.

\item [\fn{\$PSTOIMG}\texttt{ = \char34\$LATEX2HTMLDIR/pstoimg\char34;}]
Read by the \fn{install-test} \Perl{} script from \fn{latex2html.config}, 
its value is used to locate the \fn{pstoimg} \Perl{} script.

\item [\fn{\$IMAGE\_TYPE}\texttt{ = '\Meta{image-type}';}]
Set in \fn{latex2html.config}, the currently supported \Meta{image-type}s
are: \texttt{gif} and \texttt{png}.

\item [\fn{\$DVIPS}\texttt{ = 'dvips';}]
Read from \fn{latex2html.config} by \fn{install-test}, 
its value is checked to locate the \fn{dvips} program or script.

\smallskip\noindent
There could be several reasons to change the value here:
\begin{itemize}
\item
add a switch \texttt{-P}\Meta{printer} to load a specific configuration-file;\\
e.g. to use a specific set of \PS\ fonts, for improved image-generation.
\item
to prepend a path to a different version of \fn{dvips} than normally
available as the system default (e.g. the printing requirements are different).
\item
to append debugging switches, in case of poor quality images;\\
one can see which paths are being searched for fonts and other resources.
\item
to prepend commands for setting path variables that \fn{dvips} may need
in order to locate fonts or other resources.
\end{itemize}

If automatic generation of fonts is required, using \MF, the following
configuration variables are important.
%
\begin{htmllist}\htmlitemmark{OrangeBall}%
%
\item [\fn{\$PK\_GENERATION}\texttt{ = 1;}\label{pkgen}]
This variable must be set, to initiate font-generation; otherwise fonts
will be scaled from existing resources on the local system.\\
In particular this variable must \emph{not} be set, if one wishes to use
\PS\ fonts or other scalable font resources 
(see the \Cs{scalable\_fonts} \htmlref{switch}{scalefonts}).

\item [\fn{\$DVIPS\_MODE}\texttt{ = 'toshiba';}]
The mode given here must be available in the \fn{modes.mf} file,
located with the \MF\ resource files, perhaps in the \fn{misc/} subdirectory.

\item [\fn{\$METAFONT\_DPI}\texttt{ = 180;}]
The required resolution, in dots-per-inch, should be listed specifically
within the \fn{MakeTeXPK} script, called by \fn{dvips} to invoke \MF\ with
the correct parameters for the required fonts.
%
\end{htmllist}


\item [\fn{\$LATEX}\texttt{ = 'latex';}]
Read from \fn{latex2html.config} by \fn{install-test}, 
its value is checked to locate the \fn{latex} program or script.

\smallskip\noindent
If \LaTeX{} is having trouble finding style-files and/or packages, then
the default command can be prepended with other commands 
to set environment variables intended to resolve these difficulties;\\
e.g. 
\verb|$LATEX = 'setenv TEXINPUTS |\Meta{path to search}\verb| ; latex' |.

There are several variables to help control exactly which files are
read by \latextohtml{} and by \LaTeX{} when processing images:

\begin{htmllist}\htmlitemmark{OrangeBall}%
\item [\fn{\$TEXINPUTS}]
This is normally set from the environment variable of the same name.
If difficulties occur so that styles and packages are not being found, 
then extra paths can be specified here, to resolve these difficulties.

\item [\fn{\$DONT\_INCLUDE}]
This provides a list of filenames and extensions to \emph{not} include, 
even if requested to do so by an \Lc{input} or \Lc{include} command.\\
(Consult \fn{latex2html.config} for the default list.)

\item [\fn{\$DO\_INCLUDE}\texttt{ = '';}]
List of exceptions within the \fn{\$DONT\_INCLUDE} list.
These files \emph{are} to be read if requested by an \Lc{input}
or \Lc{include} command.
\end{htmllist}


\item [\fn{\$ICONSERVER}\texttt{ = '\Meta{URL}';}]
This is used to specify a URL to find the standard icons, as used for the
navigation buttons.\html{\\}
Names for the specific images size, as well as size information, 
can be found in \fn{latex2html.config}. The icons themselves can be replaced
by customised versions, provided this information is correctly updated
and the location of the customised images specified as the value
of \fn{\$ICONSERVER}.\\
When the \htmlref{\Cs{local\_icons}}{cs_localicons} switch is used,
so that a copy of the icons is placed with the \texttt{HTML} files and other
generated images, the value of \fn{\$ICONSERVER} is not needed within
the \texttt{HTML} files themselves. However it \emph{is} needed to find
the original icons to be copied to the local directory.

\item [\fn{\$NAV\_BORDER}\texttt{ = }\Meta{num}\texttt{;}]
The value given here results in a border, measured in points, around each icon.\html{\\}
A value of `\texttt{0}' is common, to maintain strict
alignment of inactive and active buttons in the control panels.


\item [\fn{\$LINKNAME}\texttt{ = '\char34index.\$EXTN\char34';}]
This is used when the \fn{\$NO\_AUTO\_LINK} \htmlref{variable}{noautolink} is empty, 
to allow a URL to the working directory to be sufficient to reach the main page 
of the completed document. 
It specifies the name of the \texttt{HTML} file which will be automatically
linked to the directory name.\html{\\}
The value of \fn{\$EXTN} is \texttt{.html} unless \fn{\$SHORTEXTN} is set, 
in which case it is \texttt{.htm}~. 

\item [\fn{\$LINKPOINT}\texttt{ = '\char34\$FILE\$EXTN\char34';}]
This specifies the name of the \texttt{HTML} file to be duplicated, or symbolically
linked, with the name specified in \fn{\$LINKNAME}.\\
At the appropriate time the value of \fn{\$FILE} is the document name, 
which usually coincides with the name of the working directory.

\index{character set}%
\index{character set!iso-latin@\texttt{\char34iso\_8859\_1\char34}}%
\item [\fn{\$CHARSET}\texttt{ = 'iso\_8859\_1';}\label{charset}]
This specifies the character set used within the \texttt{HTML} pages
produced by \latextohtml. If no value is set in a configuration
or initialisation file, the default value will be assumed.
The lowercase form \texttt{\$charset} is also recognised, but
this is overridden by the uppercase form.

\index{images!strange accents}%
\item [ \fn{\$ACCENT\_IMAGES}\texttt{ = 'large'};]
Accented characters that are not part of the ISO-Latin fonts
can be generated by making an image using \LaTeX.
This variable contains a (comma-separated) list of \LaTeX{} commands
for setting the style to be used when these images are made. 
If the value of this variable is empty then the accent is simply
ignored, using an un-accented font character (not an image) instead.
%
\end{htmllist}

\bigskip\noindent
Within the \fn{color.perl} package, the following variables are used to
identify the names of files containing specifications for named colors.
Files having these names are provided, in the \fn{\$LATEX2HTMLSTYLES}
directory, but they could be moved elsewhere, 
or replaced by alternative files having different names.
In such a case the values of these variables should be altered accordingly.
%
\begin{itemize}
\item []\fn{\$RGBCOLORFILE}\texttt{ = 'rgb.txt';}
\item []\fn{\$CRAYOLAFILE}\texttt{ = 'crayola.txt';}
\end{itemize}

\bigskip\noindent
The following variables may well be altered from the system defaults,
but this is best done using a local \fn{.latex2html-init} initialisation file,
for overall consistency of style within documents located at the same site,
or sites in close proximity.

\begin{htmllist}\htmlitemmark{PurpleBall}%
%
\item [\fn{\$default\_language}\texttt{ = 'english';}\label{isolanguage}]
This establishes which language code is to be placed within
the \HTMLtag{!DOCTYPE ... } tag that may appear at the beginning
of the \texttt{HTML} pages produced. Loading a package for
an alternative language can be expected to change the value of this variable.\html{\\}
See also the \fn{\$TITLES\_LANGUAGE} variable, described \htmlref{next}{doclanguage}.


\item [\fn{\$TITLES\_LANGUAGE}\texttt{ = 'english';}\label{doclanguage}]
This variable is used to specify the actual strings used for standard document
sections, such as ``Contents'', ``References'', ``Table of Contents'', etc.\html{\\}
Support for French and German titles is available in corresponding packages.
Loading such a package will normally alter the value of this variable, as well as 
the \fn{\$default\_language} variable described \htmlref{above}{isolanguage}.


\item [\fn{\$WORDS\_IN\_NAVIGATION\_PANEL\_TITLES}\texttt{ = 4;}]
Specifies how many words to use from section titles,
within the textual hyperlinks which accompany the navigation buttons.


\item [\fn{\$WORDS\_IN\_PAGE}\texttt{ = 450;}]
Specifies the minimum page length required before a navigation panel
is placed at the bottom of a page,
when the \fn{\$AUTO\_NAVIGATION} variable is set.


\item [\fn{\$CHILDLINE}\texttt{ = \char34<BR><HR>\char92n\char34;}]
This gives the \texttt{HTML} code to be placed between the child-links
table and the ordinary contents of the page on which it occurs.


\item [\fn{\$NETSCAPE\_HTML}\texttt{ = 0;}]
When set, this variable specifies that \texttt{HTML} code may be present which
does not conform to any official standard. 
This restricts the contents of any \HTMLtag{!DOCTYPE ... } tag which may be placed
at the beginning of the \texttt{HTML} pages produced.


\item [\fn{\$BODYTEXT}\texttt{ = '';}]
The value of this variable is used within the \HTMLtag{BODY ... } tag;
e.g. to set text and/or background colors.\html{\\}
It's value is overridden by the \verb|\bodytext| command, 
and can be added-to or parts changed using the \Lc{htmlbody} command
or \Lc{color} and \Lc{pagecolor} from the \env{color} package.


\item [\fn{\$INTERLACE}\texttt{ = 1;}\label{interlace}]
When set, interlaced images should be produced.\\
This requires \htmlref{graphics utilities}{graphicsutils} to be available 
to perform the interlacing operation.


\item [\fn{\$TRANSPARENT\_FIGURES}\texttt{ = 1;}\label{transfigs}]
When set, the background of images should be made transparent;
otherwise it is white.\html{\\}
This requires \htmlref{graphics utilities}{graphicsutils} %
to be available which can specify the color to be made transparent.


\item [\fn{\$FIGURE\_SCALE\_FACTOR}\texttt{ = 1.6;}\label{figscale}]
Scale factor applied to all images of \env{figure} and other environments, 
when being made into an image.

\smallskip\noindent
Note that this does not apply to recognised mathematics environments, 
which instead use the contents of \fn{\$MATH\_SCALE\_FACTOR} 
and \fn{\$DISP\_SCALE\_FACTOR} to specify scaling.


\item [\fn{\$MATH\_SCALE\_FACTOR}\texttt{ = 1.6;}\label{mathscale}]
Scale factor applied to all images of mathematics, both inline
and displayed. 
A value of 1.4 is a good alternative, with anti-aliased images.


\item [\fn{\$DISP\_SCALE\_FACTOR}\texttt{ = 1;}\label{dispscale}]
Extra scale factor applied to images of displayed math environments.\html{\\}
When set, this value multiplies \fn{\$MATH\_SCALE\_FACTOR}
to give the total scaling. 
A value of `1.2' is a good choice to accompany \verb|$MATH_SCALE_FACTOR = 1.4;|\,.


\item [\fn{\$EXTRA\_IMAGE\_SCALE}\label{ximagescale}]
This may hold an extra scale factor that can be applied to all generated images.\html{\\}
When set, it specifies that a scaling of \fn{\$EXTRA\_IMAGE\_SCALE} be applied 
when images are created, but to have their height and width recorded 
as the un-scaled size. This is to coax browsers into scaling the (usually larger) 
images to fit the desired size; when printed a better quality can be obtained.
Values of `1.5' and `2' give good print quality at 600dpi.


\item [\fn{\$PAPERSIZE}\texttt{ = 'a5';}\label{papersize}]
Specifies the size of a page for typesetting figures
or displayed math, when an image is to be generated.\\
This affects the lengths of lines of text within images.
Since images of text or mathematics should use larger sizes than
when printed, else clarity is lost at screen resolutions,
then a smaller paper-size is generally advisable.
This is especially so if both the \fn{\$MATH\_SCALE\_FACTOR}
and \fn{\$DISP\_SCALE\_FACTOR} scaling factors are being used,
else some images may become excessively large, including a lot
of blank space.


\item [\fn{\$LINE\_WIDTH}\texttt{ = 500;}\label{linewidth}]
Formerly specified the width of an image, when the contents were
to be right- or center-justified. (No longer used.)

\end{htmllist}

\bigskip

\noindent
The following variables are used to access the utilities required during image-generation. 
File and program locations on the local system are established by the \fn{configure-pstoimg} 
\Perl{} script and stored within \fn{\$LATEX2HTMLDIR/local.pm} as \Perl{} code, 
to be read by \fn{pstoimg} when required.

After running the \fn{configure-pstoimg} \Perl{} script it should not be necessary 
to alter the values obtained. Those shown below are what happens on the author's system;
they are for illustration only and do \emph{not} represent default values.
%

\begin{itemize}\label{graphicsutils}%
\item  \fn{\$GS\_LIB}\texttt{ = '/usr/local/share/ghostscript/4.02';}
\item  \fn{\$PNMCAT}\texttt{ = '/usr/local/bin/pnmcat';}
\item  \fn{\$PPMQUANT}\texttt{ = '/usr/local/bin/ppmquant';}
\item  \fn{\$PNMFLIP}\texttt{ = '/usr/local/bin/pnmflip';}
\item  \fn{\$PPMTOGIF}\texttt{ = '/usr/local/bin/ppmtogif';}
\item  \fn{\$HOWTO\_TRANSPARENT\_GIF}\texttt{ = 'netpbm';}
\item  \fn{\$GS\_DEVICE}\texttt{ = 'pnmraw';}
\item  \fn{\$GS}\texttt{ = '/usr/local/bin/gs';}
\item  \fn{\$PNMFILE}\texttt{ = '/usr/local/bin/pnmfile';}
\item  \fn{\$HOWTO\_INTERLACE\_GIF}\texttt{ = 'netpbm';}
\item  \fn{\$PBMMAKE}\texttt{ = '/usr/local/bin/pbmmake';}
\item  \fn{\$PNMCROP}\texttt{ = '/usr/local/bin/pnmcrop';}
\item  \fn{\$TMP}\texttt{ = '/usr/var/tmp';}
\end{itemize}

\noindent
The following variables are no longer needed, having been replaced by the
more specific information obtained using the \Perl{} script \fn{configure-pstoimg}.
\begin{itemize}
\item [] \fn{\$USENETPBM}\texttt{ = 1;}
\item [] \fn{\$PBMPLUSDIR}\texttt{ = '/usr/local/bin';}
\end{itemize}


%\filbreak
\subsection{Extending the Translator\label{sec:ext}}%
%\section{Extending the Translator\label{sec:ext}}%
\tableofchildlinks*\htmlrule
%
In \hyperref{an earlier section}{Section~}{}{sec:sty} is was seen how
the capabilities of \latextohtml{} can be extended to cope with
\LaTeX{} commands defined in packages and style-files.
This is in addition to defining simple macros and environments,
using \Lc{newcommand} and \Lc{newenvironment}.
The problem however, is that writing such extensions for packages 
requires an understanding of \Perl{} programming 
and of the way the processing in \latextohtml{} is organised. 

\index{new commands!ignore arguments}%
\index{new commands!arguments remain}%
\index{extensions!in initialisation file}
\smallskip

The default behaviour for unrecognised commands is 
for their arguments to remain in the \texttt{HTML} text,
whereas the commands themselves are passed on to \LaTeX,
in an attempt to generate an image. This is far from ideal,
for it is quite likely to lead to an error in \LaTeX{}
due to not having appropriate arguments for the command.

Currently there are several related mechanisms whereby a user 
can ask for particular commands and their arguments to be either 
\begin{itemize}
\item
\htmlref{ignored}{ignore} (via: \texttt{\&ignore\_commands}\,);
\item
passed on to \LaTeX{}  for processing \htmlref{as an image}{pass}\\
(via: \texttt{\&process\_commands\_in\_tex}\,);
\item
passed to \LaTeX{}  for processing as an image, 
\htmlref{as if an environment}{sensit}\\
(via: \texttt{\&process\_commands\_inline\_in\_tex}\,);
\item
passed on to \LaTeX, \htmlref{setting parameters}{setparam} 
to be used in subsequent images\\
(via: \texttt{\&process\_commands\_nowrap\_in\_tex}\,);
\item
processed as ``order-sensitive'', \htmlref{as if an environment}{wrapdef} 
rather than a command\\
(via: \texttt{\&process\_commands\_wrap\_deferred}\,);
\end{itemize}
The string beginning \texttt{\&.... } is the name of the \Perl{} subroutine 
that controls how the \LaTeX{} commands are to be subsequently treated during
processing by \latextohtml.
Towards the end of the \fn{latex2html} script, one finds a list
of \LaTeX{} commands to be handled by each of these subroutines.
These lists even include some common \TeX{} commands.

Analogous lists occur in most of the package extension files.
In many cases the commands are for fine-tuning the layout on
a printed page. They should simply be ignored; but any parameters
must not be allowed to cause unwanted characters to appear on
the \texttt{HTML} pages.
Customised extensions using these mechanisms may be included in the 
\fn{\$LATEX2HTMLDIR/latex2html.config} file, or in a personal 
\fn{\$HOME/.latex2html-init} initialisation file, 
as described \htmlref{next}{ignore}.



\subsubsection{Asking the Translator to Ignore Commands\label{ignore}}
%\subsection{Asking the Translator to Ignore Commands\label{ignore}}
\index{new commands!command ignored}\html{\\}%
Commands that should be ignored may be specified in the \fn{.latex2html-init}
file as input to the \verb|&ignore_commands| subroutine. 
Each command which is to be ignored should be on a separate line 
followed by compulsory or optional argument markers separated by 
{\verb|#|}'s e.g.\footnote{It is possible to add arbitrary(?) \Perl{} code
between any of the argument markers which will be executed when 
the command is processed. For this however a basic understanding of
how the translator works and of course \Perl{} is required.}:
\begin{quote}
%begin{latexonly}
\begin{small}
%end{latexonly}
\verb|<cmd_name1> # {} # [] # {} # [] ...|\\
\verb|<cmd_name2> # |\texttt{<<\emph{pattern}>>}\verb| # [] ...|
%begin{latexonly}
\end{small}
%end{latexonly}
\end{quote}
\verb|{}|'s mark compulsory arguments and \verb|[]|'s optional ones,
while \texttt{<<\emph{pattern}>>} denotes matching everything up to the
indicated string-pattern, given literally (e.g. \verb|\\endarray|);
spaces are ignored. Special characters
such as \verb|$|\,, \verb|&|\,, \verb|\| itself and perhaps some others,
need to be ``escaped'' with a preceding \verb|\|\,.


\index{new commands!arguments remain}\html{\\}%

Some commands may have arguments which should be left as text even
though the command should be ignored 
(e.g. \Lc{hbox}, \Lc{center}, etc.). 
In these cases arguments should be left unspecified.
Here is an example of how this mechanism may be used:
%begin{latexonly}
\begin{small}
%end{latexonly}
\begin{verbatim}
&ignore_commands( <<_IGNORED_CMDS_);
documentstyle # [] # {}
linebreak # []
mbox
<add your commands here>
_IGNORED_CMDS_
\end{verbatim}
%begin{latexonly}
\end{small}
%end{latexonly}


\subsubsection{Asking the Translator to Pass Commands to \LaTeX\label{pass}}%
%\subsection{Asking the Translator to Pass Commands to \LaTeX\label{pass}}%
\index{new commands!processed in LaTeX@processed in \LaTeX}\html{\\}%
Commands that should be passed on to \LaTeX{}  for processing because
there is no direct translation to \texttt{HTML} may be specified in the 
\fn{.latex2html-init}
file as input to the \verb|process_commands_in_tex| subroutine.
The format is the same as that for specifying commands to be ignored.
Here is an example:
%begin{latexonly}
\begin{small}
%end{latexonly}
\begin{verbatim}
&process_commands_in_tex (<<_RAW_ARG_CMDS_);
fbox # {}
framebox # [] # [] # {}
<add your commands here>
_RAW_ARG_CMDS_
\end{verbatim}
%begin{latexonly}
\end{small}
%end{latexonly}


\subsubsection{Handling ``order-sensitive'' Commands\label{sensit}}%
%\subsection{Handling ``order-sensitive'' Commands\label{sensit}}%
\index{new commands!order-sensitive}%
\index{new commands!order-sensitive!processed in LaTeX@processed in \LaTeX}%
\index{new commands!processed in LaTeX@processed in \LaTeX!order-sensitive}%
Some commands need to be passed to \LaTeX, but using the 
\verb|&process_commands_in_tex| subroutine gives incorrect results. 
This may occur when the command is ``order-sensitive'',
using information such as the value of a counter or a boolean expression 
(or perhaps requiring a box to have been constructed and saved).
Try using the \verb|&process_commands_inline_in_tex| subroutine instead.
Commands declared this way are first ``wrapped'' within a dummy environment,
which ensures that they are later processed in correct order with other 
environments and order-sensitive commands.

\index{order-sensitive!setting parameters}\label{setparam}\html{\\}%

Other commands may need to be passed to \LaTeX, not to create an image
themselves, but to affect the way subsequent images are created.
For example a color command such as \verb|\color{red}| should set the
text-colour to `red' for all subsequent text and images. This must
be sent to \LaTeX{} so that it is processed at exactly the right time;
i.e. before the first image required to be `red' but following any
images that are not intended to be affected by this colour-change.\html{\\}
The subroutine \verb|process_commands_nowrap_in_tex| is designed
specifically to meet such requirements.

\index{order-sensitive!wrap-deferred}\label{wrapdef}\html{\\}%

Commands can be order-sensitive without having to be passed to \LaTeX.
Indeed even if a \Perl{} subroutine has been carefully written to process
the command, it may still give wrong results if it is order-sensitive,
depending on the values of counters, say.
To handle such cases there is the \verb|&process_commands_wrap_deferred|
subroutine. This also ``wraps'' the command within a dummy environment,
but when that environment is processed the contents are not sent to \LaTeX,
as in the previous case. All of the standard \LaTeX{} commands to change, 
set or read the values of counters are handled in this way.



\subsection{Customising the Navigation Panels\label{sec:navpanel}}%
%\section{Customising the Navigation Panels\label{sec:navpanel}}%
\index{navigation panel|(}%
\index{navigation panel!customisable}\html{\\}%
The navigation panels are the strips containing ``buttons'' and text 
that appears at the top and perhaps at the bottom 
of each generated page and provides hypertext links 
to other sections of a document. 
Some of the options and variables that control whether and 
where it should appear are presented in \hyperref{an earlier section}%
{Section~}{}{navoptions}. 

\index{navigation panel!customised buttons}%
\index{htmladdtonavigation@\Lc{htmladdtonavigation}}\html{\\}%
A simple mechanism for appending customised buttons to the navigation
panel is provided by the command \Lc{htmladdtonavigation}. 
This takes one argument which \latextohtml{} appends to the navigation panel. 
For example,
%begin{latexonly}
\begin{small}
%end{latexonly}
\begin{verbatim}
\htmladdtonavigation{\htmladdnormallink
   {\htmladdimg{http://server/mybutton.gif}}{http://server/link}}
\end{verbatim}
%begin{latexonly}
\end{small}
%end{latexonly}
will add an active button \fn{mybutton.gif} pointing to the specified location.

\index{navigation panel!order of buttons}\html{\\}\noindent
Apart from these facilities it is also 
possible to specify completely what appears in the navigation panels
and in what order. As each section is processed, \latextohtml{}
assigns relevant information to a number of global variables.
These variables are used by the
subroutines \texttt{top\_navigation\_panel} and
\texttt{bottom\_navigation\_panel},
where the navigation panel is constructed as a string consisting of
these variables and some formatting information. 

\index{navigation panel!Perl subroutines@\Perl{} subroutines}\html{\\}\noindent
These subroutines can be redefined in a system- 
or user-configuration file (respectively, 
\fn{\$LATEX2HTMLDIR/latex2html.config} and \fn{\$HOME/.latex2html-init}). 
Any combination of text, \texttt{HTML} tags, 
and the variables mentioned below is acceptable.

\medskip\noindent
The control-panel variables are:\nobreak

\medskip\nobreak
\index{navigation panel!buttons}%
\index{navigation panel!iconic links}\html{\\}
\noindent
\textbf{Iconic links (buttons):}
\begin{htmllist}\htmlitemmark{RedBall}\addtolength{\leftskip}{15pt}%
\item [\fn{\$UP}]  Points up to the ``parent'' section;
\item [\fn{\$NEXT}] Points to the next section;
\item [\fn{\$NEXT\_GROUP}] Points to the next ``group'' section;
\item [\fn{\$PREVIOUS}] Points to the previous section;
\item [\fn{\$PREVIOUS\_GROUP}] Points to the previous ``group'' section;
\item [\fn{\$CONTENTS}] Points to the contents page if there is one;
\item [\fn{\$INDEX}] Points to the index page if there is one.
\end{htmllist}

\index{navigation panel!section titles}%
\index{navigation panel!textual links}\html{\\}\noindent
\textbf{Textual links (section titles):}
\begin{htmllist}\htmlitemmark{PinkBall}\latex{\addtolength{\leftskip}{15pt}}%
\item [\fn{\$UP\_TITLE}]  Points up to the ``parent'' section;
\item [\fn{\$NEXT\_TITLE}] Points to the next section;
\item [\fn{\$NEXT\_GROUP\_TITLE}] Points to the next ``group'' section;
\item [\fn{\$PREVIOUS\_TITLE}] Points to the previous section;
\item [\fn{\$PREVIOUS\_GROUP\_TITLE}] Points to the previous ``group'' section.
\end{htmllist}
If the corresponding section exists, each iconic button will contain an
active link to that section. If the corresponding section does
not exist, the button will be inactive. If the section corresponding
to a textual link does not exist then the link will be empty.

The ``next group'' and ``previous group'' are rarely used, 
since it is usually possible to determine
which are the next/previous logical pages in a document.
However these may be needed occasionally with segmented documents,
when the segments have been created with different values
for the \fn{\$MAX\_SPLIT\_DEPTH} variable. 
This is quite distinct from the segmented document effect 
in which the first page of one segment may have its `\emph{PREVIOUS}'
button artificially linked to the first page of the previous segment, 
rather than the last page.


\index{navigation panel!number of words}\html{\\}\noindent
The number of words that appears in each textual link
is controlled by the variable \fn{\$WORDS\_IN\_NAVIGATION\_PANEL\_TITLES}
which may also be changed in the configuration files.

\index{navigation panel!example}\html{\\}\noindent
Below is an example of a navigation panel for the bottom of \texttt{HTML} pages.
(Note that the ``\texttt{.}'' is \Perl{}'s string-concatenation operator 
and ``\texttt{\#}'' signifies a comment.)
%begin{latexonly}
\begin{small}
%end{latexonly}
\begin{verbatim}
sub bot_navigation_panel {
    #  Start with a horizontal rule and descriptive comment
    "<HR>\n" . "<!--Navigation Panel-->".
    # Now add a few buttons, with a space between them
       "$NEXT $UP $PREVIOUS $CONTENTS $INDEX $CUSTOM_BUTTONS" . 
    # Line break    
       "\n<BR>" .       
    # If ``next'' section exists, add its title to the navigation panel
       ($NEXT_TITLE ? "\n<B> Next:</B> $NEXT_TITLE" : undef) .   
    # Similarly with the ``up'' title ...
       ($UP_TITLE ? "\n<B>Up:</B> $UP_TITLE\n" : undef) . 
    # ... and the ``previous'' title
       ($PREVIOUS_TITLE ? "\n<B> Previous:</B> $PREVIOUS_TITLE\n" : undef) . 
}
\end{verbatim}
%begin{latexonly}
\end{small}
%end{latexonly}
Note that extra buttons may be included by defining suitable code for the container
\fn{\$CUSTOM\_BUTTONS}\,. 
The use of explicit `newline' (\verb|\n|) characters is not necessary for the
on-screen appearance of the navigation panel within a browser window. However 
it maintains an orderly organisation within the \texttt{.html} files themselves,
which is helpful if any hand-editing is later required, 
or simply to read their contents.
The corresponding subroutine for a navigation-panel at the top of a page
need not use the rule \HTMLtag{HR}, and would require a break (\HTMLtag{BR})
or two at the end, to give some visual separation from the following material.
\index{navigation panel|)}
\index{extensions|)}




             % Input counters and section
\label{sec:man}\index{manual}
\end{htmlonly}
\startdocument
%
%\section{User Manual\label{sec:man}\index{manual}}
\label{sec:man}
\index{customised layout}
\index{getting started}
\html{\\}\noindent
\latextohtml{} is a program for creating hyperlinked sets of \texttt{HTML} pages
from a document marked-up using \LaTeX{} commands.
Previous sections have discussed the results of specific \LaTeX{} commands.
In this section we discuss instead the extensive range
of command-line switches and options, and other aspects of \Perl{} code,
that affect the way the translation is performed.

\medskip\noindent
To use \latextohtml{} to translate a file~\Meta{file}\texttt{.tex }
containing \LaTeX{} commands, type:
\begin{quote}
\texttt{latex2html} \Meta{file}\texttt{.tex} 
\end{quote}

\index{browser@browser\label{IIIbrowser}}%
\index{viewer|see{\htmlref{browser}{IIIbrowser}}}%
\index{browser!NCSA Mosaic@\textsl{NCSA Mosaic}}%
\index{browser!Netscape Navigator@\textsl{Netscape Navigator}}\html{\\}%
\noindent
This will create a new directory called \Meta{file} which will contain 
the generated \texttt{HTML} files, some log files and possibly some images.
To view the result use an \texttt{HTML} browser, such as 
\appl{NCSA Mosaic} or \appl{Netscape Navigator},
on the main \texttt{HTML} file which is \texttt{\Meta{file}/\Meta{file}.html}\,. 
The file will contain navigation links 
to the other parts of the generated document.
%
\index{file suffix!optional}%
\index{file suffix!.tex@\texttt{.tex} or\texttt{.doc}}%
The \texttt{.tex} suffix is optional and will be supplied
by the program if it is omitted by the user.
Other suffixes are acceptable also, such as \texttt{.doc}\,.

\medskip
\index{options!command-line options}\html{\\}\noindent
It is possible to customise the output from \latextohtml{} using a number of 
\hyperref{command-line options}{command-line options (see Section~}{)}{options}
with which you can specify:
\begin{itemize}
\item
how to break up the document; 
\item
where to put the generated files, and what are their names;
\item
the title for the whole document; 
\item
the signature at the end of each page is; 
\item
how many navigation panels to provide, what links to put in them;
\item
what other documents this one links to;
\item 
extra information to include about the document;
\item 
whether to retain the original \LaTeX{} section-numbering scheme;
\item
and many other things that affect how the information is obtained,
processed or displayed in the resulting \texttt{.html} files and images.
\end{itemize}

\medskip
\index{manual!short on-line}\html{\\}\noindent
The \latextohtml{} script includes a short manual
which can be viewed with the command: 
\begin{verbatim}
  perldoc latex2html
\end{verbatim}


\subsection{Developing Documents using \latextohtml{}\label{devel}}
%\section{Developing Documents using \latextohtml{}\label{devel}}
%
Although any document containing \LaTeX{} commands can be translated
by the \latextohtml{} translator, the best results are obtained when
that document is itself a valid \LaTeX{} document.
Indeed it is generally a good idea to develop documents so that
they produce good readable results in both the \LaTeX{} typeset
version as well as a set of \texttt{HTML} pages.
This is not just a nicety; 
there are several good practical reasons for doing this.

\begin{htmllist}\htmlitemmark{BlueBall}
%
\item[\LaTeX{} macros: ]
The macro commands that \latextohtml{} recognises are based upon
corresponding commands for \LaTeX. If one tries to use syntax that is
incorrect for \LaTeX{} then there is no reason why \latextohtml{} should
be able to ``get it right'', by somehow recognising the true intent.


\index{error checking!using LaTeX@using \LaTeX}%
\item[error checking: ]
Processing the document first using \LaTeX{} is the easiest, 
and quickest, way to check for valid syntax.\html{\\}
Whereas \LaTeX{} stops at each error (when run in interactive mode),
allowing a fix to be made ``on the spot'' or a ``stop-fix-restart'',
\latextohtml{} does not stop when it detects an error in \LaTeX{} syntax.
Useful messages are given concerning missing or unmatched braces,
but other apparent anomalies generate only warning messages, 
which are saved to the end. 
(Some warnings are also shown immediately when the \fn{\$VERBOSITY} 
\htmlref{variable}{verbosity} is set to at least 3.)\html{\\}
In practice it can be much quicker to test for invalid syntax using \LaTeX{} before 
attempting to use the \latextohtml{} translator.

Furthermore, \LaTeX{} warns of cross-reference labels that have not been defined. 
This is useful to help avoid having hyperlinks which point to nowhere.

\index{error checking!missing braces}%
\index{error checking!unmatched brace}\html{\\}\noindent
The case of missing braces, or an unmatched opening brace, 
is an error that \latextohtml{} actually handles better than \LaTeX{} 
(or rather, the underlying \TeX{} processor).
Whereas \TeX{} only detects an error when something else goes wrong
later in the processing, \latextohtml{} shows where the
unmatched brace itself occurs. 


\index{auxiliary file}
\item[auxiliary file: ]
Some information that \latextohtml{} might need is normally
read from the \texttt{.aux} file for the document being processed;
or perhaps from the \texttt{.aux} file of a different document,
of which the current document is just a portion. 
Clearly a valid \LaTeX{} run is required to produce the \texttt{.aux} file.

\html{\\}\noindent
Of course if no information in the \texttt{.aux} file is actually used,
then this \LaTeX{} run can be neglected.
Also, if the \texttt{.aux} file has already been created, and edits
are made on the source which do not alter the information stored
within the \texttt{.aux} file, then there is no need for a fresh
\LaTeX{} run (except for the purposes of error-checking).


\index{bibliography}
\item[bibliography: ]
Suppose the document requires a bibliography, or list of references,
which is to be prepared using Bib\TeX.
\latextohtml{} reads citation information from the \texttt{.aux} file,
and can import the bibliography itself from the \texttt{.bbl} file.
However these must first be created using \LaTeX.



\index{document segmentation}
\item[document segmentation: ]
With the document segmentation technique, 
discussed fully in \hyperref{a later section}{Section~}{}{Segmentation},
it is vitally important that the full document processes correctly
in \LaTeX. The desired effect is that of a single large document,
whereas the pieces will actually be processed separately.
To achieve this, \LaTeX{} writes vital information into special
\texttt{.ptr} files. Like the \texttt{.aux} file, these files
are read by \latextohtml{} to get section-numbering and other
information to be used while processing each segment.


\index{printing!hyperlink to typeset version}%
\index{printing}\html{\\}\noindent
\item[print quality: ]
When a document contains automatically-generate images, these images
are usually bitmaps designed for viewing on-screen. In general the 
resolution of these is too poor to give a good result when printed
on a high-resolution laser-printer.
Thus if it is likely that the reader will want to obtain
a printed version of your document, then it is best to include
a hyperlink to the typeset \texttt{.dvi} version, 
or to a \PS\ conversion of the \texttt{.dvi} file. 
(In either case, a link to a compressed version is even better.)

\end{htmllist}



\subsection{Command-Line Options\label{options}}%
%\section{Command-Line Options\label{options}}%
\index{options@options\label{IIIoptions}}%
\index{switches|see{\htmlref{options}{IIIoptions}}}%
\index{options|(}%
\index{options!environment variables}%
\index{options!set in initialisation file}%
\html{\\}\noindent
The command-line options described here can be used to change the
default behaviour of \latextohtml. 
Alternatively, often there is a corresponding environment variable
whose value may be set or changed within a \fn{.latex2html-init} 
initialisation file, in order to achieve the same result.\html{\\}
There are so many options that they are listed here in groups, 
according to the nature of the effects they control.
When a large number of such options are required for the processing of a document,
it is usual to store the switches and their desired settings within a \fn{Makefile},
for use with the Unix \fn{make} utility. Now a simple command such as:
\begin{verbatim}
  make mydocument
\end{verbatim}
can initiate a call to \fn{latex2html} that would otherwise take many lines of typing.
Indeed it could instigate several such calls to \latextohtml, or to other programs
such as \LaTeX{} and Bib\TeX, \fn{dvips} and others.
The document segmentation feature, discussed in 
\hyperref{another section}{Section~}{}{Segmentation},
uses this technique extensively.


\subsubsection{Options controlling Titles, File-Names and Sectioning}
%\subsection{Options controlling Titles, File-Names and Sectioning}
\index{options!titles}\index{options!file-names}\index{options!sectioning}%
%
\index{document!title}%
%
\begin{htmllist}\htmlitemmark{PurpleBall}%
\item [ -t \Meta{top-page-title}\label{cs_toppagetitle}]
\sameas{\fn{\$TITLE}\texttt{ = \char34}\Meta{top-page-title}\texttt{\char34;}}\\
Name the document using this title.

\index{filenames!.htm@\texttt{.htm} extension}%

\item [ -short\_extn\label{cs_shortextn}]
\sameas{\fn{\$SHORTEXTN}\texttt{ = 1;}}\\
Use a filename prefix of \texttt{.htm } for the produced
\texttt{HTML} files. This is particularly useful for creating pages
to be stored on CD-ROM or other media, to be used with 
operating systems that require a 3-character extension.%

\index{filenames!long names}

\item [ -long\_titles \Meta{num}\label{cs_longtitles}]
\sameas{\fn{\$LONG\_TITLES}\texttt{ = }\Meta{num}\texttt{;}}\\
Instead of the standard  names: \texttt{node1.html}, \texttt{node2.html},... 
the filenames for each \texttt{HTML} page are constructed
from the first \Meta{num} words of the section heading
for that page, separated by the `\texttt{\_}' character.\html{\\}
Commas and common short words (\texttt{a an to by of and for the})
are omitted from both title and word-count.

\smallskip\noindent
\textbf{Warning:} Use this switch with great caution.
Currently there are no checks for uniqueness of names or overall length. 
Very long names can easily result from using this feature.%

\index{filenames!customised}%
\index{filenames!custom title hook@\texttt{custom\_title\_hook}}

\item [ -custom\_titles\label{cs_customtitles}]
\sameas{\fn{\$CUSTOM\_TITLES}\texttt{ = 1;}}\\
Instead of the standard  names: \texttt{node1.html}, \texttt{node2.html}, ... 
the filenames for each \texttt{HTML} page are constructed using a \Perl{}
subroutine named \texttt{ custom\_title\_hook}~. 
The user may define his/her own version of this subroutine, 
within a \fn{.latex2html-init} file say, 
to override the default (which uses the standard names).
This subroutine takes the section-heading as a parameter 
and must return the required name, or the empty string (default).

\index{output!redirect to directory}

\item [ -dir \Meta{output-directory}\label{cs_outputdir}]
\sameas{\fn{\$DESTDIR}\texttt{ = \char34}\Meta{output-directory}\texttt{\char34;}}\\
Redirect the output to the specified directory.\html{\\}
The default behaviour is to create (or reuse) a directory having the same
name as the prefix of the document being processed.%

\index{output!in current directory}%
\index{output!default directory}

\item [ -no\_subdir\label{cs_nosubdir}]
\sameas{\fn{\$NO\_SUBDIR}\texttt{ = 1;}}\\
Place the generated \texttt{HTML} files into the current directory.
This overrides any \fn{\$DESTDIR} setting.

\index{filenames!prefix}%
\index{temporary files}%

\item [ -prefix \Meta{filename-prefix}\label{cs_prefix}]%
\sameas{\fn{\$PREFIX}\texttt{ = \char34}\Meta{filename-prefix}\texttt{\char34;}}\\
The \Meta{filename-prefix} will be prepended to all 
\texttt{.gif}, \texttt{.pl} and \texttt{.html} files produced, 
except for the top-level \texttt{.html} file;  
it may include a (relative) directory path. 
This will enable multiple products of \latextohtml{} to
peacefully coexist in the same directory.
However, do \emph{not} attempt to \emph{simultaneously} run multiple 
instances of \latextohtml{} using the same output directory, 
else various temporary files will overwrite each other.

\index{filenames!auto-prefix}%

\item [ -auto\_prefix\label{cs_autoprefix}]
\sameas{\fn{\$AUTO\_PREFIX}\texttt{ = 1;}}\\
Constructs the prefix as `\Meta{title}--'
to be prepended to all the files produced, 
where \Meta{title} is the name of the \LaTeX{} file being processed.
(Note the `--' in this prefix.)\\
This overrides any \fn{\$PREFIX} setting.

\index{filenames!automatic link to directory name}%
\index{filenames!index.html@\fn{index.html}}%

\item [ -no\_auto\_link\label{cs_noautolink}]
\sameas{\fn{\$NO\_AUTO\_LINK}\texttt{ = 1;}}\\
If \fn{\$NO\_AUTO\_LINK} is empty and variables \fn{\$LINKPOINT} 
and \fn{\$LINKNAME} are defined appropriately
(as is the default in the \fn{latex2html.config} file),
then a hard link to the main \texttt{HTML} page is produced,
using the name supplied in \fn{\$LINKNAME}. Typically this is \fn{index.html}; 
on many systems a file of this name will be used, if it exists,
when a browser tries to view a URL which points to a directory.
On other systems a different value for \fn{\$LINKNAME} may be appropriate.
Typically \fn{\$LINKPOINT} has value \texttt{\$FILE.html}, but
this may also be changed to match whichever \texttt{HTML} page
is to become the target of the automatic link.\\
Use of the \Cs{no\_auto\_link} switch \emph{cancels} 
this automatic linking facility, 
when not required for a particular document.

\index{sections!in separate files}%

\item [ -split \Meta{num}\label{cs_splitdepth}]
\sameas{\fn{\$MAX\_SPLIT\_DEPTH}\texttt{ = }\Meta{num}\texttt{;}}
 (default is \texttt{8})\\
Stop splitting sections into separate files at this depth.
Specifying \Cs{split 0} will put the entire document into a 
single \texttt{HTML} file. See \htmlref{below}{seclevels} for
the different levels of sectioning. Also see the next item for
how to set a ``relative'' depth for splitting.%

\item [ -split +\Meta{num}\label{cs_reldepth}]
\sameas{\fn{\$MAX\_SPLIT\_DEPTH}\texttt{ = }\textrm{-}\Meta{num}\texttt{;}}
 (default is \texttt{8})\\
The level at which to stop splitting sections is calculated ``relative to'' 
the shallowest level of sectioning that occurs within the document.
For example, if the document contains \Lc{section} commands, but no \Lc{part}
or \Lc{chapter} commands, then \Cs{split +1} will cause splitting at each \Lc{section}
but not at any deeper level; whereas \Cs{split +2} or \Cs{split +3} also split down to 
\Lc{subsection} and \Lc{subsubsection} commands respectively.
Specifying \Cs{split +0} puts the entire document into a single \texttt{HTML} file.


\index{depth!child nodes revealed}%
\index{components!specify depth}

\index{child-links table}%
\item [ -link \Meta{num}\label{cs_linkdepth}]
\sameas{\fn{\$MAX\_LINK\_DEPTH}\texttt{ = }\Meta{num}\texttt{;}}
 (default is \texttt{4})\\
For each node, create links to child nodes down to this much 
deeper than the node's sectioning-level.\html{\\}
Specifying \Cs{link 0} will show no links to child nodes from that page,\html{\\}
\Cs{link 1} will show only the immediate descendents, etc.\html{\\}
A value at least as big as that of the 
\htmlref{\Cs{split \Meta{num}}}
{splitdepth} depth will
produce a mini table-of-contents (when not empty) on each page,
for the tree structure rooted at that node.%

\index{child-links table!at bottom of page}%

When the page has a sectioning-level less than the \Cs{split} depth,
so that the a mini table-of-contents has links to other \texttt{HTML} pages, 
this table is located at the bottom of the page, 
unless placed elsewhere using the \Lc{tableofchildlinks} command.%

\index{child-links table!at top of page}%
\index{child-links table!tableofchildlinks@\texttt{tableofchildlinks}}

On pages having a sectioning-level just less than the \Cs{split} depth 
the mini table-of-contents contains links to subsections etc.
occurring on the \emph{same} \texttt{HTML} page. 
Now the table is located at the \emph{top} of this page, 
unless placed elsewhere using the \Lc{tableofchildlinks} command.

\index{table of contents!depth}%

\item [ -toc\_depth  \Meta{num}\label{cs_tocdepth}]
\sameas{\fn{\$TOC\_DEPTH}\texttt{ = }\Meta{num}\texttt{;}}
 (default is \texttt{4})\\
Sectioning levels down to \Meta{num} are to be included 
within the Table-of-Contents tree.%

\index{table of contents!star-sections shown}%
\index{table of contents!star-sections not listed}

\item [ -toc\_stars\label{cs_tocstars}]
\sameas{\fn{\$TOC\_STARS}\texttt{ = 1;}}\\
Sections created using the starred-form of sectioning commands
are included within the Table-of-Contents.
As with \LaTeX, normally such sections are \emph{not} listed.
%

\index{sections!show section numbers}%
\index{sections!numbers not shown}

\item [ -show\_section\_numbers\label{cs_showsecnums}]
\sameas{\fn{\$SHOW\_SECTION\_NUMBERS}\texttt{ = 1;}}\\
Show section numbers. By default section numbers are \emph{not} shown, 
so as to encourage the use of particular sections as stand-alone documents. 
\strikeout{In order to be shown, 
section titles must be unique and must not contain inlined graphics.}

\index{segmentation!unsegment}

\item [ -unsegment\label{cs_unsegment}]
\sameas{\fn{\$UNSEGMENT}\texttt{ = 1;}}\\
Treat a segmented document (see the section about \htmlref{document
  segmentation}{Segmentation}) like it were not segmented.
This will cause the translator to concatenate all segments and process
them as a whole.
You might find this useful to check a segmented document for
consistency.
%
\end{htmllist}

\goodbreak
\index{sectioning levels}\index{levels!sectioning}\label{seclevels}%
\html{\\}\noindent
For all documents the sectioning levels referred to above are:\nobreak
\begin{center}
\begin{tabular}{ll}
\textbf{0} & document\\
\textbf{1} & part\\
\textbf{2} & chapter\\
\textbf{3} & section\\
\textbf{4} & subsection\\
\textbf{5} & subsubsection\\
\textbf{6} & paragraph\\
\textbf{7} & subparagraph\\
\textbf{8} & subsubparagraph
\end{tabular}
\end{center}
These levels apply even when the document contains no sectioning for
the shallower levels; e.g. no \Lc{part} or \Lc{chapter} commands is most common,
especially when using \LaTeX's \env{article} document-class.



\subsubsection{Options controlling Extensions and Special Features}
\index{extensions|(}%
\index{options!extensions}\index{options!special features}%
\html{\\}\noindent
The switches described here govern the type of \texttt{HTML} code that
can be generated, and how to choose between the available options 
when there are alternative strategies for implementing portions of \LaTeX{} code. 

\begin{htmllist}\htmlitemmark{PurpleBall}%
%
\index{HTML version@\texttt{HTML} version}%
\index{HTML@\texttt{HTML}!non-standard extensions}%
\item [ -html\_version \texttt{(2.0|3.0|3.2)[,(math|i18n|table)]*}\label{cs_htmlversion}]
~\\\sameas{\fn{\$HTML\_VERSION}\texttt{ = \char34 ... \char34;}}\\
This specifies both the \texttt{HTML}~\htmlref{version}{version} to generate, 
and any extra (non-standard) \texttt{HTML} features that may be required.\html{\\}
The version number corresponds to a published DTD for an \texttt{HTML} standard
(although 3.0 was never accepted and subsequently withdrawn).
A corresponding \Perl{} file in the \fn{versions/} subdirectory is loaded;
these files are named `\texttt{html}\Meta{num}\texttt{.pl}'.

Following the version number, a comma-separated list of extensions
can be given. Each corresponds to a file `\Meta{name}\texttt{.pl}' 
also located in the \fn{versions/} subdirectory. 
When such a file is loaded the resulting \texttt{HTML} code 
can no longer be expected to validate with the specified DTD. 
An exception is \Ve{math} when the \htmlref{\Cs{no\_math}}{nomath} 
switch is also used, which should still validate.

Currently, versions 2.0, 3.2 and 4.0 are available.
(and also 2.1, 2.2, 3.0 and 3.1, for hoistorical reasons). 
The extensions \Ve{i18n}, \Ve{tables}, \Ve{math}
correspond roughly to what used to be called versions `2.1', `2.2', `3.1'
respectively, in releases of \latextohtml{} up to 1996.
Now these extensions can be loaded with any of `2.0', `3.2' or `4.0' 
as the specified standard.\\  
The default version is usually set to be `3.2', within \fn{latex2html.config}\,.%

\index{extensions!disabled}%
\index{extensions!TeX defs@\TeX definitions}%

\item [ -no\_tex\_defs\label{cs_notexdefs}]
\sameas{\fn{\$TEXDEFS}\texttt{ = 0;}} (default is 1)\\
When \fn{\$TEXDEFS} is set (default) the file \fn{texdefs.perl} will be read.
This provides code to allow common \TeX{} commands like \Tc{def}, \Tc{newbox},
\Tc{newdimen} and others, to be recognised, especially within the document preamble.
In the case of \Tc{def}, the definition may even be fully interpreted,
but this requires the pattern-matching to be not too complicated.

If \fn{\$TEXDEFS} is `\texttt{0}' or empty, then \fn{texdefs.perl} will not be loaded;
the translator will make no attempt to interpret any raw \TeX{} commands.  
This feature is intended to enable sophisticated authors the ability to insert 
arbitrary \TeX{} commands in environments that are destined 
to be processed by \LaTeX{} anyway; 
e.g. \env{figure}s, \strikeout{\env{theorem}s}, \env{picture}s, etc.
However this should rarely be needed, as now there is better support for these
types of environment. There are now other methods to specify which chunks of code 
are to be passed to \LaTeX{} for explicit image-generation; 
see the discussion of the \env{makeimage} 
\hyperref[page]{environment}{environment on page~}{}{makeimage}.


\index{auxiliary file!which to read}%

\item [ -external\_file \Meta{filename}\label{cs_externalfile}] 
\sameas{\fn{\$EXTERNAL\_FILE}\texttt{ = \char34}\Meta{filename}\texttt{\char34;}}\\
Specifies the prefix of the \texttt{.aux} file that this document should read.
The \texttt{.aux} extension will be appended to this prefix 
to get the complete filename, with directory path if needed.\html{\\}  
This file could contain necessary information 
regarding citations, figure, table and section numbers from \LaTeX{} 
and perhaps other information also.
Use of this switch is vital for \htmlref{document segments}{Segmentation}, 
processed separately and linked to appear as if generated 
from a single \LaTeX{} document.%

\index{font size!for image generation}%
\index{font-size!in environments}\index{font-size!magnification}%

\item [ -font\_size \Meta{size}\label{cs_fontsize}]
\sameas{\fn{\$FONT\_SIZE}\texttt{ = \char34}\Meta{size}\texttt{\char34;}}\\
This option provides better control 
over the font size of environments made into images using \LaTeX.  
\Meta{size} must be one of the font sizes that \LaTeX{} recognizes; 
i.e. `\texttt{10pt}', `\texttt{11pt}', `\texttt{12pt}', etc.  
Default is `\texttt{10pt}', or whatever option may have been specified on the 
\Lc{documentclass} or \Lc{documentstyle} line.\html{\\}
Whatever size is selected, it will be magnified by the installation variables
\fn{\$MATH\_SCALE\_FACTOR}, \fn{\$FIGURE\_SCALE\_FACTOR}
and \fn{\$DISP\_SCALE\_FACTOR} as appropriate.

\smallskip\noindent
\textbf{Note: }This switch provides no control over the size of text
on the \texttt{HTML} pages. 
Such control is subject entirely to the user's choices
of settings for the browser windows.

\index{font size!scalable}%

\item [ -scalable\_fonts\label{cs_scalefonts}]
\sameas{\fn{\$SCALABLE\_FONTS}\texttt{ = 1;}}\\
This is used when scalable fonts, such as \PS\ versions
of the \TeX{} fonts, are available for image-generation.\html{\\}
It has the effect of setting \fn{\$PK\_GENERATION} to `\texttt{1}',
and \fn{\$DVIPS\_MODE} to be empty, 
overriding any \htmlref{previous settings}{pkgen} for these variables.

\index{simple math!cancelled}%

\item [ -no\_math\label{cs_nomath}]
\sameas{\fn{\$NO\_SIMPLE\_MATH}\texttt{ = 1;}}\\
Ordinarily simple mathematical expressions are set using the ordinary text font, 
but italiced. When part of the expression can not be represented this way,
an image is made \emph{of the whole formula}. 
This is called ``simple math''.
When \fn{\$NO\_SIMPLE\_MATH} is set, 
then \emph{all} mathematics is made into images, whether simple or not.

However, if the \Ve{math} extension is loaded, 
using the \Cs{html\_version} switch described \htmlref{earlier}{cs_htmlversion},
then specifying \Cs{no\_math} produces a quite different effect.
Now it is the special \HTMLtag{MATH} tags and entities which are cancelled. 
In their place a sophisticated scheme for parsing mathematical expressions 
is used. Images are made of those sub-parts of a formula which cannot be 
adequately expressed using (italiced) text characters and \HTMLtag{SUB} and
\HTMLtag{SUP} tags.
See \hyperref{the subsection on mathematics}{Section~}{}{maths} for more details.

\index{icons!copied to local directory}%

\item [-local\_icons\label{cs_localicons}]
\sameas{\fn{\$LOCAL\_ICONS}\texttt{ = 1;}}\\
A copy of each of the icons actually used within the document is placed in the
directory along with the \texttt{HTML} files and generated images.
This allows the whole document to be fully self-contained, within this directory;
otherwise the icons must be retrieved from a (perhaps remote) server.

\index{icons!alternative set}\index{icons!customised images}%
The icons are normally copied from a subdirectory of the \fn{\$LATEX2HTMLDIR},
set within \fn{latex2html.config}. An alternative set of icons can be used
by specifying a (relative) directory path in \fn{\$ALTERNATIVE\_ICONS}
to where the customised images can be found.


\index{initialisation!init file@\texttt{-init\_file}}%
\index{initialisation file!specified}%

\item [ -init\_file \Meta{file}\label{cs_initfile}]
Load the specified initialisation file. This \Perl{} file will be loaded after loading 
\fn{\$HOME/.latex2html-init}\,, or \fn{.latex2html-init} in the local
directory, if either file exists. It is read at the time the switch is processed,
so the contents of the file may change any of the values of any of the
variables that were previously established, as well as any default options.
More than one initialisation file can be read in this way.%

\index{index!codified links}%

\cbversion{98.1}\begin{changebar}
\index{forking!prevent}%
\item [ -no\_fork\label{cs_no_fork}]
\sameas{\fn{\$NOFORK}\texttt{ = 1;}}\\
When set this disables a feature in the early part of the processing
whereby some memory-intensive operations are performed by `forked'
child processes. Some single-task operating systems, such as \texttt{DOS},
do not support this feature. Having \fn{\$NOFORK} set then ensures
that unnecessary file-handles that are needed with the forked processes,
are not consumed unnecessarily, perhaps resulting in a fatal \Perl{} error.

\index{languages!dtd isolanguage}
\index{character set!dtd isolanguage}
\item [ -iso\_language \Meta{type}\label{cs_iso_language}]
This enables you to specify a different language type than
\texttt{'EN'} to be used in the DTD entries of the \texttt{HTML}
document, e.g. \texttt{'EN.US'}.
\end{changebar}%

\item [ -short\_index\label{cs_shortindex}]
\sameas{\fn{\$SHORT\_INDEX}\texttt{ = 1;}}\\
Creates shorter Index listings, using codified links; 
this is fully compatible with the \env{makeidx} \htmlref{package}{makeidx}.

\index{footnotes!on separate page}%
\index{footnotes!on same page}%
\index{footnotes!marker style}

\item [ -no\_footnode\label{cs_nofootnode}]
\sameas{\fn{\$NO\_FOOTNODE}\texttt{ = 1;}}\\
Suppresses use of a separate file for footnotes;
instead these are placed at the bottom of the \texttt{HTML} pages 
where the references occur.

When this option is used, it is frequently desirable to change the 
style of the marker used to indicate the presence of a footnote. 
This is done as in \LaTeX, using code such as follows.
\begin{quote}
\verb|\renewcommand{\thefootnote}{\arabic{footnote}}|
\end{quote}
All the styles \Lc{arabic}, \Lc{alph}, \Lc{roman}, \Lc{Alph} and \Lc{Roman}
are available.

\index{footnotes!numbering}
\cbversion{98.1}\begin{changebar}
\item [ -numbered\_footnotes\label{cs_numbered_footnotes}]
\sameas{\fn{\$NUMBERED\_FOOTNOTES}\texttt{ = 1;}}\\
If this is set you will get every footnote applied with a subsequent
number, to ease readability.
\end{changebar}

\index{address!signature}%
\index{address!using a subroutine}%

\item [ -address \Meta{author-address}\label{cs_address}]
\sameas{\fn{\$ADDRESS}\texttt{ = \char34}\Meta{author-address}\texttt{\char34;}}\\
Sign each page with this address.\html{\\}
See \fn{latex2html.config} for an example using \Perl{} code
to automatically include the date.

A user-defined \Perl{} subroutine called \texttt{\&custom\_address} 
can be used instead, if defined; it takes the value of \fn{\$ADDRESS} 
as a parameter, which may be used or ignored as desired.
At the time when this subroutine will be called, 
variables named \fn{\$depth}, \fn{\$title}, \fn{\$file}
hold the sectioning-level, title and filename of the \texttt{HTML} page
being produced; \fn{\$FILE} holds the name of the filename for the title-page
of the whole document.

\index{About this@\emph{About this document...}}

\item [ -info \Meta{string}\label{cs_infostring}]
\sameas{\fn{\$INFO}\texttt{ = \char34}\Meta{string}\texttt{\char34;}}\\
Generate a new section \emph{``About this document'' } containing information
about the document being translated. The default is to generate such a section 
with information on the original document, the date, the user and the translator.
An empty string (or the value `\texttt{0}') 
disables the creation of this extra section.\html{\\}
If a non-empty string is given, it will be placed as the contents of the 
\emph{``About this document'' } page instead of the default information.
%
\end{htmllist}


\subsubsection{Switches controlling Image Generation}
%\subsection[center]{Switches controlling Image Generation}
%
These switches affect whether images are created at all, whether
old images are reused on subsequent runs or new ones created afresh,
and whether anti-aliasing effects are used within the images themselves.

\begin{htmllist}\htmlitemmark{PinkBall}%
%
\index{ascii-mode}% 
\index{browser!character-based}%
\item [ -ascii\_mode \label{cs_asciimode}]
\sameas{\fn{\$ASCII\_MODE}\texttt{ = }\fn{\$EXTERNAL\_IMAGES}\texttt{ = 1;}}\\
Use only \texttt{ASCII} characters and do not include any images in the final output.
With \Cs{ascii\_mode} the output of the translator can be used on 
character-based browsers, such as \appl{lynx},
which do not support inlined images (via the \HTMLtag{IMG} tag).


\index{draft mode}%
\item [ -nolatex \label{cs_nolatex}]
\sameas{\fn{\$NOLATEX}\texttt{ = 1;}}\\
Disable the mechanism for passing unknown environments to \LaTeX{} for processing.
This can be thought of as ``draft mode'' which allows 
faster translation of the basic document structure and text, 
without fancy figures, equations or tables. 

(This option has been superseded by the \Cs{no\_images} option,
\htmlref{see below}{noimages}.)


\index{images!via hypertext links}%
\item [ -external\_images\label{cs_extimages}]
\sameas{\fn{\$EXTERNAL\_IMAGES}\texttt{ = 1;}}\\
Instead of including any generated images inside the document, 
leave them outside the document and provide hypertext links to them.


\index{images!links to PostScript@links to \PS}%
\item [ -ps\_images\label{cs_psimages}]
\sameas{\fn{\$PS\_IMAGES}\texttt{ = }\fn{\$EXTERNAL\_IMAGES}\texttt{ = 1;}}\\
Use links to external \PS\ files rather than inlined images
in the chosen graphics format.

\index{images!discard PostScript files@discard \PS\ files}%
\item [ -discard\label{cs_discard}]
\sameas{\fn{\$DISCARD\_PS}\texttt{ = 1;}}\\
The temporary \PS\ files are discarded immediately after they
have been used to create the image in the desired graphics format.

\index{images!disabled}%
\item [ -no\_images\label{cs_noimages}]
\sameas{\fn{\$NO\_IMAGES}\texttt{ = 1;}}\\
Do not attempt to produce any inlined images. 
The missing images can be generated ``off-line'' by restarting \latextohtml{}
with the option \Cs{images\_only}.

\index{images!generated off-line}%
\item [ -images\_only\label{cs_imagesonly}]
\sameas{\fn{\$IMAGES\_ONLY}\texttt{ = 1;}}\\
Try to convert any inlined images that were left over from previous
runs of \latextohtml. 


\index{image files!reuse option}%
\item [ -reuse \Meta{reuse\_option}\label{cs_reuseoptions}]
\sameas{\fn{\$REUSE}\texttt{ = }\Meta{reuse\_option}\texttt{;}}\\
This switch specifies the extent to which image files are to be shared
or recycled.\html{\\}
There are three valid options:
%
\begin{htmllist}\htmlitemmark{OrangeBall}
\index{image files!not recycled}\index{image-reuse!interactive}%
\item [\texttt{0}] 
Do not ever share or recycle image files.\\
This choice also invokes an interactive session 
prompting the user about what to do about
a pre-existing \texttt{HTML} directory, if it exists.
%
\index{image files!recycled}%
\item [\texttt{1}] 
Recycle image files from a previous run if they are available,\\
but do not share identical images that must be created in this run.
%
\index{image files!shared by default}%
\item [\texttt{2}] 
Recycle image files from a previous run and share identical
images from this run.\\
This is the default.
\end{htmllist}
\hyperref{A later section}{Section~}{}{recycling} provides 
additional information about image-reuse.


\index{image files!not reused}\index{image-reuse!interactive}%
\item [ -no\_reuse\label{cs_noreuse}]
\sameas{\fn{\$REUSE}\texttt{ = 0;}}\\
Do \emph{not} share or recycle images generated during previous translations.
This is equivalent to \Cs{reuse 0}.
(This will enable the initial interactive session during which the user is
asked whether to reuse the old directory, delete its contents or quit.)%


\index{images!anti-aliasing}\index{anti-aliasing!in figures}%
\item [ -antialias\label{cs_aalias}]
\sameas{\fn{\$ANTI\_ALIAS}\texttt{ = 1;}} (Default is 0.)\\
Generated images of \env{figure} environments and external \PS\ files
should use anti-aliasing. By default anti-aliasing is not
used with these images, since this may interfere with the contents
of the images themselves.

\index{images!anti-aliasing}\index{anti-aliasing!in math and text}%
\item [ -antialias\_text\label{cs_aaliastext}]
\sameas{\fn{\$ANTI\_ALIAS\_TEXT}\texttt{ = 1;}} (Default is 1.)\\
Generated images of typeset material such as text, mathematical formulas,
tables and the content of \env{makeimage} \htmlref{environments}{makeimage},
should use anti-aliasing effects.\html{\\}
The default is normally to use anti-aliasing for text, 
since the resulting images are much clearer on-screen. 
However the default may have been changed locally.


\index{images!no anti-aliasing}\index{anti-aliasing!turned off}%
\item [ -no\_antialias\label{cs_noaalias}]
\sameas{\fn{\$ANTI\_ALIAS}\texttt{ = 0;}} (Default is 0.)\\
Generated images of \env{figure} environments and external \PS\ files 
should not use anti-aliasing with images, 
though the local default may have been changed to use it.


\index{images!no anti-aliasing}\index{anti-aliasing!blurred printing}%
\item [ -no\_antialias\_text\label{cs_noaaliastext}]
\sameas{\fn{\$ANTI\_ALIAS\_TEXT}\texttt{ = 0;}} (Default is 1.)\\
Generated images of typeset material should \emph{not} use anti-aliasing effects.
Although on-screen images of text are definitely improved using anti-aliasing,
printed images can be badly blurred, even at 300dpi. 
Higher resolution printers do a much better job with the resulting grey-scale images.

\cbversion{98.1}\begin{changebar}
\index{images!white background}%
\item [ -white\label{cs_white}]
\sameas{\fn{\$WHITE\_BACKGROUND}\texttt{ = 1;}} (Default is 1.)\\
Ensures that images of \env{figure} environments have a white
background. Otherwise transparency effects may not work correctly.

\item [ -no\_white\label{cs_nowhite}]
\sameas{\fn{\$WHITE\_BACKGROUND}\texttt{ = '';}} (Default is 1.)\\
Cancels the requirement that \env{figure} environments have a white background.

\index{images!latex dump}%
\item [ -ldump\label{cs_ldump}]
\sameas{\fn{\$LATEX\_DUMP}\texttt{ = 1;}} (Default is 0.)\\
Use this if you want to speed up image processing during the 2nd
and subsequent runs of \latextohtml{} on the same document.
The translator now produces a \LaTeX{} format-dump of the preamble to
\fn{images.tex} which is used on subsequent runs.
This significantly reduces the startup time when
\LaTeX{} reads the \fn{images.tex} file for image-generation.

This process actually consumes additional time on the first run, 
since \LaTeX{} is called twice --- once to create the format-dump,
then again to load and use it.
The pay-off comes with the faster loading on subsequent runs. 
Approximately 1 Meg of disk space is consumed by the dump file.
\end{changebar}
%
\end{htmllist}


\subsubsection{Switches controlling Navigation Panels\label{navoptions}}
%\subsection[center]{Switches controlling Navigation Panels\label{navoptions}}
%
The following switches govern whether to include one or more navigation
panels on each \texttt{HTML} page, 
also which buttons to include within such a panel.

\begin{htmllist}\htmlitemmark{PinkBall}%

\index{navigation panel!disabled}
\item [ -no\_navigation\label{cs_nonavig}]
\sameas{\fn{\$NO\_NAVIGATION}\texttt{ = 1;}}\\
Disable the mechanism for putting navigation links in each page.\\
This overrides any settings of the \fn{\$TOP\_NAVIGATION},
\fn{\$BOTTOM\_NAVIGATION} and \fn{\$AUTO\_NAVIGATION} variables.

\index{navigation panel!at top of page}
\item [ -top\_navigation\label{cs_topnavig}]
\sameas{\fn{\$TOP\_NAVIGATION}\texttt{ = 1;}}\\
Put navigation links at the top of each page.    

\index{navigation panel!at bottom of page}%
\item [ -bottom\_navigation\label{cs_botnavig}]
\sameas{\fn{\$BOTTOM\_NAVIGATION}\texttt{ = 1;}}\\
Put navigation links at the bottom of each page as well as the top.

\index{navigation panel!according to number of words}
\item [ -auto\_navigation\label{cs_autonavig}]
\sameas{\fn{\$AUTO\_NAVIGATION}\texttt{ = 1;}}\\
Put navigation links at the top of each page. 
Also put one at the bottom of the page, if the page exceeds 
\fn{\$WORDS\_IN\_PAGE} number of words (default = 450).

\index{navigation panel!next logical page}
\item [ -next\_page\_in\_navigation\label{cs_nextinnavig}]
\sameas{\fn{\$NEXT\_PAGE\_IN\_NAVIGATION}\texttt{ = 1;}}\\
Put a link to the \textsl{next logical page} in the navigation panel.

\index{navigation panel!previous logical page}%
\item [ -previous\_page\_in\_navigation\label{cs_previnnavig}]~\\%
\sameas{\fn{\$PREVIOUS\_PAGE\_IN\_NAVIGATION}\texttt{ = 1;}}\\
Put a link to the \textsl{previous logical page} in the navigation panel.

\index{navigation panel!link to table-of-contents}
\item [ -contents\_in\_navigation\label{cs_continnavig}]
\sameas{\fn{\$CONTENTS\_IN\_NAVIGATION}\texttt{ = 1;}}\\
Put a link to the \textsl{table-of-contents} in the navigation panel if there is one.

\index{navigation panel!link to index}
\item [ -index\_in\_navigation\label{cs_indexinnavig}]
\sameas{\fn{\$INDEX\_IN\_NAVIGATION}\texttt{ = 1;}}\\
Put a link to the \textsl{index-page} in the navigation panel if there is an index.
%
\end{htmllist}


\subsubsection{Switches for Linking to other documents\label{otherdocs}}
%\subsubsection[center]{Switches for Linking to other documents\label{otherdocs}}
%
When processing a single stand-alone document, the switches described in this
section should not be needed at all, since the automatically generated navigation panels, 
described \hyperref{on the previous page}{in Section~}{, }{navoptions}
should generate all the required navigation links.
However if a document is to be regarded as part of a much larger document,
then links from its first and final pages, to locations in other parts
of the larger (virtual) document, 
need to be provided explicitly for some of the buttons in the navigation panel.

The following switches allow for such links to other documents,
by providing the title and URL for navigation panel hyperlinks.
In particular, the ``Document Segmentation''
\hyperref{feature}{feature of Section~}{}{Segmentation} 
necessarily makes great use of these switches.
It is usual for the text and targets of these navigation hyperlinks
to be recorded in a \fn{Makefile}, 
to avoid tedious typing of long command-lines having many switches.

\begin{htmllist}\htmlitemmark{PinkBall}%
%
\index{URL!universal resource locator}\index{URL!for UP@for \textsl{UP} button}%
\item [ -up\_url \Meta{URL}\label{cs_upURL}] 
\sameas{\fn{\$EXTERNAL\_UP\_LINK}\texttt{ = \char34}\Meta{URL}\texttt{\char34;}}\\
Specifies a universal resource locator (URL) to associate
with the \textsl{``UP''} button in the navigation panel(s).


\item [ -up\_title \Meta{string}\label{cs_uptitle}] 
\sameas{\fn{\$EXTERNAL\_UP\_TITLE}\texttt{ = \char34}\Meta{string}\texttt{\char34;}}\\
Specifies a title associated with this URL.%


\index{URL!for PREVIOUS@for \textsl{PREVIOUS} button}%
\item [ -prev\_url \Meta{URL}\label{cs_prevURL}] 
\sameas{\fn{\$EXTERNAL\_PREV\_LINK}\texttt{ = \char34}\Meta{URL}\texttt{\char34;}}\\
Specifies a URL to associate
with the \textsl{``PREVIOUS''} button in the navigation panel(s).%


\item [ -prev\_title \Meta{string}\label{cs_prevtitle}] 
\sameas{\fn{\$EXTERNAL\_PREV\_TITLE}\texttt{ = \char34}\Meta{string}\texttt{\char34;}}\\
Specifies a title associated with this URL.%


\index{URL!for NEXT@for \textsl{NEXT} button}%
\item [ -down\_url \Meta{URL}\label{cs_downURL}]
\sameas{\fn{\$EXTERNAL\_DOWN\_LINK}\texttt{ = \char34}\Meta{URL}\texttt{\char34;}}\\
Specifies a URL for the \textsl{``NEXT''} button in the navigation panel(s).%


\item [ -down\_title \Meta{string}\label{cs_downtitle}] 
\sameas{\fn{\$EXTERNAL\_DOWN\_TITLE}\texttt{ = \char34}\Meta{string}\texttt{\char34;}}\\
Specifies a title associated with this URL.%


\index{URL!for CONTENTS@for \textsl{CONTENTS} button}%
\item [ -contents \Meta{URL}\label{cs_contentsURL}] 
\sameas{\fn{\$EXTERNAL\_CONTENTS}\texttt{ = \char34}\Meta{URL}\texttt{\char34;}}\\
Specifies a URL for the \textsl{``CONTENTS''} button,
for document segments that would not otherwise have one.%


\index{URL!for INDEX@for \textsl{INDEX} button}%
\item [ -index \Meta{URL}\label{cs_indexURL}] 
\sameas{\fn{\$EXTERNAL\_INDEX}\texttt{ = \char34}\Meta{URL}\texttt{\char34;}}\\
Specifies a URL for the \textsl{``INDEX''} button, 
for document segments that otherwise would not have an index.

\index{URL!for external Bibliography}%
\item [ -biblio \Meta{URL}\label{cs_biblioURL}] 
\sameas{\fn{\$EXTERNAL\_BIBLIO}\texttt{ = \char34}\Meta{URL}\texttt{\char34;}}\\
Specifies the URL for the bibliography page to be used,
when not explicitly part of the document itself.
%
\end{htmllist}
\smallskip
\textbf{Warning: }
On some systems it is difficult to give text-strings \Meta{string} containing
space characters, on the command-line or via a \fn{Makefile}.
One way to overcome this is to use the corresponding variable.
Another way is to replace the spaces with underscores (\verb|_|).


\subsubsection{Switches for Help and Tracing}
%\subsubsection[center]{Switches for Help and Tracing}
%
The first two of the following switches are self-explanatory.
When problems arise in processing a document, 
the switches \Cs{debug} and \Cs{verbosity} will each
cause \latextohtml{} to generate more output to the screen.
These extra messages should help to locate the cause of the problem.

\begin{htmllist}\htmlitemmark{PinkBall}%
%
\index{options!defining temp directory}%
\item [ -tmp \Meta{path}\label{cs_tmpdir}]
Define a temporary directory to use for image generation.
If \Meta{path} is $0$, the standard temporary directory \texttt{/tmp}
is used.
%
\index{options!print option listing}%
\item [ -h(elp)\label{cs_optionlist}]
Print out the list of all command-line options.

\index{current version@current version of \latextohtml}%
\item [ -v\label{cs_showversion}] 
Print the current version of \latextohtml.

\index{debugging}\index{diagnostic information}%
\item [ -debug\label{cs_debugmode}]
\sameas{\fn{\$DEBUG}\texttt{ = 1;}}\\
Run in debug-mode, displaying messages and/or diagnostic information 
about files read, and utilities called by \latextohtml.
Shows any messages produced by these calls.%

\index{debugging!Perl@\Perl{} debugger}%

More extensive diagnostics, from the \Perl{} debugger, 
can be obtained by appending the string `\texttt{-w-}' 
to the 1st line of the \fn{latex2html} (and other) \Perl{} script(s).


\index{verbosity level}%
\item [ -verbosity \Meta{num}\label{cs_verbositylevel}] 
\sameas{\fn{\$VERBOSITY}\texttt{ = \Meta{num};}}\\
Display messages revealing certain aspects of the processing
performed by \latextohtml{} on the provided input file(s).
The \Meta{num} parameter can be an integer in the range 0 to 8.
Each higher value adds to the messages produced.

\begin{itemize}
\item[0. ]
No special tracing; as for versions of \latextohtml{} prior to \textsc{v97.1}\,.

\item[1. ] (This is the default.)
Show section-headings and the corresponding \texttt{HTML} file names,
and indicators that major stages in the processing have been completed.

\item[2. ]
Print environment names and identifier numbers, and new theorem-types.
Show warnings as they occur, and indicators for more stages of processing.
Print names of files for storing auxiliary data arrays.

\item[3. ]
Print command names as they are encountered and processed; also
any unknown commands encountered while pre-processing.
Show names of new commands, environments, theorems, counters and
counter-dependencies, for each document partition.

\item[4. ]
Indicate command-substitution the pre-process of math-environments.
Print the contents of unknown environments for processing in \LaTeX,
both before and after reverting to \LaTeX{} source.
Show all operations affecting the values of counters.
Also show links, labels and sectioning keys, at the stages of processing.

\item[5. ]
Detail the processing in the document preamble.
Show substitutions of new environments.
Show the contents of all recognised environments, both before
and after processing. Show the cached/encoded information for the image keys,
allowing two images to be tested for equality.

\item[6. ]
Show replacements of new commands, accents and wrapped commands.

\item[7. ]
Trace the processing of commands in math mode; both before and after.

\item[8. ]
Trace the processing of all commands, both before and after.

\end{itemize}
%
The command-line option sets an initial value only.
During processing the value of \fn{\$VERBOSITY} can be set
dynamically using the \verb|\htmltracing{...}| command, 
whose argument is the desired value, 
or by using the more general \Lc{HTMLset} command
as follows: \verb|\HTMLset{VERBOSITY}{|\Meta{num}\texttt{\char125}\,.
%
\end{htmllist}
\index{options|)}%
%

\subsubsection{Other Configuration Variables, without switches}
%\subsection[center]{Other Configuration Variables, without switches}
%
The configuration variables described here do not warrant having
a command-line switch to assign values.
Either they represent aspects of \latextohtml\ that are specific to the local site,
or they govern properties that should apply to \emph{all} documents,
rather than something that typically would change
for the different documents within a particular sub-directory.

Normally these variables have their value set within the \fn{latex2html.config} file.
In the following listing the defaults are shown, as the lines of \Perl{} code
used to establish these values.
If a different value is required, then these can be assigned from
a local \fn{.latex2html-init} initialisation file, 
without affecting the defaults for other users, 
or documents processed from other directories.

\begin{htmllist}\htmlitemmark{PurpleBall}%
%
\index{directory delimiter}%
\item [\fn{\$dd} ]
~holds the string to be used in file-names to delimit directories;
it is set internally to `\texttt{/}', unless the variable has already been given
a value within \fn{latex2html.config}~.

\noindent
\textbf{Note:} This value \emph{cannot} be set within a \fn{.latex2html-init} 
initialisation file, since its value \emph{needs to be known} 
in order to find such a file.

\index{installation variables!inserted@inserted by \fn{install-test}}%
\item [\fn{\$LATEX2HTMLDIR}] \latex{ }%
Read by the \fn{install-test} script from \fn{latex2html.config}, 
its value is inserted into the \fn{latex2html} \Perl{} script as part
of the installation process.

\item [\fn{\$LATEX2HTMLSTYLES}\texttt{ = \char34\$LATEX2HTMLDIR/styles\char34;}]
Read from the \fn{latex2html.config} file by \fn{install-test}\,,
its value is checked to locate the \fn{styles/} directory.

\item [\fn{\$LATEX2HTMLVERSIONS}\texttt{ = \char34\$LATEX2HTMLDIR/versions\char34;}]
The value of this variable should be set within \fn{latex2html.config} 
to specify the directory path where the version and extension files can be found.

\item [\fn{\$ALTERNATIVE\_ICONS}\texttt{ = '';}\label{alticons}]
This may contain the (relative) directory path to a set of customised icons
to be used in conjunction with the \htmlref{\Cs{local\_icons}}{cs_localicons} switch.

\index{installation variables!checked@checked by \fn{install-test}}%
\item [\fn{\$TEXEXPAND}\texttt{  = \char34\$LATEX2HTMLDIR/texexpand\char34;}]
Read by the \fn{install-test} \Perl{} script from \fn{latex2html.config}, 
its value is used to locate the \fn{texexpand} \Perl{} script.

\item [\fn{\$PSTOIMG}\texttt{ = \char34\$LATEX2HTMLDIR/pstoimg\char34;}]
Read by the \fn{install-test} \Perl{} script from \fn{latex2html.config}, 
its value is used to locate the \fn{pstoimg} \Perl{} script.

\item [\fn{\$IMAGE\_TYPE}\texttt{ = '\Meta{image-type}';}]
Set in \fn{latex2html.config}, the currently supported \Meta{image-type}s
are: \texttt{gif} and \texttt{png}.

\item [\fn{\$DVIPS}\texttt{ = 'dvips';}]
Read from \fn{latex2html.config} by \fn{install-test}, 
its value is checked to locate the \fn{dvips} program or script.

\smallskip\noindent
There could be several reasons to change the value here:
\begin{itemize}
\item
add a switch \texttt{-P}\Meta{printer} to load a specific configuration-file;\\
e.g. to use a specific set of \PS\ fonts, for improved image-generation.
\item
to prepend a path to a different version of \fn{dvips} than normally
available as the system default (e.g. the printing requirements are different).
\item
to append debugging switches, in case of poor quality images;\\
one can see which paths are being searched for fonts and other resources.
\item
to prepend commands for setting path variables that \fn{dvips} may need
in order to locate fonts or other resources.
\end{itemize}

If automatic generation of fonts is required, using \MF, the following
configuration variables are important.
%
\begin{htmllist}\htmlitemmark{OrangeBall}%
%
\item [\fn{\$PK\_GENERATION}\texttt{ = 1;}\label{pkgen}]
This variable must be set, to initiate font-generation; otherwise fonts
will be scaled from existing resources on the local system.\\
In particular this variable must \emph{not} be set, if one wishes to use
\PS\ fonts or other scalable font resources 
(see the \Cs{scalable\_fonts} \htmlref{switch}{scalefonts}).

\item [\fn{\$DVIPS\_MODE}\texttt{ = 'toshiba';}]
The mode given here must be available in the \fn{modes.mf} file,
located with the \MF\ resource files, perhaps in the \fn{misc/} subdirectory.

\item [\fn{\$METAFONT\_DPI}\texttt{ = 180;}]
The required resolution, in dots-per-inch, should be listed specifically
within the \fn{MakeTeXPK} script, called by \fn{dvips} to invoke \MF\ with
the correct parameters for the required fonts.
%
\end{htmllist}


\item [\fn{\$LATEX}\texttt{ = 'latex';}]
Read from \fn{latex2html.config} by \fn{install-test}, 
its value is checked to locate the \fn{latex} program or script.

\smallskip\noindent
If \LaTeX{} is having trouble finding style-files and/or packages, then
the default command can be prepended with other commands 
to set environment variables intended to resolve these difficulties;\\
e.g. 
\verb|$LATEX = 'setenv TEXINPUTS |\Meta{path to search}\verb| ; latex' |.

There are several variables to help control exactly which files are
read by \latextohtml{} and by \LaTeX{} when processing images:

\begin{htmllist}\htmlitemmark{OrangeBall}%
\item [\fn{\$TEXINPUTS}]
This is normally set from the environment variable of the same name.
If difficulties occur so that styles and packages are not being found, 
then extra paths can be specified here, to resolve these difficulties.

\item [\fn{\$DONT\_INCLUDE}]
This provides a list of filenames and extensions to \emph{not} include, 
even if requested to do so by an \Lc{input} or \Lc{include} command.\\
(Consult \fn{latex2html.config} for the default list.)

\item [\fn{\$DO\_INCLUDE}\texttt{ = '';}]
List of exceptions within the \fn{\$DONT\_INCLUDE} list.
These files \emph{are} to be read if requested by an \Lc{input}
or \Lc{include} command.
\end{htmllist}


\item [\fn{\$ICONSERVER}\texttt{ = '\Meta{URL}';}]
This is used to specify a URL to find the standard icons, as used for the
navigation buttons.\html{\\}
Names for the specific images size, as well as size information, 
can be found in \fn{latex2html.config}. The icons themselves can be replaced
by customised versions, provided this information is correctly updated
and the location of the customised images specified as the value
of \fn{\$ICONSERVER}.\\
When the \htmlref{\Cs{local\_icons}}{cs_localicons} switch is used,
so that a copy of the icons is placed with the \texttt{HTML} files and other
generated images, the value of \fn{\$ICONSERVER} is not needed within
the \texttt{HTML} files themselves. However it \emph{is} needed to find
the original icons to be copied to the local directory.

\item [\fn{\$NAV\_BORDER}\texttt{ = }\Meta{num}\texttt{;}]
The value given here results in a border, measured in points, around each icon.\html{\\}
A value of `\texttt{0}' is common, to maintain strict
alignment of inactive and active buttons in the control panels.


\item [\fn{\$LINKNAME}\texttt{ = '\char34index.\$EXTN\char34';}]
This is used when the \fn{\$NO\_AUTO\_LINK} \htmlref{variable}{noautolink} is empty, 
to allow a URL to the working directory to be sufficient to reach the main page 
of the completed document. 
It specifies the name of the \texttt{HTML} file which will be automatically
linked to the directory name.\html{\\}
The value of \fn{\$EXTN} is \texttt{.html} unless \fn{\$SHORTEXTN} is set, 
in which case it is \texttt{.htm}~. 

\item [\fn{\$LINKPOINT}\texttt{ = '\char34\$FILE\$EXTN\char34';}]
This specifies the name of the \texttt{HTML} file to be duplicated, or symbolically
linked, with the name specified in \fn{\$LINKNAME}.\\
At the appropriate time the value of \fn{\$FILE} is the document name, 
which usually coincides with the name of the working directory.

\index{character set}%
\index{character set!iso-latin@\texttt{\char34iso\_8859\_1\char34}}%
\item [\fn{\$CHARSET}\texttt{ = 'iso\_8859\_1';}\label{charset}]
This specifies the character set used within the \texttt{HTML} pages
produced by \latextohtml. If no value is set in a configuration
or initialisation file, the default value will be assumed.
The lowercase form \texttt{\$charset} is also recognised, but
this is overridden by the uppercase form.

\index{images!strange accents}%
\item [ \fn{\$ACCENT\_IMAGES}\texttt{ = 'large'};]
Accented characters that are not part of the ISO-Latin fonts
can be generated by making an image using \LaTeX.
This variable contains a (comma-separated) list of \LaTeX{} commands
for setting the style to be used when these images are made. 
If the value of this variable is empty then the accent is simply
ignored, using an un-accented font character (not an image) instead.
%
\end{htmllist}

\bigskip\noindent
Within the \fn{color.perl} package, the following variables are used to
identify the names of files containing specifications for named colors.
Files having these names are provided, in the \fn{\$LATEX2HTMLSTYLES}
directory, but they could be moved elsewhere, 
or replaced by alternative files having different names.
In such a case the values of these variables should be altered accordingly.
%
\begin{itemize}
\item []\fn{\$RGBCOLORFILE}\texttt{ = 'rgb.txt';}
\item []\fn{\$CRAYOLAFILE}\texttt{ = 'crayola.txt';}
\end{itemize}

\bigskip\noindent
The following variables may well be altered from the system defaults,
but this is best done using a local \fn{.latex2html-init} initialisation file,
for overall consistency of style within documents located at the same site,
or sites in close proximity.

\begin{htmllist}\htmlitemmark{PurpleBall}%
%
\item [\fn{\$default\_language}\texttt{ = 'english';}\label{isolanguage}]
This establishes which language code is to be placed within
the \HTMLtag{!DOCTYPE ... } tag that may appear at the beginning
of the \texttt{HTML} pages produced. Loading a package for
an alternative language can be expected to change the value of this variable.\html{\\}
See also the \fn{\$TITLES\_LANGUAGE} variable, described \htmlref{next}{doclanguage}.


\item [\fn{\$TITLES\_LANGUAGE}\texttt{ = 'english';}\label{doclanguage}]
This variable is used to specify the actual strings used for standard document
sections, such as ``Contents'', ``References'', ``Table of Contents'', etc.\html{\\}
Support for French and German titles is available in corresponding packages.
Loading such a package will normally alter the value of this variable, as well as 
the \fn{\$default\_language} variable described \htmlref{above}{isolanguage}.


\item [\fn{\$WORDS\_IN\_NAVIGATION\_PANEL\_TITLES}\texttt{ = 4;}]
Specifies how many words to use from section titles,
within the textual hyperlinks which accompany the navigation buttons.


\item [\fn{\$WORDS\_IN\_PAGE}\texttt{ = 450;}]
Specifies the minimum page length required before a navigation panel
is placed at the bottom of a page,
when the \fn{\$AUTO\_NAVIGATION} variable is set.


\item [\fn{\$CHILDLINE}\texttt{ = \char34<BR><HR>\char92n\char34;}]
This gives the \texttt{HTML} code to be placed between the child-links
table and the ordinary contents of the page on which it occurs.


\item [\fn{\$NETSCAPE\_HTML}\texttt{ = 0;}]
When set, this variable specifies that \texttt{HTML} code may be present which
does not conform to any official standard. 
This restricts the contents of any \HTMLtag{!DOCTYPE ... } tag which may be placed
at the beginning of the \texttt{HTML} pages produced.


\item [\fn{\$BODYTEXT}\texttt{ = '';}]
The value of this variable is used within the \HTMLtag{BODY ... } tag;
e.g. to set text and/or background colors.\html{\\}
It's value is overridden by the \verb|\bodytext| command, 
and can be added-to or parts changed using the \Lc{htmlbody} command
or \Lc{color} and \Lc{pagecolor} from the \env{color} package.


\item [\fn{\$INTERLACE}\texttt{ = 1;}\label{interlace}]
When set, interlaced images should be produced.\\
This requires \htmlref{graphics utilities}{graphicsutils} to be available 
to perform the interlacing operation.


\item [\fn{\$TRANSPARENT\_FIGURES}\texttt{ = 1;}\label{transfigs}]
When set, the background of images should be made transparent;
otherwise it is white.\html{\\}
This requires \htmlref{graphics utilities}{graphicsutils} %
to be available which can specify the color to be made transparent.


\item [\fn{\$FIGURE\_SCALE\_FACTOR}\texttt{ = 1.6;}\label{figscale}]
Scale factor applied to all images of \env{figure} and other environments, 
when being made into an image.

\smallskip\noindent
Note that this does not apply to recognised mathematics environments, 
which instead use the contents of \fn{\$MATH\_SCALE\_FACTOR} 
and \fn{\$DISP\_SCALE\_FACTOR} to specify scaling.


\item [\fn{\$MATH\_SCALE\_FACTOR}\texttt{ = 1.6;}\label{mathscale}]
Scale factor applied to all images of mathematics, both inline
and displayed. 
A value of 1.4 is a good alternative, with anti-aliased images.


\item [\fn{\$DISP\_SCALE\_FACTOR}\texttt{ = 1;}\label{dispscale}]
Extra scale factor applied to images of displayed math environments.\html{\\}
When set, this value multiplies \fn{\$MATH\_SCALE\_FACTOR}
to give the total scaling. 
A value of `1.2' is a good choice to accompany \verb|$MATH_SCALE_FACTOR = 1.4;|\,.


\item [\fn{\$EXTRA\_IMAGE\_SCALE}\label{ximagescale}]
This may hold an extra scale factor that can be applied to all generated images.\html{\\}
When set, it specifies that a scaling of \fn{\$EXTRA\_IMAGE\_SCALE} be applied 
when images are created, but to have their height and width recorded 
as the un-scaled size. This is to coax browsers into scaling the (usually larger) 
images to fit the desired size; when printed a better quality can be obtained.
Values of `1.5' and `2' give good print quality at 600dpi.


\item [\fn{\$PAPERSIZE}\texttt{ = 'a5';}\label{papersize}]
Specifies the size of a page for typesetting figures
or displayed math, when an image is to be generated.\\
This affects the lengths of lines of text within images.
Since images of text or mathematics should use larger sizes than
when printed, else clarity is lost at screen resolutions,
then a smaller paper-size is generally advisable.
This is especially so if both the \fn{\$MATH\_SCALE\_FACTOR}
and \fn{\$DISP\_SCALE\_FACTOR} scaling factors are being used,
else some images may become excessively large, including a lot
of blank space.


\item [\fn{\$LINE\_WIDTH}\texttt{ = 500;}\label{linewidth}]
Formerly specified the width of an image, when the contents were
to be right- or center-justified. (No longer used.)

\end{htmllist}

\bigskip

\noindent
The following variables are used to access the utilities required during image-generation. 
File and program locations on the local system are established by the \fn{configure-pstoimg} 
\Perl{} script and stored within \fn{\$LATEX2HTMLDIR/local.pm} as \Perl{} code, 
to be read by \fn{pstoimg} when required.

After running the \fn{configure-pstoimg} \Perl{} script it should not be necessary 
to alter the values obtained. Those shown below are what happens on the author's system;
they are for illustration only and do \emph{not} represent default values.
%

\begin{itemize}\label{graphicsutils}%
\item  \fn{\$GS\_LIB}\texttt{ = '/usr/local/share/ghostscript/4.02';}
\item  \fn{\$PNMCAT}\texttt{ = '/usr/local/bin/pnmcat';}
\item  \fn{\$PPMQUANT}\texttt{ = '/usr/local/bin/ppmquant';}
\item  \fn{\$PNMFLIP}\texttt{ = '/usr/local/bin/pnmflip';}
\item  \fn{\$PPMTOGIF}\texttt{ = '/usr/local/bin/ppmtogif';}
\item  \fn{\$HOWTO\_TRANSPARENT\_GIF}\texttt{ = 'netpbm';}
\item  \fn{\$GS\_DEVICE}\texttt{ = 'pnmraw';}
\item  \fn{\$GS}\texttt{ = '/usr/local/bin/gs';}
\item  \fn{\$PNMFILE}\texttt{ = '/usr/local/bin/pnmfile';}
\item  \fn{\$HOWTO\_INTERLACE\_GIF}\texttt{ = 'netpbm';}
\item  \fn{\$PBMMAKE}\texttt{ = '/usr/local/bin/pbmmake';}
\item  \fn{\$PNMCROP}\texttt{ = '/usr/local/bin/pnmcrop';}
\item  \fn{\$TMP}\texttt{ = '/usr/var/tmp';}
\end{itemize}

\noindent
The following variables are no longer needed, having been replaced by the
more specific information obtained using the \Perl{} script \fn{configure-pstoimg}.
\begin{itemize}
\item [] \fn{\$USENETPBM}\texttt{ = 1;}
\item [] \fn{\$PBMPLUSDIR}\texttt{ = '/usr/local/bin';}
\end{itemize}


%\filbreak
\subsection{Extending the Translator\label{sec:ext}}%
%\section{Extending the Translator\label{sec:ext}}%
\tableofchildlinks*\htmlrule
%
In \hyperref{an earlier section}{Section~}{}{sec:sty} is was seen how
the capabilities of \latextohtml{} can be extended to cope with
\LaTeX{} commands defined in packages and style-files.
This is in addition to defining simple macros and environments,
using \Lc{newcommand} and \Lc{newenvironment}.
The problem however, is that writing such extensions for packages 
requires an understanding of \Perl{} programming 
and of the way the processing in \latextohtml{} is organised. 

\index{new commands!ignore arguments}%
\index{new commands!arguments remain}%
\index{extensions!in initialisation file}
\smallskip

The default behaviour for unrecognised commands is 
for their arguments to remain in the \texttt{HTML} text,
whereas the commands themselves are passed on to \LaTeX,
in an attempt to generate an image. This is far from ideal,
for it is quite likely to lead to an error in \LaTeX{}
due to not having appropriate arguments for the command.

Currently there are several related mechanisms whereby a user 
can ask for particular commands and their arguments to be either 
\begin{itemize}
\item
\htmlref{ignored}{ignore} (via: \texttt{\&ignore\_commands}\,);
\item
passed on to \LaTeX{}  for processing \htmlref{as an image}{pass}\\
(via: \texttt{\&process\_commands\_in\_tex}\,);
\item
passed to \LaTeX{}  for processing as an image, 
\htmlref{as if an environment}{sensit}\\
(via: \texttt{\&process\_commands\_inline\_in\_tex}\,);
\item
passed on to \LaTeX, \htmlref{setting parameters}{setparam} 
to be used in subsequent images\\
(via: \texttt{\&process\_commands\_nowrap\_in\_tex}\,);
\item
processed as ``order-sensitive'', \htmlref{as if an environment}{wrapdef} 
rather than a command\\
(via: \texttt{\&process\_commands\_wrap\_deferred}\,);
\end{itemize}
The string beginning \texttt{\&.... } is the name of the \Perl{} subroutine 
that controls how the \LaTeX{} commands are to be subsequently treated during
processing by \latextohtml.
Towards the end of the \fn{latex2html} script, one finds a list
of \LaTeX{} commands to be handled by each of these subroutines.
These lists even include some common \TeX{} commands.

Analogous lists occur in most of the package extension files.
In many cases the commands are for fine-tuning the layout on
a printed page. They should simply be ignored; but any parameters
must not be allowed to cause unwanted characters to appear on
the \texttt{HTML} pages.
Customised extensions using these mechanisms may be included in the 
\fn{\$LATEX2HTMLDIR/latex2html.config} file, or in a personal 
\fn{\$HOME/.latex2html-init} initialisation file, 
as described \htmlref{next}{ignore}.



\subsubsection{Asking the Translator to Ignore Commands\label{ignore}}
%\subsection{Asking the Translator to Ignore Commands\label{ignore}}
\index{new commands!command ignored}\html{\\}%
Commands that should be ignored may be specified in the \fn{.latex2html-init}
file as input to the \verb|&ignore_commands| subroutine. 
Each command which is to be ignored should be on a separate line 
followed by compulsory or optional argument markers separated by 
{\verb|#|}'s e.g.\footnote{It is possible to add arbitrary(?) \Perl{} code
between any of the argument markers which will be executed when 
the command is processed. For this however a basic understanding of
how the translator works and of course \Perl{} is required.}:
\begin{quote}
%begin{latexonly}
\begin{small}
%end{latexonly}
\verb|<cmd_name1> # {} # [] # {} # [] ...|\\
\verb|<cmd_name2> # |\texttt{<<\emph{pattern}>>}\verb| # [] ...|
%begin{latexonly}
\end{small}
%end{latexonly}
\end{quote}
\verb|{}|'s mark compulsory arguments and \verb|[]|'s optional ones,
while \texttt{<<\emph{pattern}>>} denotes matching everything up to the
indicated string-pattern, given literally (e.g. \verb|\\endarray|);
spaces are ignored. Special characters
such as \verb|$|\,, \verb|&|\,, \verb|\| itself and perhaps some others,
need to be ``escaped'' with a preceding \verb|\|\,.


\index{new commands!arguments remain}\html{\\}%

Some commands may have arguments which should be left as text even
though the command should be ignored 
(e.g. \Lc{hbox}, \Lc{center}, etc.). 
In these cases arguments should be left unspecified.
Here is an example of how this mechanism may be used:
%begin{latexonly}
\begin{small}
%end{latexonly}
\begin{verbatim}
&ignore_commands( <<_IGNORED_CMDS_);
documentstyle # [] # {}
linebreak # []
mbox
<add your commands here>
_IGNORED_CMDS_
\end{verbatim}
%begin{latexonly}
\end{small}
%end{latexonly}


\subsubsection{Asking the Translator to Pass Commands to \LaTeX\label{pass}}%
%\subsection{Asking the Translator to Pass Commands to \LaTeX\label{pass}}%
\index{new commands!processed in LaTeX@processed in \LaTeX}\html{\\}%
Commands that should be passed on to \LaTeX{}  for processing because
there is no direct translation to \texttt{HTML} may be specified in the 
\fn{.latex2html-init}
file as input to the \verb|process_commands_in_tex| subroutine.
The format is the same as that for specifying commands to be ignored.
Here is an example:
%begin{latexonly}
\begin{small}
%end{latexonly}
\begin{verbatim}
&process_commands_in_tex (<<_RAW_ARG_CMDS_);
fbox # {}
framebox # [] # [] # {}
<add your commands here>
_RAW_ARG_CMDS_
\end{verbatim}
%begin{latexonly}
\end{small}
%end{latexonly}


\subsubsection{Handling ``order-sensitive'' Commands\label{sensit}}%
%\subsection{Handling ``order-sensitive'' Commands\label{sensit}}%
\index{new commands!order-sensitive}%
\index{new commands!order-sensitive!processed in LaTeX@processed in \LaTeX}%
\index{new commands!processed in LaTeX@processed in \LaTeX!order-sensitive}%
Some commands need to be passed to \LaTeX, but using the 
\verb|&process_commands_in_tex| subroutine gives incorrect results. 
This may occur when the command is ``order-sensitive'',
using information such as the value of a counter or a boolean expression 
(or perhaps requiring a box to have been constructed and saved).
Try using the \verb|&process_commands_inline_in_tex| subroutine instead.
Commands declared this way are first ``wrapped'' within a dummy environment,
which ensures that they are later processed in correct order with other 
environments and order-sensitive commands.

\index{order-sensitive!setting parameters}\label{setparam}\html{\\}%

Other commands may need to be passed to \LaTeX, not to create an image
themselves, but to affect the way subsequent images are created.
For example a color command such as \verb|\color{red}| should set the
text-colour to `red' for all subsequent text and images. This must
be sent to \LaTeX{} so that it is processed at exactly the right time;
i.e. before the first image required to be `red' but following any
images that are not intended to be affected by this colour-change.\html{\\}
The subroutine \verb|process_commands_nowrap_in_tex| is designed
specifically to meet such requirements.

\index{order-sensitive!wrap-deferred}\label{wrapdef}\html{\\}%

Commands can be order-sensitive without having to be passed to \LaTeX.
Indeed even if a \Perl{} subroutine has been carefully written to process
the command, it may still give wrong results if it is order-sensitive,
depending on the values of counters, say.
To handle such cases there is the \verb|&process_commands_wrap_deferred|
subroutine. This also ``wraps'' the command within a dummy environment,
but when that environment is processed the contents are not sent to \LaTeX,
as in the previous case. All of the standard \LaTeX{} commands to change, 
set or read the values of counters are handled in this way.



\subsection{Customising the Navigation Panels\label{sec:navpanel}}%
%\section{Customising the Navigation Panels\label{sec:navpanel}}%
\index{navigation panel|(}%
\index{navigation panel!customisable}\html{\\}%
The navigation panels are the strips containing ``buttons'' and text 
that appears at the top and perhaps at the bottom 
of each generated page and provides hypertext links 
to other sections of a document. 
Some of the options and variables that control whether and 
where it should appear are presented in \hyperref{an earlier section}%
{Section~}{}{navoptions}. 

\index{navigation panel!customised buttons}%
\index{htmladdtonavigation@\Lc{htmladdtonavigation}}\html{\\}%
A simple mechanism for appending customised buttons to the navigation
panel is provided by the command \Lc{htmladdtonavigation}. 
This takes one argument which \latextohtml{} appends to the navigation panel. 
For example,
%begin{latexonly}
\begin{small}
%end{latexonly}
\begin{verbatim}
\htmladdtonavigation{\htmladdnormallink
   {\htmladdimg{http://server/mybutton.gif}}{http://server/link}}
\end{verbatim}
%begin{latexonly}
\end{small}
%end{latexonly}
will add an active button \fn{mybutton.gif} pointing to the specified location.

\index{navigation panel!order of buttons}\html{\\}\noindent
Apart from these facilities it is also 
possible to specify completely what appears in the navigation panels
and in what order. As each section is processed, \latextohtml{}
assigns relevant information to a number of global variables.
These variables are used by the
subroutines \texttt{top\_navigation\_panel} and
\texttt{bottom\_navigation\_panel},
where the navigation panel is constructed as a string consisting of
these variables and some formatting information. 

\index{navigation panel!Perl subroutines@\Perl{} subroutines}\html{\\}\noindent
These subroutines can be redefined in a system- 
or user-configuration file (respectively, 
\fn{\$LATEX2HTMLDIR/latex2html.config} and \fn{\$HOME/.latex2html-init}). 
Any combination of text, \texttt{HTML} tags, 
and the variables mentioned below is acceptable.

\medskip\noindent
The control-panel variables are:\nobreak

\medskip\nobreak
\index{navigation panel!buttons}%
\index{navigation panel!iconic links}\html{\\}
\noindent
\textbf{Iconic links (buttons):}
\begin{htmllist}\htmlitemmark{RedBall}\addtolength{\leftskip}{15pt}%
\item [\fn{\$UP}]  Points up to the ``parent'' section;
\item [\fn{\$NEXT}] Points to the next section;
\item [\fn{\$NEXT\_GROUP}] Points to the next ``group'' section;
\item [\fn{\$PREVIOUS}] Points to the previous section;
\item [\fn{\$PREVIOUS\_GROUP}] Points to the previous ``group'' section;
\item [\fn{\$CONTENTS}] Points to the contents page if there is one;
\item [\fn{\$INDEX}] Points to the index page if there is one.
\end{htmllist}

\index{navigation panel!section titles}%
\index{navigation panel!textual links}\html{\\}\noindent
\textbf{Textual links (section titles):}
\begin{htmllist}\htmlitemmark{PinkBall}\latex{\addtolength{\leftskip}{15pt}}%
\item [\fn{\$UP\_TITLE}]  Points up to the ``parent'' section;
\item [\fn{\$NEXT\_TITLE}] Points to the next section;
\item [\fn{\$NEXT\_GROUP\_TITLE}] Points to the next ``group'' section;
\item [\fn{\$PREVIOUS\_TITLE}] Points to the previous section;
\item [\fn{\$PREVIOUS\_GROUP\_TITLE}] Points to the previous ``group'' section.
\end{htmllist}
If the corresponding section exists, each iconic button will contain an
active link to that section. If the corresponding section does
not exist, the button will be inactive. If the section corresponding
to a textual link does not exist then the link will be empty.

The ``next group'' and ``previous group'' are rarely used, 
since it is usually possible to determine
which are the next/previous logical pages in a document.
However these may be needed occasionally with segmented documents,
when the segments have been created with different values
for the \fn{\$MAX\_SPLIT\_DEPTH} variable. 
This is quite distinct from the segmented document effect 
in which the first page of one segment may have its `\emph{PREVIOUS}'
button artificially linked to the first page of the previous segment, 
rather than the last page.


\index{navigation panel!number of words}\html{\\}\noindent
The number of words that appears in each textual link
is controlled by the variable \fn{\$WORDS\_IN\_NAVIGATION\_PANEL\_TITLES}
which may also be changed in the configuration files.

\index{navigation panel!example}\html{\\}\noindent
Below is an example of a navigation panel for the bottom of \texttt{HTML} pages.
(Note that the ``\texttt{.}'' is \Perl{}'s string-concatenation operator 
and ``\texttt{\#}'' signifies a comment.)
%begin{latexonly}
\begin{small}
%end{latexonly}
\begin{verbatim}
sub bot_navigation_panel {
    #  Start with a horizontal rule and descriptive comment
    "<HR>\n" . "<!--Navigation Panel-->".
    # Now add a few buttons, with a space between them
       "$NEXT $UP $PREVIOUS $CONTENTS $INDEX $CUSTOM_BUTTONS" . 
    # Line break    
       "\n<BR>" .       
    # If ``next'' section exists, add its title to the navigation panel
       ($NEXT_TITLE ? "\n<B> Next:</B> $NEXT_TITLE" : undef) .   
    # Similarly with the ``up'' title ...
       ($UP_TITLE ? "\n<B>Up:</B> $UP_TITLE\n" : undef) . 
    # ... and the ``previous'' title
       ($PREVIOUS_TITLE ? "\n<B> Previous:</B> $PREVIOUS_TITLE\n" : undef) . 
}
\end{verbatim}
%begin{latexonly}
\end{small}
%end{latexonly}
Note that extra buttons may be included by defining suitable code for the container
\fn{\$CUSTOM\_BUTTONS}\,. 
The use of explicit `newline' (\verb|\n|) characters is not necessary for the
on-screen appearance of the navigation panel within a browser window. However 
it maintains an orderly organisation within the \texttt{.html} files themselves,
which is helpful if any hand-editing is later required, 
or simply to read their contents.
The corresponding subroutine for a navigation-panel at the top of a page
need not use the rule \HTMLtag{HR}, and would require a break (\HTMLtag{BR})
or two at the end, to give some visual separation from the following material.
\index{navigation panel|)}
\index{extensions|)}




             % Input counters and section
\label{sec:man}\index{manual}
\end{htmlonly}
\startdocument
%
%\section{User Manual\label{sec:man}\index{manual}}
\label{sec:man}
\index{customised layout}
\index{getting started}
\html{\\}\noindent
\latextohtml{} is a program for creating hyperlinked sets of \texttt{HTML} pages
from a document marked-up using \LaTeX{} commands.
Previous sections have discussed the results of specific \LaTeX{} commands.
In this section we discuss instead the extensive range
of command-line switches and options, and other aspects of \Perl{} code,
that affect the way the translation is performed.

\medskip\noindent
To use \latextohtml{} to translate a file~\Meta{file}\texttt{.tex }
containing \LaTeX{} commands, type:
\begin{quote}
\texttt{latex2html} \Meta{file}\texttt{.tex} 
\end{quote}

\index{browser@browser\label{IIIbrowser}}%
\index{viewer|see{\htmlref{browser}{IIIbrowser}}}%
\index{browser!NCSA Mosaic@\textsl{NCSA Mosaic}}%
\index{browser!Netscape Navigator@\textsl{Netscape Navigator}}\html{\\}%
\noindent
This will create a new directory called \Meta{file} which will contain 
the generated \texttt{HTML} files, some log files and possibly some images.
To view the result use an \texttt{HTML} browser, such as 
\appl{NCSA Mosaic} or \appl{Netscape Navigator},
on the main \texttt{HTML} file which is \texttt{\Meta{file}/\Meta{file}.html}\,. 
The file will contain navigation links 
to the other parts of the generated document.
%
\index{file suffix!optional}%
\index{file suffix!.tex@\texttt{.tex} or\texttt{.doc}}%
The \texttt{.tex} suffix is optional and will be supplied
by the program if it is omitted by the user.
Other suffixes are acceptable also, such as \texttt{.doc}\,.

\medskip
\index{options!command-line options}\html{\\}\noindent
It is possible to customise the output from \latextohtml{} using a number of 
\hyperref{command-line options}{command-line options (see Section~}{)}{options}
with which you can specify:
\begin{itemize}
\item
how to break up the document; 
\item
where to put the generated files, and what are their names;
\item
the title for the whole document; 
\item
the signature at the end of each page is; 
\item
how many navigation panels to provide, what links to put in them;
\item
what other documents this one links to;
\item 
extra information to include about the document;
\item 
whether to retain the original \LaTeX{} section-numbering scheme;
\item
and many other things that affect how the information is obtained,
processed or displayed in the resulting \texttt{.html} files and images.
\end{itemize}

\medskip
\index{manual!short on-line}\html{\\}\noindent
The \latextohtml{} script includes a short manual
which can be viewed with the command: 
\begin{verbatim}
  perldoc latex2html
\end{verbatim}


\subsection{Developing Documents using \latextohtml{}\label{devel}}
%\section{Developing Documents using \latextohtml{}\label{devel}}
%
Although any document containing \LaTeX{} commands can be translated
by the \latextohtml{} translator, the best results are obtained when
that document is itself a valid \LaTeX{} document.
Indeed it is generally a good idea to develop documents so that
they produce good readable results in both the \LaTeX{} typeset
version as well as a set of \texttt{HTML} pages.
This is not just a nicety; 
there are several good practical reasons for doing this.

\begin{htmllist}\htmlitemmark{BlueBall}
%
\item[\LaTeX{} macros: ]
The macro commands that \latextohtml{} recognises are based upon
corresponding commands for \LaTeX. If one tries to use syntax that is
incorrect for \LaTeX{} then there is no reason why \latextohtml{} should
be able to ``get it right'', by somehow recognising the true intent.


\index{error checking!using LaTeX@using \LaTeX}%
\item[error checking: ]
Processing the document first using \LaTeX{} is the easiest, 
and quickest, way to check for valid syntax.\html{\\}
Whereas \LaTeX{} stops at each error (when run in interactive mode),
allowing a fix to be made ``on the spot'' or a ``stop-fix-restart'',
\latextohtml{} does not stop when it detects an error in \LaTeX{} syntax.
Useful messages are given concerning missing or unmatched braces,
but other apparent anomalies generate only warning messages, 
which are saved to the end. 
(Some warnings are also shown immediately when the \fn{\$VERBOSITY} 
\htmlref{variable}{verbosity} is set to at least 3.)\html{\\}
In practice it can be much quicker to test for invalid syntax using \LaTeX{} before 
attempting to use the \latextohtml{} translator.

Furthermore, \LaTeX{} warns of cross-reference labels that have not been defined. 
This is useful to help avoid having hyperlinks which point to nowhere.

\index{error checking!missing braces}%
\index{error checking!unmatched brace}\html{\\}\noindent
The case of missing braces, or an unmatched opening brace, 
is an error that \latextohtml{} actually handles better than \LaTeX{} 
(or rather, the underlying \TeX{} processor).
Whereas \TeX{} only detects an error when something else goes wrong
later in the processing, \latextohtml{} shows where the
unmatched brace itself occurs. 


\index{auxiliary file}
\item[auxiliary file: ]
Some information that \latextohtml{} might need is normally
read from the \texttt{.aux} file for the document being processed;
or perhaps from the \texttt{.aux} file of a different document,
of which the current document is just a portion. 
Clearly a valid \LaTeX{} run is required to produce the \texttt{.aux} file.

\html{\\}\noindent
Of course if no information in the \texttt{.aux} file is actually used,
then this \LaTeX{} run can be neglected.
Also, if the \texttt{.aux} file has already been created, and edits
are made on the source which do not alter the information stored
within the \texttt{.aux} file, then there is no need for a fresh
\LaTeX{} run (except for the purposes of error-checking).


\index{bibliography}
\item[bibliography: ]
Suppose the document requires a bibliography, or list of references,
which is to be prepared using Bib\TeX.
\latextohtml{} reads citation information from the \texttt{.aux} file,
and can import the bibliography itself from the \texttt{.bbl} file.
However these must first be created using \LaTeX.



\index{document segmentation}
\item[document segmentation: ]
With the document segmentation technique, 
discussed fully in \hyperref{a later section}{Section~}{}{Segmentation},
it is vitally important that the full document processes correctly
in \LaTeX. The desired effect is that of a single large document,
whereas the pieces will actually be processed separately.
To achieve this, \LaTeX{} writes vital information into special
\texttt{.ptr} files. Like the \texttt{.aux} file, these files
are read by \latextohtml{} to get section-numbering and other
information to be used while processing each segment.


\index{printing!hyperlink to typeset version}%
\index{printing}\html{\\}\noindent
\item[print quality: ]
When a document contains automatically-generate images, these images
are usually bitmaps designed for viewing on-screen. In general the 
resolution of these is too poor to give a good result when printed
on a high-resolution laser-printer.
Thus if it is likely that the reader will want to obtain
a printed version of your document, then it is best to include
a hyperlink to the typeset \texttt{.dvi} version, 
or to a \PS\ conversion of the \texttt{.dvi} file. 
(In either case, a link to a compressed version is even better.)

\end{htmllist}



\subsection{Command-Line Options\label{options}}%
%\section{Command-Line Options\label{options}}%
\index{options@options\label{IIIoptions}}%
\index{switches|see{\htmlref{options}{IIIoptions}}}%
\index{options|(}%
\index{options!environment variables}%
\index{options!set in initialisation file}%
\html{\\}\noindent
The command-line options described here can be used to change the
default behaviour of \latextohtml. 
Alternatively, often there is a corresponding environment variable
whose value may be set or changed within a \fn{.latex2html-init} 
initialisation file, in order to achieve the same result.\html{\\}
There are so many options that they are listed here in groups, 
according to the nature of the effects they control.
When a large number of such options are required for the processing of a document,
it is usual to store the switches and their desired settings within a \fn{Makefile},
for use with the Unix \fn{make} utility. Now a simple command such as:
\begin{verbatim}
  make mydocument
\end{verbatim}
can initiate a call to \fn{latex2html} that would otherwise take many lines of typing.
Indeed it could instigate several such calls to \latextohtml, or to other programs
such as \LaTeX{} and Bib\TeX, \fn{dvips} and others.
The document segmentation feature, discussed in 
\hyperref{another section}{Section~}{}{Segmentation},
uses this technique extensively.


\subsubsection{Options controlling Titles, File-Names and Sectioning}
%\subsection{Options controlling Titles, File-Names and Sectioning}
\index{options!titles}\index{options!file-names}\index{options!sectioning}%
%
\index{document!title}%
%
\begin{htmllist}\htmlitemmark{PurpleBall}%
\item [ -t \Meta{top-page-title}\label{cs_toppagetitle}]
\sameas{\fn{\$TITLE}\texttt{ = \char34}\Meta{top-page-title}\texttt{\char34;}}\\
Name the document using this title.

\index{filenames!.htm@\texttt{.htm} extension}%

\item [ -short\_extn\label{cs_shortextn}]
\sameas{\fn{\$SHORTEXTN}\texttt{ = 1;}}\\
Use a filename prefix of \texttt{.htm } for the produced
\texttt{HTML} files. This is particularly useful for creating pages
to be stored on CD-ROM or other media, to be used with 
operating systems that require a 3-character extension.%

\index{filenames!long names}

\item [ -long\_titles \Meta{num}\label{cs_longtitles}]
\sameas{\fn{\$LONG\_TITLES}\texttt{ = }\Meta{num}\texttt{;}}\\
Instead of the standard  names: \texttt{node1.html}, \texttt{node2.html},... 
the filenames for each \texttt{HTML} page are constructed
from the first \Meta{num} words of the section heading
for that page, separated by the `\texttt{\_}' character.\html{\\}
Commas and common short words (\texttt{a an to by of and for the})
are omitted from both title and word-count.

\smallskip\noindent
\textbf{Warning:} Use this switch with great caution.
Currently there are no checks for uniqueness of names or overall length. 
Very long names can easily result from using this feature.%

\index{filenames!customised}%
\index{filenames!custom title hook@\texttt{custom\_title\_hook}}

\item [ -custom\_titles\label{cs_customtitles}]
\sameas{\fn{\$CUSTOM\_TITLES}\texttt{ = 1;}}\\
Instead of the standard  names: \texttt{node1.html}, \texttt{node2.html}, ... 
the filenames for each \texttt{HTML} page are constructed using a \Perl{}
subroutine named \texttt{ custom\_title\_hook}~. 
The user may define his/her own version of this subroutine, 
within a \fn{.latex2html-init} file say, 
to override the default (which uses the standard names).
This subroutine takes the section-heading as a parameter 
and must return the required name, or the empty string (default).

\index{output!redirect to directory}

\item [ -dir \Meta{output-directory}\label{cs_outputdir}]
\sameas{\fn{\$DESTDIR}\texttt{ = \char34}\Meta{output-directory}\texttt{\char34;}}\\
Redirect the output to the specified directory.\html{\\}
The default behaviour is to create (or reuse) a directory having the same
name as the prefix of the document being processed.%

\index{output!in current directory}%
\index{output!default directory}

\item [ -no\_subdir\label{cs_nosubdir}]
\sameas{\fn{\$NO\_SUBDIR}\texttt{ = 1;}}\\
Place the generated \texttt{HTML} files into the current directory.
This overrides any \fn{\$DESTDIR} setting.

\index{filenames!prefix}%
\index{temporary files}%

\item [ -prefix \Meta{filename-prefix}\label{cs_prefix}]%
\sameas{\fn{\$PREFIX}\texttt{ = \char34}\Meta{filename-prefix}\texttt{\char34;}}\\
The \Meta{filename-prefix} will be prepended to all 
\texttt{.gif}, \texttt{.pl} and \texttt{.html} files produced, 
except for the top-level \texttt{.html} file;  
it may include a (relative) directory path. 
This will enable multiple products of \latextohtml{} to
peacefully coexist in the same directory.
However, do \emph{not} attempt to \emph{simultaneously} run multiple 
instances of \latextohtml{} using the same output directory, 
else various temporary files will overwrite each other.

\index{filenames!auto-prefix}%

\item [ -auto\_prefix\label{cs_autoprefix}]
\sameas{\fn{\$AUTO\_PREFIX}\texttt{ = 1;}}\\
Constructs the prefix as `\Meta{title}--'
to be prepended to all the files produced, 
where \Meta{title} is the name of the \LaTeX{} file being processed.
(Note the `--' in this prefix.)\\
This overrides any \fn{\$PREFIX} setting.

\index{filenames!automatic link to directory name}%
\index{filenames!index.html@\fn{index.html}}%

\item [ -no\_auto\_link\label{cs_noautolink}]
\sameas{\fn{\$NO\_AUTO\_LINK}\texttt{ = 1;}}\\
If \fn{\$NO\_AUTO\_LINK} is empty and variables \fn{\$LINKPOINT} 
and \fn{\$LINKNAME} are defined appropriately
(as is the default in the \fn{latex2html.config} file),
then a hard link to the main \texttt{HTML} page is produced,
using the name supplied in \fn{\$LINKNAME}. Typically this is \fn{index.html}; 
on many systems a file of this name will be used, if it exists,
when a browser tries to view a URL which points to a directory.
On other systems a different value for \fn{\$LINKNAME} may be appropriate.
Typically \fn{\$LINKPOINT} has value \texttt{\$FILE.html}, but
this may also be changed to match whichever \texttt{HTML} page
is to become the target of the automatic link.\\
Use of the \Cs{no\_auto\_link} switch \emph{cancels} 
this automatic linking facility, 
when not required for a particular document.

\index{sections!in separate files}%

\item [ -split \Meta{num}\label{cs_splitdepth}]
\sameas{\fn{\$MAX\_SPLIT\_DEPTH}\texttt{ = }\Meta{num}\texttt{;}}
 (default is \texttt{8})\\
Stop splitting sections into separate files at this depth.
Specifying \Cs{split 0} will put the entire document into a 
single \texttt{HTML} file. See \htmlref{below}{seclevels} for
the different levels of sectioning. Also see the next item for
how to set a ``relative'' depth for splitting.%

\item [ -split +\Meta{num}\label{cs_reldepth}]
\sameas{\fn{\$MAX\_SPLIT\_DEPTH}\texttt{ = }\textrm{-}\Meta{num}\texttt{;}}
 (default is \texttt{8})\\
The level at which to stop splitting sections is calculated ``relative to'' 
the shallowest level of sectioning that occurs within the document.
For example, if the document contains \Lc{section} commands, but no \Lc{part}
or \Lc{chapter} commands, then \Cs{split +1} will cause splitting at each \Lc{section}
but not at any deeper level; whereas \Cs{split +2} or \Cs{split +3} also split down to 
\Lc{subsection} and \Lc{subsubsection} commands respectively.
Specifying \Cs{split +0} puts the entire document into a single \texttt{HTML} file.


\index{depth!child nodes revealed}%
\index{components!specify depth}

\index{child-links table}%
\item [ -link \Meta{num}\label{cs_linkdepth}]
\sameas{\fn{\$MAX\_LINK\_DEPTH}\texttt{ = }\Meta{num}\texttt{;}}
 (default is \texttt{4})\\
For each node, create links to child nodes down to this much 
deeper than the node's sectioning-level.\html{\\}
Specifying \Cs{link 0} will show no links to child nodes from that page,\html{\\}
\Cs{link 1} will show only the immediate descendents, etc.\html{\\}
A value at least as big as that of the 
\htmlref{\Cs{split \Meta{num}}}
{splitdepth} depth will
produce a mini table-of-contents (when not empty) on each page,
for the tree structure rooted at that node.%

\index{child-links table!at bottom of page}%

When the page has a sectioning-level less than the \Cs{split} depth,
so that the a mini table-of-contents has links to other \texttt{HTML} pages, 
this table is located at the bottom of the page, 
unless placed elsewhere using the \Lc{tableofchildlinks} command.%

\index{child-links table!at top of page}%
\index{child-links table!tableofchildlinks@\texttt{tableofchildlinks}}

On pages having a sectioning-level just less than the \Cs{split} depth 
the mini table-of-contents contains links to subsections etc.
occurring on the \emph{same} \texttt{HTML} page. 
Now the table is located at the \emph{top} of this page, 
unless placed elsewhere using the \Lc{tableofchildlinks} command.

\index{table of contents!depth}%

\item [ -toc\_depth  \Meta{num}\label{cs_tocdepth}]
\sameas{\fn{\$TOC\_DEPTH}\texttt{ = }\Meta{num}\texttt{;}}
 (default is \texttt{4})\\
Sectioning levels down to \Meta{num} are to be included 
within the Table-of-Contents tree.%

\index{table of contents!star-sections shown}%
\index{table of contents!star-sections not listed}

\item [ -toc\_stars\label{cs_tocstars}]
\sameas{\fn{\$TOC\_STARS}\texttt{ = 1;}}\\
Sections created using the starred-form of sectioning commands
are included within the Table-of-Contents.
As with \LaTeX, normally such sections are \emph{not} listed.
%

\index{sections!show section numbers}%
\index{sections!numbers not shown}

\item [ -show\_section\_numbers\label{cs_showsecnums}]
\sameas{\fn{\$SHOW\_SECTION\_NUMBERS}\texttt{ = 1;}}\\
Show section numbers. By default section numbers are \emph{not} shown, 
so as to encourage the use of particular sections as stand-alone documents. 
\strikeout{In order to be shown, 
section titles must be unique and must not contain inlined graphics.}

\index{segmentation!unsegment}

\item [ -unsegment\label{cs_unsegment}]
\sameas{\fn{\$UNSEGMENT}\texttt{ = 1;}}\\
Treat a segmented document (see the section about \htmlref{document
  segmentation}{Segmentation}) like it were not segmented.
This will cause the translator to concatenate all segments and process
them as a whole.
You might find this useful to check a segmented document for
consistency.
%
\end{htmllist}

\goodbreak
\index{sectioning levels}\index{levels!sectioning}\label{seclevels}%
\html{\\}\noindent
For all documents the sectioning levels referred to above are:\nobreak
\begin{center}
\begin{tabular}{ll}
\textbf{0} & document\\
\textbf{1} & part\\
\textbf{2} & chapter\\
\textbf{3} & section\\
\textbf{4} & subsection\\
\textbf{5} & subsubsection\\
\textbf{6} & paragraph\\
\textbf{7} & subparagraph\\
\textbf{8} & subsubparagraph
\end{tabular}
\end{center}
These levels apply even when the document contains no sectioning for
the shallower levels; e.g. no \Lc{part} or \Lc{chapter} commands is most common,
especially when using \LaTeX's \env{article} document-class.



\subsubsection{Options controlling Extensions and Special Features}
\index{extensions|(}%
\index{options!extensions}\index{options!special features}%
\html{\\}\noindent
The switches described here govern the type of \texttt{HTML} code that
can be generated, and how to choose between the available options 
when there are alternative strategies for implementing portions of \LaTeX{} code. 

\begin{htmllist}\htmlitemmark{PurpleBall}%
%
\index{HTML version@\texttt{HTML} version}%
\index{HTML@\texttt{HTML}!non-standard extensions}%
\item [ -html\_version \texttt{(2.0|3.0|3.2)[,(math|i18n|table)]*}\label{cs_htmlversion}]
~\\\sameas{\fn{\$HTML\_VERSION}\texttt{ = \char34 ... \char34;}}\\
This specifies both the \texttt{HTML}~\htmlref{version}{version} to generate, 
and any extra (non-standard) \texttt{HTML} features that may be required.\html{\\}
The version number corresponds to a published DTD for an \texttt{HTML} standard
(although 3.0 was never accepted and subsequently withdrawn).
A corresponding \Perl{} file in the \fn{versions/} subdirectory is loaded;
these files are named `\texttt{html}\Meta{num}\texttt{.pl}'.

Following the version number, a comma-separated list of extensions
can be given. Each corresponds to a file `\Meta{name}\texttt{.pl}' 
also located in the \fn{versions/} subdirectory. 
When such a file is loaded the resulting \texttt{HTML} code 
can no longer be expected to validate with the specified DTD. 
An exception is \Ve{math} when the \htmlref{\Cs{no\_math}}{nomath} 
switch is also used, which should still validate.

Currently, versions 2.0, 3.2 and 4.0 are available.
(and also 2.1, 2.2, 3.0 and 3.1, for hoistorical reasons). 
The extensions \Ve{i18n}, \Ve{tables}, \Ve{math}
correspond roughly to what used to be called versions `2.1', `2.2', `3.1'
respectively, in releases of \latextohtml{} up to 1996.
Now these extensions can be loaded with any of `2.0', `3.2' or `4.0' 
as the specified standard.\\  
The default version is usually set to be `3.2', within \fn{latex2html.config}\,.%

\index{extensions!disabled}%
\index{extensions!TeX defs@\TeX definitions}%

\item [ -no\_tex\_defs\label{cs_notexdefs}]
\sameas{\fn{\$TEXDEFS}\texttt{ = 0;}} (default is 1)\\
When \fn{\$TEXDEFS} is set (default) the file \fn{texdefs.perl} will be read.
This provides code to allow common \TeX{} commands like \Tc{def}, \Tc{newbox},
\Tc{newdimen} and others, to be recognised, especially within the document preamble.
In the case of \Tc{def}, the definition may even be fully interpreted,
but this requires the pattern-matching to be not too complicated.

If \fn{\$TEXDEFS} is `\texttt{0}' or empty, then \fn{texdefs.perl} will not be loaded;
the translator will make no attempt to interpret any raw \TeX{} commands.  
This feature is intended to enable sophisticated authors the ability to insert 
arbitrary \TeX{} commands in environments that are destined 
to be processed by \LaTeX{} anyway; 
e.g. \env{figure}s, \strikeout{\env{theorem}s}, \env{picture}s, etc.
However this should rarely be needed, as now there is better support for these
types of environment. There are now other methods to specify which chunks of code 
are to be passed to \LaTeX{} for explicit image-generation; 
see the discussion of the \env{makeimage} 
\hyperref[page]{environment}{environment on page~}{}{makeimage}.


\index{auxiliary file!which to read}%

\item [ -external\_file \Meta{filename}\label{cs_externalfile}] 
\sameas{\fn{\$EXTERNAL\_FILE}\texttt{ = \char34}\Meta{filename}\texttt{\char34;}}\\
Specifies the prefix of the \texttt{.aux} file that this document should read.
The \texttt{.aux} extension will be appended to this prefix 
to get the complete filename, with directory path if needed.\html{\\}  
This file could contain necessary information 
regarding citations, figure, table and section numbers from \LaTeX{} 
and perhaps other information also.
Use of this switch is vital for \htmlref{document segments}{Segmentation}, 
processed separately and linked to appear as if generated 
from a single \LaTeX{} document.%

\index{font size!for image generation}%
\index{font-size!in environments}\index{font-size!magnification}%

\item [ -font\_size \Meta{size}\label{cs_fontsize}]
\sameas{\fn{\$FONT\_SIZE}\texttt{ = \char34}\Meta{size}\texttt{\char34;}}\\
This option provides better control 
over the font size of environments made into images using \LaTeX.  
\Meta{size} must be one of the font sizes that \LaTeX{} recognizes; 
i.e. `\texttt{10pt}', `\texttt{11pt}', `\texttt{12pt}', etc.  
Default is `\texttt{10pt}', or whatever option may have been specified on the 
\Lc{documentclass} or \Lc{documentstyle} line.\html{\\}
Whatever size is selected, it will be magnified by the installation variables
\fn{\$MATH\_SCALE\_FACTOR}, \fn{\$FIGURE\_SCALE\_FACTOR}
and \fn{\$DISP\_SCALE\_FACTOR} as appropriate.

\smallskip\noindent
\textbf{Note: }This switch provides no control over the size of text
on the \texttt{HTML} pages. 
Such control is subject entirely to the user's choices
of settings for the browser windows.

\index{font size!scalable}%

\item [ -scalable\_fonts\label{cs_scalefonts}]
\sameas{\fn{\$SCALABLE\_FONTS}\texttt{ = 1;}}\\
This is used when scalable fonts, such as \PS\ versions
of the \TeX{} fonts, are available for image-generation.\html{\\}
It has the effect of setting \fn{\$PK\_GENERATION} to `\texttt{1}',
and \fn{\$DVIPS\_MODE} to be empty, 
overriding any \htmlref{previous settings}{pkgen} for these variables.

\index{simple math!cancelled}%

\item [ -no\_math\label{cs_nomath}]
\sameas{\fn{\$NO\_SIMPLE\_MATH}\texttt{ = 1;}}\\
Ordinarily simple mathematical expressions are set using the ordinary text font, 
but italiced. When part of the expression can not be represented this way,
an image is made \emph{of the whole formula}. 
This is called ``simple math''.
When \fn{\$NO\_SIMPLE\_MATH} is set, 
then \emph{all} mathematics is made into images, whether simple or not.

However, if the \Ve{math} extension is loaded, 
using the \Cs{html\_version} switch described \htmlref{earlier}{cs_htmlversion},
then specifying \Cs{no\_math} produces a quite different effect.
Now it is the special \HTMLtag{MATH} tags and entities which are cancelled. 
In their place a sophisticated scheme for parsing mathematical expressions 
is used. Images are made of those sub-parts of a formula which cannot be 
adequately expressed using (italiced) text characters and \HTMLtag{SUB} and
\HTMLtag{SUP} tags.
See \hyperref{the subsection on mathematics}{Section~}{}{maths} for more details.

\index{icons!copied to local directory}%

\item [-local\_icons\label{cs_localicons}]
\sameas{\fn{\$LOCAL\_ICONS}\texttt{ = 1;}}\\
A copy of each of the icons actually used within the document is placed in the
directory along with the \texttt{HTML} files and generated images.
This allows the whole document to be fully self-contained, within this directory;
otherwise the icons must be retrieved from a (perhaps remote) server.

\index{icons!alternative set}\index{icons!customised images}%
The icons are normally copied from a subdirectory of the \fn{\$LATEX2HTMLDIR},
set within \fn{latex2html.config}. An alternative set of icons can be used
by specifying a (relative) directory path in \fn{\$ALTERNATIVE\_ICONS}
to where the customised images can be found.


\index{initialisation!init file@\texttt{-init\_file}}%
\index{initialisation file!specified}%

\item [ -init\_file \Meta{file}\label{cs_initfile}]
Load the specified initialisation file. This \Perl{} file will be loaded after loading 
\fn{\$HOME/.latex2html-init}\,, or \fn{.latex2html-init} in the local
directory, if either file exists. It is read at the time the switch is processed,
so the contents of the file may change any of the values of any of the
variables that were previously established, as well as any default options.
More than one initialisation file can be read in this way.%

\index{index!codified links}%

\cbversion{98.1}\begin{changebar}
\index{forking!prevent}%
\item [ -no\_fork\label{cs_no_fork}]
\sameas{\fn{\$NOFORK}\texttt{ = 1;}}\\
When set this disables a feature in the early part of the processing
whereby some memory-intensive operations are performed by `forked'
child processes. Some single-task operating systems, such as \texttt{DOS},
do not support this feature. Having \fn{\$NOFORK} set then ensures
that unnecessary file-handles that are needed with the forked processes,
are not consumed unnecessarily, perhaps resulting in a fatal \Perl{} error.

\index{languages!dtd isolanguage}
\index{character set!dtd isolanguage}
\item [ -iso\_language \Meta{type}\label{cs_iso_language}]
This enables you to specify a different language type than
\texttt{'EN'} to be used in the DTD entries of the \texttt{HTML}
document, e.g. \texttt{'EN.US'}.
\end{changebar}%

\item [ -short\_index\label{cs_shortindex}]
\sameas{\fn{\$SHORT\_INDEX}\texttt{ = 1;}}\\
Creates shorter Index listings, using codified links; 
this is fully compatible with the \env{makeidx} \htmlref{package}{makeidx}.

\index{footnotes!on separate page}%
\index{footnotes!on same page}%
\index{footnotes!marker style}

\item [ -no\_footnode\label{cs_nofootnode}]
\sameas{\fn{\$NO\_FOOTNODE}\texttt{ = 1;}}\\
Suppresses use of a separate file for footnotes;
instead these are placed at the bottom of the \texttt{HTML} pages 
where the references occur.

When this option is used, it is frequently desirable to change the 
style of the marker used to indicate the presence of a footnote. 
This is done as in \LaTeX, using code such as follows.
\begin{quote}
\verb|\renewcommand{\thefootnote}{\arabic{footnote}}|
\end{quote}
All the styles \Lc{arabic}, \Lc{alph}, \Lc{roman}, \Lc{Alph} and \Lc{Roman}
are available.

\index{footnotes!numbering}
\cbversion{98.1}\begin{changebar}
\item [ -numbered\_footnotes\label{cs_numbered_footnotes}]
\sameas{\fn{\$NUMBERED\_FOOTNOTES}\texttt{ = 1;}}\\
If this is set you will get every footnote applied with a subsequent
number, to ease readability.
\end{changebar}

\index{address!signature}%
\index{address!using a subroutine}%

\item [ -address \Meta{author-address}\label{cs_address}]
\sameas{\fn{\$ADDRESS}\texttt{ = \char34}\Meta{author-address}\texttt{\char34;}}\\
Sign each page with this address.\html{\\}
See \fn{latex2html.config} for an example using \Perl{} code
to automatically include the date.

A user-defined \Perl{} subroutine called \texttt{\&custom\_address} 
can be used instead, if defined; it takes the value of \fn{\$ADDRESS} 
as a parameter, which may be used or ignored as desired.
At the time when this subroutine will be called, 
variables named \fn{\$depth}, \fn{\$title}, \fn{\$file}
hold the sectioning-level, title and filename of the \texttt{HTML} page
being produced; \fn{\$FILE} holds the name of the filename for the title-page
of the whole document.

\index{About this@\emph{About this document...}}

\item [ -info \Meta{string}\label{cs_infostring}]
\sameas{\fn{\$INFO}\texttt{ = \char34}\Meta{string}\texttt{\char34;}}\\
Generate a new section \emph{``About this document'' } containing information
about the document being translated. The default is to generate such a section 
with information on the original document, the date, the user and the translator.
An empty string (or the value `\texttt{0}') 
disables the creation of this extra section.\html{\\}
If a non-empty string is given, it will be placed as the contents of the 
\emph{``About this document'' } page instead of the default information.
%
\end{htmllist}


\subsubsection{Switches controlling Image Generation}
%\subsection[center]{Switches controlling Image Generation}
%
These switches affect whether images are created at all, whether
old images are reused on subsequent runs or new ones created afresh,
and whether anti-aliasing effects are used within the images themselves.

\begin{htmllist}\htmlitemmark{PinkBall}%
%
\index{ascii-mode}% 
\index{browser!character-based}%
\item [ -ascii\_mode \label{cs_asciimode}]
\sameas{\fn{\$ASCII\_MODE}\texttt{ = }\fn{\$EXTERNAL\_IMAGES}\texttt{ = 1;}}\\
Use only \texttt{ASCII} characters and do not include any images in the final output.
With \Cs{ascii\_mode} the output of the translator can be used on 
character-based browsers, such as \appl{lynx},
which do not support inlined images (via the \HTMLtag{IMG} tag).


\index{draft mode}%
\item [ -nolatex \label{cs_nolatex}]
\sameas{\fn{\$NOLATEX}\texttt{ = 1;}}\\
Disable the mechanism for passing unknown environments to \LaTeX{} for processing.
This can be thought of as ``draft mode'' which allows 
faster translation of the basic document structure and text, 
without fancy figures, equations or tables. 

(This option has been superseded by the \Cs{no\_images} option,
\htmlref{see below}{noimages}.)


\index{images!via hypertext links}%
\item [ -external\_images\label{cs_extimages}]
\sameas{\fn{\$EXTERNAL\_IMAGES}\texttt{ = 1;}}\\
Instead of including any generated images inside the document, 
leave them outside the document and provide hypertext links to them.


\index{images!links to PostScript@links to \PS}%
\item [ -ps\_images\label{cs_psimages}]
\sameas{\fn{\$PS\_IMAGES}\texttt{ = }\fn{\$EXTERNAL\_IMAGES}\texttt{ = 1;}}\\
Use links to external \PS\ files rather than inlined images
in the chosen graphics format.

\index{images!discard PostScript files@discard \PS\ files}%
\item [ -discard\label{cs_discard}]
\sameas{\fn{\$DISCARD\_PS}\texttt{ = 1;}}\\
The temporary \PS\ files are discarded immediately after they
have been used to create the image in the desired graphics format.

\index{images!disabled}%
\item [ -no\_images\label{cs_noimages}]
\sameas{\fn{\$NO\_IMAGES}\texttt{ = 1;}}\\
Do not attempt to produce any inlined images. 
The missing images can be generated ``off-line'' by restarting \latextohtml{}
with the option \Cs{images\_only}.

\index{images!generated off-line}%
\item [ -images\_only\label{cs_imagesonly}]
\sameas{\fn{\$IMAGES\_ONLY}\texttt{ = 1;}}\\
Try to convert any inlined images that were left over from previous
runs of \latextohtml. 


\index{image files!reuse option}%
\item [ -reuse \Meta{reuse\_option}\label{cs_reuseoptions}]
\sameas{\fn{\$REUSE}\texttt{ = }\Meta{reuse\_option}\texttt{;}}\\
This switch specifies the extent to which image files are to be shared
or recycled.\html{\\}
There are three valid options:
%
\begin{htmllist}\htmlitemmark{OrangeBall}
\index{image files!not recycled}\index{image-reuse!interactive}%
\item [\texttt{0}] 
Do not ever share or recycle image files.\\
This choice also invokes an interactive session 
prompting the user about what to do about
a pre-existing \texttt{HTML} directory, if it exists.
%
\index{image files!recycled}%
\item [\texttt{1}] 
Recycle image files from a previous run if they are available,\\
but do not share identical images that must be created in this run.
%
\index{image files!shared by default}%
\item [\texttt{2}] 
Recycle image files from a previous run and share identical
images from this run.\\
This is the default.
\end{htmllist}
\hyperref{A later section}{Section~}{}{recycling} provides 
additional information about image-reuse.


\index{image files!not reused}\index{image-reuse!interactive}%
\item [ -no\_reuse\label{cs_noreuse}]
\sameas{\fn{\$REUSE}\texttt{ = 0;}}\\
Do \emph{not} share or recycle images generated during previous translations.
This is equivalent to \Cs{reuse 0}.
(This will enable the initial interactive session during which the user is
asked whether to reuse the old directory, delete its contents or quit.)%


\index{images!anti-aliasing}\index{anti-aliasing!in figures}%
\item [ -antialias\label{cs_aalias}]
\sameas{\fn{\$ANTI\_ALIAS}\texttt{ = 1;}} (Default is 0.)\\
Generated images of \env{figure} environments and external \PS\ files
should use anti-aliasing. By default anti-aliasing is not
used with these images, since this may interfere with the contents
of the images themselves.

\index{images!anti-aliasing}\index{anti-aliasing!in math and text}%
\item [ -antialias\_text\label{cs_aaliastext}]
\sameas{\fn{\$ANTI\_ALIAS\_TEXT}\texttt{ = 1;}} (Default is 1.)\\
Generated images of typeset material such as text, mathematical formulas,
tables and the content of \env{makeimage} \htmlref{environments}{makeimage},
should use anti-aliasing effects.\html{\\}
The default is normally to use anti-aliasing for text, 
since the resulting images are much clearer on-screen. 
However the default may have been changed locally.


\index{images!no anti-aliasing}\index{anti-aliasing!turned off}%
\item [ -no\_antialias\label{cs_noaalias}]
\sameas{\fn{\$ANTI\_ALIAS}\texttt{ = 0;}} (Default is 0.)\\
Generated images of \env{figure} environments and external \PS\ files 
should not use anti-aliasing with images, 
though the local default may have been changed to use it.


\index{images!no anti-aliasing}\index{anti-aliasing!blurred printing}%
\item [ -no\_antialias\_text\label{cs_noaaliastext}]
\sameas{\fn{\$ANTI\_ALIAS\_TEXT}\texttt{ = 0;}} (Default is 1.)\\
Generated images of typeset material should \emph{not} use anti-aliasing effects.
Although on-screen images of text are definitely improved using anti-aliasing,
printed images can be badly blurred, even at 300dpi. 
Higher resolution printers do a much better job with the resulting grey-scale images.

\cbversion{98.1}\begin{changebar}
\index{images!white background}%
\item [ -white\label{cs_white}]
\sameas{\fn{\$WHITE\_BACKGROUND}\texttt{ = 1;}} (Default is 1.)\\
Ensures that images of \env{figure} environments have a white
background. Otherwise transparency effects may not work correctly.

\item [ -no\_white\label{cs_nowhite}]
\sameas{\fn{\$WHITE\_BACKGROUND}\texttt{ = '';}} (Default is 1.)\\
Cancels the requirement that \env{figure} environments have a white background.

\index{images!latex dump}%
\item [ -ldump\label{cs_ldump}]
\sameas{\fn{\$LATEX\_DUMP}\texttt{ = 1;}} (Default is 0.)\\
Use this if you want to speed up image processing during the 2nd
and subsequent runs of \latextohtml{} on the same document.
The translator now produces a \LaTeX{} format-dump of the preamble to
\fn{images.tex} which is used on subsequent runs.
This significantly reduces the startup time when
\LaTeX{} reads the \fn{images.tex} file for image-generation.

This process actually consumes additional time on the first run, 
since \LaTeX{} is called twice --- once to create the format-dump,
then again to load and use it.
The pay-off comes with the faster loading on subsequent runs. 
Approximately 1 Meg of disk space is consumed by the dump file.
\end{changebar}
%
\end{htmllist}


\subsubsection{Switches controlling Navigation Panels\label{navoptions}}
%\subsection[center]{Switches controlling Navigation Panels\label{navoptions}}
%
The following switches govern whether to include one or more navigation
panels on each \texttt{HTML} page, 
also which buttons to include within such a panel.

\begin{htmllist}\htmlitemmark{PinkBall}%

\index{navigation panel!disabled}
\item [ -no\_navigation\label{cs_nonavig}]
\sameas{\fn{\$NO\_NAVIGATION}\texttt{ = 1;}}\\
Disable the mechanism for putting navigation links in each page.\\
This overrides any settings of the \fn{\$TOP\_NAVIGATION},
\fn{\$BOTTOM\_NAVIGATION} and \fn{\$AUTO\_NAVIGATION} variables.

\index{navigation panel!at top of page}
\item [ -top\_navigation\label{cs_topnavig}]
\sameas{\fn{\$TOP\_NAVIGATION}\texttt{ = 1;}}\\
Put navigation links at the top of each page.    

\index{navigation panel!at bottom of page}%
\item [ -bottom\_navigation\label{cs_botnavig}]
\sameas{\fn{\$BOTTOM\_NAVIGATION}\texttt{ = 1;}}\\
Put navigation links at the bottom of each page as well as the top.

\index{navigation panel!according to number of words}
\item [ -auto\_navigation\label{cs_autonavig}]
\sameas{\fn{\$AUTO\_NAVIGATION}\texttt{ = 1;}}\\
Put navigation links at the top of each page. 
Also put one at the bottom of the page, if the page exceeds 
\fn{\$WORDS\_IN\_PAGE} number of words (default = 450).

\index{navigation panel!next logical page}
\item [ -next\_page\_in\_navigation\label{cs_nextinnavig}]
\sameas{\fn{\$NEXT\_PAGE\_IN\_NAVIGATION}\texttt{ = 1;}}\\
Put a link to the \textsl{next logical page} in the navigation panel.

\index{navigation panel!previous logical page}%
\item [ -previous\_page\_in\_navigation\label{cs_previnnavig}]~\\%
\sameas{\fn{\$PREVIOUS\_PAGE\_IN\_NAVIGATION}\texttt{ = 1;}}\\
Put a link to the \textsl{previous logical page} in the navigation panel.

\index{navigation panel!link to table-of-contents}
\item [ -contents\_in\_navigation\label{cs_continnavig}]
\sameas{\fn{\$CONTENTS\_IN\_NAVIGATION}\texttt{ = 1;}}\\
Put a link to the \textsl{table-of-contents} in the navigation panel if there is one.

\index{navigation panel!link to index}
\item [ -index\_in\_navigation\label{cs_indexinnavig}]
\sameas{\fn{\$INDEX\_IN\_NAVIGATION}\texttt{ = 1;}}\\
Put a link to the \textsl{index-page} in the navigation panel if there is an index.
%
\end{htmllist}


\subsubsection{Switches for Linking to other documents\label{otherdocs}}
%\subsubsection[center]{Switches for Linking to other documents\label{otherdocs}}
%
When processing a single stand-alone document, the switches described in this
section should not be needed at all, since the automatically generated navigation panels, 
described \hyperref{on the previous page}{in Section~}{, }{navoptions}
should generate all the required navigation links.
However if a document is to be regarded as part of a much larger document,
then links from its first and final pages, to locations in other parts
of the larger (virtual) document, 
need to be provided explicitly for some of the buttons in the navigation panel.

The following switches allow for such links to other documents,
by providing the title and URL for navigation panel hyperlinks.
In particular, the ``Document Segmentation''
\hyperref{feature}{feature of Section~}{}{Segmentation} 
necessarily makes great use of these switches.
It is usual for the text and targets of these navigation hyperlinks
to be recorded in a \fn{Makefile}, 
to avoid tedious typing of long command-lines having many switches.

\begin{htmllist}\htmlitemmark{PinkBall}%
%
\index{URL!universal resource locator}\index{URL!for UP@for \textsl{UP} button}%
\item [ -up\_url \Meta{URL}\label{cs_upURL}] 
\sameas{\fn{\$EXTERNAL\_UP\_LINK}\texttt{ = \char34}\Meta{URL}\texttt{\char34;}}\\
Specifies a universal resource locator (URL) to associate
with the \textsl{``UP''} button in the navigation panel(s).


\item [ -up\_title \Meta{string}\label{cs_uptitle}] 
\sameas{\fn{\$EXTERNAL\_UP\_TITLE}\texttt{ = \char34}\Meta{string}\texttt{\char34;}}\\
Specifies a title associated with this URL.%


\index{URL!for PREVIOUS@for \textsl{PREVIOUS} button}%
\item [ -prev\_url \Meta{URL}\label{cs_prevURL}] 
\sameas{\fn{\$EXTERNAL\_PREV\_LINK}\texttt{ = \char34}\Meta{URL}\texttt{\char34;}}\\
Specifies a URL to associate
with the \textsl{``PREVIOUS''} button in the navigation panel(s).%


\item [ -prev\_title \Meta{string}\label{cs_prevtitle}] 
\sameas{\fn{\$EXTERNAL\_PREV\_TITLE}\texttt{ = \char34}\Meta{string}\texttt{\char34;}}\\
Specifies a title associated with this URL.%


\index{URL!for NEXT@for \textsl{NEXT} button}%
\item [ -down\_url \Meta{URL}\label{cs_downURL}]
\sameas{\fn{\$EXTERNAL\_DOWN\_LINK}\texttt{ = \char34}\Meta{URL}\texttt{\char34;}}\\
Specifies a URL for the \textsl{``NEXT''} button in the navigation panel(s).%


\item [ -down\_title \Meta{string}\label{cs_downtitle}] 
\sameas{\fn{\$EXTERNAL\_DOWN\_TITLE}\texttt{ = \char34}\Meta{string}\texttt{\char34;}}\\
Specifies a title associated with this URL.%


\index{URL!for CONTENTS@for \textsl{CONTENTS} button}%
\item [ -contents \Meta{URL}\label{cs_contentsURL}] 
\sameas{\fn{\$EXTERNAL\_CONTENTS}\texttt{ = \char34}\Meta{URL}\texttt{\char34;}}\\
Specifies a URL for the \textsl{``CONTENTS''} button,
for document segments that would not otherwise have one.%


\index{URL!for INDEX@for \textsl{INDEX} button}%
\item [ -index \Meta{URL}\label{cs_indexURL}] 
\sameas{\fn{\$EXTERNAL\_INDEX}\texttt{ = \char34}\Meta{URL}\texttt{\char34;}}\\
Specifies a URL for the \textsl{``INDEX''} button, 
for document segments that otherwise would not have an index.

\index{URL!for external Bibliography}%
\item [ -biblio \Meta{URL}\label{cs_biblioURL}] 
\sameas{\fn{\$EXTERNAL\_BIBLIO}\texttt{ = \char34}\Meta{URL}\texttt{\char34;}}\\
Specifies the URL for the bibliography page to be used,
when not explicitly part of the document itself.
%
\end{htmllist}
\smallskip
\textbf{Warning: }
On some systems it is difficult to give text-strings \Meta{string} containing
space characters, on the command-line or via a \fn{Makefile}.
One way to overcome this is to use the corresponding variable.
Another way is to replace the spaces with underscores (\verb|_|).


\subsubsection{Switches for Help and Tracing}
%\subsubsection[center]{Switches for Help and Tracing}
%
The first two of the following switches are self-explanatory.
When problems arise in processing a document, 
the switches \Cs{debug} and \Cs{verbosity} will each
cause \latextohtml{} to generate more output to the screen.
These extra messages should help to locate the cause of the problem.

\begin{htmllist}\htmlitemmark{PinkBall}%
%
\index{options!defining temp directory}%
\item [ -tmp \Meta{path}\label{cs_tmpdir}]
Define a temporary directory to use for image generation.
If \Meta{path} is $0$, the standard temporary directory \texttt{/tmp}
is used.
%
\index{options!print option listing}%
\item [ -h(elp)\label{cs_optionlist}]
Print out the list of all command-line options.

\index{current version@current version of \latextohtml}%
\item [ -v\label{cs_showversion}] 
Print the current version of \latextohtml.

\index{debugging}\index{diagnostic information}%
\item [ -debug\label{cs_debugmode}]
\sameas{\fn{\$DEBUG}\texttt{ = 1;}}\\
Run in debug-mode, displaying messages and/or diagnostic information 
about files read, and utilities called by \latextohtml.
Shows any messages produced by these calls.%

\index{debugging!Perl@\Perl{} debugger}%

More extensive diagnostics, from the \Perl{} debugger, 
can be obtained by appending the string `\texttt{-w-}' 
to the 1st line of the \fn{latex2html} (and other) \Perl{} script(s).


\index{verbosity level}%
\item [ -verbosity \Meta{num}\label{cs_verbositylevel}] 
\sameas{\fn{\$VERBOSITY}\texttt{ = \Meta{num};}}\\
Display messages revealing certain aspects of the processing
performed by \latextohtml{} on the provided input file(s).
The \Meta{num} parameter can be an integer in the range 0 to 8.
Each higher value adds to the messages produced.

\begin{itemize}
\item[0. ]
No special tracing; as for versions of \latextohtml{} prior to \textsc{v97.1}\,.

\item[1. ] (This is the default.)
Show section-headings and the corresponding \texttt{HTML} file names,
and indicators that major stages in the processing have been completed.

\item[2. ]
Print environment names and identifier numbers, and new theorem-types.
Show warnings as they occur, and indicators for more stages of processing.
Print names of files for storing auxiliary data arrays.

\item[3. ]
Print command names as they are encountered and processed; also
any unknown commands encountered while pre-processing.
Show names of new commands, environments, theorems, counters and
counter-dependencies, for each document partition.

\item[4. ]
Indicate command-substitution the pre-process of math-environments.
Print the contents of unknown environments for processing in \LaTeX,
both before and after reverting to \LaTeX{} source.
Show all operations affecting the values of counters.
Also show links, labels and sectioning keys, at the stages of processing.

\item[5. ]
Detail the processing in the document preamble.
Show substitutions of new environments.
Show the contents of all recognised environments, both before
and after processing. Show the cached/encoded information for the image keys,
allowing two images to be tested for equality.

\item[6. ]
Show replacements of new commands, accents and wrapped commands.

\item[7. ]
Trace the processing of commands in math mode; both before and after.

\item[8. ]
Trace the processing of all commands, both before and after.

\end{itemize}
%
The command-line option sets an initial value only.
During processing the value of \fn{\$VERBOSITY} can be set
dynamically using the \verb|\htmltracing{...}| command, 
whose argument is the desired value, 
or by using the more general \Lc{HTMLset} command
as follows: \verb|\HTMLset{VERBOSITY}{|\Meta{num}\texttt{\char125}\,.
%
\end{htmllist}
\index{options|)}%
%

\subsubsection{Other Configuration Variables, without switches}
%\subsection[center]{Other Configuration Variables, without switches}
%
The configuration variables described here do not warrant having
a command-line switch to assign values.
Either they represent aspects of \latextohtml\ that are specific to the local site,
or they govern properties that should apply to \emph{all} documents,
rather than something that typically would change
for the different documents within a particular sub-directory.

Normally these variables have their value set within the \fn{latex2html.config} file.
In the following listing the defaults are shown, as the lines of \Perl{} code
used to establish these values.
If a different value is required, then these can be assigned from
a local \fn{.latex2html-init} initialisation file, 
without affecting the defaults for other users, 
or documents processed from other directories.

\begin{htmllist}\htmlitemmark{PurpleBall}%
%
\index{directory delimiter}%
\item [\fn{\$dd} ]
~holds the string to be used in file-names to delimit directories;
it is set internally to `\texttt{/}', unless the variable has already been given
a value within \fn{latex2html.config}~.

\noindent
\textbf{Note:} This value \emph{cannot} be set within a \fn{.latex2html-init} 
initialisation file, since its value \emph{needs to be known} 
in order to find such a file.

\index{installation variables!inserted@inserted by \fn{install-test}}%
\item [\fn{\$LATEX2HTMLDIR}] \latex{ }%
Read by the \fn{install-test} script from \fn{latex2html.config}, 
its value is inserted into the \fn{latex2html} \Perl{} script as part
of the installation process.

\item [\fn{\$LATEX2HTMLSTYLES}\texttt{ = \char34\$LATEX2HTMLDIR/styles\char34;}]
Read from the \fn{latex2html.config} file by \fn{install-test}\,,
its value is checked to locate the \fn{styles/} directory.

\item [\fn{\$LATEX2HTMLVERSIONS}\texttt{ = \char34\$LATEX2HTMLDIR/versions\char34;}]
The value of this variable should be set within \fn{latex2html.config} 
to specify the directory path where the version and extension files can be found.

\item [\fn{\$ALTERNATIVE\_ICONS}\texttt{ = '';}\label{alticons}]
This may contain the (relative) directory path to a set of customised icons
to be used in conjunction with the \htmlref{\Cs{local\_icons}}{cs_localicons} switch.

\index{installation variables!checked@checked by \fn{install-test}}%
\item [\fn{\$TEXEXPAND}\texttt{  = \char34\$LATEX2HTMLDIR/texexpand\char34;}]
Read by the \fn{install-test} \Perl{} script from \fn{latex2html.config}, 
its value is used to locate the \fn{texexpand} \Perl{} script.

\item [\fn{\$PSTOIMG}\texttt{ = \char34\$LATEX2HTMLDIR/pstoimg\char34;}]
Read by the \fn{install-test} \Perl{} script from \fn{latex2html.config}, 
its value is used to locate the \fn{pstoimg} \Perl{} script.

\item [\fn{\$IMAGE\_TYPE}\texttt{ = '\Meta{image-type}';}]
Set in \fn{latex2html.config}, the currently supported \Meta{image-type}s
are: \texttt{gif} and \texttt{png}.

\item [\fn{\$DVIPS}\texttt{ = 'dvips';}]
Read from \fn{latex2html.config} by \fn{install-test}, 
its value is checked to locate the \fn{dvips} program or script.

\smallskip\noindent
There could be several reasons to change the value here:
\begin{itemize}
\item
add a switch \texttt{-P}\Meta{printer} to load a specific configuration-file;\\
e.g. to use a specific set of \PS\ fonts, for improved image-generation.
\item
to prepend a path to a different version of \fn{dvips} than normally
available as the system default (e.g. the printing requirements are different).
\item
to append debugging switches, in case of poor quality images;\\
one can see which paths are being searched for fonts and other resources.
\item
to prepend commands for setting path variables that \fn{dvips} may need
in order to locate fonts or other resources.
\end{itemize}

If automatic generation of fonts is required, using \MF, the following
configuration variables are important.
%
\begin{htmllist}\htmlitemmark{OrangeBall}%
%
\item [\fn{\$PK\_GENERATION}\texttt{ = 1;}\label{pkgen}]
This variable must be set, to initiate font-generation; otherwise fonts
will be scaled from existing resources on the local system.\\
In particular this variable must \emph{not} be set, if one wishes to use
\PS\ fonts or other scalable font resources 
(see the \Cs{scalable\_fonts} \htmlref{switch}{scalefonts}).

\item [\fn{\$DVIPS\_MODE}\texttt{ = 'toshiba';}]
The mode given here must be available in the \fn{modes.mf} file,
located with the \MF\ resource files, perhaps in the \fn{misc/} subdirectory.

\item [\fn{\$METAFONT\_DPI}\texttt{ = 180;}]
The required resolution, in dots-per-inch, should be listed specifically
within the \fn{MakeTeXPK} script, called by \fn{dvips} to invoke \MF\ with
the correct parameters for the required fonts.
%
\end{htmllist}


\item [\fn{\$LATEX}\texttt{ = 'latex';}]
Read from \fn{latex2html.config} by \fn{install-test}, 
its value is checked to locate the \fn{latex} program or script.

\smallskip\noindent
If \LaTeX{} is having trouble finding style-files and/or packages, then
the default command can be prepended with other commands 
to set environment variables intended to resolve these difficulties;\\
e.g. 
\verb|$LATEX = 'setenv TEXINPUTS |\Meta{path to search}\verb| ; latex' |.

There are several variables to help control exactly which files are
read by \latextohtml{} and by \LaTeX{} when processing images:

\begin{htmllist}\htmlitemmark{OrangeBall}%
\item [\fn{\$TEXINPUTS}]
This is normally set from the environment variable of the same name.
If difficulties occur so that styles and packages are not being found, 
then extra paths can be specified here, to resolve these difficulties.

\item [\fn{\$DONT\_INCLUDE}]
This provides a list of filenames and extensions to \emph{not} include, 
even if requested to do so by an \Lc{input} or \Lc{include} command.\\
(Consult \fn{latex2html.config} for the default list.)

\item [\fn{\$DO\_INCLUDE}\texttt{ = '';}]
List of exceptions within the \fn{\$DONT\_INCLUDE} list.
These files \emph{are} to be read if requested by an \Lc{input}
or \Lc{include} command.
\end{htmllist}


\item [\fn{\$ICONSERVER}\texttt{ = '\Meta{URL}';}]
This is used to specify a URL to find the standard icons, as used for the
navigation buttons.\html{\\}
Names for the specific images size, as well as size information, 
can be found in \fn{latex2html.config}. The icons themselves can be replaced
by customised versions, provided this information is correctly updated
and the location of the customised images specified as the value
of \fn{\$ICONSERVER}.\\
When the \htmlref{\Cs{local\_icons}}{cs_localicons} switch is used,
so that a copy of the icons is placed with the \texttt{HTML} files and other
generated images, the value of \fn{\$ICONSERVER} is not needed within
the \texttt{HTML} files themselves. However it \emph{is} needed to find
the original icons to be copied to the local directory.

\item [\fn{\$NAV\_BORDER}\texttt{ = }\Meta{num}\texttt{;}]
The value given here results in a border, measured in points, around each icon.\html{\\}
A value of `\texttt{0}' is common, to maintain strict
alignment of inactive and active buttons in the control panels.


\item [\fn{\$LINKNAME}\texttt{ = '\char34index.\$EXTN\char34';}]
This is used when the \fn{\$NO\_AUTO\_LINK} \htmlref{variable}{noautolink} is empty, 
to allow a URL to the working directory to be sufficient to reach the main page 
of the completed document. 
It specifies the name of the \texttt{HTML} file which will be automatically
linked to the directory name.\html{\\}
The value of \fn{\$EXTN} is \texttt{.html} unless \fn{\$SHORTEXTN} is set, 
in which case it is \texttt{.htm}~. 

\item [\fn{\$LINKPOINT}\texttt{ = '\char34\$FILE\$EXTN\char34';}]
This specifies the name of the \texttt{HTML} file to be duplicated, or symbolically
linked, with the name specified in \fn{\$LINKNAME}.\\
At the appropriate time the value of \fn{\$FILE} is the document name, 
which usually coincides with the name of the working directory.

\index{character set}%
\index{character set!iso-latin@\texttt{\char34iso\_8859\_1\char34}}%
\item [\fn{\$CHARSET}\texttt{ = 'iso\_8859\_1';}\label{charset}]
This specifies the character set used within the \texttt{HTML} pages
produced by \latextohtml. If no value is set in a configuration
or initialisation file, the default value will be assumed.
The lowercase form \texttt{\$charset} is also recognised, but
this is overridden by the uppercase form.

\index{images!strange accents}%
\item [ \fn{\$ACCENT\_IMAGES}\texttt{ = 'large'};]
Accented characters that are not part of the ISO-Latin fonts
can be generated by making an image using \LaTeX.
This variable contains a (comma-separated) list of \LaTeX{} commands
for setting the style to be used when these images are made. 
If the value of this variable is empty then the accent is simply
ignored, using an un-accented font character (not an image) instead.
%
\end{htmllist}

\bigskip\noindent
Within the \fn{color.perl} package, the following variables are used to
identify the names of files containing specifications for named colors.
Files having these names are provided, in the \fn{\$LATEX2HTMLSTYLES}
directory, but they could be moved elsewhere, 
or replaced by alternative files having different names.
In such a case the values of these variables should be altered accordingly.
%
\begin{itemize}
\item []\fn{\$RGBCOLORFILE}\texttt{ = 'rgb.txt';}
\item []\fn{\$CRAYOLAFILE}\texttt{ = 'crayola.txt';}
\end{itemize}

\bigskip\noindent
The following variables may well be altered from the system defaults,
but this is best done using a local \fn{.latex2html-init} initialisation file,
for overall consistency of style within documents located at the same site,
or sites in close proximity.

\begin{htmllist}\htmlitemmark{PurpleBall}%
%
\item [\fn{\$default\_language}\texttt{ = 'english';}\label{isolanguage}]
This establishes which language code is to be placed within
the \HTMLtag{!DOCTYPE ... } tag that may appear at the beginning
of the \texttt{HTML} pages produced. Loading a package for
an alternative language can be expected to change the value of this variable.\html{\\}
See also the \fn{\$TITLES\_LANGUAGE} variable, described \htmlref{next}{doclanguage}.


\item [\fn{\$TITLES\_LANGUAGE}\texttt{ = 'english';}\label{doclanguage}]
This variable is used to specify the actual strings used for standard document
sections, such as ``Contents'', ``References'', ``Table of Contents'', etc.\html{\\}
Support for French and German titles is available in corresponding packages.
Loading such a package will normally alter the value of this variable, as well as 
the \fn{\$default\_language} variable described \htmlref{above}{isolanguage}.


\item [\fn{\$WORDS\_IN\_NAVIGATION\_PANEL\_TITLES}\texttt{ = 4;}]
Specifies how many words to use from section titles,
within the textual hyperlinks which accompany the navigation buttons.


\item [\fn{\$WORDS\_IN\_PAGE}\texttt{ = 450;}]
Specifies the minimum page length required before a navigation panel
is placed at the bottom of a page,
when the \fn{\$AUTO\_NAVIGATION} variable is set.


\item [\fn{\$CHILDLINE}\texttt{ = \char34<BR><HR>\char92n\char34;}]
This gives the \texttt{HTML} code to be placed between the child-links
table and the ordinary contents of the page on which it occurs.


\item [\fn{\$NETSCAPE\_HTML}\texttt{ = 0;}]
When set, this variable specifies that \texttt{HTML} code may be present which
does not conform to any official standard. 
This restricts the contents of any \HTMLtag{!DOCTYPE ... } tag which may be placed
at the beginning of the \texttt{HTML} pages produced.


\item [\fn{\$BODYTEXT}\texttt{ = '';}]
The value of this variable is used within the \HTMLtag{BODY ... } tag;
e.g. to set text and/or background colors.\html{\\}
It's value is overridden by the \verb|\bodytext| command, 
and can be added-to or parts changed using the \Lc{htmlbody} command
or \Lc{color} and \Lc{pagecolor} from the \env{color} package.


\item [\fn{\$INTERLACE}\texttt{ = 1;}\label{interlace}]
When set, interlaced images should be produced.\\
This requires \htmlref{graphics utilities}{graphicsutils} to be available 
to perform the interlacing operation.


\item [\fn{\$TRANSPARENT\_FIGURES}\texttt{ = 1;}\label{transfigs}]
When set, the background of images should be made transparent;
otherwise it is white.\html{\\}
This requires \htmlref{graphics utilities}{graphicsutils} %
to be available which can specify the color to be made transparent.


\item [\fn{\$FIGURE\_SCALE\_FACTOR}\texttt{ = 1.6;}\label{figscale}]
Scale factor applied to all images of \env{figure} and other environments, 
when being made into an image.

\smallskip\noindent
Note that this does not apply to recognised mathematics environments, 
which instead use the contents of \fn{\$MATH\_SCALE\_FACTOR} 
and \fn{\$DISP\_SCALE\_FACTOR} to specify scaling.


\item [\fn{\$MATH\_SCALE\_FACTOR}\texttt{ = 1.6;}\label{mathscale}]
Scale factor applied to all images of mathematics, both inline
and displayed. 
A value of 1.4 is a good alternative, with anti-aliased images.


\item [\fn{\$DISP\_SCALE\_FACTOR}\texttt{ = 1;}\label{dispscale}]
Extra scale factor applied to images of displayed math environments.\html{\\}
When set, this value multiplies \fn{\$MATH\_SCALE\_FACTOR}
to give the total scaling. 
A value of `1.2' is a good choice to accompany \verb|$MATH_SCALE_FACTOR = 1.4;|\,.


\item [\fn{\$EXTRA\_IMAGE\_SCALE}\label{ximagescale}]
This may hold an extra scale factor that can be applied to all generated images.\html{\\}
When set, it specifies that a scaling of \fn{\$EXTRA\_IMAGE\_SCALE} be applied 
when images are created, but to have their height and width recorded 
as the un-scaled size. This is to coax browsers into scaling the (usually larger) 
images to fit the desired size; when printed a better quality can be obtained.
Values of `1.5' and `2' give good print quality at 600dpi.


\item [\fn{\$PAPERSIZE}\texttt{ = 'a5';}\label{papersize}]
Specifies the size of a page for typesetting figures
or displayed math, when an image is to be generated.\\
This affects the lengths of lines of text within images.
Since images of text or mathematics should use larger sizes than
when printed, else clarity is lost at screen resolutions,
then a smaller paper-size is generally advisable.
This is especially so if both the \fn{\$MATH\_SCALE\_FACTOR}
and \fn{\$DISP\_SCALE\_FACTOR} scaling factors are being used,
else some images may become excessively large, including a lot
of blank space.


\item [\fn{\$LINE\_WIDTH}\texttt{ = 500;}\label{linewidth}]
Formerly specified the width of an image, when the contents were
to be right- or center-justified. (No longer used.)

\end{htmllist}

\bigskip

\noindent
The following variables are used to access the utilities required during image-generation. 
File and program locations on the local system are established by the \fn{configure-pstoimg} 
\Perl{} script and stored within \fn{\$LATEX2HTMLDIR/local.pm} as \Perl{} code, 
to be read by \fn{pstoimg} when required.

After running the \fn{configure-pstoimg} \Perl{} script it should not be necessary 
to alter the values obtained. Those shown below are what happens on the author's system;
they are for illustration only and do \emph{not} represent default values.
%

\begin{itemize}\label{graphicsutils}%
\item  \fn{\$GS\_LIB}\texttt{ = '/usr/local/share/ghostscript/4.02';}
\item  \fn{\$PNMCAT}\texttt{ = '/usr/local/bin/pnmcat';}
\item  \fn{\$PPMQUANT}\texttt{ = '/usr/local/bin/ppmquant';}
\item  \fn{\$PNMFLIP}\texttt{ = '/usr/local/bin/pnmflip';}
\item  \fn{\$PPMTOGIF}\texttt{ = '/usr/local/bin/ppmtogif';}
\item  \fn{\$HOWTO\_TRANSPARENT\_GIF}\texttt{ = 'netpbm';}
\item  \fn{\$GS\_DEVICE}\texttt{ = 'pnmraw';}
\item  \fn{\$GS}\texttt{ = '/usr/local/bin/gs';}
\item  \fn{\$PNMFILE}\texttt{ = '/usr/local/bin/pnmfile';}
\item  \fn{\$HOWTO\_INTERLACE\_GIF}\texttt{ = 'netpbm';}
\item  \fn{\$PBMMAKE}\texttt{ = '/usr/local/bin/pbmmake';}
\item  \fn{\$PNMCROP}\texttt{ = '/usr/local/bin/pnmcrop';}
\item  \fn{\$TMP}\texttt{ = '/usr/var/tmp';}
\end{itemize}

\noindent
The following variables are no longer needed, having been replaced by the
more specific information obtained using the \Perl{} script \fn{configure-pstoimg}.
\begin{itemize}
\item [] \fn{\$USENETPBM}\texttt{ = 1;}
\item [] \fn{\$PBMPLUSDIR}\texttt{ = '/usr/local/bin';}
\end{itemize}


%\filbreak
\subsection{Extending the Translator\label{sec:ext}}%
%\section{Extending the Translator\label{sec:ext}}%
\tableofchildlinks*\htmlrule
%
In \hyperref{an earlier section}{Section~}{}{sec:sty} is was seen how
the capabilities of \latextohtml{} can be extended to cope with
\LaTeX{} commands defined in packages and style-files.
This is in addition to defining simple macros and environments,
using \Lc{newcommand} and \Lc{newenvironment}.
The problem however, is that writing such extensions for packages 
requires an understanding of \Perl{} programming 
and of the way the processing in \latextohtml{} is organised. 

\index{new commands!ignore arguments}%
\index{new commands!arguments remain}%
\index{extensions!in initialisation file}
\smallskip

The default behaviour for unrecognised commands is 
for their arguments to remain in the \texttt{HTML} text,
whereas the commands themselves are passed on to \LaTeX,
in an attempt to generate an image. This is far from ideal,
for it is quite likely to lead to an error in \LaTeX{}
due to not having appropriate arguments for the command.

Currently there are several related mechanisms whereby a user 
can ask for particular commands and their arguments to be either 
\begin{itemize}
\item
\htmlref{ignored}{ignore} (via: \texttt{\&ignore\_commands}\,);
\item
passed on to \LaTeX{}  for processing \htmlref{as an image}{pass}\\
(via: \texttt{\&process\_commands\_in\_tex}\,);
\item
passed to \LaTeX{}  for processing as an image, 
\htmlref{as if an environment}{sensit}\\
(via: \texttt{\&process\_commands\_inline\_in\_tex}\,);
\item
passed on to \LaTeX, \htmlref{setting parameters}{setparam} 
to be used in subsequent images\\
(via: \texttt{\&process\_commands\_nowrap\_in\_tex}\,);
\item
processed as ``order-sensitive'', \htmlref{as if an environment}{wrapdef} 
rather than a command\\
(via: \texttt{\&process\_commands\_wrap\_deferred}\,);
\end{itemize}
The string beginning \texttt{\&.... } is the name of the \Perl{} subroutine 
that controls how the \LaTeX{} commands are to be subsequently treated during
processing by \latextohtml.
Towards the end of the \fn{latex2html} script, one finds a list
of \LaTeX{} commands to be handled by each of these subroutines.
These lists even include some common \TeX{} commands.

Analogous lists occur in most of the package extension files.
In many cases the commands are for fine-tuning the layout on
a printed page. They should simply be ignored; but any parameters
must not be allowed to cause unwanted characters to appear on
the \texttt{HTML} pages.
Customised extensions using these mechanisms may be included in the 
\fn{\$LATEX2HTMLDIR/latex2html.config} file, or in a personal 
\fn{\$HOME/.latex2html-init} initialisation file, 
as described \htmlref{next}{ignore}.



\subsubsection{Asking the Translator to Ignore Commands\label{ignore}}
%\subsection{Asking the Translator to Ignore Commands\label{ignore}}
\index{new commands!command ignored}\html{\\}%
Commands that should be ignored may be specified in the \fn{.latex2html-init}
file as input to the \verb|&ignore_commands| subroutine. 
Each command which is to be ignored should be on a separate line 
followed by compulsory or optional argument markers separated by 
{\verb|#|}'s e.g.\footnote{It is possible to add arbitrary(?) \Perl{} code
between any of the argument markers which will be executed when 
the command is processed. For this however a basic understanding of
how the translator works and of course \Perl{} is required.}:
\begin{quote}
%begin{latexonly}
\begin{small}
%end{latexonly}
\verb|<cmd_name1> # {} # [] # {} # [] ...|\\
\verb|<cmd_name2> # |\texttt{<<\emph{pattern}>>}\verb| # [] ...|
%begin{latexonly}
\end{small}
%end{latexonly}
\end{quote}
\verb|{}|'s mark compulsory arguments and \verb|[]|'s optional ones,
while \texttt{<<\emph{pattern}>>} denotes matching everything up to the
indicated string-pattern, given literally (e.g. \verb|\\endarray|);
spaces are ignored. Special characters
such as \verb|$|\,, \verb|&|\,, \verb|\| itself and perhaps some others,
need to be ``escaped'' with a preceding \verb|\|\,.


\index{new commands!arguments remain}\html{\\}%

Some commands may have arguments which should be left as text even
though the command should be ignored 
(e.g. \Lc{hbox}, \Lc{center}, etc.). 
In these cases arguments should be left unspecified.
Here is an example of how this mechanism may be used:
%begin{latexonly}
\begin{small}
%end{latexonly}
\begin{verbatim}
&ignore_commands( <<_IGNORED_CMDS_);
documentstyle # [] # {}
linebreak # []
mbox
<add your commands here>
_IGNORED_CMDS_
\end{verbatim}
%begin{latexonly}
\end{small}
%end{latexonly}


\subsubsection{Asking the Translator to Pass Commands to \LaTeX\label{pass}}%
%\subsection{Asking the Translator to Pass Commands to \LaTeX\label{pass}}%
\index{new commands!processed in LaTeX@processed in \LaTeX}\html{\\}%
Commands that should be passed on to \LaTeX{}  for processing because
there is no direct translation to \texttt{HTML} may be specified in the 
\fn{.latex2html-init}
file as input to the \verb|process_commands_in_tex| subroutine.
The format is the same as that for specifying commands to be ignored.
Here is an example:
%begin{latexonly}
\begin{small}
%end{latexonly}
\begin{verbatim}
&process_commands_in_tex (<<_RAW_ARG_CMDS_);
fbox # {}
framebox # [] # [] # {}
<add your commands here>
_RAW_ARG_CMDS_
\end{verbatim}
%begin{latexonly}
\end{small}
%end{latexonly}


\subsubsection{Handling ``order-sensitive'' Commands\label{sensit}}%
%\subsection{Handling ``order-sensitive'' Commands\label{sensit}}%
\index{new commands!order-sensitive}%
\index{new commands!order-sensitive!processed in LaTeX@processed in \LaTeX}%
\index{new commands!processed in LaTeX@processed in \LaTeX!order-sensitive}%
Some commands need to be passed to \LaTeX, but using the 
\verb|&process_commands_in_tex| subroutine gives incorrect results. 
This may occur when the command is ``order-sensitive'',
using information such as the value of a counter or a boolean expression 
(or perhaps requiring a box to have been constructed and saved).
Try using the \verb|&process_commands_inline_in_tex| subroutine instead.
Commands declared this way are first ``wrapped'' within a dummy environment,
which ensures that they are later processed in correct order with other 
environments and order-sensitive commands.

\index{order-sensitive!setting parameters}\label{setparam}\html{\\}%

Other commands may need to be passed to \LaTeX, not to create an image
themselves, but to affect the way subsequent images are created.
For example a color command such as \verb|\color{red}| should set the
text-colour to `red' for all subsequent text and images. This must
be sent to \LaTeX{} so that it is processed at exactly the right time;
i.e. before the first image required to be `red' but following any
images that are not intended to be affected by this colour-change.\html{\\}
The subroutine \verb|process_commands_nowrap_in_tex| is designed
specifically to meet such requirements.

\index{order-sensitive!wrap-deferred}\label{wrapdef}\html{\\}%

Commands can be order-sensitive without having to be passed to \LaTeX.
Indeed even if a \Perl{} subroutine has been carefully written to process
the command, it may still give wrong results if it is order-sensitive,
depending on the values of counters, say.
To handle such cases there is the \verb|&process_commands_wrap_deferred|
subroutine. This also ``wraps'' the command within a dummy environment,
but when that environment is processed the contents are not sent to \LaTeX,
as in the previous case. All of the standard \LaTeX{} commands to change, 
set or read the values of counters are handled in this way.



\subsection{Customising the Navigation Panels\label{sec:navpanel}}%
%\section{Customising the Navigation Panels\label{sec:navpanel}}%
\index{navigation panel|(}%
\index{navigation panel!customisable}\html{\\}%
The navigation panels are the strips containing ``buttons'' and text 
that appears at the top and perhaps at the bottom 
of each generated page and provides hypertext links 
to other sections of a document. 
Some of the options and variables that control whether and 
where it should appear are presented in \hyperref{an earlier section}%
{Section~}{}{navoptions}. 

\index{navigation panel!customised buttons}%
\index{htmladdtonavigation@\Lc{htmladdtonavigation}}\html{\\}%
A simple mechanism for appending customised buttons to the navigation
panel is provided by the command \Lc{htmladdtonavigation}. 
This takes one argument which \latextohtml{} appends to the navigation panel. 
For example,
%begin{latexonly}
\begin{small}
%end{latexonly}
\begin{verbatim}
\htmladdtonavigation{\htmladdnormallink
   {\htmladdimg{http://server/mybutton.gif}}{http://server/link}}
\end{verbatim}
%begin{latexonly}
\end{small}
%end{latexonly}
will add an active button \fn{mybutton.gif} pointing to the specified location.

\index{navigation panel!order of buttons}\html{\\}\noindent
Apart from these facilities it is also 
possible to specify completely what appears in the navigation panels
and in what order. As each section is processed, \latextohtml{}
assigns relevant information to a number of global variables.
These variables are used by the
subroutines \texttt{top\_navigation\_panel} and
\texttt{bottom\_navigation\_panel},
where the navigation panel is constructed as a string consisting of
these variables and some formatting information. 

\index{navigation panel!Perl subroutines@\Perl{} subroutines}\html{\\}\noindent
These subroutines can be redefined in a system- 
or user-configuration file (respectively, 
\fn{\$LATEX2HTMLDIR/latex2html.config} and \fn{\$HOME/.latex2html-init}). 
Any combination of text, \texttt{HTML} tags, 
and the variables mentioned below is acceptable.

\medskip\noindent
The control-panel variables are:\nobreak

\medskip\nobreak
\index{navigation panel!buttons}%
\index{navigation panel!iconic links}\html{\\}
\noindent
\textbf{Iconic links (buttons):}
\begin{htmllist}\htmlitemmark{RedBall}\addtolength{\leftskip}{15pt}%
\item [\fn{\$UP}]  Points up to the ``parent'' section;
\item [\fn{\$NEXT}] Points to the next section;
\item [\fn{\$NEXT\_GROUP}] Points to the next ``group'' section;
\item [\fn{\$PREVIOUS}] Points to the previous section;
\item [\fn{\$PREVIOUS\_GROUP}] Points to the previous ``group'' section;
\item [\fn{\$CONTENTS}] Points to the contents page if there is one;
\item [\fn{\$INDEX}] Points to the index page if there is one.
\end{htmllist}

\index{navigation panel!section titles}%
\index{navigation panel!textual links}\html{\\}\noindent
\textbf{Textual links (section titles):}
\begin{htmllist}\htmlitemmark{PinkBall}\latex{\addtolength{\leftskip}{15pt}}%
\item [\fn{\$UP\_TITLE}]  Points up to the ``parent'' section;
\item [\fn{\$NEXT\_TITLE}] Points to the next section;
\item [\fn{\$NEXT\_GROUP\_TITLE}] Points to the next ``group'' section;
\item [\fn{\$PREVIOUS\_TITLE}] Points to the previous section;
\item [\fn{\$PREVIOUS\_GROUP\_TITLE}] Points to the previous ``group'' section.
\end{htmllist}
If the corresponding section exists, each iconic button will contain an
active link to that section. If the corresponding section does
not exist, the button will be inactive. If the section corresponding
to a textual link does not exist then the link will be empty.

The ``next group'' and ``previous group'' are rarely used, 
since it is usually possible to determine
which are the next/previous logical pages in a document.
However these may be needed occasionally with segmented documents,
when the segments have been created with different values
for the \fn{\$MAX\_SPLIT\_DEPTH} variable. 
This is quite distinct from the segmented document effect 
in which the first page of one segment may have its `\emph{PREVIOUS}'
button artificially linked to the first page of the previous segment, 
rather than the last page.


\index{navigation panel!number of words}\html{\\}\noindent
The number of words that appears in each textual link
is controlled by the variable \fn{\$WORDS\_IN\_NAVIGATION\_PANEL\_TITLES}
which may also be changed in the configuration files.

\index{navigation panel!example}\html{\\}\noindent
Below is an example of a navigation panel for the bottom of \texttt{HTML} pages.
(Note that the ``\texttt{.}'' is \Perl{}'s string-concatenation operator 
and ``\texttt{\#}'' signifies a comment.)
%begin{latexonly}
\begin{small}
%end{latexonly}
\begin{verbatim}
sub bot_navigation_panel {
    #  Start with a horizontal rule and descriptive comment
    "<HR>\n" . "<!--Navigation Panel-->".
    # Now add a few buttons, with a space between them
       "$NEXT $UP $PREVIOUS $CONTENTS $INDEX $CUSTOM_BUTTONS" . 
    # Line break    
       "\n<BR>" .       
    # If ``next'' section exists, add its title to the navigation panel
       ($NEXT_TITLE ? "\n<B> Next:</B> $NEXT_TITLE" : undef) .   
    # Similarly with the ``up'' title ...
       ($UP_TITLE ? "\n<B>Up:</B> $UP_TITLE\n" : undef) . 
    # ... and the ``previous'' title
       ($PREVIOUS_TITLE ? "\n<B> Previous:</B> $PREVIOUS_TITLE\n" : undef) . 
}
\end{verbatim}
%begin{latexonly}
\end{small}
%end{latexonly}
Note that extra buttons may be included by defining suitable code for the container
\fn{\$CUSTOM\_BUTTONS}\,. 
The use of explicit `newline' (\verb|\n|) characters is not necessary for the
on-screen appearance of the navigation panel within a browser window. However 
it maintains an orderly organisation within the \texttt{.html} files themselves,
which is helpful if any hand-editing is later required, 
or simply to read their contents.
The corresponding subroutine for a navigation-panel at the top of a page
need not use the rule \HTMLtag{HR}, and would require a break (\HTMLtag{BR})
or two at the end, to give some visual separation from the following material.
\index{navigation panel|)}
\index{extensions|)}




             % Input counters and section
\label{sec:man}\index{manual}
\end{htmlonly}
\startdocument
%
%\section{User Manual\label{sec:man}\index{manual}}
\label{sec:man}
\index{customised layout}
\index{getting started}
\html{\\}\noindent
\latextohtml{} is a program for creating hyperlinked sets of \texttt{HTML} pages
from a document marked-up using \LaTeX{} commands.
Previous sections have discussed the results of specific \LaTeX{} commands.
In this section we discuss instead the extensive range
of command-line switches and options, and other aspects of \Perl{} code,
that affect the way the translation is performed.

\medskip\noindent
To use \latextohtml{} to translate a file~\Meta{file}\texttt{.tex }
containing \LaTeX{} commands, type:
\begin{quote}
\texttt{latex2html} \Meta{file}\texttt{.tex} 
\end{quote}

\index{browser@browser\label{IIIbrowser}}%
\index{viewer|see{\htmlref{browser}{IIIbrowser}}}%
\index{browser!NCSA Mosaic@\textsl{NCSA Mosaic}}%
\index{browser!Netscape Navigator@\textsl{Netscape Navigator}}\html{\\}%
\noindent
This will create a new directory called \Meta{file} which will contain 
the generated \texttt{HTML} files, some log files and possibly some images.
To view the result use an \texttt{HTML} browser, such as 
\appl{NCSA Mosaic} or \appl{Netscape Navigator},
on the main \texttt{HTML} file which is \texttt{\Meta{file}/\Meta{file}.html}\,. 
The file will contain navigation links 
to the other parts of the generated document.
%
\index{file suffix!optional}%
\index{file suffix!.tex@\texttt{.tex} or\texttt{.doc}}%
The \texttt{.tex} suffix is optional and will be supplied
by the program if it is omitted by the user.
Other suffixes are acceptable also, such as \texttt{.doc}\,.

\medskip
\index{options!command-line options}\html{\\}\noindent
It is possible to customise the output from \latextohtml{} using a number of 
\hyperref{command-line options}{command-line options (see Section~}{)}{options}
with which you can specify:
\begin{itemize}
\item
how to break up the document; 
\item
where to put the generated files, and what are their names;
\item
the title for the whole document; 
\item
the signature at the end of each page is; 
\item
how many navigation panels to provide, what links to put in them;
\item
what other documents this one links to;
\item 
extra information to include about the document;
\item 
whether to retain the original \LaTeX{} section-numbering scheme;
\item
and many other things that affect how the information is obtained,
processed or displayed in the resulting \texttt{.html} files and images.
\end{itemize}

\medskip
\index{manual!short on-line}\html{\\}\noindent
The \latextohtml{} script includes a short manual
which can be viewed with the command: 
\begin{verbatim}
  perldoc latex2html
\end{verbatim}


\subsection{Developing Documents using \latextohtml{}\label{devel}}
%\section{Developing Documents using \latextohtml{}\label{devel}}
%
Although any document containing \LaTeX{} commands can be translated
by the \latextohtml{} translator, the best results are obtained when
that document is itself a valid \LaTeX{} document.
Indeed it is generally a good idea to develop documents so that
they produce good readable results in both the \LaTeX{} typeset
version as well as a set of \texttt{HTML} pages.
This is not just a nicety; 
there are several good practical reasons for doing this.

\begin{htmllist}\htmlitemmark{BlueBall}
%
\item[\LaTeX{} macros: ]
The macro commands that \latextohtml{} recognises are based upon
corresponding commands for \LaTeX. If one tries to use syntax that is
incorrect for \LaTeX{} then there is no reason why \latextohtml{} should
be able to ``get it right'', by somehow recognising the true intent.


\index{error checking!using LaTeX@using \LaTeX}%
\item[error checking: ]
Processing the document first using \LaTeX{} is the easiest, 
and quickest, way to check for valid syntax.\html{\\}
Whereas \LaTeX{} stops at each error (when run in interactive mode),
allowing a fix to be made ``on the spot'' or a ``stop-fix-restart'',
\latextohtml{} does not stop when it detects an error in \LaTeX{} syntax.
Useful messages are given concerning missing or unmatched braces,
but other apparent anomalies generate only warning messages, 
which are saved to the end. 
(Some warnings are also shown immediately when the \fn{\$VERBOSITY} 
\htmlref{variable}{verbosity} is set to at least 3.)\html{\\}
In practice it can be much quicker to test for invalid syntax using \LaTeX{} before 
attempting to use the \latextohtml{} translator.

Furthermore, \LaTeX{} warns of cross-reference labels that have not been defined. 
This is useful to help avoid having hyperlinks which point to nowhere.

\index{error checking!missing braces}%
\index{error checking!unmatched brace}\html{\\}\noindent
The case of missing braces, or an unmatched opening brace, 
is an error that \latextohtml{} actually handles better than \LaTeX{} 
(or rather, the underlying \TeX{} processor).
Whereas \TeX{} only detects an error when something else goes wrong
later in the processing, \latextohtml{} shows where the
unmatched brace itself occurs. 


\index{auxiliary file}
\item[auxiliary file: ]
Some information that \latextohtml{} might need is normally
read from the \texttt{.aux} file for the document being processed;
or perhaps from the \texttt{.aux} file of a different document,
of which the current document is just a portion. 
Clearly a valid \LaTeX{} run is required to produce the \texttt{.aux} file.

\html{\\}\noindent
Of course if no information in the \texttt{.aux} file is actually used,
then this \LaTeX{} run can be neglected.
Also, if the \texttt{.aux} file has already been created, and edits
are made on the source which do not alter the information stored
within the \texttt{.aux} file, then there is no need for a fresh
\LaTeX{} run (except for the purposes of error-checking).


\index{bibliography}
\item[bibliography: ]
Suppose the document requires a bibliography, or list of references,
which is to be prepared using Bib\TeX.
\latextohtml{} reads citation information from the \texttt{.aux} file,
and can import the bibliography itself from the \texttt{.bbl} file.
However these must first be created using \LaTeX.



\index{document segmentation}
\item[document segmentation: ]
With the document segmentation technique, 
discussed fully in \hyperref{a later section}{Section~}{}{Segmentation},
it is vitally important that the full document processes correctly
in \LaTeX. The desired effect is that of a single large document,
whereas the pieces will actually be processed separately.
To achieve this, \LaTeX{} writes vital information into special
\texttt{.ptr} files. Like the \texttt{.aux} file, these files
are read by \latextohtml{} to get section-numbering and other
information to be used while processing each segment.


\index{printing!hyperlink to typeset version}%
\index{printing}\html{\\}\noindent
\item[print quality: ]
When a document contains automatically-generate images, these images
are usually bitmaps designed for viewing on-screen. In general the 
resolution of these is too poor to give a good result when printed
on a high-resolution laser-printer.
Thus if it is likely that the reader will want to obtain
a printed version of your document, then it is best to include
a hyperlink to the typeset \texttt{.dvi} version, 
or to a \PS\ conversion of the \texttt{.dvi} file. 
(In either case, a link to a compressed version is even better.)

\end{htmllist}



\subsection{Command-Line Options\label{options}}%
%\section{Command-Line Options\label{options}}%
\index{options@options\label{IIIoptions}}%
\index{switches|see{\htmlref{options}{IIIoptions}}}%
\index{options|(}%
\index{options!environment variables}%
\index{options!set in initialisation file}%
\html{\\}\noindent
The command-line options described here can be used to change the
default behaviour of \latextohtml. 
Alternatively, often there is a corresponding environment variable
whose value may be set or changed within a \fn{.latex2html-init} 
initialisation file, in order to achieve the same result.\html{\\}
There are so many options that they are listed here in groups, 
according to the nature of the effects they control.
When a large number of such options are required for the processing of a document,
it is usual to store the switches and their desired settings within a \fn{Makefile},
for use with the Unix \fn{make} utility. Now a simple command such as:
\begin{verbatim}
  make mydocument
\end{verbatim}
can initiate a call to \fn{latex2html} that would otherwise take many lines of typing.
Indeed it could instigate several such calls to \latextohtml, or to other programs
such as \LaTeX{} and Bib\TeX, \fn{dvips} and others.
The document segmentation feature, discussed in 
\hyperref{another section}{Section~}{}{Segmentation},
uses this technique extensively.


\subsubsection{Options controlling Titles, File-Names and Sectioning}
%\subsection{Options controlling Titles, File-Names and Sectioning}
\index{options!titles}\index{options!file-names}\index{options!sectioning}%
%
\index{document!title}%
%
\begin{htmllist}\htmlitemmark{PurpleBall}%
\item [ -t \Meta{top-page-title}\label{cs_toppagetitle}]
\sameas{\fn{\$TITLE}\texttt{ = \char34}\Meta{top-page-title}\texttt{\char34;}}\\
Name the document using this title.

\index{filenames!.htm@\texttt{.htm} extension}%

\item [ -short\_extn\label{cs_shortextn}]
\sameas{\fn{\$SHORTEXTN}\texttt{ = 1;}}\\
Use a filename prefix of \texttt{.htm } for the produced
\texttt{HTML} files. This is particularly useful for creating pages
to be stored on CD-ROM or other media, to be used with 
operating systems that require a 3-character extension.%

\index{filenames!long names}

\item [ -long\_titles \Meta{num}\label{cs_longtitles}]
\sameas{\fn{\$LONG\_TITLES}\texttt{ = }\Meta{num}\texttt{;}}\\
Instead of the standard  names: \texttt{node1.html}, \texttt{node2.html},... 
the filenames for each \texttt{HTML} page are constructed
from the first \Meta{num} words of the section heading
for that page, separated by the `\texttt{\_}' character.\html{\\}
Commas and common short words (\texttt{a an to by of and for the})
are omitted from both title and word-count.

\smallskip\noindent
\textbf{Warning:} Use this switch with great caution.
Currently there are no checks for uniqueness of names or overall length. 
Very long names can easily result from using this feature.%

\index{filenames!customised}%
\index{filenames!custom title hook@\texttt{custom\_title\_hook}}

\item [ -custom\_titles\label{cs_customtitles}]
\sameas{\fn{\$CUSTOM\_TITLES}\texttt{ = 1;}}\\
Instead of the standard  names: \texttt{node1.html}, \texttt{node2.html}, ... 
the filenames for each \texttt{HTML} page are constructed using a \Perl{}
subroutine named \texttt{ custom\_title\_hook}~. 
The user may define his/her own version of this subroutine, 
within a \fn{.latex2html-init} file say, 
to override the default (which uses the standard names).
This subroutine takes the section-heading as a parameter 
and must return the required name, or the empty string (default).

\index{output!redirect to directory}

\item [ -dir \Meta{output-directory}\label{cs_outputdir}]
\sameas{\fn{\$DESTDIR}\texttt{ = \char34}\Meta{output-directory}\texttt{\char34;}}\\
Redirect the output to the specified directory.\html{\\}
The default behaviour is to create (or reuse) a directory having the same
name as the prefix of the document being processed.%

\index{output!in current directory}%
\index{output!default directory}

\item [ -no\_subdir\label{cs_nosubdir}]
\sameas{\fn{\$NO\_SUBDIR}\texttt{ = 1;}}\\
Place the generated \texttt{HTML} files into the current directory.
This overrides any \fn{\$DESTDIR} setting.

\index{filenames!prefix}%
\index{temporary files}%

\item [ -prefix \Meta{filename-prefix}\label{cs_prefix}]%
\sameas{\fn{\$PREFIX}\texttt{ = \char34}\Meta{filename-prefix}\texttt{\char34;}}\\
The \Meta{filename-prefix} will be prepended to all 
\texttt{.gif}, \texttt{.pl} and \texttt{.html} files produced, 
except for the top-level \texttt{.html} file;  
it may include a (relative) directory path. 
This will enable multiple products of \latextohtml{} to
peacefully coexist in the same directory.
However, do \emph{not} attempt to \emph{simultaneously} run multiple 
instances of \latextohtml{} using the same output directory, 
else various temporary files will overwrite each other.

\index{filenames!auto-prefix}%

\item [ -auto\_prefix\label{cs_autoprefix}]
\sameas{\fn{\$AUTO\_PREFIX}\texttt{ = 1;}}\\
Constructs the prefix as `\Meta{title}--'
to be prepended to all the files produced, 
where \Meta{title} is the name of the \LaTeX{} file being processed.
(Note the `--' in this prefix.)\\
This overrides any \fn{\$PREFIX} setting.

\index{filenames!automatic link to directory name}%
\index{filenames!index.html@\fn{index.html}}%

\item [ -no\_auto\_link\label{cs_noautolink}]
\sameas{\fn{\$NO\_AUTO\_LINK}\texttt{ = 1;}}\\
If \fn{\$NO\_AUTO\_LINK} is empty and variables \fn{\$LINKPOINT} 
and \fn{\$LINKNAME} are defined appropriately
(as is the default in the \fn{latex2html.config} file),
then a hard link to the main \texttt{HTML} page is produced,
using the name supplied in \fn{\$LINKNAME}. Typically this is \fn{index.html}; 
on many systems a file of this name will be used, if it exists,
when a browser tries to view a URL which points to a directory.
On other systems a different value for \fn{\$LINKNAME} may be appropriate.
Typically \fn{\$LINKPOINT} has value \texttt{\$FILE.html}, but
this may also be changed to match whichever \texttt{HTML} page
is to become the target of the automatic link.\\
Use of the \Cs{no\_auto\_link} switch \emph{cancels} 
this automatic linking facility, 
when not required for a particular document.

\index{sections!in separate files}%

\item [ -split \Meta{num}\label{cs_splitdepth}]
\sameas{\fn{\$MAX\_SPLIT\_DEPTH}\texttt{ = }\Meta{num}\texttt{;}}
 (default is \texttt{8})\\
Stop splitting sections into separate files at this depth.
Specifying \Cs{split 0} will put the entire document into a 
single \texttt{HTML} file. See \htmlref{below}{seclevels} for
the different levels of sectioning. Also see the next item for
how to set a ``relative'' depth for splitting.%

\item [ -split +\Meta{num}\label{cs_reldepth}]
\sameas{\fn{\$MAX\_SPLIT\_DEPTH}\texttt{ = }\textrm{-}\Meta{num}\texttt{;}}
 (default is \texttt{8})\\
The level at which to stop splitting sections is calculated ``relative to'' 
the shallowest level of sectioning that occurs within the document.
For example, if the document contains \Lc{section} commands, but no \Lc{part}
or \Lc{chapter} commands, then \Cs{split +1} will cause splitting at each \Lc{section}
but not at any deeper level; whereas \Cs{split +2} or \Cs{split +3} also split down to 
\Lc{subsection} and \Lc{subsubsection} commands respectively.
Specifying \Cs{split +0} puts the entire document into a single \texttt{HTML} file.


\index{depth!child nodes revealed}%
\index{components!specify depth}

\index{child-links table}%
\item [ -link \Meta{num}\label{cs_linkdepth}]
\sameas{\fn{\$MAX\_LINK\_DEPTH}\texttt{ = }\Meta{num}\texttt{;}}
 (default is \texttt{4})\\
For each node, create links to child nodes down to this much 
deeper than the node's sectioning-level.\html{\\}
Specifying \Cs{link 0} will show no links to child nodes from that page,\html{\\}
\Cs{link 1} will show only the immediate descendents, etc.\html{\\}
A value at least as big as that of the 
\htmlref{\Cs{split \Meta{num}}}
{splitdepth} depth will
produce a mini table-of-contents (when not empty) on each page,
for the tree structure rooted at that node.%

\index{child-links table!at bottom of page}%

When the page has a sectioning-level less than the \Cs{split} depth,
so that the a mini table-of-contents has links to other \texttt{HTML} pages, 
this table is located at the bottom of the page, 
unless placed elsewhere using the \Lc{tableofchildlinks} command.%

\index{child-links table!at top of page}%
\index{child-links table!tableofchildlinks@\texttt{tableofchildlinks}}

On pages having a sectioning-level just less than the \Cs{split} depth 
the mini table-of-contents contains links to subsections etc.
occurring on the \emph{same} \texttt{HTML} page. 
Now the table is located at the \emph{top} of this page, 
unless placed elsewhere using the \Lc{tableofchildlinks} command.

\index{table of contents!depth}%

\item [ -toc\_depth  \Meta{num}\label{cs_tocdepth}]
\sameas{\fn{\$TOC\_DEPTH}\texttt{ = }\Meta{num}\texttt{;}}
 (default is \texttt{4})\\
Sectioning levels down to \Meta{num} are to be included 
within the Table-of-Contents tree.%

\index{table of contents!star-sections shown}%
\index{table of contents!star-sections not listed}

\item [ -toc\_stars\label{cs_tocstars}]
\sameas{\fn{\$TOC\_STARS}\texttt{ = 1;}}\\
Sections created using the starred-form of sectioning commands
are included within the Table-of-Contents.
As with \LaTeX, normally such sections are \emph{not} listed.
%

\index{sections!show section numbers}%
\index{sections!numbers not shown}

\item [ -show\_section\_numbers\label{cs_showsecnums}]
\sameas{\fn{\$SHOW\_SECTION\_NUMBERS}\texttt{ = 1;}}\\
Show section numbers. By default section numbers are \emph{not} shown, 
so as to encourage the use of particular sections as stand-alone documents. 
\strikeout{In order to be shown, 
section titles must be unique and must not contain inlined graphics.}

\index{segmentation!unsegment}

\item [ -unsegment\label{cs_unsegment}]
\sameas{\fn{\$UNSEGMENT}\texttt{ = 1;}}\\
Treat a segmented document (see the section about \htmlref{document
  segmentation}{Segmentation}) like it were not segmented.
This will cause the translator to concatenate all segments and process
them as a whole.
You might find this useful to check a segmented document for
consistency.
%
\end{htmllist}

\goodbreak
\index{sectioning levels}\index{levels!sectioning}\label{seclevels}%
\html{\\}\noindent
For all documents the sectioning levels referred to above are:\nobreak
\begin{center}
\begin{tabular}{ll}
\textbf{0} & document\\
\textbf{1} & part\\
\textbf{2} & chapter\\
\textbf{3} & section\\
\textbf{4} & subsection\\
\textbf{5} & subsubsection\\
\textbf{6} & paragraph\\
\textbf{7} & subparagraph\\
\textbf{8} & subsubparagraph
\end{tabular}
\end{center}
These levels apply even when the document contains no sectioning for
the shallower levels; e.g. no \Lc{part} or \Lc{chapter} commands is most common,
especially when using \LaTeX's \env{article} document-class.



\subsubsection{Options controlling Extensions and Special Features}
\index{extensions|(}%
\index{options!extensions}\index{options!special features}%
\html{\\}\noindent
The switches described here govern the type of \texttt{HTML} code that
can be generated, and how to choose between the available options 
when there are alternative strategies for implementing portions of \LaTeX{} code. 

\begin{htmllist}\htmlitemmark{PurpleBall}%
%
\index{HTML version@\texttt{HTML} version}%
\index{HTML@\texttt{HTML}!non-standard extensions}%
\item [ -html\_version \texttt{(2.0|3.0|3.2)[,(math|i18n|table)]*}\label{cs_htmlversion}]
~\\\sameas{\fn{\$HTML\_VERSION}\texttt{ = \char34 ... \char34;}}\\
This specifies both the \texttt{HTML}~\htmlref{version}{version} to generate, 
and any extra (non-standard) \texttt{HTML} features that may be required.\html{\\}
The version number corresponds to a published DTD for an \texttt{HTML} standard
(although 3.0 was never accepted and subsequently withdrawn).
A corresponding \Perl{} file in the \fn{versions/} subdirectory is loaded;
these files are named `\texttt{html}\Meta{num}\texttt{.pl}'.

Following the version number, a comma-separated list of extensions
can be given. Each corresponds to a file `\Meta{name}\texttt{.pl}' 
also located in the \fn{versions/} subdirectory. 
When such a file is loaded the resulting \texttt{HTML} code 
can no longer be expected to validate with the specified DTD. 
An exception is \Ve{math} when the \htmlref{\Cs{no\_math}}{nomath} 
switch is also used, which should still validate.

Currently, versions 2.0, 3.2 and 4.0 are available.
(and also 2.1, 2.2, 3.0 and 3.1, for hoistorical reasons). 
The extensions \Ve{i18n}, \Ve{tables}, \Ve{math}
correspond roughly to what used to be called versions `2.1', `2.2', `3.1'
respectively, in releases of \latextohtml{} up to 1996.
Now these extensions can be loaded with any of `2.0', `3.2' or `4.0' 
as the specified standard.\\  
The default version is usually set to be `3.2', within \fn{latex2html.config}\,.%

\index{extensions!disabled}%
\index{extensions!TeX defs@\TeX definitions}%

\item [ -no\_tex\_defs\label{cs_notexdefs}]
\sameas{\fn{\$TEXDEFS}\texttt{ = 0;}} (default is 1)\\
When \fn{\$TEXDEFS} is set (default) the file \fn{texdefs.perl} will be read.
This provides code to allow common \TeX{} commands like \Tc{def}, \Tc{newbox},
\Tc{newdimen} and others, to be recognised, especially within the document preamble.
In the case of \Tc{def}, the definition may even be fully interpreted,
but this requires the pattern-matching to be not too complicated.

If \fn{\$TEXDEFS} is `\texttt{0}' or empty, then \fn{texdefs.perl} will not be loaded;
the translator will make no attempt to interpret any raw \TeX{} commands.  
This feature is intended to enable sophisticated authors the ability to insert 
arbitrary \TeX{} commands in environments that are destined 
to be processed by \LaTeX{} anyway; 
e.g. \env{figure}s, \strikeout{\env{theorem}s}, \env{picture}s, etc.
However this should rarely be needed, as now there is better support for these
types of environment. There are now other methods to specify which chunks of code 
are to be passed to \LaTeX{} for explicit image-generation; 
see the discussion of the \env{makeimage} 
\hyperref[page]{environment}{environment on page~}{}{makeimage}.


\index{auxiliary file!which to read}%

\item [ -external\_file \Meta{filename}\label{cs_externalfile}] 
\sameas{\fn{\$EXTERNAL\_FILE}\texttt{ = \char34}\Meta{filename}\texttt{\char34;}}\\
Specifies the prefix of the \texttt{.aux} file that this document should read.
The \texttt{.aux} extension will be appended to this prefix 
to get the complete filename, with directory path if needed.\html{\\}  
This file could contain necessary information 
regarding citations, figure, table and section numbers from \LaTeX{} 
and perhaps other information also.
Use of this switch is vital for \htmlref{document segments}{Segmentation}, 
processed separately and linked to appear as if generated 
from a single \LaTeX{} document.%

\index{font size!for image generation}%
\index{font-size!in environments}\index{font-size!magnification}%

\item [ -font\_size \Meta{size}\label{cs_fontsize}]
\sameas{\fn{\$FONT\_SIZE}\texttt{ = \char34}\Meta{size}\texttt{\char34;}}\\
This option provides better control 
over the font size of environments made into images using \LaTeX.  
\Meta{size} must be one of the font sizes that \LaTeX{} recognizes; 
i.e. `\texttt{10pt}', `\texttt{11pt}', `\texttt{12pt}', etc.  
Default is `\texttt{10pt}', or whatever option may have been specified on the 
\Lc{documentclass} or \Lc{documentstyle} line.\html{\\}
Whatever size is selected, it will be magnified by the installation variables
\fn{\$MATH\_SCALE\_FACTOR}, \fn{\$FIGURE\_SCALE\_FACTOR}
and \fn{\$DISP\_SCALE\_FACTOR} as appropriate.

\smallskip\noindent
\textbf{Note: }This switch provides no control over the size of text
on the \texttt{HTML} pages. 
Such control is subject entirely to the user's choices
of settings for the browser windows.

\index{font size!scalable}%

\item [ -scalable\_fonts\label{cs_scalefonts}]
\sameas{\fn{\$SCALABLE\_FONTS}\texttt{ = 1;}}\\
This is used when scalable fonts, such as \PS\ versions
of the \TeX{} fonts, are available for image-generation.\html{\\}
It has the effect of setting \fn{\$PK\_GENERATION} to `\texttt{1}',
and \fn{\$DVIPS\_MODE} to be empty, 
overriding any \htmlref{previous settings}{pkgen} for these variables.

\index{simple math!cancelled}%

\item [ -no\_math\label{cs_nomath}]
\sameas{\fn{\$NO\_SIMPLE\_MATH}\texttt{ = 1;}}\\
Ordinarily simple mathematical expressions are set using the ordinary text font, 
but italiced. When part of the expression can not be represented this way,
an image is made \emph{of the whole formula}. 
This is called ``simple math''.
When \fn{\$NO\_SIMPLE\_MATH} is set, 
then \emph{all} mathematics is made into images, whether simple or not.

However, if the \Ve{math} extension is loaded, 
using the \Cs{html\_version} switch described \htmlref{earlier}{cs_htmlversion},
then specifying \Cs{no\_math} produces a quite different effect.
Now it is the special \HTMLtag{MATH} tags and entities which are cancelled. 
In their place a sophisticated scheme for parsing mathematical expressions 
is used. Images are made of those sub-parts of a formula which cannot be 
adequately expressed using (italiced) text characters and \HTMLtag{SUB} and
\HTMLtag{SUP} tags.
See \hyperref{the subsection on mathematics}{Section~}{}{maths} for more details.

\index{icons!copied to local directory}%

\item [-local\_icons\label{cs_localicons}]
\sameas{\fn{\$LOCAL\_ICONS}\texttt{ = 1;}}\\
A copy of each of the icons actually used within the document is placed in the
directory along with the \texttt{HTML} files and generated images.
This allows the whole document to be fully self-contained, within this directory;
otherwise the icons must be retrieved from a (perhaps remote) server.

\index{icons!alternative set}\index{icons!customised images}%
The icons are normally copied from a subdirectory of the \fn{\$LATEX2HTMLDIR},
set within \fn{latex2html.config}. An alternative set of icons can be used
by specifying a (relative) directory path in \fn{\$ALTERNATIVE\_ICONS}
to where the customised images can be found.


\index{initialisation!init file@\texttt{-init\_file}}%
\index{initialisation file!specified}%

\item [ -init\_file \Meta{file}\label{cs_initfile}]
Load the specified initialisation file. This \Perl{} file will be loaded after loading 
\fn{\$HOME/.latex2html-init}\,, or \fn{.latex2html-init} in the local
directory, if either file exists. It is read at the time the switch is processed,
so the contents of the file may change any of the values of any of the
variables that were previously established, as well as any default options.
More than one initialisation file can be read in this way.%

\index{index!codified links}%

\cbversion{98.1}\begin{changebar}
\index{forking!prevent}%
\item [ -no\_fork\label{cs_no_fork}]
\sameas{\fn{\$NOFORK}\texttt{ = 1;}}\\
When set this disables a feature in the early part of the processing
whereby some memory-intensive operations are performed by `forked'
child processes. Some single-task operating systems, such as \texttt{DOS},
do not support this feature. Having \fn{\$NOFORK} set then ensures
that unnecessary file-handles that are needed with the forked processes,
are not consumed unnecessarily, perhaps resulting in a fatal \Perl{} error.

\index{languages!dtd isolanguage}
\index{character set!dtd isolanguage}
\item [ -iso\_language \Meta{type}\label{cs_iso_language}]
This enables you to specify a different language type than
\texttt{'EN'} to be used in the DTD entries of the \texttt{HTML}
document, e.g. \texttt{'EN.US'}.
\end{changebar}%

\item [ -short\_index\label{cs_shortindex}]
\sameas{\fn{\$SHORT\_INDEX}\texttt{ = 1;}}\\
Creates shorter Index listings, using codified links; 
this is fully compatible with the \env{makeidx} \htmlref{package}{makeidx}.

\index{footnotes!on separate page}%
\index{footnotes!on same page}%
\index{footnotes!marker style}

\item [ -no\_footnode\label{cs_nofootnode}]
\sameas{\fn{\$NO\_FOOTNODE}\texttt{ = 1;}}\\
Suppresses use of a separate file for footnotes;
instead these are placed at the bottom of the \texttt{HTML} pages 
where the references occur.

When this option is used, it is frequently desirable to change the 
style of the marker used to indicate the presence of a footnote. 
This is done as in \LaTeX, using code such as follows.
\begin{quote}
\verb|\renewcommand{\thefootnote}{\arabic{footnote}}|
\end{quote}
All the styles \Lc{arabic}, \Lc{alph}, \Lc{roman}, \Lc{Alph} and \Lc{Roman}
are available.

\index{footnotes!numbering}
\cbversion{98.1}\begin{changebar}
\item [ -numbered\_footnotes\label{cs_numbered_footnotes}]
\sameas{\fn{\$NUMBERED\_FOOTNOTES}\texttt{ = 1;}}\\
If this is set you will get every footnote applied with a subsequent
number, to ease readability.
\end{changebar}

\index{address!signature}%
\index{address!using a subroutine}%

\item [ -address \Meta{author-address}\label{cs_address}]
\sameas{\fn{\$ADDRESS}\texttt{ = \char34}\Meta{author-address}\texttt{\char34;}}\\
Sign each page with this address.\html{\\}
See \fn{latex2html.config} for an example using \Perl{} code
to automatically include the date.

A user-defined \Perl{} subroutine called \texttt{\&custom\_address} 
can be used instead, if defined; it takes the value of \fn{\$ADDRESS} 
as a parameter, which may be used or ignored as desired.
At the time when this subroutine will be called, 
variables named \fn{\$depth}, \fn{\$title}, \fn{\$file}
hold the sectioning-level, title and filename of the \texttt{HTML} page
being produced; \fn{\$FILE} holds the name of the filename for the title-page
of the whole document.

\index{About this@\emph{About this document...}}

\item [ -info \Meta{string}\label{cs_infostring}]
\sameas{\fn{\$INFO}\texttt{ = \char34}\Meta{string}\texttt{\char34;}}\\
Generate a new section \emph{``About this document'' } containing information
about the document being translated. The default is to generate such a section 
with information on the original document, the date, the user and the translator.
An empty string (or the value `\texttt{0}') 
disables the creation of this extra section.\html{\\}
If a non-empty string is given, it will be placed as the contents of the 
\emph{``About this document'' } page instead of the default information.
%
\end{htmllist}


\subsubsection{Switches controlling Image Generation}
%\subsection[center]{Switches controlling Image Generation}
%
These switches affect whether images are created at all, whether
old images are reused on subsequent runs or new ones created afresh,
and whether anti-aliasing effects are used within the images themselves.

\begin{htmllist}\htmlitemmark{PinkBall}%
%
\index{ascii-mode}% 
\index{browser!character-based}%
\item [ -ascii\_mode \label{cs_asciimode}]
\sameas{\fn{\$ASCII\_MODE}\texttt{ = }\fn{\$EXTERNAL\_IMAGES}\texttt{ = 1;}}\\
Use only \texttt{ASCII} characters and do not include any images in the final output.
With \Cs{ascii\_mode} the output of the translator can be used on 
character-based browsers, such as \appl{lynx},
which do not support inlined images (via the \HTMLtag{IMG} tag).


\index{draft mode}%
\item [ -nolatex \label{cs_nolatex}]
\sameas{\fn{\$NOLATEX}\texttt{ = 1;}}\\
Disable the mechanism for passing unknown environments to \LaTeX{} for processing.
This can be thought of as ``draft mode'' which allows 
faster translation of the basic document structure and text, 
without fancy figures, equations or tables. 

(This option has been superseded by the \Cs{no\_images} option,
\htmlref{see below}{noimages}.)


\index{images!via hypertext links}%
\item [ -external\_images\label{cs_extimages}]
\sameas{\fn{\$EXTERNAL\_IMAGES}\texttt{ = 1;}}\\
Instead of including any generated images inside the document, 
leave them outside the document and provide hypertext links to them.


\index{images!links to PostScript@links to \PS}%
\item [ -ps\_images\label{cs_psimages}]
\sameas{\fn{\$PS\_IMAGES}\texttt{ = }\fn{\$EXTERNAL\_IMAGES}\texttt{ = 1;}}\\
Use links to external \PS\ files rather than inlined images
in the chosen graphics format.

\index{images!discard PostScript files@discard \PS\ files}%
\item [ -discard\label{cs_discard}]
\sameas{\fn{\$DISCARD\_PS}\texttt{ = 1;}}\\
The temporary \PS\ files are discarded immediately after they
have been used to create the image in the desired graphics format.

\index{images!disabled}%
\item [ -no\_images\label{cs_noimages}]
\sameas{\fn{\$NO\_IMAGES}\texttt{ = 1;}}\\
Do not attempt to produce any inlined images. 
The missing images can be generated ``off-line'' by restarting \latextohtml{}
with the option \Cs{images\_only}.

\index{images!generated off-line}%
\item [ -images\_only\label{cs_imagesonly}]
\sameas{\fn{\$IMAGES\_ONLY}\texttt{ = 1;}}\\
Try to convert any inlined images that were left over from previous
runs of \latextohtml. 


\index{image files!reuse option}%
\item [ -reuse \Meta{reuse\_option}\label{cs_reuseoptions}]
\sameas{\fn{\$REUSE}\texttt{ = }\Meta{reuse\_option}\texttt{;}}\\
This switch specifies the extent to which image files are to be shared
or recycled.\html{\\}
There are three valid options:
%
\begin{htmllist}\htmlitemmark{OrangeBall}
\index{image files!not recycled}\index{image-reuse!interactive}%
\item [\texttt{0}] 
Do not ever share or recycle image files.\\
This choice also invokes an interactive session 
prompting the user about what to do about
a pre-existing \texttt{HTML} directory, if it exists.
%
\index{image files!recycled}%
\item [\texttt{1}] 
Recycle image files from a previous run if they are available,\\
but do not share identical images that must be created in this run.
%
\index{image files!shared by default}%
\item [\texttt{2}] 
Recycle image files from a previous run and share identical
images from this run.\\
This is the default.
\end{htmllist}
\hyperref{A later section}{Section~}{}{recycling} provides 
additional information about image-reuse.


\index{image files!not reused}\index{image-reuse!interactive}%
\item [ -no\_reuse\label{cs_noreuse}]
\sameas{\fn{\$REUSE}\texttt{ = 0;}}\\
Do \emph{not} share or recycle images generated during previous translations.
This is equivalent to \Cs{reuse 0}.
(This will enable the initial interactive session during which the user is
asked whether to reuse the old directory, delete its contents or quit.)%


\index{images!anti-aliasing}\index{anti-aliasing!in figures}%
\item [ -antialias\label{cs_aalias}]
\sameas{\fn{\$ANTI\_ALIAS}\texttt{ = 1;}} (Default is 0.)\\
Generated images of \env{figure} environments and external \PS\ files
should use anti-aliasing. By default anti-aliasing is not
used with these images, since this may interfere with the contents
of the images themselves.

\index{images!anti-aliasing}\index{anti-aliasing!in math and text}%
\item [ -antialias\_text\label{cs_aaliastext}]
\sameas{\fn{\$ANTI\_ALIAS\_TEXT}\texttt{ = 1;}} (Default is 1.)\\
Generated images of typeset material such as text, mathematical formulas,
tables and the content of \env{makeimage} \htmlref{environments}{makeimage},
should use anti-aliasing effects.\html{\\}
The default is normally to use anti-aliasing for text, 
since the resulting images are much clearer on-screen. 
However the default may have been changed locally.


\index{images!no anti-aliasing}\index{anti-aliasing!turned off}%
\item [ -no\_antialias\label{cs_noaalias}]
\sameas{\fn{\$ANTI\_ALIAS}\texttt{ = 0;}} (Default is 0.)\\
Generated images of \env{figure} environments and external \PS\ files 
should not use anti-aliasing with images, 
though the local default may have been changed to use it.


\index{images!no anti-aliasing}\index{anti-aliasing!blurred printing}%
\item [ -no\_antialias\_text\label{cs_noaaliastext}]
\sameas{\fn{\$ANTI\_ALIAS\_TEXT}\texttt{ = 0;}} (Default is 1.)\\
Generated images of typeset material should \emph{not} use anti-aliasing effects.
Although on-screen images of text are definitely improved using anti-aliasing,
printed images can be badly blurred, even at 300dpi. 
Higher resolution printers do a much better job with the resulting grey-scale images.

\cbversion{98.1}\begin{changebar}
\index{images!white background}%
\item [ -white\label{cs_white}]
\sameas{\fn{\$WHITE\_BACKGROUND}\texttt{ = 1;}} (Default is 1.)\\
Ensures that images of \env{figure} environments have a white
background. Otherwise transparency effects may not work correctly.

\item [ -no\_white\label{cs_nowhite}]
\sameas{\fn{\$WHITE\_BACKGROUND}\texttt{ = '';}} (Default is 1.)\\
Cancels the requirement that \env{figure} environments have a white background.

\index{images!latex dump}%
\item [ -ldump\label{cs_ldump}]
\sameas{\fn{\$LATEX\_DUMP}\texttt{ = 1;}} (Default is 0.)\\
Use this if you want to speed up image processing during the 2nd
and subsequent runs of \latextohtml{} on the same document.
The translator now produces a \LaTeX{} format-dump of the preamble to
\fn{images.tex} which is used on subsequent runs.
This significantly reduces the startup time when
\LaTeX{} reads the \fn{images.tex} file for image-generation.

This process actually consumes additional time on the first run, 
since \LaTeX{} is called twice --- once to create the format-dump,
then again to load and use it.
The pay-off comes with the faster loading on subsequent runs. 
Approximately 1 Meg of disk space is consumed by the dump file.
\end{changebar}
%
\end{htmllist}


\subsubsection{Switches controlling Navigation Panels\label{navoptions}}
%\subsection[center]{Switches controlling Navigation Panels\label{navoptions}}
%
The following switches govern whether to include one or more navigation
panels on each \texttt{HTML} page, 
also which buttons to include within such a panel.

\begin{htmllist}\htmlitemmark{PinkBall}%

\index{navigation panel!disabled}
\item [ -no\_navigation\label{cs_nonavig}]
\sameas{\fn{\$NO\_NAVIGATION}\texttt{ = 1;}}\\
Disable the mechanism for putting navigation links in each page.\\
This overrides any settings of the \fn{\$TOP\_NAVIGATION},
\fn{\$BOTTOM\_NAVIGATION} and \fn{\$AUTO\_NAVIGATION} variables.

\index{navigation panel!at top of page}
\item [ -top\_navigation\label{cs_topnavig}]
\sameas{\fn{\$TOP\_NAVIGATION}\texttt{ = 1;}}\\
Put navigation links at the top of each page.    

\index{navigation panel!at bottom of page}%
\item [ -bottom\_navigation\label{cs_botnavig}]
\sameas{\fn{\$BOTTOM\_NAVIGATION}\texttt{ = 1;}}\\
Put navigation links at the bottom of each page as well as the top.

\index{navigation panel!according to number of words}
\item [ -auto\_navigation\label{cs_autonavig}]
\sameas{\fn{\$AUTO\_NAVIGATION}\texttt{ = 1;}}\\
Put navigation links at the top of each page. 
Also put one at the bottom of the page, if the page exceeds 
\fn{\$WORDS\_IN\_PAGE} number of words (default = 450).

\index{navigation panel!next logical page}
\item [ -next\_page\_in\_navigation\label{cs_nextinnavig}]
\sameas{\fn{\$NEXT\_PAGE\_IN\_NAVIGATION}\texttt{ = 1;}}\\
Put a link to the \textsl{next logical page} in the navigation panel.

\index{navigation panel!previous logical page}%
\item [ -previous\_page\_in\_navigation\label{cs_previnnavig}]~\\%
\sameas{\fn{\$PREVIOUS\_PAGE\_IN\_NAVIGATION}\texttt{ = 1;}}\\
Put a link to the \textsl{previous logical page} in the navigation panel.

\index{navigation panel!link to table-of-contents}
\item [ -contents\_in\_navigation\label{cs_continnavig}]
\sameas{\fn{\$CONTENTS\_IN\_NAVIGATION}\texttt{ = 1;}}\\
Put a link to the \textsl{table-of-contents} in the navigation panel if there is one.

\index{navigation panel!link to index}
\item [ -index\_in\_navigation\label{cs_indexinnavig}]
\sameas{\fn{\$INDEX\_IN\_NAVIGATION}\texttt{ = 1;}}\\
Put a link to the \textsl{index-page} in the navigation panel if there is an index.
%
\end{htmllist}


\subsubsection{Switches for Linking to other documents\label{otherdocs}}
%\subsubsection[center]{Switches for Linking to other documents\label{otherdocs}}
%
When processing a single stand-alone document, the switches described in this
section should not be needed at all, since the automatically generated navigation panels, 
described \hyperref{on the previous page}{in Section~}{, }{navoptions}
should generate all the required navigation links.
However if a document is to be regarded as part of a much larger document,
then links from its first and final pages, to locations in other parts
of the larger (virtual) document, 
need to be provided explicitly for some of the buttons in the navigation panel.

The following switches allow for such links to other documents,
by providing the title and URL for navigation panel hyperlinks.
In particular, the ``Document Segmentation''
\hyperref{feature}{feature of Section~}{}{Segmentation} 
necessarily makes great use of these switches.
It is usual for the text and targets of these navigation hyperlinks
to be recorded in a \fn{Makefile}, 
to avoid tedious typing of long command-lines having many switches.

\begin{htmllist}\htmlitemmark{PinkBall}%
%
\index{URL!universal resource locator}\index{URL!for UP@for \textsl{UP} button}%
\item [ -up\_url \Meta{URL}\label{cs_upURL}] 
\sameas{\fn{\$EXTERNAL\_UP\_LINK}\texttt{ = \char34}\Meta{URL}\texttt{\char34;}}\\
Specifies a universal resource locator (URL) to associate
with the \textsl{``UP''} button in the navigation panel(s).


\item [ -up\_title \Meta{string}\label{cs_uptitle}] 
\sameas{\fn{\$EXTERNAL\_UP\_TITLE}\texttt{ = \char34}\Meta{string}\texttt{\char34;}}\\
Specifies a title associated with this URL.%


\index{URL!for PREVIOUS@for \textsl{PREVIOUS} button}%
\item [ -prev\_url \Meta{URL}\label{cs_prevURL}] 
\sameas{\fn{\$EXTERNAL\_PREV\_LINK}\texttt{ = \char34}\Meta{URL}\texttt{\char34;}}\\
Specifies a URL to associate
with the \textsl{``PREVIOUS''} button in the navigation panel(s).%


\item [ -prev\_title \Meta{string}\label{cs_prevtitle}] 
\sameas{\fn{\$EXTERNAL\_PREV\_TITLE}\texttt{ = \char34}\Meta{string}\texttt{\char34;}}\\
Specifies a title associated with this URL.%


\index{URL!for NEXT@for \textsl{NEXT} button}%
\item [ -down\_url \Meta{URL}\label{cs_downURL}]
\sameas{\fn{\$EXTERNAL\_DOWN\_LINK}\texttt{ = \char34}\Meta{URL}\texttt{\char34;}}\\
Specifies a URL for the \textsl{``NEXT''} button in the navigation panel(s).%


\item [ -down\_title \Meta{string}\label{cs_downtitle}] 
\sameas{\fn{\$EXTERNAL\_DOWN\_TITLE}\texttt{ = \char34}\Meta{string}\texttt{\char34;}}\\
Specifies a title associated with this URL.%


\index{URL!for CONTENTS@for \textsl{CONTENTS} button}%
\item [ -contents \Meta{URL}\label{cs_contentsURL}] 
\sameas{\fn{\$EXTERNAL\_CONTENTS}\texttt{ = \char34}\Meta{URL}\texttt{\char34;}}\\
Specifies a URL for the \textsl{``CONTENTS''} button,
for document segments that would not otherwise have one.%


\index{URL!for INDEX@for \textsl{INDEX} button}%
\item [ -index \Meta{URL}\label{cs_indexURL}] 
\sameas{\fn{\$EXTERNAL\_INDEX}\texttt{ = \char34}\Meta{URL}\texttt{\char34;}}\\
Specifies a URL for the \textsl{``INDEX''} button, 
for document segments that otherwise would not have an index.

\index{URL!for external Bibliography}%
\item [ -biblio \Meta{URL}\label{cs_biblioURL}] 
\sameas{\fn{\$EXTERNAL\_BIBLIO}\texttt{ = \char34}\Meta{URL}\texttt{\char34;}}\\
Specifies the URL for the bibliography page to be used,
when not explicitly part of the document itself.
%
\end{htmllist}
\smallskip
\textbf{Warning: }
On some systems it is difficult to give text-strings \Meta{string} containing
space characters, on the command-line or via a \fn{Makefile}.
One way to overcome this is to use the corresponding variable.
Another way is to replace the spaces with underscores (\verb|_|).


\subsubsection{Switches for Help and Tracing}
%\subsubsection[center]{Switches for Help and Tracing}
%
The first two of the following switches are self-explanatory.
When problems arise in processing a document, 
the switches \Cs{debug} and \Cs{verbosity} will each
cause \latextohtml{} to generate more output to the screen.
These extra messages should help to locate the cause of the problem.

\begin{htmllist}\htmlitemmark{PinkBall}%
%
\index{options!defining temp directory}%
\item [ -tmp \Meta{path}\label{cs_tmpdir}]
Define a temporary directory to use for image generation.
If \Meta{path} is $0$, the standard temporary directory \texttt{/tmp}
is used.
%
\index{options!print option listing}%
\item [ -h(elp)\label{cs_optionlist}]
Print out the list of all command-line options.

\index{current version@current version of \latextohtml}%
\item [ -v\label{cs_showversion}] 
Print the current version of \latextohtml.

\index{debugging}\index{diagnostic information}%
\item [ -debug\label{cs_debugmode}]
\sameas{\fn{\$DEBUG}\texttt{ = 1;}}\\
Run in debug-mode, displaying messages and/or diagnostic information 
about files read, and utilities called by \latextohtml.
Shows any messages produced by these calls.%

\index{debugging!Perl@\Perl{} debugger}%

More extensive diagnostics, from the \Perl{} debugger, 
can be obtained by appending the string `\texttt{-w-}' 
to the 1st line of the \fn{latex2html} (and other) \Perl{} script(s).


\index{verbosity level}%
\item [ -verbosity \Meta{num}\label{cs_verbositylevel}] 
\sameas{\fn{\$VERBOSITY}\texttt{ = \Meta{num};}}\\
Display messages revealing certain aspects of the processing
performed by \latextohtml{} on the provided input file(s).
The \Meta{num} parameter can be an integer in the range 0 to 8.
Each higher value adds to the messages produced.

\begin{itemize}
\item[0. ]
No special tracing; as for versions of \latextohtml{} prior to \textsc{v97.1}\,.

\item[1. ] (This is the default.)
Show section-headings and the corresponding \texttt{HTML} file names,
and indicators that major stages in the processing have been completed.

\item[2. ]
Print environment names and identifier numbers, and new theorem-types.
Show warnings as they occur, and indicators for more stages of processing.
Print names of files for storing auxiliary data arrays.

\item[3. ]
Print command names as they are encountered and processed; also
any unknown commands encountered while pre-processing.
Show names of new commands, environments, theorems, counters and
counter-dependencies, for each document partition.

\item[4. ]
Indicate command-substitution the pre-process of math-environments.
Print the contents of unknown environments for processing in \LaTeX,
both before and after reverting to \LaTeX{} source.
Show all operations affecting the values of counters.
Also show links, labels and sectioning keys, at the stages of processing.

\item[5. ]
Detail the processing in the document preamble.
Show substitutions of new environments.
Show the contents of all recognised environments, both before
and after processing. Show the cached/encoded information for the image keys,
allowing two images to be tested for equality.

\item[6. ]
Show replacements of new commands, accents and wrapped commands.

\item[7. ]
Trace the processing of commands in math mode; both before and after.

\item[8. ]
Trace the processing of all commands, both before and after.

\end{itemize}
%
The command-line option sets an initial value only.
During processing the value of \fn{\$VERBOSITY} can be set
dynamically using the \verb|\htmltracing{...}| command, 
whose argument is the desired value, 
or by using the more general \Lc{HTMLset} command
as follows: \verb|\HTMLset{VERBOSITY}{|\Meta{num}\texttt{\char125}\,.
%
\end{htmllist}
\index{options|)}%
%

\subsubsection{Other Configuration Variables, without switches}
%\subsection[center]{Other Configuration Variables, without switches}
%
The configuration variables described here do not warrant having
a command-line switch to assign values.
Either they represent aspects of \latextohtml\ that are specific to the local site,
or they govern properties that should apply to \emph{all} documents,
rather than something that typically would change
for the different documents within a particular sub-directory.

Normally these variables have their value set within the \fn{latex2html.config} file.
In the following listing the defaults are shown, as the lines of \Perl{} code
used to establish these values.
If a different value is required, then these can be assigned from
a local \fn{.latex2html-init} initialisation file, 
without affecting the defaults for other users, 
or documents processed from other directories.

\begin{htmllist}\htmlitemmark{PurpleBall}%
%
\index{directory delimiter}%
\item [\fn{\$dd} ]
~holds the string to be used in file-names to delimit directories;
it is set internally to `\texttt{/}', unless the variable has already been given
a value within \fn{latex2html.config}~.

\noindent
\textbf{Note:} This value \emph{cannot} be set within a \fn{.latex2html-init} 
initialisation file, since its value \emph{needs to be known} 
in order to find such a file.

\index{installation variables!inserted@inserted by \fn{install-test}}%
\item [\fn{\$LATEX2HTMLDIR}] \latex{ }%
Read by the \fn{install-test} script from \fn{latex2html.config}, 
its value is inserted into the \fn{latex2html} \Perl{} script as part
of the installation process.

\item [\fn{\$LATEX2HTMLSTYLES}\texttt{ = \char34\$LATEX2HTMLDIR/styles\char34;}]
Read from the \fn{latex2html.config} file by \fn{install-test}\,,
its value is checked to locate the \fn{styles/} directory.

\item [\fn{\$LATEX2HTMLVERSIONS}\texttt{ = \char34\$LATEX2HTMLDIR/versions\char34;}]
The value of this variable should be set within \fn{latex2html.config} 
to specify the directory path where the version and extension files can be found.

\item [\fn{\$ALTERNATIVE\_ICONS}\texttt{ = '';}\label{alticons}]
This may contain the (relative) directory path to a set of customised icons
to be used in conjunction with the \htmlref{\Cs{local\_icons}}{cs_localicons} switch.

\index{installation variables!checked@checked by \fn{install-test}}%
\item [\fn{\$TEXEXPAND}\texttt{  = \char34\$LATEX2HTMLDIR/texexpand\char34;}]
Read by the \fn{install-test} \Perl{} script from \fn{latex2html.config}, 
its value is used to locate the \fn{texexpand} \Perl{} script.

\item [\fn{\$PSTOIMG}\texttt{ = \char34\$LATEX2HTMLDIR/pstoimg\char34;}]
Read by the \fn{install-test} \Perl{} script from \fn{latex2html.config}, 
its value is used to locate the \fn{pstoimg} \Perl{} script.

\item [\fn{\$IMAGE\_TYPE}\texttt{ = '\Meta{image-type}';}]
Set in \fn{latex2html.config}, the currently supported \Meta{image-type}s
are: \texttt{gif} and \texttt{png}.

\item [\fn{\$DVIPS}\texttt{ = 'dvips';}]
Read from \fn{latex2html.config} by \fn{install-test}, 
its value is checked to locate the \fn{dvips} program or script.

\smallskip\noindent
There could be several reasons to change the value here:
\begin{itemize}
\item
add a switch \texttt{-P}\Meta{printer} to load a specific configuration-file;\\
e.g. to use a specific set of \PS\ fonts, for improved image-generation.
\item
to prepend a path to a different version of \fn{dvips} than normally
available as the system default (e.g. the printing requirements are different).
\item
to append debugging switches, in case of poor quality images;\\
one can see which paths are being searched for fonts and other resources.
\item
to prepend commands for setting path variables that \fn{dvips} may need
in order to locate fonts or other resources.
\end{itemize}

If automatic generation of fonts is required, using \MF, the following
configuration variables are important.
%
\begin{htmllist}\htmlitemmark{OrangeBall}%
%
\item [\fn{\$PK\_GENERATION}\texttt{ = 1;}\label{pkgen}]
This variable must be set, to initiate font-generation; otherwise fonts
will be scaled from existing resources on the local system.\\
In particular this variable must \emph{not} be set, if one wishes to use
\PS\ fonts or other scalable font resources 
(see the \Cs{scalable\_fonts} \htmlref{switch}{scalefonts}).

\item [\fn{\$DVIPS\_MODE}\texttt{ = 'toshiba';}]
The mode given here must be available in the \fn{modes.mf} file,
located with the \MF\ resource files, perhaps in the \fn{misc/} subdirectory.

\item [\fn{\$METAFONT\_DPI}\texttt{ = 180;}]
The required resolution, in dots-per-inch, should be listed specifically
within the \fn{MakeTeXPK} script, called by \fn{dvips} to invoke \MF\ with
the correct parameters for the required fonts.
%
\end{htmllist}


\item [\fn{\$LATEX}\texttt{ = 'latex';}]
Read from \fn{latex2html.config} by \fn{install-test}, 
its value is checked to locate the \fn{latex} program or script.

\smallskip\noindent
If \LaTeX{} is having trouble finding style-files and/or packages, then
the default command can be prepended with other commands 
to set environment variables intended to resolve these difficulties;\\
e.g. 
\verb|$LATEX = 'setenv TEXINPUTS |\Meta{path to search}\verb| ; latex' |.

There are several variables to help control exactly which files are
read by \latextohtml{} and by \LaTeX{} when processing images:

\begin{htmllist}\htmlitemmark{OrangeBall}%
\item [\fn{\$TEXINPUTS}]
This is normally set from the environment variable of the same name.
If difficulties occur so that styles and packages are not being found, 
then extra paths can be specified here, to resolve these difficulties.

\item [\fn{\$DONT\_INCLUDE}]
This provides a list of filenames and extensions to \emph{not} include, 
even if requested to do so by an \Lc{input} or \Lc{include} command.\\
(Consult \fn{latex2html.config} for the default list.)

\item [\fn{\$DO\_INCLUDE}\texttt{ = '';}]
List of exceptions within the \fn{\$DONT\_INCLUDE} list.
These files \emph{are} to be read if requested by an \Lc{input}
or \Lc{include} command.
\end{htmllist}


\item [\fn{\$ICONSERVER}\texttt{ = '\Meta{URL}';}]
This is used to specify a URL to find the standard icons, as used for the
navigation buttons.\html{\\}
Names for the specific images size, as well as size information, 
can be found in \fn{latex2html.config}. The icons themselves can be replaced
by customised versions, provided this information is correctly updated
and the location of the customised images specified as the value
of \fn{\$ICONSERVER}.\\
When the \htmlref{\Cs{local\_icons}}{cs_localicons} switch is used,
so that a copy of the icons is placed with the \texttt{HTML} files and other
generated images, the value of \fn{\$ICONSERVER} is not needed within
the \texttt{HTML} files themselves. However it \emph{is} needed to find
the original icons to be copied to the local directory.

\item [\fn{\$NAV\_BORDER}\texttt{ = }\Meta{num}\texttt{;}]
The value given here results in a border, measured in points, around each icon.\html{\\}
A value of `\texttt{0}' is common, to maintain strict
alignment of inactive and active buttons in the control panels.


\item [\fn{\$LINKNAME}\texttt{ = '\char34index.\$EXTN\char34';}]
This is used when the \fn{\$NO\_AUTO\_LINK} \htmlref{variable}{noautolink} is empty, 
to allow a URL to the working directory to be sufficient to reach the main page 
of the completed document. 
It specifies the name of the \texttt{HTML} file which will be automatically
linked to the directory name.\html{\\}
The value of \fn{\$EXTN} is \texttt{.html} unless \fn{\$SHORTEXTN} is set, 
in which case it is \texttt{.htm}~. 

\item [\fn{\$LINKPOINT}\texttt{ = '\char34\$FILE\$EXTN\char34';}]
This specifies the name of the \texttt{HTML} file to be duplicated, or symbolically
linked, with the name specified in \fn{\$LINKNAME}.\\
At the appropriate time the value of \fn{\$FILE} is the document name, 
which usually coincides with the name of the working directory.

\index{character set}%
\index{character set!iso-latin@\texttt{\char34iso\_8859\_1\char34}}%
\item [\fn{\$CHARSET}\texttt{ = 'iso\_8859\_1';}\label{charset}]
This specifies the character set used within the \texttt{HTML} pages
produced by \latextohtml. If no value is set in a configuration
or initialisation file, the default value will be assumed.
The lowercase form \texttt{\$charset} is also recognised, but
this is overridden by the uppercase form.

\index{images!strange accents}%
\item [ \fn{\$ACCENT\_IMAGES}\texttt{ = 'large'};]
Accented characters that are not part of the ISO-Latin fonts
can be generated by making an image using \LaTeX.
This variable contains a (comma-separated) list of \LaTeX{} commands
for setting the style to be used when these images are made. 
If the value of this variable is empty then the accent is simply
ignored, using an un-accented font character (not an image) instead.
%
\end{htmllist}

\bigskip\noindent
Within the \fn{color.perl} package, the following variables are used to
identify the names of files containing specifications for named colors.
Files having these names are provided, in the \fn{\$LATEX2HTMLSTYLES}
directory, but they could be moved elsewhere, 
or replaced by alternative files having different names.
In such a case the values of these variables should be altered accordingly.
%
\begin{itemize}
\item []\fn{\$RGBCOLORFILE}\texttt{ = 'rgb.txt';}
\item []\fn{\$CRAYOLAFILE}\texttt{ = 'crayola.txt';}
\end{itemize}

\bigskip\noindent
The following variables may well be altered from the system defaults,
but this is best done using a local \fn{.latex2html-init} initialisation file,
for overall consistency of style within documents located at the same site,
or sites in close proximity.

\begin{htmllist}\htmlitemmark{PurpleBall}%
%
\item [\fn{\$default\_language}\texttt{ = 'english';}\label{isolanguage}]
This establishes which language code is to be placed within
the \HTMLtag{!DOCTYPE ... } tag that may appear at the beginning
of the \texttt{HTML} pages produced. Loading a package for
an alternative language can be expected to change the value of this variable.\html{\\}
See also the \fn{\$TITLES\_LANGUAGE} variable, described \htmlref{next}{doclanguage}.


\item [\fn{\$TITLES\_LANGUAGE}\texttt{ = 'english';}\label{doclanguage}]
This variable is used to specify the actual strings used for standard document
sections, such as ``Contents'', ``References'', ``Table of Contents'', etc.\html{\\}
Support for French and German titles is available in corresponding packages.
Loading such a package will normally alter the value of this variable, as well as 
the \fn{\$default\_language} variable described \htmlref{above}{isolanguage}.


\item [\fn{\$WORDS\_IN\_NAVIGATION\_PANEL\_TITLES}\texttt{ = 4;}]
Specifies how many words to use from section titles,
within the textual hyperlinks which accompany the navigation buttons.


\item [\fn{\$WORDS\_IN\_PAGE}\texttt{ = 450;}]
Specifies the minimum page length required before a navigation panel
is placed at the bottom of a page,
when the \fn{\$AUTO\_NAVIGATION} variable is set.


\item [\fn{\$CHILDLINE}\texttt{ = \char34<BR><HR>\char92n\char34;}]
This gives the \texttt{HTML} code to be placed between the child-links
table and the ordinary contents of the page on which it occurs.


\item [\fn{\$NETSCAPE\_HTML}\texttt{ = 0;}]
When set, this variable specifies that \texttt{HTML} code may be present which
does not conform to any official standard. 
This restricts the contents of any \HTMLtag{!DOCTYPE ... } tag which may be placed
at the beginning of the \texttt{HTML} pages produced.


\item [\fn{\$BODYTEXT}\texttt{ = '';}]
The value of this variable is used within the \HTMLtag{BODY ... } tag;
e.g. to set text and/or background colors.\html{\\}
It's value is overridden by the \verb|\bodytext| command, 
and can be added-to or parts changed using the \Lc{htmlbody} command
or \Lc{color} and \Lc{pagecolor} from the \env{color} package.


\item [\fn{\$INTERLACE}\texttt{ = 1;}\label{interlace}]
When set, interlaced images should be produced.\\
This requires \htmlref{graphics utilities}{graphicsutils} to be available 
to perform the interlacing operation.


\item [\fn{\$TRANSPARENT\_FIGURES}\texttt{ = 1;}\label{transfigs}]
When set, the background of images should be made transparent;
otherwise it is white.\html{\\}
This requires \htmlref{graphics utilities}{graphicsutils} %
to be available which can specify the color to be made transparent.


\item [\fn{\$FIGURE\_SCALE\_FACTOR}\texttt{ = 1.6;}\label{figscale}]
Scale factor applied to all images of \env{figure} and other environments, 
when being made into an image.

\smallskip\noindent
Note that this does not apply to recognised mathematics environments, 
which instead use the contents of \fn{\$MATH\_SCALE\_FACTOR} 
and \fn{\$DISP\_SCALE\_FACTOR} to specify scaling.


\item [\fn{\$MATH\_SCALE\_FACTOR}\texttt{ = 1.6;}\label{mathscale}]
Scale factor applied to all images of mathematics, both inline
and displayed. 
A value of 1.4 is a good alternative, with anti-aliased images.


\item [\fn{\$DISP\_SCALE\_FACTOR}\texttt{ = 1;}\label{dispscale}]
Extra scale factor applied to images of displayed math environments.\html{\\}
When set, this value multiplies \fn{\$MATH\_SCALE\_FACTOR}
to give the total scaling. 
A value of `1.2' is a good choice to accompany \verb|$MATH_SCALE_FACTOR = 1.4;|\,.


\item [\fn{\$EXTRA\_IMAGE\_SCALE}\label{ximagescale}]
This may hold an extra scale factor that can be applied to all generated images.\html{\\}
When set, it specifies that a scaling of \fn{\$EXTRA\_IMAGE\_SCALE} be applied 
when images are created, but to have their height and width recorded 
as the un-scaled size. This is to coax browsers into scaling the (usually larger) 
images to fit the desired size; when printed a better quality can be obtained.
Values of `1.5' and `2' give good print quality at 600dpi.


\item [\fn{\$PAPERSIZE}\texttt{ = 'a5';}\label{papersize}]
Specifies the size of a page for typesetting figures
or displayed math, when an image is to be generated.\\
This affects the lengths of lines of text within images.
Since images of text or mathematics should use larger sizes than
when printed, else clarity is lost at screen resolutions,
then a smaller paper-size is generally advisable.
This is especially so if both the \fn{\$MATH\_SCALE\_FACTOR}
and \fn{\$DISP\_SCALE\_FACTOR} scaling factors are being used,
else some images may become excessively large, including a lot
of blank space.


\item [\fn{\$LINE\_WIDTH}\texttt{ = 500;}\label{linewidth}]
Formerly specified the width of an image, when the contents were
to be right- or center-justified. (No longer used.)

\end{htmllist}

\bigskip

\noindent
The following variables are used to access the utilities required during image-generation. 
File and program locations on the local system are established by the \fn{configure-pstoimg} 
\Perl{} script and stored within \fn{\$LATEX2HTMLDIR/local.pm} as \Perl{} code, 
to be read by \fn{pstoimg} when required.

After running the \fn{configure-pstoimg} \Perl{} script it should not be necessary 
to alter the values obtained. Those shown below are what happens on the author's system;
they are for illustration only and do \emph{not} represent default values.
%

\begin{itemize}\label{graphicsutils}%
\item  \fn{\$GS\_LIB}\texttt{ = '/usr/local/share/ghostscript/4.02';}
\item  \fn{\$PNMCAT}\texttt{ = '/usr/local/bin/pnmcat';}
\item  \fn{\$PPMQUANT}\texttt{ = '/usr/local/bin/ppmquant';}
\item  \fn{\$PNMFLIP}\texttt{ = '/usr/local/bin/pnmflip';}
\item  \fn{\$PPMTOGIF}\texttt{ = '/usr/local/bin/ppmtogif';}
\item  \fn{\$HOWTO\_TRANSPARENT\_GIF}\texttt{ = 'netpbm';}
\item  \fn{\$GS\_DEVICE}\texttt{ = 'pnmraw';}
\item  \fn{\$GS}\texttt{ = '/usr/local/bin/gs';}
\item  \fn{\$PNMFILE}\texttt{ = '/usr/local/bin/pnmfile';}
\item  \fn{\$HOWTO\_INTERLACE\_GIF}\texttt{ = 'netpbm';}
\item  \fn{\$PBMMAKE}\texttt{ = '/usr/local/bin/pbmmake';}
\item  \fn{\$PNMCROP}\texttt{ = '/usr/local/bin/pnmcrop';}
\item  \fn{\$TMP}\texttt{ = '/usr/var/tmp';}
\end{itemize}

\noindent
The following variables are no longer needed, having been replaced by the
more specific information obtained using the \Perl{} script \fn{configure-pstoimg}.
\begin{itemize}
\item [] \fn{\$USENETPBM}\texttt{ = 1;}
\item [] \fn{\$PBMPLUSDIR}\texttt{ = '/usr/local/bin';}
\end{itemize}


%\filbreak
\subsection{Extending the Translator\label{sec:ext}}%
%\section{Extending the Translator\label{sec:ext}}%
\tableofchildlinks*\htmlrule
%
In \hyperref{an earlier section}{Section~}{}{sec:sty} is was seen how
the capabilities of \latextohtml{} can be extended to cope with
\LaTeX{} commands defined in packages and style-files.
This is in addition to defining simple macros and environments,
using \Lc{newcommand} and \Lc{newenvironment}.
The problem however, is that writing such extensions for packages 
requires an understanding of \Perl{} programming 
and of the way the processing in \latextohtml{} is organised. 

\index{new commands!ignore arguments}%
\index{new commands!arguments remain}%
\index{extensions!in initialisation file}
\smallskip

The default behaviour for unrecognised commands is 
for their arguments to remain in the \texttt{HTML} text,
whereas the commands themselves are passed on to \LaTeX,
in an attempt to generate an image. This is far from ideal,
for it is quite likely to lead to an error in \LaTeX{}
due to not having appropriate arguments for the command.

Currently there are several related mechanisms whereby a user 
can ask for particular commands and their arguments to be either 
\begin{itemize}
\item
\htmlref{ignored}{ignore} (via: \texttt{\&ignore\_commands}\,);
\item
passed on to \LaTeX{}  for processing \htmlref{as an image}{pass}\\
(via: \texttt{\&process\_commands\_in\_tex}\,);
\item
passed to \LaTeX{}  for processing as an image, 
\htmlref{as if an environment}{sensit}\\
(via: \texttt{\&process\_commands\_inline\_in\_tex}\,);
\item
passed on to \LaTeX, \htmlref{setting parameters}{setparam} 
to be used in subsequent images\\
(via: \texttt{\&process\_commands\_nowrap\_in\_tex}\,);
\item
processed as ``order-sensitive'', \htmlref{as if an environment}{wrapdef} 
rather than a command\\
(via: \texttt{\&process\_commands\_wrap\_deferred}\,);
\end{itemize}
The string beginning \texttt{\&.... } is the name of the \Perl{} subroutine 
that controls how the \LaTeX{} commands are to be subsequently treated during
processing by \latextohtml.
Towards the end of the \fn{latex2html} script, one finds a list
of \LaTeX{} commands to be handled by each of these subroutines.
These lists even include some common \TeX{} commands.

Analogous lists occur in most of the package extension files.
In many cases the commands are for fine-tuning the layout on
a printed page. They should simply be ignored; but any parameters
must not be allowed to cause unwanted characters to appear on
the \texttt{HTML} pages.
Customised extensions using these mechanisms may be included in the 
\fn{\$LATEX2HTMLDIR/latex2html.config} file, or in a personal 
\fn{\$HOME/.latex2html-init} initialisation file, 
as described \htmlref{next}{ignore}.



\subsubsection{Asking the Translator to Ignore Commands\label{ignore}}
%\subsection{Asking the Translator to Ignore Commands\label{ignore}}
\index{new commands!command ignored}\html{\\}%
Commands that should be ignored may be specified in the \fn{.latex2html-init}
file as input to the \verb|&ignore_commands| subroutine. 
Each command which is to be ignored should be on a separate line 
followed by compulsory or optional argument markers separated by 
{\verb|#|}'s e.g.\footnote{It is possible to add arbitrary(?) \Perl{} code
between any of the argument markers which will be executed when 
the command is processed. For this however a basic understanding of
how the translator works and of course \Perl{} is required.}:
\begin{quote}
%begin{latexonly}
\begin{small}
%end{latexonly}
\verb|<cmd_name1> # {} # [] # {} # [] ...|\\
\verb|<cmd_name2> # |\texttt{<<\emph{pattern}>>}\verb| # [] ...|
%begin{latexonly}
\end{small}
%end{latexonly}
\end{quote}
\verb|{}|'s mark compulsory arguments and \verb|[]|'s optional ones,
while \texttt{<<\emph{pattern}>>} denotes matching everything up to the
indicated string-pattern, given literally (e.g. \verb|\\endarray|);
spaces are ignored. Special characters
such as \verb|$|\,, \verb|&|\,, \verb|\| itself and perhaps some others,
need to be ``escaped'' with a preceding \verb|\|\,.


\index{new commands!arguments remain}\html{\\}%

Some commands may have arguments which should be left as text even
though the command should be ignored 
(e.g. \Lc{hbox}, \Lc{center}, etc.). 
In these cases arguments should be left unspecified.
Here is an example of how this mechanism may be used:
%begin{latexonly}
\begin{small}
%end{latexonly}
\begin{verbatim}
&ignore_commands( <<_IGNORED_CMDS_);
documentstyle # [] # {}
linebreak # []
mbox
<add your commands here>
_IGNORED_CMDS_
\end{verbatim}
%begin{latexonly}
\end{small}
%end{latexonly}


\subsubsection{Asking the Translator to Pass Commands to \LaTeX\label{pass}}%
%\subsection{Asking the Translator to Pass Commands to \LaTeX\label{pass}}%
\index{new commands!processed in LaTeX@processed in \LaTeX}\html{\\}%
Commands that should be passed on to \LaTeX{}  for processing because
there is no direct translation to \texttt{HTML} may be specified in the 
\fn{.latex2html-init}
file as input to the \verb|process_commands_in_tex| subroutine.
The format is the same as that for specifying commands to be ignored.
Here is an example:
%begin{latexonly}
\begin{small}
%end{latexonly}
\begin{verbatim}
&process_commands_in_tex (<<_RAW_ARG_CMDS_);
fbox # {}
framebox # [] # [] # {}
<add your commands here>
_RAW_ARG_CMDS_
\end{verbatim}
%begin{latexonly}
\end{small}
%end{latexonly}


\subsubsection{Handling ``order-sensitive'' Commands\label{sensit}}%
%\subsection{Handling ``order-sensitive'' Commands\label{sensit}}%
\index{new commands!order-sensitive}%
\index{new commands!order-sensitive!processed in LaTeX@processed in \LaTeX}%
\index{new commands!processed in LaTeX@processed in \LaTeX!order-sensitive}%
Some commands need to be passed to \LaTeX, but using the 
\verb|&process_commands_in_tex| subroutine gives incorrect results. 
This may occur when the command is ``order-sensitive'',
using information such as the value of a counter or a boolean expression 
(or perhaps requiring a box to have been constructed and saved).
Try using the \verb|&process_commands_inline_in_tex| subroutine instead.
Commands declared this way are first ``wrapped'' within a dummy environment,
which ensures that they are later processed in correct order with other 
environments and order-sensitive commands.

\index{order-sensitive!setting parameters}\label{setparam}\html{\\}%

Other commands may need to be passed to \LaTeX, not to create an image
themselves, but to affect the way subsequent images are created.
For example a color command such as \verb|\color{red}| should set the
text-colour to `red' for all subsequent text and images. This must
be sent to \LaTeX{} so that it is processed at exactly the right time;
i.e. before the first image required to be `red' but following any
images that are not intended to be affected by this colour-change.\html{\\}
The subroutine \verb|process_commands_nowrap_in_tex| is designed
specifically to meet such requirements.

\index{order-sensitive!wrap-deferred}\label{wrapdef}\html{\\}%

Commands can be order-sensitive without having to be passed to \LaTeX.
Indeed even if a \Perl{} subroutine has been carefully written to process
the command, it may still give wrong results if it is order-sensitive,
depending on the values of counters, say.
To handle such cases there is the \verb|&process_commands_wrap_deferred|
subroutine. This also ``wraps'' the command within a dummy environment,
but when that environment is processed the contents are not sent to \LaTeX,
as in the previous case. All of the standard \LaTeX{} commands to change, 
set or read the values of counters are handled in this way.



\subsection{Customising the Navigation Panels\label{sec:navpanel}}%
%\section{Customising the Navigation Panels\label{sec:navpanel}}%
\index{navigation panel|(}%
\index{navigation panel!customisable}\html{\\}%
The navigation panels are the strips containing ``buttons'' and text 
that appears at the top and perhaps at the bottom 
of each generated page and provides hypertext links 
to other sections of a document. 
Some of the options and variables that control whether and 
where it should appear are presented in \hyperref{an earlier section}%
{Section~}{}{navoptions}. 

\index{navigation panel!customised buttons}%
\index{htmladdtonavigation@\Lc{htmladdtonavigation}}\html{\\}%
A simple mechanism for appending customised buttons to the navigation
panel is provided by the command \Lc{htmladdtonavigation}. 
This takes one argument which \latextohtml{} appends to the navigation panel. 
For example,
%begin{latexonly}
\begin{small}
%end{latexonly}
\begin{verbatim}
\htmladdtonavigation{\htmladdnormallink
   {\htmladdimg{http://server/mybutton.gif}}{http://server/link}}
\end{verbatim}
%begin{latexonly}
\end{small}
%end{latexonly}
will add an active button \fn{mybutton.gif} pointing to the specified location.

\index{navigation panel!order of buttons}\html{\\}\noindent
Apart from these facilities it is also 
possible to specify completely what appears in the navigation panels
and in what order. As each section is processed, \latextohtml{}
assigns relevant information to a number of global variables.
These variables are used by the
subroutines \texttt{top\_navigation\_panel} and
\texttt{bottom\_navigation\_panel},
where the navigation panel is constructed as a string consisting of
these variables and some formatting information. 

\index{navigation panel!Perl subroutines@\Perl{} subroutines}\html{\\}\noindent
These subroutines can be redefined in a system- 
or user-configuration file (respectively, 
\fn{\$LATEX2HTMLDIR/latex2html.config} and \fn{\$HOME/.latex2html-init}). 
Any combination of text, \texttt{HTML} tags, 
and the variables mentioned below is acceptable.

\medskip\noindent
The control-panel variables are:\nobreak

\medskip\nobreak
\index{navigation panel!buttons}%
\index{navigation panel!iconic links}\html{\\}
\noindent
\textbf{Iconic links (buttons):}
\begin{htmllist}\htmlitemmark{RedBall}\addtolength{\leftskip}{15pt}%
\item [\fn{\$UP}]  Points up to the ``parent'' section;
\item [\fn{\$NEXT}] Points to the next section;
\item [\fn{\$NEXT\_GROUP}] Points to the next ``group'' section;
\item [\fn{\$PREVIOUS}] Points to the previous section;
\item [\fn{\$PREVIOUS\_GROUP}] Points to the previous ``group'' section;
\item [\fn{\$CONTENTS}] Points to the contents page if there is one;
\item [\fn{\$INDEX}] Points to the index page if there is one.
\end{htmllist}

\index{navigation panel!section titles}%
\index{navigation panel!textual links}\html{\\}\noindent
\textbf{Textual links (section titles):}
\begin{htmllist}\htmlitemmark{PinkBall}\latex{\addtolength{\leftskip}{15pt}}%
\item [\fn{\$UP\_TITLE}]  Points up to the ``parent'' section;
\item [\fn{\$NEXT\_TITLE}] Points to the next section;
\item [\fn{\$NEXT\_GROUP\_TITLE}] Points to the next ``group'' section;
\item [\fn{\$PREVIOUS\_TITLE}] Points to the previous section;
\item [\fn{\$PREVIOUS\_GROUP\_TITLE}] Points to the previous ``group'' section.
\end{htmllist}
If the corresponding section exists, each iconic button will contain an
active link to that section. If the corresponding section does
not exist, the button will be inactive. If the section corresponding
to a textual link does not exist then the link will be empty.

The ``next group'' and ``previous group'' are rarely used, 
since it is usually possible to determine
which are the next/previous logical pages in a document.
However these may be needed occasionally with segmented documents,
when the segments have been created with different values
for the \fn{\$MAX\_SPLIT\_DEPTH} variable. 
This is quite distinct from the segmented document effect 
in which the first page of one segment may have its `\emph{PREVIOUS}'
button artificially linked to the first page of the previous segment, 
rather than the last page.


\index{navigation panel!number of words}\html{\\}\noindent
The number of words that appears in each textual link
is controlled by the variable \fn{\$WORDS\_IN\_NAVIGATION\_PANEL\_TITLES}
which may also be changed in the configuration files.

\index{navigation panel!example}\html{\\}\noindent
Below is an example of a navigation panel for the bottom of \texttt{HTML} pages.
(Note that the ``\texttt{.}'' is \Perl{}'s string-concatenation operator 
and ``\texttt{\#}'' signifies a comment.)
%begin{latexonly}
\begin{small}
%end{latexonly}
\begin{verbatim}
sub bot_navigation_panel {
    #  Start with a horizontal rule and descriptive comment
    "<HR>\n" . "<!--Navigation Panel-->".
    # Now add a few buttons, with a space between them
       "$NEXT $UP $PREVIOUS $CONTENTS $INDEX $CUSTOM_BUTTONS" . 
    # Line break    
       "\n<BR>" .       
    # If ``next'' section exists, add its title to the navigation panel
       ($NEXT_TITLE ? "\n<B> Next:</B> $NEXT_TITLE" : undef) .   
    # Similarly with the ``up'' title ...
       ($UP_TITLE ? "\n<B>Up:</B> $UP_TITLE\n" : undef) . 
    # ... and the ``previous'' title
       ($PREVIOUS_TITLE ? "\n<B> Previous:</B> $PREVIOUS_TITLE\n" : undef) . 
}
\end{verbatim}
%begin{latexonly}
\end{small}
%end{latexonly}
Note that extra buttons may be included by defining suitable code for the container
\fn{\$CUSTOM\_BUTTONS}\,. 
The use of explicit `newline' (\verb|\n|) characters is not necessary for the
on-screen appearance of the navigation panel within a browser window. However 
it maintains an orderly organisation within the \texttt{.html} files themselves,
which is helpful if any hand-editing is later required, 
or simply to read their contents.
The corresponding subroutine for a navigation-panel at the top of a page
need not use the rule \HTMLtag{HR}, and would require a break (\HTMLtag{BR})
or two at the end, to give some visual separation from the following material.
\index{navigation panel|)}
\index{extensions|)}




